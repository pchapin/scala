
\section{Mininess}
\label{section-mininess}

To simplify our language design and to support strong type safety not
present in the full nesC programming language, we have defined a
reduced dialect of nesC we call \newterm{Mininess}. While Mininess
retains much of the expression and statement structure of nesC, it
simplifies nesC primarily in the way in which interfaces to modules
are defined and used.

In Mininess a module's uses-provides list can contain only bare
command declarations. Commands that are provided form the
\textit{exports} of the module, and commands that are used form the
\textit{imports} of the module. NesC-style events are not permitted
but can be encoded with suitably defined commands (e.g.~a used
interface event is semantically equivalent to a provided command). In
addition each module has a module type which is the signature of that
module's imports and exports taken together. Generic components are
supported at the Scalaness level as described in
\autoref{section-scalaness}. In particular, Mininess components
support a form of bounded parametric polymorphism \cite{FramedML}.

As an example, the modules in
\autoref{figure-mininess-message-builders} constitute part of a secure
messaging system. The \code{MessageBuilderC} module builds a message
consisting of a header and a block of message data. It then calls a
\code{computeToken} command to obtain an authorization token over the
completed message. The type \code{HeaderType} and value
\code{messageSize} are not declared in the Mininess code but instead
are provided during the execution of the first stage program using a
mechanism described in \autoref{section-scalaness}. Notice that
\code{AuthorizeC} does not require parameterization.

\begin{fpfig}[tbhp]{Mininess Message Building Modules}{figure-mininess-message-builders}
{
\begin{Verbatim}[fontsize=\small]
module MessageBuilderC {
  uses command computeToken(
    void *data, size_t size);
}
implementation {
  struct Message {
    HeaderType header;
    uint8_t    message[messageSize];
  };
  struct Message mes;
  ...
  call computeToken(&mes, sizeof(mes));
  ...
}


module AuthorizeC {
  provides command computeToken(
    void *data, size_t size);
}
implementation { ... }
\end{Verbatim}
}
\end{fpfig}

Mininess also defines a subtype relation between primitive integer
types in an obvious way and a restricted form of record width
subtyping. To make programming less error prone Mininess disallows
implicit conversions from wide integers to narrow integers or between
signed and unsigned integers.

% ... and also array stuff. We can leave the details for another day.

% TMI (Too Much Information)
% ==========================
%Mininess also defines subtype relations between the primitive integral
%types. Let \code{<:} indicate the subtype relation. In Mininess we
%have
%\vspace{1.0ex}
%\begin{Verbatim}[commandchars=\\\{\}, fontsize=\small]
%\centerline{int8_t  <: int16_t  <: int32_t }
%\centerline{uint8_t <: uint16_t <: uint32_t}
%\end{Verbatim}
%\vspace{1.0ex}
%
%The standard types \code{int} and \code{unsigned int} are treated as
%aliases for appropriate exact width types. The Scalaness compiler is
%configurable so that the aliases can be defined depending on the
%target platform.
%
%Mininess also provides a restricted form of record width subtyping; a
%pointer to structure $S_1$ is a subtype of a pointer to structure
%$S_2$ if and only if the member declarations of $S_2$ is prefix of the
%member declarations in $S_1$.
%
%Notice that there is no subtype relationship between the signed and
%unsigned types. Mininess does not provide any implicit conversions
%between signed and unsigned integers nor from a wide integer to a
%narrower integer. Such conversions are error prone.
%
%Mininess also forbids parameterized interfaces (since it forbids all
%interfaces) and generic components. These features are recovered in a
%more powerful form in the first stage where Scala's support for
%generic and parameterized classes can be used instead.
