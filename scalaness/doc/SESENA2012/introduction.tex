\section{Introduction}
\label{section-introduction}

It is well known that wireless sensor network (WSN) programming is
challenging because of the severe resource constraints imposed by
sensor network nodes. Often, WSN software benefits from iterative
refinement over time as it adapts to environmental and network
conditions. \newterm{Staged programming}
\cite{Taha-MetaML,Sheard-TemplateHaskell,Mainland-Flask-2008,FramedML}
is a technique that supports programmatic specification of code
deployment cycles and code refinement. In staged programming, code is
a datatype that can be dynamically modified, composed, and
specialized. Thus, one program ``stage'' can specify how to generate
code for ``later'' stages. This idea has been explored previously for
embedded systems programming, since it allows to obtain highly
efficient code based on dynamically-determined conditions
\cite{Taha03generatingheap-bounded,DBLP:conf/icess/Taha04}.

We visualize a \emph{two} stage programming system for wireless sensor
networks where the first stage runs possibly \emph{in the field} on a
relatively powerful base station or hand-held device. The first stage
could make use of inputs known only during deployment or by querying
the network after it has been deployed. For example, using information
about the actual the number of neighbors each node has, the first
stage program could optimize routing tables, or other neighbor
specific data structures, to the smallest possible size. The resulting
specialized node-level program could then be (re)deployed to the
network. This process could even be automated and executed
periodically to allow the network to adapt to changing conditions
while remaining highly optimized. In this way our system differs from
macroprogramming systems such as Kairos
\cite{springerlink:10.1007/1150259312}, and Regiment
\cite{Newton:2007:RMS:1236360.1236422} which support full network
programming but no facility for iteratively generating code.

In our envisioned scenario it is critical that in all cases the first
stage program generates a sensible, type correct residual program. The
user of the first stage program will not be in a position to fix, or
perhaps even understand, error messages produced by the second stage
compiler. This is particularly true in the case where the first stage
executes automatically without direct human oversight.

Existing staged programming systems are not suitable for this
application domain. Many systems assume that the first and second
stage languages are the same or that both stages execute in the same
address space. In our scenario the two stages are widely separated
with the first stage executing on a powerful machine under the control
of a conventional operating system and the second stage executing on
small embedded devices. The dramatic difference between those
platforms requires that significantly different programming languages
be used for the two stages.

Also in existing staged programming environments the type correctness
of second stage programs is verified either during the running of the
first stage program or, even later, during the compilation of the
second stage program \cite{Mainland-Flask-2008}. This is too late for
our envisioned sensor network applications. Instead the type
correctness of all possible residual programs must be verified during
the compilation of the first stage program. Previous work on $\langle
\text{ML} \rangle$, a staged language calculus, provides this
guarantee in the context of stages executing as separate processes
\cite{FramedML}. $\langle \text{ML} \rangle$ was developed explicitly
to provide a static type safety argument and theoretical foundation
for the more practical programming system introduced here.

In our system the programmer writes first stage programs in a dialect
of Scala, a powerful functional and object oriented hybrid language
for the Java Virtual Machine \cite{PiS2}. The first stage program
specializes and composes modules of code written in a simplified
dialect of nesC, a language commonly used on wireless sensor network
nodes \cite{Gay-nesC-2003}. We call our system \newterm{Scalaness}.

The remainder of this paper is organized as follows. In Sec.~II we
give an overview of Mininess, the language in which node-level module
code is written. In Sec.~III we describe the extensions to the Scala
programming language that allow Mininess modules to be specialized and
composed. In Sec.~IV we focus on how our system allows programs to be
constructed from a mixture of staged components and pre-built
components. In Sec.~V we discuss our implementation. Finally in
Sec.~VI we conclude.
