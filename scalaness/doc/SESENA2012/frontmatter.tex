\title{Dynamic Configuration of nesC Components in Scala}
\author{
% 1st. author
\IEEEauthorblockN{Peter Chapin}
\IEEEauthorblockA{The University of Vermont\\
Burlington, Vermont\\
Email: pchapin@cs.uvm.edu}
% 2nd. author
\and
\IEEEauthorblockN{Christian Skalka}
\IEEEauthorblockA{The University of Vermont\\
Burlington, Vermont\\
Email: skalka@cs.uvm.edu}
% 3rd. author
\and
\IEEEauthorblockN{Michael Watson}
\IEEEauthorblockA{The University of Vermont\\
Burlington, Vermont\\
Email: mpwatson@uvm.edu}
}
\maketitle

\begin{abstract}
  Building programs for constrained embedded devices is challenging
  due to severe limitations on processing speed, memory, and network
  bandwidth. Staged programming can help bridge the gap between high
  level code refinement techniques and device level programs by
  allowing a first stage program to programmatically specialize device
  level code. Here we introduce \textit{Scalaness}, a two stage
  programming system for wireless sensor networks. With Scalaness the
  first stage program is written in an extended dialect of Scala where
  components written in a reduced dialect of nesC are composed and
  specialized. This allows dynamic configuration of nesC components in
  Scala in a well-typed manner. We give an overview of Scalaness and
  focus particularly on how it allows second stage programs to be
  constructed from a mixture of generated components and library
  components.
\end{abstract}

\begin{IEEEkeywords}
staged programming, wireless sensor networks, scala;
\end{IEEEkeywords}
