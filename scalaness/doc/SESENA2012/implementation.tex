
\section{Implementation}
\label{section-implementation}

In this section we outline our implementation of Scalaness.

The standard Scala compiler is organized as a pipeline of
\textit{phases} that progressively lower Scala source code to JVM
bytecode. The compiler has an architecture that allows plugins to
inject new phases into that pipeline. The new phases have access to
internal compiler artifacts, such as the abstract syntax tree,
generated by the standard phases.

\autoref{figure-compiler-phases} shows the phase structure of the
Scala compiler with the Scalaness plugin activated.

\begin{figure}[htbp]
  \expandafter\ifx\csname graph\endcsname\relax
   \csname newbox\expandafter\endcsname\csname graph\endcsname
\fi
\ifx\graphtemp\undefined
  \csname newdimen\endcsname\graphtemp
\fi
\expandafter\setbox\csname graph\endcsname
 =\vtop{\vskip 0pt\hbox{%
    \special{pn 8}%
    \special{pa 0 3000}%
    \special{pa 1750 3000}%
    \special{pa 1750 200}%
    \special{pa 0 200}%
    \special{pa 0 3000}%
    \special{fp}%
    \special{pa 250 650}%
    \special{pa 1500 650}%
    \special{pa 1500 400}%
    \special{pa 250 400}%
    \special{pa 250 650}%
    \special{fp}%
    \graphtemp=.5ex
    \advance\graphtemp by 0.525in
    \rlap{\kern 0.875in\lower\graphtemp\hbox to 0pt{\hss Scala Parser\hss}}%
    \special{pa 250 1100}%
    \special{pa 1500 1100}%
    \special{pa 1500 850}%
    \special{pa 250 850}%
    \special{pa 250 1100}%
    \special{da 0.050}%
    \graphtemp=.5ex
    \advance\graphtemp by 0.975in
    \rlap{\kern 0.875in\lower\graphtemp\hbox to 0pt{\hss Scalaness to Scala\hss}}%
    \special{pa 250 1800}%
    \special{pa 1500 1800}%
    \special{pa 1500 1300}%
    \special{pa 250 1300}%
    \special{pa 250 1800}%
    \special{fp}%
    \graphtemp=\baselineskip
    \multiply\graphtemp by -1
    \divide\graphtemp by 2
    \advance\graphtemp by .5ex
    \advance\graphtemp by 1.550in
    \rlap{\kern 0.875in\lower\graphtemp\hbox to 0pt{\hss Scala\hss}}%
    \graphtemp=\baselineskip
    \multiply\graphtemp by 1
    \divide\graphtemp by 2
    \advance\graphtemp by .5ex
    \advance\graphtemp by 1.550in
    \rlap{\kern 0.875in\lower\graphtemp\hbox to 0pt{\hss Type Analysis\hss}}%
    \special{pa 250 2500}%
    \special{pa 1500 2500}%
    \special{pa 1500 2000}%
    \special{pa 250 2000}%
    \special{pa 250 2500}%
    \special{da 0.050}%
    \graphtemp=\baselineskip
    \multiply\graphtemp by -1
    \divide\graphtemp by 2
    \advance\graphtemp by .5ex
    \advance\graphtemp by 2.250in
    \rlap{\kern 0.875in\lower\graphtemp\hbox to 0pt{\hss Scalaness\hss}}%
    \graphtemp=\baselineskip
    \multiply\graphtemp by 1
    \divide\graphtemp by 2
    \advance\graphtemp by .5ex
    \advance\graphtemp by 2.250in
    \rlap{\kern 0.875in\lower\graphtemp\hbox to 0pt{\hss Type Analysis\hss}}%
    \special{sh 1.000}%
    \special{pn 1}%
    \special{pa 900 300}%
    \special{pa 875 400}%
    \special{pa 850 300}%
    \special{pa 900 300}%
    \special{fp}%
    \special{pn 8}%
    \special{pa 875 0}%
    \special{pa 875 300}%
    \special{fp}%
    \special{sh 1.000}%
    \special{pn 1}%
    \special{pa 900 750}%
    \special{pa 875 850}%
    \special{pa 850 750}%
    \special{pa 900 750}%
    \special{fp}%
    \special{pn 8}%
    \special{pa 875 650}%
    \special{pa 875 750}%
    \special{fp}%
    \special{sh 1.000}%
    \special{pn 1}%
    \special{pa 900 1200}%
    \special{pa 875 1300}%
    \special{pa 850 1200}%
    \special{pa 900 1200}%
    \special{fp}%
    \special{pn 8}%
    \special{pa 875 1100}%
    \special{pa 875 1200}%
    \special{fp}%
    \special{sh 1.000}%
    \special{pn 1}%
    \special{pa 900 1900}%
    \special{pa 875 2000}%
    \special{pa 850 1900}%
    \special{pa 900 1900}%
    \special{fp}%
    \special{pn 8}%
    \special{pa 875 1800}%
    \special{pa 875 1900}%
    \special{fp}%
    \special{sh 1.000}%
    \special{pn 1}%
    \special{pa 900 3100}%
    \special{pa 875 3200}%
    \special{pa 850 3100}%
    \special{pa 900 3100}%
    \special{fp}%
    \special{pn 8}%
    \special{pa 875 2500}%
    \special{pa 875 3100}%
    \special{da 0.050}%
    \graphtemp=.5ex
    \advance\graphtemp by 0.100in
    \rlap{\kern 0.875in\lower\graphtemp\hbox to 0pt{\hss Scalaness Source }}%
    \graphtemp=.5ex
    \advance\graphtemp by 3.100in
    \rlap{\kern 0.875in\lower\graphtemp\hbox to 0pt{\hss JVM Bytecode }}%
    \graphtemp=\baselineskip
    \multiply\graphtemp by -1
    \divide\graphtemp by 2
    \advance\graphtemp by .5ex
    \advance\graphtemp by 0.750in
    \rlap{\kern 2.150in\lower\graphtemp\hbox to 0pt{ Standard\hss}}%
    \graphtemp=\baselineskip
    \multiply\graphtemp by 1
    \divide\graphtemp by 2
    \advance\graphtemp by .5ex
    \advance\graphtemp by 0.750in
    \rlap{\kern 2.150in\lower\graphtemp\hbox to 0pt{ Phases\hss}}%
    \special{sh 1.000}%
    \special{pn 1}%
    \special{pa 1586 581}%
    \special{pa 1500 525}%
    \special{pa 1603 534}%
    \special{pa 1586 581}%
    \special{fp}%
    \special{pn 8}%
    \special{pa 2150 750}%
    \special{pa 1594 558}%
    \special{fp}%
    \special{sh 1.000}%
    \special{pn 1}%
    \special{pa 1582 1488}%
    \special{pa 1500 1550}%
    \special{pa 1544 1457}%
    \special{pa 1582 1488}%
    \special{fp}%
    \special{pn 8}%
    \special{pa 2150 750}%
    \special{pa 1563 1472}%
    \special{fp}%
    \graphtemp=\baselineskip
    \multiply\graphtemp by -1
    \divide\graphtemp by 2
    \advance\graphtemp by .5ex
    \advance\graphtemp by 1.900in
    \rlap{\kern 2.150in\lower\graphtemp\hbox to 0pt{ Scalaness\hss}}%
    \graphtemp=\baselineskip
    \multiply\graphtemp by 1
    \divide\graphtemp by 2
    \advance\graphtemp by .5ex
    \advance\graphtemp by 1.900in
    \rlap{\kern 2.150in\lower\graphtemp\hbox to 0pt{ Phases\hss}}%
    \special{sh 1.000}%
    \special{pn 1}%
    \special{pa 1537 1071}%
    \special{pa 1500 975}%
    \special{pa 1578 1042}%
    \special{pa 1537 1071}%
    \special{fp}%
    \special{pn 8}%
    \special{pa 2150 1900}%
    \special{pa 1557 1057}%
    \special{fp}%
    \special{sh 1.000}%
    \special{pn 1}%
    \special{pa 1600 2225}%
    \special{pa 1500 2250}%
    \special{pa 1576 2181}%
    \special{pa 1600 2225}%
    \special{fp}%
    \special{pn 8}%
    \special{pa 2150 1900}%
    \special{pa 1588 2203}%
    \special{fp}%
    \hbox{\vrule depth3.200in width0pt height 0pt}%
    \kern 2.150in
  }%
}%

  \centerline{\raise 1em\box\graph}
  \caption{Scalaness Compiler Phases}
  \label{figure-compiler-phases}
\end{figure}

The Scalaness to Scala conversion phase exists to handle
\code{Typedef} instances that are used as types and type parameters to
Mininess modules. The conversion phase replaces such instances with a
token type that will be acceptable to the Scala type checker.
Additional information about the \code{Typedef} instance is recorded
for later use by the Scalaness type checker.

The Scalaness type checker also parses and type checks the Mininess
code associated with Scala \code{NesCComponent} classes. For each
Mininess module the initial type environment includes the parameters
of the module declared in the Scalaness program. The type checking is
done before specialization; if it succeeds all instances of the module
will be type correct \cite{FramedML}.

The Scalaness type checker phase is also responsible for interface
unwrapping, as described in section
\autoref{section-external-libraries}, and performs additional type
checking of the Scalaness code based on the module types it infers.
Notice that Scalaness type checking is done after Scala type checking.
This simplifies the implementation by allowing the pure Scala parts of
the program to be type checked using the existing, well tested Scala
type checker. Scalaness typing is strictly more precise than Scala
typing; a program that is Scalaness-typable is guaranteed to be
Scala-typable.

Finally a runtime environment, made available through a library jar
file, provides support for component composition. The runtime
environment also performs the actual work of specializing the nesC
code and outputting the residual program.

Progress on the implementation is ongoing. Currently the type checking
of Mininess components, using type and term parameters provided by the
Scalaness program, is complete. Mininess module types are inferred and
are made available to the Scalaness type checker. Interface unwrapping
as described in \autoref{section-external-libraries} is partially
complete. Type checking at the Scalaness level is currently being 
implemented.
