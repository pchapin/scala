
\section{Scalaness}
\label{section-scalaness}

In this section we describe the first stage language as an extension
of Scala. We do not intend to restrict Scala in any way; the first
stage programmer has full access to the entire language and supporting
libraries. We extend the language by providing a way for the first
stage programmer to declare, specialize, and manipulate Mininess
components.

\subsection{nesC Mappable Types}

A Scalaness program manipulates values that will ultimately be used as
parameters to generic Mininess modules. These values cross the
boundary between Scalaness and Mininess and are subject to special
handling by the system. We define a \newterm{nesC mappable type} as a
Scala type that has a corresponding nesC type. In a Scalaness program
all nesC mappable types must be subtypes of the marker trait
\code{NesCType}. NesC mappable types are also first class values in
Scalaness programs as further discussed in section
\autoref{section-first-class-types}.

The primitive nesC types are mapped from Scalaness by classes with
names such as, for example, \code{Int32}, \code{UInt16}, and
\code{Void}. Instances of these classes can be operated on in the
expected way, allowing the Scalaness program to compute values that
will be passed to the second stage Mininess program.

Structure types in Mininess are mapped from Scala classes and can be
defined by the programmer. A user defined class $C$ is nesC mappable
if it obeys the following inductive rules.
\begin{enumerate}
\item $C$ is not generic.
\item $C$ is a subtype of \code{NesCType}.
\item All of $C$'s fields are nesC mappable.
\item All of $C$'s supertypes (except for \code{AnyRef} and
  \code{ScalaObject}) are nesC mappable.
\end{enumerate}

There are otherwise no restrictions on the definition or use of nesC
mappable classes in the Scalaness program. In particular, they are
able to have convenience methods, although the methods of a nesC
mappable class have no manifestation in the generated nesC code.

\autoref{figure-nesc-mappable-types} shows two nesC mappable classes
that correspond to Mininess structures used by the secure messaging
system introduced in \autoref{figure-mininess-message-builders}. The
\code{Header} class contains a TinyOS node ID and the identifier of a
component on that node where the message will be delivered. The
\code{TimeStampedHeader} subclass includes a time stamp on the header.

\begin{fpfig}[tbhp]{nesC Mappable Class Types}{figure-nesc-mappable-types}
{
\begin{Verbatim}[fontsize=\small]
class Header(
  nodeID     : nodeIDType,
  componentID: UInt8) extends NesCType

class TimeStampedHeader(timeStamp: UInt16)
  extends Header
\end{Verbatim}
}
\end{fpfig}

The type used for the node ID, \code{nodeIDType}, is computed during
the execution of the first stage program as described in
\autoref{section-first-class-types}.

\subsection{nesC Components}
\label{section-nesc-components}

A Scalaness program treats Mininess components as entire units. Each
component is parameterized by upper bounded types or by typed values,
which allows module genericity to be expressed. The types used with
these parameters must be nesC mappable types.

A Mininess module is represented in a Scalaness program as a Scala
class or object that extends the marker trait \code{NesCComponet}.
Type and value parameters of the module are provided using the usual
Scala syntax; a Scala object is only permitted when the represented
Mininess module is not generic.

The last expression in the class's (or object's) primary constructor
must be a string literal that names a file containing the Mininess
code of the represented module.
\autoref{figure-scalaness-message-builders} shows the Scalaness
representation of the Mininess modules in
\autoref{figure-mininess-message-builders}.

\begin{fpfig}[tbhp]{Scalaness Message Building Modules}{figure-scalaness-message-builders}
{
\begin{Verbatim}[fontsize=\small]
class MessageBuilderC[HeaderType <: Header]
  (messageSize: UInt8)
    extends NesCComponent
{ "MessageBuilderC.nc" }

object AuthorizeC extends NesCComponent
{ "AuthorizeC.nc" }

\end{Verbatim}
}
\end{fpfig}

The \code{HeaderType} type parameter and \code{messageSize} value
parameter are used in \code{MessageBuilderC}'s Mininess code. Each
instance of the class corresponds to a separate instance of the
module, appropriately specialized. The \code{AuthorizeC} module, in
contrast, is not generic and can be represented by a singleton Scala
object directly.

\autoref{figure-module-instantiation} illustrates how generic Mininess
modules can be instantiated using Scala's ordinary syntax for doing
so. For illustrative purposes the payload size is dynamically
specified in one case and statically in the other.

\begin{fpfig}[tbhp]{Scalaness Module Instantiation}{figure-module-instantiation}
{
\begin{Verbatim}[fontsize=\small]
val basicBuilder =
  new MessageBuilderC[Header](payloadSize)

val timeStampedBuilder =
  new MessageBuilderC[TimeStampedHeader](16)
\end{Verbatim}
}
\end{fpfig}

The Scalaness compiler infers the module type for the instances. Note
that there is currently no syntax for Scalaness module types in
Scalaness programs. This imposes some limitations on how modules can
be used since there are places in Scala where type annotations are
necessary.

\subsection{Component Composition}
\label{section-component-composition}

NesC programs consist of components that are wired together in
\newterm{configurations}. Like a module a configuration also has
imports and exports, and thus a Scalaness module type, but it is
implemented by connecting together other components. Scalaness allows
two components with module types to be composed with the \code{+}
operator. This creates a new configuration from the components and
joins exports of one component to the compatible imports of the other.
It is important to note that it is a type error if such a composition
can't be done.

\autoref{figure-component-composition} shows the basic builder module
being connected to the authorizer component. The resulting
configuration is then composed with the timestamped builder module.
Note that the exports of \code{AuthorizeC} are still available for
connection after composition allowing both message builders to use the
same authorizer.

\begin{fpfig}[tbhp]{Component Composition}{figure-component-composition}
{
\begin{Verbatim}[fontsize=\small]
val configuration =
  timeStampedBuilder +
    (basicBuilder + AuthorizeC)

validate: configuration
\end{Verbatim}
}
\end{fpfig}

The \code{validate:} method applied to a component causes the
specialized Mininess program to be generated, along with any
supporting nesC configurations required. Following the theory
developed in \cite{FramedML} if the \code{validate} expression type
checks then the generated nesC program will be type correct.

\subsection{Mappable Type Computation}
\label{section-first-class-types}

An important feature of Scalaness is its ability to treat nesC
mappable types as first class values dung the execution of the first
stage. This allows a Scalaness program to dynamically compute which
types are most appropriate to use in the second stage code. Type
computation in the first stage is done with the help of instances of
the \code{Typedef} class defined in \autoref{figure-typedef}.

\begin{fpfig}[tbhp]{Representation of a nesC Type}{figure-typedef}
{
\begin{Verbatim}[fontsize=\small]
class Typedef[+T <: NesCType]
  (t: CovariantClass[T]) extends NesCType

class CovariantClass[+T](t: Class[T])
\end{Verbatim}
}
\end{fpfig}

Each \code{Typedef} instance wraps a normal Java \code{Class} instance
representing a nesC mappable type. Its type parameter is an upper
bound on the types that a particular instance may wrap. A
\code{Typedef} instance may then be used syntactically as a type in
Scalaness to, for example, specify the type of a field in a nesC
mappable class as in \autoref{figure-nesc-mappable-types}, or passed
as a type argument to a Mininess module.

\autoref{figure-compute-with-types} shows an example of a node ID type
being selected based on dynamic information about the size of the
network. Note that there is an implicit conversion from \code{Class}
to \code{CovariantClass}.

\begin{fpfig}[tbhp]{Dynamic Computation with nesC Types}{figure-compute-with-types}
{
\begin{Verbatim}[fontsize=\small]
val nodeIDType = new Typedef[UInt16](
  if (topology.size >= 256)
    Class[UInt16]
  else
    Class[UInt8]
)
\end{Verbatim}
}
\end{fpfig}
