\section{The Scalaness Language}
\label{section-scalaness}

Scalaness serves as the language for nesT module composition in the same manner as nesC
configurations serve to compose nesC modules, but Scalaness is a more powerful metalanguage
since modules are treated as a new category of first class values in Scalaness. Instantiation,
composition (aka wiring), and imaging of modules are defined as operations on module values.
Because instantiation of modules with both types and values is allowed, values and types may
migrate from the Scalaness level to the nesT level after programmatic refinement, realizing a
disciplined form of code specialization.

%\snote{Scalaness formalized vs Scalaness implemented was not clear,
%  reworded this to bring out the difference.  May need some polishing,
%  take a pass at it.}  
Our goal in this Section is to describe the Scalaness syntax and semantics realized in our
implementation, and justify our claims of type safety. Since Scala as implemented is too large
to easily formalize, we formalize here a subset of Scala expressed as an extension of
Featherweight Java \cite{FJ}, a core calculus subsumed by Scala. A formalized core calculus and
type analysis for Scala exists \cite{Cremet:2006:CCS:2135978.2135980}, but FJ is adequate and
simpler. We thus elide many features of Scala in the Scalaness formalization, but we adapt all
Scala features unchanged in our Scalaness implementation. Here we primarily focus on the module
metaprogramming operations that we add. The presentation here ``cleans up'' some implementation
details, but is otherwise an accurate description of the module operation semantics and
especially the module operation typing rules.

% Scalaness type checking is grounded in a previous foundational
% work, as discussed in \autoref{section-framedml}. The type checking
% algorithm presented here statically supports type safety at both
% programmatic levels, i.e.~if a Scalaness program type checks, then
% running it will not cause a type failure, \emph{and} residual nesT
% code will also be type-safe when run on its own. 

\subsection{Syntax of Scalaness}

\scalanesssyntaxfig The Scalaness language syntax is presented in
\autoref{figure-scalanesssyntax}. To represent an adequate core calculus of Scala, it subsumes
two Featherweight Java variants: Featherweight Generic Java (FGJ) \cite{FJ} and Assignment
Featherweight Java (AFJ) \cite{AFJ}. The generic class types of FGJ are needed to model type
construction, and the mutation in AFJ is essential to consider since one of our main concerns is
nesT code specialization; nesT programs are run in a separate process space, so specialization
with stateful values, a likely common idiom in a Scala setting, presents a challenge.

We refer the reader to \cite{FJ,AFJ} for details on the FGJ and AFJ object oriented calculi,
which are represented in the languages of class definitions, constructors, methods, and the
first line of expression forms defined in \autoref{figure-scalanesssyntax}. Scalaness extends
these features with a typed variable declaration form $\jdef{x}{T}{e_1}{e_2}$ where the scope of
$\ttt{x}$ is $\tt{e_2}$, a dynamic type construction form $\jtlet{x}{T}{e_1}{e_2}$ with similar
scoping rules, and several features for module definition and manipulation. First, we include
nesT modules $\jmodval$ in the Scalaness expression and value spaces: instantiation is obtained
via the form $\jinst{e_1}{\ttvec{e}_1; \ttvec{e}_2}$, where $\ttvec{e}_\tt{1}$ are type
parameters and $\ttvec{e}_\tt{2}$ are value parameters. Wiring of modules is denoted
$\jwire{\tt{e_1}}{\tt{e_2}}$. Imaging of modules, denoted $\jimage{\tt{e}}$, ensures that
$\tt{e}$ computes to a runnable module, in the sense of \autoref{def-runnable}.

\subsection{Semantics of Scalaness}

The semantics of Scalaness is an extension of the semantics of AFJ and FGJ to incorporate nesT
modules and operations. Computations assume a fixed class table $\CT$ allowing access to class
definitions via class names, which always decorate an object's type. A \emph{store} $\jstore$ is
a function from memory locations $\tt{p}$ to object representations. Objects are represented in
memory by lists of object references $\ttvec{l}$, which refer to the locations of the objects
stored in mutable field values. A reference $\tt{l}$ is a pair $\jref{\tt{p}}{N}$ where $\tt{p}$
is the memory location of an object representation and $\tt{N}$ is the nominal type of the
object, including its class name. Hence, given an object reference
$\jref{\tt{p}}{\jinst{C}{\ttvec{T}}}$, we can access and mutate its fields $\ttvec{l} =
\jstore(\tt{p})$, and access and use its methods via the definition $\CT(\tt{C})$.

Following AFJ, the semantics of Scalaness is defined as a \emph{labeled transition system},
where transitions are of the form $\tt{e} - \lc s = \jstore, s' = \jstore' \rc \rightarrow
\tt{e}' $. Intuitively, this denotes that given an initial store $\jstore$ and expression
$\tt{e}$, one step of evaluation results in a modified store $\jstore'$ and contractum
$\tt{e}'$. We write $\tt{e} \rightarrow \tt{e}'$ as an abbreviation when the store is not
altered.
% space saver - SS
%  When discussing the
% semantics of module operations in Scalaness we will use this
% abbreviated form, since they are purely functional.

% We refer the reader to \cite{XXX} for details on the semantics of AFJ
% and FGJ subsumed by Scalaness.
The primary novelty of Scalaness over FGJ/AFJ is the formal semantics of type and module
construction. We begin with type construction, which is provided to allow programmers to
dynamically construct module type instances. The appropriate behavior is obtained by treating
dynamically constructed types as extensions of a basic class of objects, and declarations of
nesT level types via a \tt{typedef} construct as syntactic sugar for ordinary object
construction. We define a \tt{LiftableType} class as the supertype of all types of objects which
can be used to instantiate a module, and dynamically constructed types are defined as instances
of a generic \tt{MetaType} class.
\begin{definition}
  Any Scalaness class table $\CT$ comprises the following definitions:
$$
\begin{array}{l}
\CT(\tt{LiftableType}) = \gclass{\tt{LiftableType}}{}{Object}{\ldots}\\
\CT(\tt{MetaType}) = \gclass{\tt{MetaType}}{\tt{X <: LiftableType}}{Object}{ \ldots }
\end{array}
$$
And we take as given the following syntactic sugar:
$$
\jtlet{x}{T}{e_1}{e_2}\ \defeq\ \jdef{x}{\jinst{MetaType}{T}}{e_1}{e_2}
$$
Class type $\tt{MetaType}$ is generalized on a single type variable. For brevity of 
notation, we define:
$$
\jinst{\tt{MetaType}}{\ttvec{T}}\quad \defeq \quad \overline{\jinst{MetaType}{T}}
$$
\end{definition}
A crucial fact of Scalaness type construction is that any dynamically constructed type cannot be
treated as a type at the Scalaness level. This is a more restrictive mechanism than envisioned
in our foundational model \cite{FramedML,FramedMLworkshop}, however it allows us to define
Scalaness as a straightforward extension to Scala, especially in terms of type checking.

Module instantiation is the only point where specialization of nesT modules is allowed.
% , since non-local variables are disallowed in module
% definitions
Since Scalaness and nesT are two different language spaces, some sort of transformation must
occur when values migrate from Scalaness to nesT via module instantiation. This \emph{lifting}
transformation involves both data mapping and serialization since the process spaces also
differ. We aim to be flexible and allow the user to specify how values are lifted and how types
are transformed. We only require that lifting and type transformation are coherent, in the sense
that the lifting of an object should be typeable at the object's type transformation. We
formalize this in the following definition.% Note that lifting is defined on references,
% since this is how objects are referred to at run-time and we never
% want to allow lifting of modules.
\begin{definition}
\label{def-lifting}
We assume given a relation $\ser{\jstore}$ which transforms a Scalaness reference $\tt{l}$ into
nesT declarations $\vect{d}$ and expression $e$. We also assume given a Scalaness-to-nesT
transformation of types $\codt{\cdot}$. To preserve type safety, we require in all cases that
$\jref{p}{N} \ser{\jstore} \vect{d}, e$ implies both of the following for some type environment
$G$:
$$
\varnothing, \varnothing \vdash \vect{d} : \tenv \qquad \text{ and} \qquad
 \tenv, \varnothing \vdash e : \codt{\tt{N}} 
$$
\end{definition}
The full definition of serialization and an example are given and discussed below in
\autoref{section-serialization}.
% \snote{check that cut here is OK, just trying to save space.}
%  In brief, when a
% module $\mu$ is instantiated, serialization will bind the value
% parameters of $\mu$ to the lifted values of their instances in a
% series of declarations that are added to its own. This is specified in
% the \TirName{ModInst} rule in \autoref{figure-jmodsemantics}. Another
% important detail of the \TirName{ModInst} rule is that only type
% information in type parameters is used, and migrates into the module
% via type transformation and ordinary substititution.  

\jmodsemanticsfig

Module wiring is given a standard component composition semantics. We only allow wiring of
instantiated modules, which is consistent with nesC and simpler to implement. In a wiring
$\jwire{\tt{e}_1}{\tt{e}_2}$, the exports of $\tt{e}_1$ are wired to the imports of $\tt{e}_2$.
This is specified in the \TirName{ModWire} rule in \autoref{figure-jmodsemantics}, which relies
on the following auxiliary definition of operations for combining mappings.
% space saver - SS
%  of various
% sorts (e.g.~lists of imports and exports)
\begin{definition}[Special Mapping Operations]
  Let $m$ range over vectors with mapping interpretations, in particular~$\tpdecl$, $\vpdecl$,
  $\imports$, and $\exports$. Binary operator $m_1\maploosemerge m_2$ represents (non-exclusive)
  map merge, i.e.~$m_1 \maploosemerge m_2 = m_1 @ m_2$ with the requirement that $\identifier\in
  \dom(m_1)\cap\dom(m_2)$ implies $m_1(\identifier) = m_2(\identifier)$.
%Binary operator $\mu_1\uplus\mu_2$
%represents mutually exclusive map merge; for this operation to be
%defined it must be that $\dom(\mu_1)\cap\dom(\mu_2) = \emptyset$.
%Given $\vect{\mu} = [\mu_1,\ldots,\mu_n]$ we write
%$\maploosemerge\vect{\mu}$ to denote $\mu_1 \maploosemerge \cdots
%\maploosemerge \mu_n$, similarly $\uplus\vect{\mu}$. 
  The mapping $m / S$ is the same as $m$ except undefined on domain elements in set $S$, and the
  mapping $\restrict{m}{S}$ is the same as $m$ except undefined on elements not in ${S}$.
\end{definition}
Finally, the $\TirName{ModImage}$ rule in \autoref{figure-jmodsemantics} shows that imaging it
is an assertion requiring its arguments to be a runnable module.

\begin{exmp}
\label{example-scalanesssemantics}
Assume given the definitions in \autoref{figure-example}, a module $\tt{radioC : radioT}$ and an
invocation $\tt{nodeSpecialize}(1,2,50,\tt{radioC})$. Then $\tt{ncode} \ltimes \tt{scode}$ will
evaluate to the following module:
$$
\margs{}\lc\ ; \ldots; \tt{error\_t\ \ main()} \lc \tt{call}\ \ \tt{send}((\tt{uint8})1, (\tt{uint8})2, "\tt{hello}") \rc\rc
$$  
where the elided declarations include a specialized radio and message type:
$$
\tt{error\_t\ \ radio(\lc \tt{src : uint8}; \tt{dest : uint8}; \tt{data : uint8[\,]} \rc)} \lc \ldots \rc
$$
\end{exmp}

\subsection{Serialization and Lifting}
\label{section-serialization}

Serialization generates a flattened nesT source code version of a Scalaness object in memory. At
the top level, serialization binds the value parameters of a module to the results of
flattening, aka lifting, via a sequence of declarations. Here is the precise definition.
\begin{definition}[Serialization]
\label{def-serialization}
Assume given a store $\jstore$ which is implicit in the following definitions. We define
serialization of Scalaness references as follows, along with an extension of the user defined
lifting relation to sequences of references:
\begin{mathpar}
\inferrule%[Serialize]
{\vect{\tt{l}} \ser{\bm} \vect{\decl},\vect{e}}
{\serialize(\vect{x}, \vect{\t}, \vect{\tt{l}}) = \vect{\decl} @\ {\vect{\t}\ \vect{x} = \vect{e}}}

\inferrule
{}
{\varnothing \ser{\jstore} \varnothing, \varnothing}

\inferrule
{\tt{l} \ser{\jstore} \vect{d}, e \\ \ttvec{l} \ser{\jstore} \vect{d'}, \vect{e}}
{\tt{l}\vect{\tt{l}} \ser{\jstore} \vect{d} @ \vect{d'}, e\vect{e}}
\end{mathpar}
\end{definition}
Although lifting is user defined, a standard strategy is to introduce a new declared variable
for each memory reference in the lifted object, and bind the variable to the lifted referent.
Hence, lifting will typically be defined recursively. In our implementation, we have adapted a
``default'' lifting which follows this strategy, and also transforms objects by just
transforming the fields into a representative struct, and ignoring methods. We will illustrate
this with an example in \autoref{section-implementation}.
% , which also shows how base values such 
% as integers are lifted
We can formally capture the essence of this transformation with the following definitions. It is
easy to see that these definitions will satisfy the requirements of \autoref{def-lifting}.
\begin{exmp} In this example we allow lifting of any object references,
and transform the object $o$ into a structure containing the transformed 
fields of $o$. Methods are disregarded by the transformation. Here is the 
specification of the type transformation:
\begin{mathpar}
\inferrule[ChapinT] {\CT(\ttt{C}) =
  \gclass{C}{\gbounds{X}{S}}{N}{\fieldvec{R}{f};\ K\ \ttvec{M}}}
          {\codt{\jinst{C}{\ttvec{T}}} = \lc \vect{\ttt{f}} :
            \codt{\ttvec{R}[\ttvec{T}/\ttvec{X}]} \rc}
\end{mathpar}
and here is the specification of lifting.
\begin{mathpar}
\inferrule[Chapin]
{\jstore(\tt{p}) = \ttvec{l} \\ \fields{C} = \tdecls{T}{f} \\ 
 \ttvec{l} \ser{\jstore} \vect{d}, \vect{e} \\ x \ \text{ fresh}} 
{\jref{p}{\jinst{C}{\ttvec{R}}} \ser{\jstore} \vect{\decl} 
   @ (\xlet{\codt{\jinst{C}{\ttvec{R}}}}{x}{\lc \vect{\ttt{f}} = \vect{e}\rc}) , x}
\end{mathpar}
\end{exmp}


\subsection{Scalaness Type Checking}
\label{section-typing}

\scalanesstypingfig
% boilerplate cut to save space - SS
% The Scalaness Type judgements are of the form $\Gamma \vdash \tt{e} :
% \tt{T}$, where $\Gamma$ is a free variable typing environment and
% judgement validity is obtained via a deduction system. 
We adapt the typing rules of FGJ in their entirety, and refer the reader to \cite{FJ} for
relevant details. Since type construction via \texttt{typedef} is syntactic sugar for normal
object construction, that is covered by those rules as well. It remains to define typing rules
for nesT modules and module operations.

%\begin{theorem}[AFGJ Type Safety]
%If $\varnothing \vdash \tt{e} : \tt{N}$ and $\tt{e} \rightarrow^*
%\tt{e}'$ with $\tt{e}'$ irreducible\footnote{This is defined as ``normal
%form'' in \cite{XXX}.}, then $\tt{e}'$ is either a
%reference $\jref{\tt{p}}{N'}$ with $\tt{N'} <: \tt{N}$, or an
%expression containing $(\tt{N}')\tt{\jref{\tt{p}}{N''}}$ where
%$\tt{N''}\not<:\tt{N}'$.
%\end{theorem}

The nesT module type form at the Scalaness level is $\jmodt{\tpdecl}{\jmodtcat}$, where
$\jmodtcat$ is a nesT module type. The $\tpdecl$ in this form represents the type bounds of
dynamically constructed types that have been used to instantiate the module; we refer to this
part of the type as the \emph{instance coercion}. Because these types are dynamically
constructed, their identity is not known statically, hence the need to treat them as
upper-bounded type names in the static type analysis. It is important to note that the type
names in $\tpdecl$ will be fully resolved at run time, so that any module generated by a
Scalaness program execution will have a fully reified nesT type, i.e.~an empty instance
coercion.

This is reflected in the \TirName{ModT} rule in \autoref{figure-scalanesstyping}, which connects
the nesT typing system with the Scalaness type system. Since in this case we are typing an
uninstantiated module definition its instance coercion is empty. An instance coercion in a
module type is directly populated when a module is instantiated, as in the \TirName{ModInstT}
rule. Here, the type instances $\ttvec{e}_1$ are all dynamically constructed, so they define the
upper bounds of the instantiated module's instance coercion. We also expect all type and value
parameters to respect the typing bounds specified in the module definition.
%A subtle but significant detail in this rule is the
%consequence of dynamically constructed types having no meaning ``as
%types'' at the Scalaness level. This means that no Scalaness value of
%that type can be constructed, so dynamically constructed type names do
%not occur in the typings of value parameters. This detail is of
%practical importance, and will be discussed at greater length in
%\autoref{section-futurework}.
The \TirName{ModWireT} typing rule for module wiring is a straightforward reflection of the
operational rule for module wiring, as is the \TirName{ModImageT} rule for module runnability
imaging.
\begin{exmp}
\label{example-scalanesstyping}
Returning to the code in \autoref{figure-example}, we may assign the following typing, where the
relevant type environment $G$ contains typing for free variables within the function
\tt{nodeSpecialize}:
$$
  G \vdash \jinst{radioC}{adt}\ :\  
    \jmodt{\tt{adt} \subtype \tt{int32}}{\margs{}\lc\ ;
      \tt{error\_t\ radio(\t)} \rc }
$$
$$
\text{where}\ \t = \lc \tt{src : adt}; \tt{dest : adt}; \tt{data : uint8[\,]} \rc 
$$
%and
%$$    
%  G \vdash \jinst{sendC}{adt} \ltimes \jinst{radioC}{adt}\ :\   
%    \jmodt{\tt{adt} \subtype \tt{int32}}{\margs{}\lc \ 
%      ;\,
%      \tt{error\_t\ send(\tt{adt\ s}, \tt{adt\ d}, \tt{data\ uint8[\,]})} \rc}
%$$
%$$
%\text{where}\ \t = \lc \tt{src : adt}; \tt{dest : adt}; \tt{data : uint8[\,]} \rc 
%$$
\end{exmp}

\subsection{Foundational Insights and Type Safety} 
\label{section-framedml}

Type checking of modules and operations is inspired by the type theory and metatheory developed
for the language \fml\ \cite{FramedML}. Scalaness module instantiation in particular can be
decomposed into a set of \fml\ operations, and typeablity of module instantiation follows from
the typeablity of their composition. The language \fml\ is obtained by extending system \fsub\
with state, dynamic type construction, and staging features. The expression $\langle e \rangle$
is a code value, and the $\mathrm{lift}$ operation takes a value at one stage and ``lifts'' it
to the next, by turning it into code and performing any necessary serialization.

Given this, a Scalaness module with a value and type parameter can be modeled in \fml\ as a
term:
$$\lambda x : \s_1 . \Lambda t \subtype \s_2 . \langle e \rangle$$
where $x$ and $t$ are value and type parameters for the block of code $\langle e \rangle$. Then,
module instantiation can be modeled as the application of this term to a type and value
parameter, where the latter must be lifted into the next stage:
$$
(\lambda x : \s_1 . \Lambda t \subtype \s_2 . \langle e \rangle)\ (\mathrm{lift}\ e)\ \t
$$ 
% This interpretation of modules and module operations for the
% purposes of typing is evidenced by the Scalaness type form
% $\jmodt{\tpdecl}{\jmodtcat}$, where $\tpdecl$ defines the type bounds
% for dynamically constructed types used to instantiate a module. This
% is directly analogous to $\exists$ type bindings in $\fml$ types,
% which statically define the upper bounds of dynamically constructed
% types. 

Observing that AFJ, FGJ, and \fml\ are all proven type safe, and that Scalaness is in essence an
orthogonal composition of these three languages, we conjecture that type safety is maintained in
this composition.
% Furthermore, since static type safety in \fml\ implies
% type safety of all computational stages, we expect that static type
% safety in Scalaness implies type safety of any residual nesT code in
% Scalaness computations. This conjecture can be stated formally as
% follows.

\begin{conject}[Scalaness Type Safety]
  If $\varnothing \vdash \tt{e} : \tt{T}$ and $\tt{e} \rightarrow^* \jimage{\mu}$, then $\mu$ is
  runnable and does not go wrong.
\end{conject}

