\section{Conclusion}
\label{section-conclusion}

We have introduced Scalaness/nesT, a two stage programming system for wireless sensor networks.
Our system provides a powerful programming environment for dynamically specializing and
composing nesC modules in a type safe way; any type correct Scalaness program will generate only
type correct residual programs.

\subsection{Related Work}
%\label{sec:related}
%\label{section-related}
We do not review the broader topic of sensor network programming here; the reader is referred to
\cite{Mottola:2011:PWS:1922649.1922656} for a broader perspective.

The potential of applying metaprogramming to sensor networks was explored in the functional
sensor language Flask \cite{Mainland-Flask-2008}. The main motivation for designing Flask was to
allow FRP-based
%\cite{frp} -- trimmed this reference to save space
stream combinators to be pre-computed before network deployment. The Flask designers did not
focus on computing precise types for the object stage code at the meta stage, so cross-stage
static type checking is not performed -- it is possible to generate ill-typed Flask object code.
MetaML \cite{Taha-MetaML,DBLP:conf/icess/Taha04} and MetaHaskell \cite{mainland12} have each
been promoted as effective foundations for embedded systems programming with type safety, but
neither addresses type specialization or dynamic type construction. MetaHaskell does support
heterogenous language staging, the lower stage language is defined by a plug-in and several
instantiations have been defined including one for a low-level C-like language.
%   Like our approach they guarantee type safety of all
% lower-stage code produced.  They use a more traditional
% metaprogramming model, however, not the process-separated model
% needed for sensor metaprogramming and do not address the issues of
% metaprogramming module composition and type specialization.
\snote{Tweaked the above MetaHaskell remarks.}

Actor based sensor metaprogramming has been studied in \cite{cheong07}; this work shares our
focus on high level dynamic reprogrammability but is untyped. More broadly, meta programming is
known to be useful for increasing the efficiency of systems applications. One example is Tempo
\cite{289140}, a system that integrates partial evaluation and type specialization for
increasing efficiency of systems applications. Ur \cite{UrPLDI10} allows for type safe meta
programming for web applications. We refer readers to our $\lb \textrm{ML} \rb$ paper
\cite{FramedML} for a more complete review of meta programming systems/calculi.

%foundational meta programming calculi such as based on modal
%logic \cite{pfenningjacm}, higher-order abstract syntax
%\cite{guarded-xi}, and first-order abstract syntax with deBruijn
%indices \cite{xi-metatypeful}. 

The units of staged code composition in nesT programming are \emph{modules}. Countless different
module systems exist, but they are primarily designed to achieve separate compilation and sound
linking
\cite{Cardelli-1997}. %\dnote{removed several words here}%, and to not be composed at runtime.
Our different design goal leads to different design choices in nesT modules. For example, data
crossing nesT module boundaries needs to conform to the property of process separation, a
non-issue in standard module system designs. In addition, nesT modules allow values/types across
the boundary of modules to be flexibly constructed, including dynamic construction of types, to
achieve maximal flexibility of cross-stage specialization. Module systems such as ML modules
\cite{macqueen84} %,ocaml-modules
and Units \cite{flatt98units} allow types to be imported/exported as we also support; there are
several features of ML modules including type hiding that we do not aim to
support. % This feature can be encoded by
% nesT modules: a Units-style type import can be encoded as a nesT
% formal type parameter, and a type export as real arguments applied to
% the module instantiation, and module linking as module instantiation.
nesT modules are more expressive in their support of first class modules as values and the
possibility of dynamic construction of ``type exports.'' That said, first class modules are not
new \cite{99620,ancona01calculus}, % cut: lsecoop2004,Shields01first-classmodules,
we only claim novelty in their application to program staging and the incorporation of dynamic
type construction.
%\snote{David, I reworded
%  this pgph.  Please check to make sure I didn't falsify any claim.  I
%  removed the details of how to do the units embedding, the committee members
%  are more ML people anyway.}
%\dnote{May need to talk about phase distinction of ML modules.}

The type parametricity of System F and F$_\le$ \cite{Cardelli-1985}, and the practical type
systems it inspired such as Java's generics, do not treat types as first class values as we do.
C++ templates support types as meta values in template expansion, but type safety of generated
code is not guaranteed without full template expansion. Concepts \cite{gregor06:_concepts}
improves on this, but types are still not first class values.
% space saving cut for CC here - SS
%  Many systems with types as fully
% first-class values are undecidable, and we wish to avoid that route
% since systems programmers require decidable type systems. Type systems
% with runtime type values can be decidable when various restrictions
% are placed; examples of such systems include
% \cite{Condit07dependenttypes,1375602}. %trimmed Harper95compilingpolymorphism


% \snote{not sure why the following is necessary to mention, its all
%   untyped stuff, right? -- Cite Macros. Felleisen + Flatt;
%   Fully-parameterized, first-class modules with hygienic macros,
%   dissertation by Martin Gasbichler,
%   2006.
% {}
% }
%Several authors have explored the interaction of program
%staging and type dependence to support compiler construction
%\cite{1173724} and interpreters \cite{Pasalic02taglessstaged}.  Also
%related is work on program generation formalisms for compiler
%construction that leverage first class types and intensional
%polymorphism \cite{968395}.  Perhaps the system most closely related to
%ours is Monnier and Shao's \cite{inlining-shao}, where type
%abstraction as a language construct is supported in a staged program
%calculus albeit following a standard System F$_\le$ style (i.e.~types
%are not treated as expressions).  The integration of staging
%abstractions and side effects is another dimension of our work that
%has been considered by previous authors.  Kameyama et al.~have studied
%staging in the presence of side effects as a way to optimize
%algorithms that exploit mutation \cite{1480962}.  Moggi and Fagorzi
%have established a monadic foundation for integrating staging with
%arbitrary side effects in a highly general and mathematically rigorous
%fashion \cite{DBLP:conf/fossacs/MoggiF03}.  But in addition to various
%technical differences, these systems are contrasted with ours in that
%none have considered embedded systems as an application space.

% moved this cite to nest where it is briefly mentioned only. - SS
%
% Our formal definition of a ``safe C'' subset and proof of subject
% reduction follows in the footsteps of other safe approaches to C
% operational semantics \cite{Leroy-compcert-06,grossman03}.  Since we
% are not interested in supporting pointer arithmetic we model the C
% memory at a level abstracted from low-level details such as contiguous 
% layout of arrays and structs.

%Dynamic code update is a key component of our application.  This topic
%has been well-studied (see e.g. \cite{hicks-dynamic-update06}), but
%the primary focus in previous work has been on changing out
%\emph{parts} of existing applications at run-time; here we have a more 
%modest goal which involves rebooting the mote.  Changing just a part
%of a mote program is very appealing since less code will need to be
%sent over-the-air; our approach to this is to use an independent layer
%to send over a binary difference -- only the changed code needs to be
%sent.
