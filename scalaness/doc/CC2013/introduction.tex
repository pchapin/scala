\section{Introduction}

Programming wireless sensor networks (WSNs) and other small embedded devices presents multiple
challenges due to severe resource constraints. This paper describes a programming language
system supporting type safe dynamic code generation for WSNs. It features programming
abstractions for specializing sensor code, allowing on-the-fly adaptation to current WSN
deployment conditions that can achieve greater
efficiency. % The system has been implemented as an extension to Scala \cite{PiS2},
 % through modification of the Scala compiler. 
% everyone knows Scala -SS
%Scala is a powerful
% functional and object oriented hybrid language for the Java Virtual
% Machine .

Our system targets a scenario where a relatively powerful hub device can automatically combine
dynamically specialized libraries and deploy them to a WSN. We use a restricted form of
\emph{staging} \cite{Taha-MetaML,DBLP:conf/icess/Taha04,289140} to achieve well founded dynamic
program generation. \emph{First stage} code is written in an extended version of Scala, called
Scalaness, which is programmer friendly and suitable for running on powerful hubs. Scalaness
program execution yields a residual \emph{second stage} WSN node program written in nesT, a type
safe variant of the popular nesC WSN programming language \cite{Gay-nesC-2003}. The second stage
program is constructed from module components treated as first class values, which may be type
and value specialized during the course of first stage computation.

A central contribution of Scalaness is static type safety. The Scalaness implementation
incorporates a type checking algorithm ensuring that typeable Scalaness programs will always
generate type safe nesT code. Since type generalization is allowed to be cross-stage, our
technology supports a novel form of cross-stage type specialization. In existing strongly typed
staged sensor programming environments the type correctness of second stage programs must be
verified after execution of first stage code \cite{Mainland-Flask-2008}, and could in fact error
which would invalidate the deployment. Such type errors are always caught at first stage
compilation time in our system. Our previous work on the staged programming calculus $\langle
\text{ML} \rangle$ \cite{FramedML} provides a theoretical foundation for Scalaness/nesT type
safety.

\subsection{Scalaness/nesT Implementations}

The goal of our work is to build a practical programming system for writing real WSN
applications. Here we present language designs and proof-of-concept implementations that are
contributions in this direction. Scala is an appropriate choice as a basis for our Scalaness
first stage language because its compiler is open source and easy to modify and maintain, and
the Scala language offers a flexible and user friendly feature set.
% Commented out bits about Scalas formal foundation, its not
%  verified so seems best to keep silent as its not a major selling
%   point. - SS
%
%  Scala also has a strong formal foundation. In particular,
% although full Scala has not been formally proven type safe in the
% manner of its Java subset (via Featherweight Java \cite{XXX}), its
% type checking algorithm has been formalized and shown to be decidable
% \cite{XXX}. 
Our nesT WSN language is implemented by translation into nesC,
which can in turn be compiled for TinyOS platforms.
% nesC characterized previously - SS
%  This is an appropriate
% choice due to the popularity of TinyOS in WSN settings, and the domain
% specificity of nesC.
% We also allow nesT code to reference nesC libraries in our
% implementation. 
nesT type checking has been implemented from the ground up since it attains stronger type safety
guarantees than nesC. Scalaness can compose and specialize nesT modules programmatically, and
the compiler integrates type checking for nesT. nesT is by necessity a lower level language than
Scala, arguably a required gap since sensors are too small to support VMs or automated memory
management.

\subsection{Overview of Scalaness/nesT Design}

We introduce our system through an informal discussion of its novel features, illustrated by the
example in \autoref{figure-example}. This toy example highlights how staged programming
techniques can reduce WSN energy consumption by allowing network node messaging code to be
specialized on the hub. In particular, we focus on how radio packets can be specialized; since
each bit of transmission is known to consume energy similar to 800 instructions \cite{tag}, this
can lead to big savings. The example is written in a user friendly form that is a minor
variation on the formal abstract syntax of Scalaness and nesT that we subsequently present.

In the example, the Scalaness function \tt{node\_specialize} configures a simple application to
be run on individual sensor nodes-- to tell a one-hop neighbor ``hello''. To support
configuration of the the application for a variety of node platforms, the function is abstract
wrt the nesT radio module \tt{radioC} which defines the physical layer interface, although the
required type signature \tt{radioT} of any particular radio module is known. The nesT \tt{sendC}
module is explicitly defined, which implements a message send abstraction on the node, as is the
nesT \tt{nodeC} module which implements the top level node application. The
\tt{node\_specialize} code instantiates the \tt{nodeC} module with source and destination
addresses, as well as an address type \tt{adt} which is determined dynamically based on the
value \tt{nmax} which is a neighborhood size; we assume \emph{a priori} that the address space
size is bounded by \tt{nmax}. Hence, instantiation of \tt{nodeC} enables a type specialization
to use the minimal needed bit sizes.
% (note that all addresses are appropriately type cast leading to
% reduced space consumption for those parameters). 
%As a first
%approximations, TinyOS pracitioners can view Scalaness programs as a
%significantly enriched form of nesC module configuration.

\subsubsection{Modules as Staging Elements.}

In Scalaness, nesT modules can be treated as data to be composed, following traditional staged
programming languages \cite{Taha-MetaML}. The so-called ``runnable'' modules -- ones without
imports or generic parameters -- define an initial machine configuration. This supports a TinyOS
mote reprogramming model, where the entire OS is recompiled and target nodes are reimaged and
rebooted. We define an \tt{image} operation (invoked in line \ref{l:image} of the example) which
asserts its argument to be runnable and dumps the module code for subsequent sensor deployment.

nesT modules specify a list of imported function signatures, and a list of exported functions
implemented by the module. Module genericity is obtained via a sequence of type and value
parameter definitions. For example, as specified in \autoref{figure-example}, any module
\tt{radioC} satisfying \tt{radioT} has an address type parameter \tt{adt} which specializes the
message type declared in the exported \tt{radio} function.
%, whereas
%\tt{sendC} imports a \tt{radio} function and exports a \tt{send}
%function.  
All generic type parameters are assigned an upper
\emph{type bound} via the subtyping symbol $\subtype$. 
%The \tt{nodeC}
%module additionally has value parameters \tt{self} and \tt{neighbor},
%which are type cast during the call to the imported \tt{send}
%function. 
(The concrete syntax used in \autoref{figure-example} precedes each import/export definition
with keyword $\tt{import}$/$\tt{export}$ for a more readable presentation; this is not part of
the formal nesT syntax.)
  
The other Scalaness operations on nesT modules are instantiation and composition. In line
\ref{l:lift}, module \tt{nodeC} is instantiated with arguments \tt{adt}, \tt{self}, and
\tt{neighbor}. In lines \ref{l:scode} and \ref{l:image}, modules are composed using the
$\ltimes$ operator. The semantics of module composition $\mu_1 \ltimes \mu_2$ is standard
\cite{Cardelli-1997}; imports of one module are connected to exports of the other. nesT module
composition is analogous to nesC \emph{configuration
  wiring}.% In \autoref{example-scalanesssemantics}
% we describe these semantics in detail.
\begin{fpfig*}[t]{An Example \tt{Scalaness}/{\bf \texttt{nesT}} Program; fonts distinguish the two grammars}{figure-example}
\lstset{numbers=left, numberstyle=\tiny, stepnumber=1, numbersep=5pt, basicstyle=\ttfamily} 
%\lstset{classoffset=1, 
%morekeywords={post, typedef,image,uint,uint8,uint4,uint16,uint64,error_t,import,export,def,val},keywordstyle=\color{black}\bfseries, classoffset=0}
\lstset{moredelim=*[is][\color{red}]{(((}{)))}} 
\lstset{escapeinside={(*@}{@*)}}

{\scriptsize\bf
$$
\tt{radioT} \defeq \texttt{<adt $\subtype$ uint> \{ export error\_t radio ({src:adt; dest:adt; data:uint8[]})\}} 
$$
\begin{lstlisting}
 (*@\tt{sendC\ \  =}@*)  <adt (*@$\subtype$@*) uint>  
           { import error_t radio( {src:adt; dest:adt; data: uint8[]} );       
             export error_t send (adt s, adt d, uint8[] data) 
               { radio({src = s, dest = d, data = data}); (*@\label{l:struct}@*) }
           }
 
 (*@\tt{nodeC\ \  =}@*) < adt (*@$\subtype$@*) uint, uint self, uint neighbor>.
          { import error_t send (adt s, adt d, uint8[] data); 
            export error_t main() 
              { call send((adt)self, (adt)neighbor, "hello"); }   
          }
 
 (*@\tt{def\ nodeSpecialize(self: uint, neighbor: uint, nmax: uint16, radioC: radioT)}@*) {
   (*@\tt{typedef\ adt <: uint16 =\ if\ (nmax <= 256)\ uint8\ else\ uint16;}@*)  (*@\label{l:typedef}@*)   
   (*@\tt{val\ scode = \jinst{sendC}{adt} \ltimes\ \jinst{radioC}{adt};}@*)  (*@\label{l:scode}@*)    
   (*@\tt{val\ ncode = \jinst{nodeC}{adt, self, neighbor};}@*) (*@\label{l:lift}@*)    
   (*@\tt{image(ncode  \ltimes scode);}@*) (*@\label{l:image}@*)   
 }
\end{lstlisting}}%small
\end{fpfig*}


\subsubsection{Typing.} 

The most novel feature of our type system is dynamic type construction. We allow dynamic
construction of nesT types at the Scalaness level for module instantiation and specialization.
On line \ref{l:typedef} in \autoref{figure-example}, the address type \tt{adt} is dynamically
constructed via a conditional expression.

% Scalaness incorporates two distinct type checking algorithms-- one for the first stage
% extended Scala language, and one for nesT code.
Scalaness type checking is an extension of Scala type checking; we introduce a new module type
form to the Scala type language, and add type checking cases for the three module operations:
instantiation, composition, and imaging. No other part of Scala type checking needs to be
modified. NesT type checking is defined as a standalone type system, and yields first stage
module types.

% stuff about soundness cut, was already mentioned in intro and is in
% scalaness section below as well. SS
%
% Scala type checking has been formally studied and shown to be
% decidable \cite{Cremet:2006:CCS:2135978.2135980}.
% % , and in this work it
% % is also conjectured that Scala enjoys type safety.
% % no need to say - SS
% % , which is supported
% % by practical experience
% The typing rules we add to accommodate staged module
% operations are based on our previous foundational work
% \cite{FramedML}, lending strong support for the safety of Scalaness typing.
% % Not sure what this review of nesT is doing here, nesT was already
% % briefly described earlier. - SS
% %
% % The nesT type system is a
% % mostly standard procedural language type system, but with type safety
% % restrictions on features (e.g.~no pointer arithmetic) and bounded
% % subtyping polymorphism for greater precision of the analysis.  We
% % illustrate nesT and Scalaness type checking in
% % \autoref{example-nesttyping} and \autoref{example-scalanesstyping},
% % respectively.



%\subsubsection{TinyOS Compliant Programming Model.} 
%
%Like nesC, the nesT langage comprises a subset of the C language of
%expressions and statements to define sequential computations. The
%model also includes pseudo-concurrent task postings with a
%run-to-completion semantics. Indeed, all nesT code has an
%interpretation in nesC, and residual nesT code is translated to nesC
%by the Scalaness compiler for subsequent compilation and deployment.
%
%Although Scalaness is intended for real applications, in this document
%and in our implementation we have focused on a minimal language to
%isolate fundamental issues and simplify implementation details. Thus
%for example, our module language does not separate interfaces and
%module implementations as in nesC. Nevertheless it is an adequate
%``featherweight'' representation of the nesC module language and
%integrates naturally with the TinyOS programming model.
%
%Since it is not realistic (or convenient) to imagine that all TinyOS
%libraries can be recreated in nesT, it is essential to be able to use
%nesC libraries as needed. In \autoref{section-libraries} we describe a
%nesT mechanism in the implementation for importing nesC modules.

\subsubsection{Cross-Stage Migration of Types and Values.} 

A crucial feature of our programming model is \emph{process separation} between stages
\cite{FramedML}. Since first and second stage code are to be run on separate devices, state is
not shared between these stages. Thus, serialization may be required when modules are
instantiated. Furthermore, types and base values may be represented differently on the first and
second stages, requiring some sort of transformation during module instantiation. An example
transformation is discussed in \autoref{section-serialization}.

\subsection{Remainder of the Paper} 

In this paper we specify the features of Scalaness/nesT and to report on its implementation. We
develop core calculi for nesT in \autoref{section-nest} and for Scalaness in
\autoref{section-scalaness}.
% which is  
% a characteristic subset of Scalaness suitable for formal
% consideration, in the style of Featherweight Java. In
% \autoref{section-nest} we describe the nesT syntax, semantics, and
% type checking discipline. In \autoref{section-scalaness} we describe
% the first stage Scalaness language in featherweight form, including
% its syntax, semantics, and type checking discipline. 
In \autoref{section-implementation} we describe our implementation. We conclude with remarks on
related work in \autoref{section-conclusion}.

