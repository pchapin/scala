\title{Scalaness/nesT: Type Specialized Staged Programming for Sensor Networks}

\author{
Peter Chapin\inst{1} \and 
Yu David Liu\inst{2} \and 
Christian Skalka\inst{1} \and 
Scott Smith\inst{3} \and 
Michael Watson\inst{1}
\institute{
University of Vermont, \texttt{\{pchapin,skalka,mpwatson\}@cems.uvm.edu} \and
SUNY Binghamton, \texttt{davidl@cs.binghamton.edu} \and
The Johns Hopkins University, \texttt{scott@cs.jhu.edu}
}
}

\maketitle

\begin{abstract}
  Building programs for constrained embedded device networks is challenging due to severe
  limitations on processing speed, memory, and bandwidth. Staged programming can help bridge the
  gap between high level code refinement techniques and device level programs by allowing a
  first stage program to programmatically specialize device level code. Here we introduce a two
  stage programming system for wireless sensor networks. The first stage program is written in
  an extended dialect of Scala, called Scalaness, where components written in a type safe
  dialect of nesC, called nesT, are composed and specialized. The residue of Scalaness program
  evaluation is a TinyOS-compliant device image that can be compiled as a nesC program, but
  enjoys additional type safety properties. Scalaness has been implemented through direct
  modification and plugin extension of the Scala compiler.
\end{abstract}

