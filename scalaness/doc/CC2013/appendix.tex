\section*{Appendix A: Scalaness Code Sample}

This code sample defines and creates two modules, one that knows how to compute checksums and
another that formats messages and calculates a checksum on them. These two modules are
parameterized by the size of the message block to use and by the type used to hold the checksum.
A special Library component is defined that wraps the TinyOS libraries, specifically the
\tt{MainC} component and a Timer. Upon booting the program formats a message every second.

{\scriptsize
\begin{verbatim}
//-----------------------------------------------------------------------
// FILE    : Main.scala
// SUBJECT : The main object for the Scalaness Checksum sample.
//-----------------------------------------------------------------------
object Main {

  import edu.uvm.mininess._
  import edu.uvm.scalaness._
  import LiftableTypes._

  // The Scalaness representation of the nesC Library.
  @ModuleType("""{}
                 <;>
                 { booted(): Void,
                   fired(): Void;
                   startPeriodic(period: UInt32): Void }""")
  object LibraryC extends NesTModule {
    external("LibraryC.nc")
  }

  // A component for computing checksums.
  @ModuleType("""{ }
                 <checksumType <: UInt32; size : UInt16> 
                 { ; compute_checksum(data: Array[UInt8]): checksumType }""")
  class ChecksumC extends NesTModule {
    var checksumType : Metatype[LiftableType] = _
    var size : LiftableType = _
    def instantiate(size: LiftableType, checksumType: MetaType[LiftableType])
       { val result = new ChecksumC; 
         result.size = size; 
         result.checksumType = checksumType; 
         result }
    "ChecksumC.nt"
  }

  // A component for creating messages.
  @ModuleType("""{}
                 <checksumType <: UInt32; size : UInt16>
                 { compute_checksum(data: Array[UInt8]): checksumType,
                   startPeriodic(period: UInt32): Void;
                   booted(): Void,
                   fired(): Void }""")
  class MessageFormatterC extends NesTModule {
    var checksumType : Metatype[LiftableType] = _
    var size : LiftableType = _
    def instantiate(size: LiftableType, checksumType: MetaType[LiftableType])
       { val result = new ChecksumC; 
         result.size = size; 
         result.checksumType = checksumType; 
         result }
    "MessageFormatterC.nt"
  }

  /**
   * The following method returns a fully instantiated nesT component for 
   * computing checksums. The precise component created depends on runtime 
   * information.
   *
   * @param size The size of the data block the component will process.
   * @checksumType The type used for the final checksum.
   */
  @ModuleType("""{ checksumType <: UInt32 }
                 <;>
                 { ; compute_checksum(data: Array[UInt8]): checksumType }""")
  def getChecksummer(size: UInt16, checksumType: MetaType[UInt32]) = {
    // Arbitrary computation...
    (new ChecksumC).instantiate(size, checksumType)
  }
    
  /**
   * The main method obtains configuration information from the command line 
   * and composes the final program accordingly.
   * 
   */ 
  def main(args: Array[String]) {

    def getChecksumType(args: Array[String]) = {
      args(0).toInt match {
        case  8 => 
          println("Selecting 8 bit checksums")   
          new MetaType[UInt32](NesTTypes.UInt8)
        
        case 16 =>
          println("Selecting 16 bit checksums") 
          new MetaType[UInt32](NesTTypes.UInt16)

        case 32 =>
          println("Selecting 32 bit checksums")
          new MetaType[UInt32](NesTTypes.UInt32)
      }
    }

    def getSize(args: Array[String]) = {
      val size = args(1).toInt
      println(s"Selecting $size byte message blocks")
      new UInt16(size)
    }

    if (args.length != 2) println("Usage: Main checksum_bit_length block_size")
    else {
      val desiredChecksumType = getChecksumType(args)
      val desiredSize = getSize(args)

      @ModuleType("""{checksumType <: UInt32}
                     <;>
                     { compute_checksum(data: Array[UInt8]): checksumType,
                       startPeriodic(period: UInt32): Void;
                       booted(): Void,
                       fired(): Void }""")
      val formattingModule =
        (new MessageFormatterC).instantiate(desiredSize, desiredChecksumType)

      @ModuleType("""{checksumType <: UInt32}
                     <;>
                     { ; }""")
      val resultModule =
        LibraryC +>  
        formattingModule +> 
        getChecksummer(desiredSize, desiredChecksumType)

      resultModule.image()
    }
  }   
}
\end{verbatim}       
}

\section*{Appendix B: nesT Code Sample}

Here is some of the nesT source code associated with the preceding Scalaness program. Note that
module definitions refer to type and value parameters declared in Scalaness code.

{\scriptsize
\begin{verbatim}
//-----------------------------------------------------------------------
// FILE    : MessageFormatterC.nt
// SUBJECT : Formats messages for checksumming
//-----------------------------------------------------------------------
// Type  parameter: checksumType: The type used to hold a checksum.
// Value parameter: size: The size of the data array to process.

// Main program of the node.
module MessageFormatterC {
    
    // Convenience feature. Library interface automatically "unwrapped."
    uses interface Boot;
    
    import command checksumType compute_checksum( uint8_t data[] );

    // Current can't unwrap the Timer interface automatically.
    import command void startPeriodic( uint32_t period );
    export command void fired( );
}
implementation {
    
    // Unwrapping converts this event to appropriate command.
    event void Boot.booted( )
    {
        // Casting from int16_t to uint32_t explicitly enabled 
        // in configuration.
        call startPeriodic( (uint32_t)1000 );
    }
    
    // Called once per second.
    command void fired( )
    {
        uint8_t raw[size];
        int16_t i;
        checksumType checksum;
        
        // Construct message.
        for( i = 0; i < (int16_t)size; ++i ) {
            raw[i] = (i & 0x00FF);
        }
        checksum = call compute_checksum( raw );
        
        // Other program components used to send message with checksum.
    }    
}
//-----------------------------------------------------------------------
// FILE    : ChecksumC.nt
// SUBJECT : Checksum computation code
//-----------------------------------------------------------------------
// Type  parameter: checksumType: The type used to hold a checksum.
// Value parameter: size: The size of the data array to process.

module ChecksumC {
    export command checksumType compute_checksum(uint8_t data[]);
}
implementation {
   
    // Computes a simple checksum over the data array.
    command checksumType compute_checksum(uint8_t data[])
    {
        checksumType sum = 0;
        int16_t i;
        
        // Casting from uint16_t to int16_t is explicitly enabled 
        // in configuration.
        for( i = 0; i < (int16_t)size; ++i ) {
            sum += data[i];
        }
        return sum;
    }   
}
\end{verbatim}
}
