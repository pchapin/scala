%%%%%%%%%%%%%%%%%%%%%%%%%%%%%%%%%%%%%%%%%%%%%%%%%%%%%%%%%%%%%%%%%%%%%%%%%%%%
% FILE    : macros.tex
% AUTHOR  : (C) Copyright 2013 by Peter C. Chapin
% SUBJECT : Various macro definitions that can be used in my dissertation.
%%%%%%%%%%%%%%%%%%%%%%%%%%%%%%%%%%%%%%%%%%%%%%%%%%%%%%%%%%%%%%%%%%%%%%%%%%%%

%\newtheorem{condition}{Condition}[section]
%\newtheorem{corollary}{Corollary}[section]
\newtheorem{definition}{Definition}[section]
%\newtheorem{example}{Example}[section]
%\newtheorem{lemma}{Lemma}[section]
%\newtheorem{proposition}{Proposition}[section]
%\newtheorem{theorem}{Theorem}[section]

% I want to typeset newly introduced terms in a special way.
\newcommand{\newterm}[1]{\emph{#1}}

% Used for bits of code in-line.
\newcommand{\inlinecode}[1]{\texttt{#1}}
\newcommand{\filename}[1]{\texttt{#1}}

\newcommand{\code}[1]{\texttt{#1}}
\newcommand{\RT}{\text{RT}}
\newcommand{\cedge}[1]{\stackrel{#1}{\longleftarrow}}
\newcommand{\cred}[3]{\mathit{#1} \cedge{#3} \mathit{#2}}
\newcommand{\creds}{\mathcal{C}}
\newcommand{\prolog}{\text{Prolog}}
\newcommand{\datalog}{\text{Datalog}}
\newcommand{\datalogc}{$\text{Datalog}_\mathcal{C}$}
\newcommand{\semantics}[1]{\llbracket #1 \rrbracket}
\newcommand{\activation}[2]{#1\,\mathit{as}\,#2}
\newcommand{\delcred}[3]{#1 \stackrel{#3}{\longrightarrow} #2}
\newcommand{\mathgraph}[1]{\mathcal{G}_{#1}}
\newcommand{\Sprocket}{Sprocket$_{RT}$}

% The \note macro is useful for creating easy to see notes.
\long\def\note#1{\marginpar{NT}{\small \ \ $\langle\langle\langle$\
{#1}\
    $\rangle\rangle\rangle$\ \ }} 


% I really should put this in a package file! Note how I set up some parameters before opening
% the listing. This used to be necessary when I was using verbatim environments. Is it still
% necessary with the listings package? Are those other settings just being overridden?
%
\newsavebox{\savebigbox}
\newenvironment{mybigbox}{\begin{lrbox}{\savebigbox}
  \begin{minipage}{0.95\columnwidth}%
    \small\setlength{\baselineskip}{0.9\baselineskip}}
{\end{minipage}\end{lrbox}\fbox{\usebox{\savebigbox}}}
% The definition of 'bigbox' above appears to conflict with something in the stmaryrd package.
% I thus changed it to 'mybigbox.' Uses of \begin{bigbox} in the text should probably be changed
% to \begin{mybigbox} to reflect this. Note that I also changed 'wbigbox' to 'mywbigbox' for
% consistency.


% This version should be used for full width boxes.
\newsavebox{\savewbigbox}
\newenvironment{mywbigbox}{\begin{lrbox}{\savewbigbox}
  \begin{minipage}{0.9\textwidth}%
    \small\setlength{\baselineskip}{0.9\baselineskip}}
{\end{minipage}\end{lrbox}\fbox{\usebox{\savewbigbox}}}


% This macro is for text figures (program listings). Use as follows:
% \begin{figure}[htbp]
% \begin{textbox}{3in}  % The width of the text.
% \begin{Verbatim}
% ...
% \end{Verbatim}
% \end{textbox}
% \caption{...}
% \label{...}
% \end{figure}
%
\newsavebox{\savetextbigbox}
\newenvironment{textbox}[1]{
  \begin{lrbox}{\savetextbigbox}
  \begin{minipage}{#1}
  \vspace{0.6em}
}
{
  \vspace{0.3em}
  \end{minipage}
  \end{lrbox}\framebox[\textwidth]{\hfill\usebox{\savetextbigbox}\hfill}
}
% Surround the \usebox{} call with \fbox{} to make the centering boxes visible.
