%%%%%%%%%%%%%%%%%%%%%%%%%%%%%%%%%%%%%%%%%%%%%%%%%%%%%%%%%%%%%%%%%%%%%%%%%%%%
% FILE    : macros.tex
% AUTHOR  : (C) Copyright 2013 by Peter C. Chapin
% SUBJECT : Various macro definitions that can be used in my dissertation.
%%%%%%%%%%%%%%%%%%%%%%%%%%%%%%%%%%%%%%%%%%%%%%%%%%%%%%%%%%%%%%%%%%%%%%%%%%%%

% I want to typeset newly introduced terms in a special way.
\newcommand{\newterm}[1]{\emph{#1}}

% Used for bits of code in-line.
\newcommand{\inlinecode}[1]{\texttt{#1}}

\newcommand{\filename}[1]{\texttt{#1}}

% The \note macro is useful for creating easy to see notes.
\long\def\note#1{\marginpar{NT}{\small \ \ $\langle\langle\langle$\
{#1}\
    $\rangle\rangle\rangle$\ \ }} 

% I really should put this in a package file! Note how I set up some parameters before opening
% the listing. This used to be necessary when I was using verbatim environments. Is it still
% necessary with the listings package? Are those other settings just being overridden?
%
\newsavebox{\savebigbox}
\newenvironment{bigbox}{\begin{lrbox}{\savebigbox}
  \begin{minipage}{0.95\columnwidth}%
    \small\setlength{\baselineskip}{0.9\baselineskip}}
{\end{minipage}\end{lrbox}\fbox{\usebox{\savebigbox}}}

% This version should be used for full width boxes.
\newsavebox{\savewbigbox}
\newenvironment{wbigbox}{\begin{lrbox}{\savewbigbox}
  \begin{minipage}{0.9\textwidth}%
    \small\setlength{\baselineskip}{0.9\baselineskip}}
{\end{minipage}\end{lrbox}\fbox{\usebox{\savewbigbox}}}
