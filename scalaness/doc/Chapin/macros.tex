%%%%%%%%%%%%%%%%%%%%%%%%%%%%%%%%%%%%%%%%%%%%%%%%%%%%%%%%%%%%%%%%%%%%%%%%%%%%
% FILE    : macros.tex
% AUTHOR  : (C) Copyright 2013 by Peter C. Chapin
% SUBJECT : Various macro definitions that can be used in my dissertation.
%%%%%%%%%%%%%%%%%%%%%%%%%%%%%%%%%%%%%%%%%%%%%%%%%%%%%%%%%%%%%%%%%%%%%%%%%%%%

\newtheorem{condition}{Condition}[section]
\newtheorem{conject}{Conjecture}[section]
\newtheorem{corollary}{Corollary}[section]
\newtheorem{definition}{Definition}[section]
\newtheorem{example}{Example}[section]
%\newtheorem{exmp}{Example}[section]
\newtheorem{lemma}{Lemma}[section]
\newtheorem{proposition}{Proposition}[section]
\newtheorem{theorem}{Theorem}[section]

% The definition of \arr doesn't quite work. The vertical bars are typeset as horizontal.
%\def\arr#1{\textup{\textbf{[}}\mskip -.5mu\textup{\textbf{|}}\, #1 \,\textup{\textbf{|}}\mskip -.5 mu\textup{\textbf{]}}}
\def\arr#1{[\,#1\,]}

%\def\abbrv{\ \stackrel{\rm encoded}{=}\ }
\def\blockno{m}
\def\blok#1{\textbf{(}#1\textbf{)}}
\def\castto#1#2{(#1)\,#2}
\def\defassign{::=}
\def\dom{{\rm Dom}}
\def\envletter{E}
\def\figsize{\small}                      % Use in figures to set the size of the figure.
\def\fundef#1{\mathit{#1}}
\def\lc{\textup{\textbf{\{}}}             % Set brackets used in code.
\def\mapidx#1{{(\mskip -2.5mu #1\mskip -2.5mu)}}
\def\maploosemerge{\curlyveedownarrow}
\def\margs#1{\mathrm{<}#1\mathrm{>}}
\def\neight{n^{8}}
\def\nsixtn{n^{16}}
\def\ran{{\rm Ran}}
\def\rc{\textup{\textbf{\}}}}
\def\s{\varsigma}
\def\t{\tau }
\def\undefv{\ttbf{uninit}}
\def\VAR{\textit{x} }

\newcommand{\abbrvt}[3]{\tt{abbrvt\ #1 = #2\ in\ #3}}
\newcommand{\abs}[1]{|#1|}
\newcommand{\activation}[2]{#1\,\mathit{as}\,#2}
\newcommand{\addt}{\mathit{add}}
\newcommand{\bit}{\ttbf{bit}}
\newcommand{\blockmem}{M}
\newcommand{\bm}{\blockmem}
\newcommand{\bn}{\blockno}
\newcommand{\bootload}{\fundef{bootload}}
\newcommand{\bootseq}[1]{\mathbf{boot}(#1)}
\newcommand{\cast}[2]{\tt{(#1)#2}}
\newcommand{\cedge}[1]{\stackrel{#1}{\longleftarrow}}

% The \code macro should only be used when italic font is needed in inline code snippets, for
% example to deal with metavariables in code examples. Otherwise the listings package should
% be used (the short inline form should be turned on in most chapters). The listings package
% is more configurable and automatically bolds keywords.
\newcommand{\code}[1]{\texttt{#1}}

\newcommand{\codt}[1]{\llbracket #1 \rrbracket}
\newcommand{\compatible}[2]{\mathit{compatible}(#1,#2)}
\newcommand{\compute}{\leadsto}
\newcommand{\context}[2]{#1\lc#2\rc}
\newcommand{\cred}[3]{\mathit{#1} \cedge{#3} \mathit{#2}}
\newcommand{\creds}{\mathcal{C}}
\newcommand{\CT}{{CT}}
\newcommand{\cval}[2]{\lfloor #1,#2 \rfloor}
\newcommand{\datalogc}{$\text{Datalog}_\mathcal{C}$}
\newcommand{\datalog}{\text{Datalog}}
\newcommand{\decl}{d}
\newcommand{\decls}{\vect{\decl};}
\newcommand{\defeq}{\triangleq}
\newcommand{\defvec}[2]{\vect{#1} = \vect{#2}}
\newcommand{\delcred}[3]{#1 \stackrel{#3}{\longrightarrow} #2}
\newcommand{\docast}[3]{\fundef{docast}(#1,#2,#3)}
\newcommand{\exportsty}{\varepsilon}
\newcommand{\exports}{\xi}
\newcommand{\fdname}{\textsf{l}}
\newcommand{\fields}[1]{\mathit{fields}(\tt{#1})}
\newcommand{\fieldvec}[2]{\ttvec{#1}\ \ttvec{#2}}
\newcommand{\filename}[1]{\texttt{#1}}    % File names.
\newcommand{\flash}{F}
\newcommand{\fml}{\ensuremath{\langle \text{ML} \rangle}}
\newcommand{\fname}{\textsf{f}}
\newcommand{\fsub}{\ensuremath{F_\le}}
\newcommand{\gbounds}[2]{\ttvec{#1} <: {\ttvec{#2}}}
\newcommand{\gclass}[4]{\tt{class\ #1\langle #2\rangle\ extends\ #3\ \{#4 \}}}
\newcommand{\gdesc}[1]{\text{\textit{#1}}}
\newcommand{\gnew}[3]{\tt{new\ #1\langle#2\rangle(#3)}}
\newcommand{\identifier}{\mathit{id}}
\newcommand{\id}{\identifier}
\newcommand{\idx}[1]{[#1]}
\newcommand{\imports}{\iota}
\newcommand{\init}[3]{\tt{#1(#2)\{ #3 \}}}
\newcommand{\inteight}{\ttbf{uint64}}
\newcommand{\intfour}{\ttbf{uint32}}
\newcommand{\inthalf}{\ttbf{uint4}}
\newcommand{\intone}{\ttbf{uint8}}
\newcommand{\intt}{\ttbf{uint}}
\newcommand{\inttwo}{\ttbf{uint16}}
\newcommand{\jdef}[4]{\tt{def\ #1 : #2 = #3\ in\ #4}}
\newcommand{\jimage}[1]{\tt{image}\ #1}
\newcommand{\jinst}[2]{\tt{#1\ensuremath \langle #2 \ensuremath \rangle}}
\newcommand{\jmodt}[2]{#1 \circ #2}
\newcommand{\jmodtcat}{\mu\!\tau}
\newcommand{\jmodval}{\mu}
\newcommand{\jref}[2]{\tt{(}#1,\tt{#2)}}
\newcommand{\jstore}{ST}
\newcommand{\jtlet}[4]{\tt{typedef\ #1 <: #2 = #3\ in\ #4}}
\newcommand{\jwire}[2]{#1 \ltimes #2}
\newcommand{\kwelse}{\ttbf{else}}
\newcommand{\kwif}{\ttbf{if}}
\newcommand{\kwpost}{\ttbf{post}}
\newcommand{\kwreturn}{\ttbf{return}}
\newcommand{\kwstar}{\texttt{*}}
\newcommand{\kwthen}{\ttbf{then}}
\newcommand{\kwtlet}{\ttbf{typedef}}
\newcommand{\kwtypet}{\ttbf{type}}
\newcommand{\kwwhile}{\ttbf{while}}
\newcommand{\lvalue}{\ell e}
\newcommand{\mathgraph}[1]{\mathcal{G}_{#1}}
\newcommand{\meth}[4]{\tt{#1\ #2(#3)\{ #4 \}}}
\newcommand{\mutate}[2]{\tt{#1 = #2}}
\newcommand{\mv}{{\nu}}
\newcommand{\nesT}{\text{nesT}}
\newcommand{\newterm}[1]{\emph{#1}}                  % Newly introduced terms.
\newcommand{\nextt}{\mathit{next}}
\newcommand{\op}{\ \textit{op}\ }
\newcommand{\prolog}{\text{Prolog}}
\newcommand{\promote}{\ll}
\newcommand{\restrict}[2]{#1\!\mid\!_{#2}}
\newcommand{\return}[1]{\tt{return\ #1}}
\newcommand{\RT}{\textit{RT}}
\newcommand{\runseq}[1]{\mathbf{run}(#1)}
\newcommand{\select}[2]{\tt{#1.#2}}
\newcommand{\semantics}[1]{\llbracket #1 \rrbracket}
\newcommand{\send}[3]{\tt{#1.#2(#3)}}
\newcommand{\ser}[1]{\overset{\text{lift}}{\hookrightarrow}}
\newcommand{\serialize}{\mathrm{serialize}}
\newcommand{\Sprocket}{Sprocket}                     % The name of my SpartanRPC implementation.
\newcommand{\subjudge}[3]{#1 \vdash #2 \subtype #3}
\newcommand{\subtvec}[2]{\vect{#1} \subtype \vect{#2}}
\newcommand{\subtype}{\preccurlyeq}
\newcommand{\super}{\tt{super}}
\newcommand{\tasks}{P}
\newcommand{\tbindvec}[2]{\vect{#1} : \vect{#2}}
\newcommand{\tcompute}[1]{\stackrel{#1}\leadsto}
\newcommand{\tdecls}[2]{\ttvec{#1}\ \ttvec{#2}}
\newcommand{\tdefvec}[3]{\vect{#1} : \vect{#2} = \vect{#3}}
\newcommand{\tenv}{G}
\newcommand{\this}{\tt{this}}
\newcommand{\tpdecl}{\textit{T}}
\newcommand{\ttbf}[1]{\mbox{\bf \texttt{#1}}}        % New version to match lstlisting.
\newcommand{\ttt}[1]{{\tt #1}}
\newcommand{\ttvec}[1]{{\tt{\bar{#1}}}}
\newcommand{\TVAR}{\textit{t}}
\newcommand{\vect}[1]{\overline{#1}}
\newcommand{\vpdecl}{\textit{V}}
\newcommand{\xlet}[3]{#1\ #2\ =\ #3}

\renewcommand{\tt}[1]{\ensuremath{\mathtt{#1}}}

% The \note macro is useful for creating easy to see notes.
\long\def\note#1{\marginpar{NT}{\small \ \ $\langle\langle\langle$\
{#1}\
    $\rangle\rangle\rangle$\ \ }} 


% The macro below was being used for some of the program listings. However, all listings are now
% being handled by the listings package so the macro below is no longer needed. I'm keeping it
% for possible future reference in case, for some reason, it proves useful again (perhaps for
% other kind of Verbatim displays?). It is commented out to be sure it is not used accidentally.

% This macro is for text figures (program listings). Use as follows:
% \begin{figure}[htbp]
% \begin{textbox}{3in}  % The width of the text.
% \begin{Verbatim}
% ...
% \end{Verbatim}
% \end{textbox}
% \caption{...}
% \label{...}
% \end{figure}
%
%\newsavebox{\savetextbigbox}
%\newenvironment{textbox}[1]{
%  \begin{lrbox}{\savetextbigbox}
%  \begin{minipage}{#1}
%  \vspace{0.6em}
%}
%{
%  \vspace{0.3em}
%  \end{minipage}
%  \end{lrbox}\framebox[\textwidth]{\hfill\usebox{\savetextbigbox}\hfill}
%}
% Surround the \usebox{} call with \fbox{} to make the centering boxes visible.
