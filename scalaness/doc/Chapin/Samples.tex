
\chapter{Scalaness/nesT Sample}
\label{chapter-sample}

% Keep the listings limited to column 62 at most (60 is better).

In this appendix I show a simple Scalaness/nesT sample. The sample composes two nesT modules
being used as part of a hypothetical communication protocol. One module formats messages for
transmission and the other module computes checksums.

It is reasonable for the checksum module to be separate since different applications may wish to
use different checksum algorithms. In fact, the Scalaness program could dynamically select one
of several candidate checksum modules as it composes the overall application, although that
feature is not demonstrated here.

I begin with the Scalaness program itself. In this simple example the entire program is
contained in a single Scala object holding the main method. The program begins by declaring
objects representing the wrapped nesC libraries needed. It then declares classes representing
the nesT components to be used, defines some helper methods, and executes the main body. Notice
that the type abbreviation binders \cite{watson-masters-2013} are used to introduce convenience
objects holding module type information.

This program demonstrates dynamic type construction, nesT module type and value parameters,
module instantiation, and wiring. The program also demonstrates returning a dynamically
constructed type from a Scala method.

The full source code is below.

\singlespace
\vspace{1.0ex}
\begin{lstlisting}[language=scalaness]
//------------------------------------------------------------
// FILE    : Main.scala
// SUBJECT : The main object for the Scalaness Checksum sample
//------------------------------------------------------------

object Main {

  import edu.uvm.nest._
  import edu.uvm.scalaness._
  import LiftableTypes._ 

  // The Scalaness representation of the library imports.
  @ModuleType(
    """{}
       <;>
       { booted(): Void,
         fired(): Void; }""")
  object LibraryIC extends NesTComponent {
    external("LibraryIC.nc")
  }


  // The Scalaness representation of the library exports.
  @ModuleType(
    """{}
       <;>
       { ; startPeriodic(period: UInt32): Void }""")
  object LibraryEC extends NesTComponent {
    external("LibraryEC.nc")
  }


  // A component for computing checksums.
  @ModuleType(
    """{}
       < checksumType <: UInt32; size: UInt16 >
       { ; compute_checksum(
             data: Array[UInt8]): checksumType }""")
  class ChecksumC extends NesTComponent {
    "ChecksumC.nc"
  }
 

  // A component for creating messages.
  @ModuleType(
    """{}
       < checksumType <: UInt32; size: UInt16 >
       { compute_checksum(
           data: Array[UInt8]): checksumType,
         startPeriodic(period: UInt32): Void;
         booted(): Void,
         fired(): Void }""")
  class MessageFormatterC extends NesTComponent {
    "MessageFormatterC.nc"
  }


  /**
   * The following method returns a fully instantiated nesT
   * module for computing checksums. The precise module
   * created depends on runtime information.
   *
   * @param size The size of the data block the module will
   * process.
   * @param checksumType The type used for the final checksum.
   */
  def getChecksummer(
    size: UInt16, checksumType: MetaType[UInt32]) = {
  
    @ModuleType(
      """{}
         < checksumType <: UInt32; size: UInt16 >
         { ; compute_checksum(
               data: Array[UInt8]): checksumType }""")
    val CheckSummer = new ChecksumC
    
    @ModuleType(
      """{ checksumType <: UInt32 }
           <;>
           { ; compute_checksum(
                 data: Array[UInt8]): checksumType }""")
    val instCheckSummer =
      CheckSummer.instantiate(size, checksumType)
    
    instCheckSummer
  }

    
  /**
   * The main method obtains configuration information from
   * the command line and composes the final program.
   * 
   * @param args The command line arguments
   */ 
  def main(args: Array[String]) {

    // Create type abbreviations for convenience.

    // An uninstantiated module type.
    val MesgT = new TypeAbbreviation(
      """{}
         < checksumType <: UInt32; size: UInt16 >
         { startPeriodic(period: UInt32): Void,
           compute_checksum(
             data: Array[UInt8]): checksumType;
           fired(): Void,
           booted(): Void }""", List())

    // An instantiated module type.
    val FormT = new TypeAbbreviation(
      """{ checksumType <: UInt32 }
         <;>
         { compute_checksum(
             data: Array[UInt8]): checksumType,
           startPeriodic(period: UInt32): Void;
           booted(): Void,
           fired(): Void }""", List())

    // An instantiated module type.
    val CheckT = new TypeAbbreviation(
      """{ checksumType <: UInt32 }
         <;>
         { ;
           compute_checksum(
             data: Array[UInt8]): checksumType }""", List())

    // A runnable module type.
    val ResultT = new TypeAbbreviation(
      """{ checksumType <: UInt32 }
         <;>
         { ; }""", List())

    // Return a MetaType based on command line argument.
    def getChecksumType(args: Array[String]) = {
      args(0).toInt match {
        case  8 => 
          println("Selecting 8 bit checksums")   
          new MetaType[UInt32](NesTTypes.UInt8)
        
        case 16 =>
          println("Selecting 16 bit checksums") 
          new MetaType[UInt32](NesTTypes.UInt16)

        case 32 =>
          println("Selecting 32 bit checksums")
          new MetaType[UInt32](NesTTypes.UInt32)
      }
    }

    // Method that returns a liftable value.
    def getSize(args: Array[String]) = {
      val size = args(1).toInt
      println(s"Selecting $size byte message blocks")
      new UInt16(size)
    }


    if (args.length != 2)
      println("Usage: Main checksum_bit_length block_size")
    else {
 
      // Get run time information about types/values.
      val desiredChecksumType = getChecksumType(args)
      val desiredSize = getSize(args)

      // Uninstantiated message formatter.
      @TypeAbbr(MesgT)
      val MessageFormatter =
        new MessageFormatterC

      // Instantiated message formatter.
      @TypeAbbr(FormT)
      val formattingModule =
          MessageFormatter.instantiate(
            desiredSize, desiredChecksumType)

      // Compute appropriate checking module.
      @TypeAbbr(CheckT)
      val checkingModule =
          getChecksummer(desiredSize, desiredChecksumType)
      
      // Wire things together.
      @TypeAbbr(ResultT)
      val resultModule =
          LibraryIC +>
            formattingModule +> checkingModule +> LibraryEC

      // Generate the nesT/nesC.
      resultModule.image()
    }
  }

}       
\end{lstlisting}
\vspace{1.0ex}
\primaryspacing

Next I show the nesT implementation of the checksum module. This version uses a simple
arithmetic summation. It is parameterized by the type used to hold the checksum and by a value
representing the size of the array to be processed. Thus specializations of this module can only
operate on fixed sized arrays, presummably the size of some standard message format.

\singlespace
\vspace{1.0ex}
\begin{lstlisting}[language=nesC]
// Type : checksumType: The type used to hold a checksum.
// Value: size: The size of the data array to process.
module ChecksumC {
  provides
    command checksumType compute_checksum(uint8_t data[]);
}
implementation {
   
  // Computes a simple checksum over the data array.
  command checksumType compute_checksum(uint8_t data[])
  {
    checksumType sum = 0;
    int16_t i;
        
    // Casting from uint16_t to int16_t is explicitly
    // enabled in type compatibility relation. The
    // compiler uses a built-in implementation of this
    // conversion.
    //
    for( i = 0; i < (int16_t)size; ++i ) {
      sum += data[i];
    }
    return sum;
  }
    
}
\end{lstlisting}
\vspace{1.0ex}
\primaryspacing

The listing below illustrates how the Scalaness compiler rewrites the checksum module to pure
nesC in the example program given. This listing also shows the result of type and value
specialization, in this instance implemented by way of simple substitution. This version of the
module was created for 8~bit checksums on eight element data arrays.

Notice the addition of compiler generated variables to hold dynamic size information for the
array expressions. These variables are checked to ensure memory safety as described in
\autoref{section-array-operations}.

\singlespace
\vspace{1.0ex}
\begin{lstlisting}[language=nesC]
module ChecksumC  {
  provides {
    command uint8_t compute_checksum(
      uint8_t data[], uint16_t _sc_data_SIZE );
  }
  uses {
    command void boundsCheckFailed( );
  }
}
implementation {
  command uint8_t compute_checksum(
    uint8_t data[], uint16_t _sc_data_SIZE) 
  {
    uint8_t sum = 0;
    int16_t i;
    for( i = 0; i < (int16_t)(8); ++i )
    {
    {
      int _sc_2 = i;
      if( _sc_2 >= _sc_data_SIZE )
        call boundsCheckFailed( );
      sum += data[_sc_2];
    }
    }
    return sum;
  }

}
\end{lstlisting}
\vspace{1.0ex}
\primaryspacing

The message formatting module constructs a ``message'' consisting of ascending byte values and
then computes a checksum over that message. Nothing more is done with the message in this simple
demonstration. A more realistic program would then send the message to an underlying
communication module for transmission.

\singlespace
\vspace{1.0ex}
\begin{lstlisting}[language=nesC]

// Type : checksumType: The type used to hold a checksum.
// Value: size: The size of the data array to process.
//
// Main program of the node.
module MessageFormatterC {
    
  uses command checksumType compute_checksum(uint8_t data[]);
  uses command void startPeriodic( uint32_t period );
  provides command void booted( );
  provides command void fired( );
}
implementation {
    
  command void booted( )
  {
    // Casting from int16_t to uint32_t is explicitly
    // enabled in type compatibility relation.
    //
    call startPeriodic( (uint32_t)1000 );
  }
    
  // Called once per second.
  command void fired( )
  {
    uint8_t  raw[size];
    uint16_t i;
    checksumType checksum;
        
    // Construct message.
    for( i = 0U; i < size; ++i ) {
      raw[i] = (i & 0x00FF);
    }
    checksum = call compute_checksum( raw );
        
    // Other program components used to send message with checksum.
  }
    
}
\end{lstlisting}
\vspace{1.0ex}
\primaryspacing

The specialized and rewritten nesC shows the dynamic size of an array expression being passed to
\texttt{compute\_checksum}.

\singlespace
\vspace{1.0ex}
\begin{lstlisting}[language=nesC]
module MessageFormatterC  {
  provides {
    command void fired( );
    command void booted( );
  }
  uses {
    command void startPeriodic( uint32_t period );
    command uint8_t compute_checksum(
      uint8_t data[], uint16_t _sc_data_SIZE );
    command void boundsCheckFailed( );
  }
}
implementation {
  command void booted( ) 
  {
    call startPeriodic( (uint32_t)( 1000 ) );
  }

  command void fired( ) 
  {
    uint8_t  raw[8 ];
    uint16_t i;
    uint8_t  checksum;
    for( i = 0U; i < 8; ++i )
    {
    {
      int _sc_1 = i;
      if( _sc_1 >= 8 )
        call boundsCheckFailed( );
      raw[_sc_1] = ( i & 0x00FF );
    }
    }
    checksum = call compute_checksum( raw, 8 );
  }

}
\end{lstlisting}
\vspace{1.0ex}
\primaryspacing

