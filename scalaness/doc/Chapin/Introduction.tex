\chapter{Introduction}
\label{chapter-introduction}

Embedded systems in general, and wireless sensor networks in particular, often entail
programming small, resource constrained systems. A typical node in a wireless sensor network
might have only 48 KiB of program ROM and 8 KiB of RAM (or less), and be using a small, 16~bit
microcontroller running at only 8 MHz \cite{XXX}. Yet embedded systems applications are
increasing in complexity. In addition to application logic, such systems are now expected to
support various network protocol stacks, etc, etc, and elaborate security services, etc
\cite{XXX}.

One approach to writing highly efficient programs is to use \newterm{staging}. In a staged
programming environment, a first stage program is used to compose and specialize a lower level,
second stage program. Specialized code can often be considerably more optimized. However,
flexibility is retained because the first stage program can be re-executed at a later time to
re-specialize the second stage code.

Staging is a widely researched topic
\cite{Taha-MetaML,Sheard-TemplateHaskell,Mainland-Flask-2008,FramedML}. \note{Probably should
  only mention the seminal papers here.} Many of these systems are quite general and allow an
arbitrary number of stages where each stage can manipulate small fragments of next-stage code.
Furthermore most existing systems (except\ldots) use the same language in all stages. Finally
the type correctness of the stage $n+1$ programs are normally checked during their compilation
or during the execution of the stage $n$ programs that produce them.

I focus here on the specific problem of using staged programming to improve the efficiency of
node level programs for wireless sensor networks and other similar embedded systems. In such an
environment the first stage program is envisioned to run on a powerful base station or even a
hand-held device such as a smart phone. Using information gathered in the field, the first stage
program computes an appropriately specialized second stage program that is then deployed to the
network using an in-situ distributed deployment system such as Deluge \cite{XXX}. Feedback
provided by the network allows the first stage program to recompute an updated second stage
program that can then be redeployed. The second stage program running on the sensor nodes
remains small and efficient and yet long term flexibility is retained.

In this environment only two stages are necessary so a language that provides an arbitrary
number of stages is not required. Furthermore the practical realities of sensor network
programming means that the second stage language needs to be either C or nesC
\cite{Gay-nesC-2003} or some dialect of these languages. However, nesC, especially, is an
embedded systems language and not suitable for creating high level applications. Thus the first
stage language can and should be something different.

In this dissertation I describe \newterm{Scalaness}, an extension of Scala \cite{PiS2} with
features that allow the programmer to compose and specialize components written in a reduced
dialect of nesC called \newterm{Mininess}. One novel feature of Scalaness is that the type
system of Scala has been modified so that a well-typed Scalaness program will always generate a
well-typed Mininess program. No type errors will arise in the field when a compiled Scalaness
program is used to repeatedly generate optimized node programs. This is essential since the user
of the Scalaness program won't be a position to deal with type errors in the nesC generated by
that program.

\note{Talk about the ``deliverables'' of this work. Talk about trust management. Talk about
  SpartanRPC. Give organization of the rest of the document.}

%%% Local Variables: 
%%% mode: LaTeX
%%% TeX-master: "main"
%%% End: 
