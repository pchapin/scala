
The on-node cryptographic computations required by our system are digital signature
verification, session key negotiation using Diffie-Hellman key exchange, and message
authentication code (MAC) computation using a session key. Of these three operations the first
two involve complex public key cryptography. The MAC computation occurs much more frequently but
is much cheaper since it uses hardware accelerated symmetric key cryptography. In fact, the
motivation behind creating session keys is to avoid public key operations for every message.

ECC can achieve much higher security for a given number of key bits saving memory and network
bandwidth relative to other public key cryptosystems. In our implementation, described in detail
in \autoref{section-implementation}, we use 160 bit ECC keys, providing a security similar to
1024 bit RSA keys \cite{lenstra-verheul-2001}. We believe this is a reasonable level of security
for the anticipated applications.
