
\subsection{External Libraries}
\label{section-external-libraries-implementation}

\note{I need a little more introduction.} The Scalaness compiler parses the component
specification of the external component and unwraps the interfaces used and provided by that
component. For each such interface the Scalaness compiler generates a shim module that forwards
nesT commands to the full nesC interface. When generic interfaces are used by the external
component these shim modules are themselves generic.

\autoref{figure-shim-modules} shows the shims generated for a library component
\inlinecode{LibraryC} that uses interface \inlinecode{U} and provides interface \inlinecode{P}.

\singlespace
\begin{fpfig}[tbhp]{Shim Modules}{figure-shim-modules}
{
\begin{lstlisting}[language=nesC]
module Sclnss_U1 {
  provides interface U;
  uses     command void Uc(int);
  provides command void Ue(int);
}
implementation {
  command void Ue(int i)
    { signal U.Ue(i); }

  command void U.Uc(int i)
    { call Uc(i); }
}

module Sclnss_P1 {
  uses     interface P;
  provides command void Pc(int i);
  uses     command void Pe(int i);
}
implementation {
  event void P.Pe(int i)
    { call Pe(i); }

  command void Pc(int i)
    { call P.Pc(i); }
}
\end{lstlisting}
}
\end{fpfig}
\primaryspacing

The precise names used for these components are internal identifiers generated by Scalaness and
are not intended to be used by the programmer. Instead Scalaness wraps the original library
component and the shims into a configuration as shown in \autoref{figure-library-configuration}.
If the original component was generic or used generic interfaces, then the wrapping
configuration is also generic so that it properly corresponds to its representation in
Scalaness.

\singlespace
\begin{fpfig}[tbhp]{Configuration Wrapping Library Component}{figure-library-configuration}
{
\begin{lstlisting}[language=nesC]
configuration Sclnss_LibraryC {
  uses     command void Uc(int);
  provides command void Ue(int);
  provides command void Pc(int);
  uses     command void Pe(int);
}
implementation {
  components Sclnss_P1, Sclnss_U1, LibraryC;
  LibraryC.U  -> Sclnss_U1;
  Sclnss_P1.P -> LibraryC;

  Sclnss_P1.Pe -> Pe;
  Pc -> Sclnss_P1.Pc;
  
  Sclnss_U1.Uc -> Uc;
  Ue -> Sclnss_U1.Ue;
}
\end{lstlisting}
}
\end{fpfig}
\primaryspacing

In effect the Scalaness compiler automatically converts an external library component into a
nesT component that can be manipulated by the first stage program. The Scalaness compiler
extracts a module type for the component, based on its imports and exports, as usual. The
component can then participate in the process of component composition allowed by the Scalaness
language.

In the current implementation the Scalaness compiler does not attempt to type check the external
library component. Presumably external components have been previously well tested. In effect,
Scalaness treats them as entirely static; staging is only applied to a portion of the final
application and to its overall configuration.

% So it's possible to use an external generic component that fails for certain instantiations
% and Scalaness won't notice that. Fixing this would entail extending the type system to cover
% full nesC since the contents of the external components are \emph{not} nesT.

As I've described so far each external library component is wrapped in a separate configuration.
However, a programmer could create a larger configuration of library components manually and
then treat that entire configuration as a single entity inside the Scalaness program. The choice
depends on the amount of control the programmer wants the first stage program to have on how
library components are specified and composed. In some applications it may make sense for an
entire subsystem of library components to be configured into a single entity ahead of time. In
other applications the programmer may wish to control library configuration as part of the first
stage execution.
