
\chapter{DScalaness/DnesT}
\label{chapter-dscalaness-dnest}

% Theory from the CC and GPCE papers...

This chapter describes a programming language system supporting type safe dynamic code
generation for sensor networks. It features programming abstractions for specializing sensor
code, allowing on-the-fly adaptation to current sensor network deployment conditions that can
achieve greater efficiency. The system has been implemented as an extension to Scala
\cite{PiS2}, through modification of the Scala compiler.

I consider scenarios where a relatively powerful hub device can automatically combine
dynamically specialized libraries and deploy them to a sensor network. To this end a restricted
form of \emph{staging} \cite{Taha-MetaML,DBLP:conf/icess/Taha04,289140} is used to achieve well
founded dynamic program generation. \emph{First stage} code is written in an extended version of
Scala, called Scalaness, which is programmer friendly and suitable for running on powerful hubs.
The execution of a Scalaness program yields a residual \emph{second stage} node program written
in nesT, a type safe variant of the nesC WSN programming language. The second stage program is
constructed from module components treated as first class values, which may be type and value
specialized during the course of first stage computation.

A central contribution of Scalaness is static type safety. The Scalaness implementation
incorporates a type checking algorithm ensuring that typeable Scalaness programs will always
generate type safe nesT code. Since type generalization is allowed to be cross-stage, the
technology supports a novel form of cross-stage type specialization. In existing strongly typed
staged sensor programming environments the type correctness of second stage programs must be
verified after execution of first stage code \cite{Mainland-Flask-2008}, and could in fact
produce an error which would invalidate the deployment. Such type errors are always caught at
first stage compilation time with Scalaness. Previous work on the staged programming calculus
$\langle \text{ML} \rangle$ \cite{FramedML} provides a theoretical foundation for Scalaness/nesT
type safety.

\section{Overview of Scalaness/nesT Design}

The goal of Scalaness is to build a practical programming system for writing arbitrary sensor
network applications. In that respect it is a much more general system than SpartanRPC described
earlier. In fact, Scalaness could potentially be used for other kinds of embedded systems
programming beyond sensor networks. However, my work did not investigate the usability of
Scalaness in those more general contexts.

Scala is an appropriate choice as a basis for the first stage language because its compiler is
open source and easy to modify and maintain. Also the Scala language offers a rich, flexible,
and user friendly set of features familiar to application programmers working in traditional
(desktop, server) environments. Furthermore Scala has an active user community with a growing
collection of tools and other supporting resources.

However, Scala is not appropriate in the extremely resource constrained environment of a sensor
network node. In contrast nesT is implemented by translation into nesC, which can in turn be
compiled for TinyOS platforms. The nesT language is roughly (although not exactly) a subset of
nesC and shares with nesC many features appropriate for sensor node programming.

The nesT type checking has been implemented from the ground up since it attains stronger type
safety guarantees than nesC. Scalaness can compose and specialize nesT modules programmatically,
and the compiler integrates type checking for nesT. nesT is by necessity a lower level language
than Scala, arguably a required gap since sensors are too small to support VMs or automated
memory management.

I introduce the system through an informal discussion of its novel features, illustrated by the
example in \autoref{figure-example}. This toy example highlights how staged programming
techniques can reduce sensor network energy consumption by allowing network node messaging code
to be specialized on the hub. In particular, the example focuses on how radio packets can be
specialized; since each bit of transmission is known to consume energy similar to 800
instructions \cite{tag}, this can lead to big savings. The example is written in a user friendly
form that is a minor variation on the formal abstract syntax of Scalaness and nesT that I
subsequently present.

In the example, the Scalaness function \texttt{node\_specialize} configures a simple application
to be run on individual sensor nodes---to tell a one-hop neighbor ``hello.'' To support
configuration of the the application for a variety of node platforms, the function is abstract
with respect to the nesT radio module \texttt{radioC} which defines the physical layer
interface, although the required type signature \texttt{radioT} of any particular radio module
is known. The nesT \texttt{sendC} module is explicitly defined, which implements a message send
abstraction on the node, as is the nesT \texttt{nodeC} module which implements the top level
node application. The \texttt{node\_specialize} code instantiates the \texttt{nodeC} module with
source and destination addresses, as well as an address type \texttt{adt} which is determined
dynamically based on the value \texttt{nmax} which is a neighborhood size; we assume a priori
that the address space size is bounded by \texttt{nmax}. Hence, instantiation of \texttt{nodeC}
enables a type specialization to use the minimal needed bit sizes. As a first approximations,
TinyOS pracitioners can view Scalaness programs as a significantly enriched form of nesC module
configuration.

\subsection{Modules as Staging Elements}

In Scalaness, nesT modules can be treated as data to be composed, following traditional staged
programming languages \cite{Taha-MetaML}. The so-called ``runnable'' modules---ones without
imports or generic parameters---define an initial machine configuration. This supports a TinyOS
mote reprogramming model, where the entire OS is recompiled and target nodes are reimaged and
rebooted. We define an \texttt{image} operation (invoked in line \ref{l:image} of the example)
which asserts its argument to be runnable and dumps the module code for subsequent sensor
deployment.

NesT modules specify a list of imported function signatures, and a list of exported functions
implemented by the module. Module genericity is obtained via a sequence of type and value
parameter definitions. For example, as specified in \autoref{figure-example}, any module
\texttt{radioC} satisfying \texttt{radioT} has an address type parameter \texttt{adt} which
specializes the message type declared in the exported \texttt{radio} function.

All generic type parameters are assigned an upper \emph{type bound} via the subtyping symbol
$\subtype$. The \texttt{nodeC} module additionally has value parameters \texttt{self} and
\texttt{neighbor}, which are type cast during the call to the imported \texttt{send} function.
The concrete syntax used in \autoref{figure-example} precedes each import/export definition with
keyword $\texttt{import}$/$\texttt{export}$ for a more readable presentation; this is not part
of the formal nesT syntax.
  
The other Scalaness operations on nesT modules are instantiation and composition. In line
\ref{l:lift}, module \texttt{nodeC} is instantiated with arguments \texttt{adt}, \texttt{self},
and \texttt{neighbor}. In lines \ref{l:scode} and \ref{l:image}, modules are composed using the
$\ltimes$ operator. The semantics of module composition $\mu_1 \ltimes \mu_2$ is standard
\cite{Cardelli-1997}; imports of one module are connected to exports of the other. NesT module
composition is analogous to nesC \emph{configuration wiring}.

\begin{fpfig*}[t]{An Example \texttt{Scalaness}/{\bf \texttt{nesT}} Program; fonts distinguish the two grammars}{figure-example}
\lstset{numbers=left, numberstyle=\tiny, stepnumber=1, numbersep=5pt, basicstyle=\ttfamily} 
\lstset{moredelim=*[is][\color{red}]{(((}{)))}} 
\lstset{escapeinside={(*@}{@*)}}

% TODO: Clean up this figure!
{\scriptsize\bf
$$
\texttt{radioT} \defeq \texttt{<adt $\subtype$ uint> \{ export error\_t radio ({src:adt; dest:adt; data:uint8[]})\}} 
$$
\begin{lstlisting}
 (*@\texttt{sendC\ \  =}@*)  <adt (*@$\subtype$@*) uint>  
           { import error_t radio( {src:adt; dest:adt; data: uint8[]} );       
             export error_t send (adt s, adt d, uint8[] data) 
               { radio({src = s, dest = d, data = data}); (*@\label{l:struct}@*) }
           }
 
 (*@\texttt{nodeC\ \  =}@*) < adt (*@$\subtype$@*) uint, uint self, uint neighbor>.
          { import error_t send (adt s, adt d, uint8[] data); 
            export error_t main() 
              { call send((adt)self, (adt)neighbor, "hello"); }   
          }
 
 (*@\texttt{def\ nodeSpecialize(self: uint, neighbor: uint, nmax: uint16, radioC: radioT)}@*) {
   (*@\texttt{typedef\ adt <: uint16 =\ if\ (nmax <= 256)\ uint8\ else\ uint16;}@*)  (*@\label{l:typedef}@*)   
   (*@\tt{val\ scode = \jinst{sendC}{adt} \ltimes \jinst{radioC}{adt};}@*)  (*@\label{l:scode}@*)    
   (*@\tt{val\ ncode = \jinst{nodeC}{adt, self, neighbor};}@*) (*@\label{l:lift}@*)    
   (*@\tt{image(ncode  \ltimes scode);}@*) (*@\label{l:image}@*)   
 }
\end{lstlisting}
}
\end{fpfig*}


\subsection{Typing} 

The most novel feature of the Scalaness type system is dynamic type construction. Dynamic
construction of nesT types is allowed at the Scalaness level for module instantiation and
specialization. On line \ref{l:typedef} in \autoref{figure-example}, the address type
\texttt{adt} is dynamically constructed via a conditional expression.

Scala type checking has been formally studied and shown to be decidable
\cite{Cremet:2006:CCS:2135978.2135980}. Scalaness type checking is an extension of Scala type
checking; a new module type form is introduced to the Scala type language, type checking cases
have been added for the three module operations: instantiation, composition, and imaging. No
other part of Scala type checking needs to be modified. NesT type checking is defined as a
standalone type system, and yields first stage module types.

\subsection{TinyOS Compliant Programming Model}

Like nesC, the nesT langage comprises a subset of the C language of expressions and statements
to define sequential computations. The model also includes pseudo-concurrent task postings with
a run-to-completion semantics. Indeed, all nesT code has an interpretation in nesC, and residual
nesT code is translated to nesC by the Scalaness compiler for subsequent compilation and
deployment.

Although Scalaness is intended for real applications, in this dissertation and in the
implementation I have focused on a minimal language to isolate fundamental issues and simplify
implementation details. Thus for example, the module language does not separate interfaces and
module implementations as in nesC. Nevertheless it is an adequate ``featherweight''
representation of the nesC module language and integrates naturally with the TinyOS programming
model.

Since it is not realistic (or convenient) to recreate all TinyOS libraries in nesT, it is
essential to be able to use nesC libraries as needed. In \autoref{section-external-libraries} I
describe a mechanism in the implementation for importing nesC modules.

\subsection{Cross-Stage Migration of Types and Values.} 

A crucial feature of my programming model is \emph{process separation} between stages
\cite{FramedML}. Since first and second stage code are to be run on separate devices, state is
not shared between these stages. Thus, serialization may be required when modules are
instantiated. Furthermore, types and base values may be represented differently on the first and
second stages, requiring some sort of transformation during module instantiation. An example
transformation is discussed in \autoref{section-serialization}.

\section{The nesT language}
\label{section-nest-theory}
 
\syntaxfig

The nesT design aims to distill a production language, nesC, into its fundamental elements,
yielding a language that is amenable to formal analysis but also practical. However, since the
focus here is on type safety, nesT enhances, and in some instances restricts, those fundamental
elements to obtain a type safe language, as discussed below.

\subsubsection{Notation.} \emph{Sequences} are notated $x_1,\ldots,x_n$, and are abbreviated
$\vect{x}$; $\vect{x}_\mapidx{i}$ is the $i$-th element, $\emptyset$ denotes the empty sequence,
and $\abs{\vect{x}}$ is the size. I write $x \in \vect{x}$ to denote membership in sequences,
and $x\vect{x}$ denotes a sequence with head $x$ and tail $\vect{x}$. We denote append as
$\vect{x}@\vect{y}$. For relational symbols $R \in \{ \subtype, =, : \}$, we use the
abbreviation: $\vect{x}\, R\, \vect{y} = x_1\, R\, y_1, \ldots, x_n\, R\, y_n$. So for example,
$\tbindvec{x}{\t} = x_1 : \t_1,\ldots,x_n : \t_n$.

I also use the following naming conventions for various language constructs. I use metavariable
$\fname$ (of set $\mathcal{F}$) for function names, $\fdname$ (of set $\mathcal{L}$) for field
names, $\VAR$ (of set $\mathcal{V}$) for term variables, $\TVAR$ (of set $\mathcal{T}$) for type
variables, $ \blockno$ (of set $\mathcal{M}$) for memory locations, $\neight$ (of set
$\mathbb{Z}_{2^8}$) for 8-bit unsigned integers, and $\nsixtn$ (of set $\mathbb{Z}_{2^{16}}$)
for 16-bit unsigned integers. I use $n$ to range over both types of integers when their type is
irrelevant.

\subsection{Syntax and Features of nesT}

The syntax of nesT is presented in \autoref{figure-syntax}. It comprises a core language of
expressions for defining computations, a language of declarations for defining variables and
functions, and a language for defining modules.

\semanticssyntaxfig

\coresemanticsfig

\subsubsection{Expressions.}
The nesT language includes standard C-like constructs, such as conditional branching, looping,
sequencing of expressions, and function calls, arrays, structs, numeric base datatypes (and
operations on them). Familiar syntax $e \idx e$ and $e.\fdname$ is used for array indexing and
struct field selection, respectively.

A ``null'' value $\undefv$ is also provided. Function definitions and calls are available only
in a limited nullary form to simplify the model, in particular the expression $\blok{e}$ is a
function body. Since functions are not recursive there is no expressiveness lost, a global
variable can be used to encode a function parameter. Memory locations $\blockno$ are program
values (\emph{\'a la} pointers). As in all C dialects, assignment can only be performed on
so-called \emph{l-values} $\lvalue$, a restricted subset of expression syntax.

As in nesC, no dynamic memory allocation is possible; all memory layout is established by static
variable declarations. Also, the core language we present here does not allow pointer arithmetic
to support type safety. Type casting and array access have run time checks imposed, as I explain
in \autoref{section-nestsemantics}.

Also as in nesC, a $\kwpost$ operation is provided for posting tasks. The semantics will account
for tasks using the ``run-to-completion'' model of TinyOS. Interrupts are omitted from our
language to avoid concurrency issues in the semantics. Typical sensor network applications do
not need interrupts; they are only needed in a few low level nesC libraries.

\subsubsection{Declarations.}
Programs in nesT may refer to declarations of values and functions which are scoped at the
module level and establish the statically fixed memory layout of nesT. A convenient form for
explicit initialization of array and struct values is provided, though neither arrays nor struct
values may be treated anonymously (i.e.~they must be declared and referred to by name). Besides
convenience, declarations are useful to support serialization of program objects passed in from
the Scala level in Scalaness, as we discuss in \autoref{section-serialization}.

\subsubsection{Modules.}
NesT Modules are written $\margs{\tpdecl; \vpdecl}\lc \imports; \decls \exports \rc $ with
$\tpdecl$ and $\vpdecl$ being generic type and term parameters, $\vect{d}$ being module scope
identifier declarations, including function definitions, and $\imports$ and $\exports$ being
imports and exports. Exports are explicitly defind in the module. A ``runnable'' module---one
without imports or generic parameters---defines an initial machine configuration. In
\autoref{figure-example}, the wiring $\texttt{ncode} \ltimes \texttt{scode}$ on line
\ref{l:image} yields a runnable module.

\tasksemanticsfig

\declsemanticsfig

\subsection{Semantics of nesT} 
\label{section-nestsemantics}

The operational semantics for nesT is defined as a small step relation $\compute$ on dynamic
configurations in \autoref{figure-semanticssyntax}. The semantics are decomposed into several
distinct $\compute$ relations; each computation ``sub-relation'' can be distinguished by the
arity of the relevant configurations. The semantics for the non-module portions follows standard
C-like language formalizations \cite{Leroy-compcert-06,grossman03}.

\subsubsection{Semantics of Expressions.}
At the heart of nesT is a C-like language of expressions built from l- and r-values. An l-value
is an object in memory that can be assigned to, in particular a variable, a struct or array
field, or a pointer. An r-value is a value resulting from expression computation and may be a
value that is not in memory.

The nesT syntax for necessary dynamic entities is given in \autoref{figure-semanticssyntax}. In
nesT, computed values are represented as pairs $\cval{\pi}{\mv}$, where $\mv$ is a base,
pointer, array, or struct value, and $\pi$ is a tag indicating whether or not the value is in
memory. Computed r-values not in memory are denoted $\cval{\circ}{\mv}$, e.g. $\cval{\circ}{2}$
is the result of computing 1 + 1. The l-value object $\cval{\bn}{\mv}$ indicates that the value
$\mv$ is in memory at location $\bn$.

Memories are modeled as sequences of definitions $\bn : \t = \mv$; observe that each memory
location $\bn$ is typed at $\t$ and assigned a value $\mv$. Memories are interpreted as
mappings, writing $\blockmem(\bn) = \mv$ when there exists some $\t$ such that $\bn : \t = \mv$
is the leftmost definition of $\bn$ in $\blockmem$, and writing $\blockmem[\bn \mapsto \mv]$ to
denote $(\bn : \t = \mv)$ where $\bn : \t = \mv'$ is the leftmost definition of $\bn$ in
$\blockmem$.

Evaluation rules for selected expressions are given in \autoref{figure-coresemantics}. Here,
computation is on pairs of memories and expressions. I assume existence of a function
$\fundef{docast}$ which performs a casting conversion. This function is allowed to be defined by
users, and in certain cases may be a no-op (e.g.~casting pointers to arrays when the latter are
contiguous in memory). In any case, if $\docast{\t}{\mv}{\blockmem}$ is defined, the cast
conversion is required to be type safe, in that the result must be of type $\t$. This is
discussed more in \autoref{section-nesttyping}.

Note that a pointer is modeled by an object of the form $\cval{\pi}{\bn}$. The operation
$\kwstar\cval{\pi}{\bn}$ looks up the value at address $\bn$ in memory. The operation
$\&\cval{\bn}{\mv}$ returns the address $\bn$ of the object in memory as an r-value. Functions
are defined in an assumed-given codebase $\flash$ with a lookup semantics defined similarly to
that for memories $\bm$.

\bootloadsemanticsfig

\subsubsection{Semantics of Tasks.}
NesC uses a simple scheduling model of serial, run-to-completion execution of queued
\emph{tasks}. The base semantics of nesT are thus supplemented with a corresponding \emph{task
  collection} $\tasks$ of the tasks yet to run, and defined with a single step transition
relation on configurations extended with task collections. The definition of task collections is
left undetermined, and also how tasks are added and retrieved---this because it is unspecified
whether tasks are treated in e.g.~a FIFO manner by the scheduler. I let
$\addt(\tasks,\fname\blok{})$ denote $\tasks'$ which is $\tasks$ plus the task consisting of the
function call $\fname\blok{}$, and let $\nextt(\tasks)$ denote a pair $\tasks',\fname\blok{}$
which comprises the ``next'' task $\fname\blok{}$ in $\tasks$, and $\tasks'$ which is $\tasks$
with $\fname\blok{}$ removed. The task semantics, integrated with the expression semantics
defined previously, are defined in \autoref{figure-tasksemantics}. As for expressions, the
existence of a given codebase $\flash$ is assumed. When it is necessary to be explicit about
which codebase is given for a computation, I will write $\flash \vdash \tasks, \bm, e \compute
\tasks', \bm', e'$.

\subsubsection{Semantics of Declarations.}
The operational behavior of declarations is fairly straightforward. Functions and first class
mutable variables may be declared and initialized. At run time mutable variables are bound (via
substitution) to an l-value $\cval{\bn}{\mv}$, where $\bn$ is the address of the variable. Thus,
for base and function type declarations we have the following rules, respectively:
$$
\small{
\inferrule[FDecl]
{}
{\flash, \tasks, \bm, (\fname : \t = \blok{e}) \vect{\decl} \compute 
 (\fname : \t = \blok{e}) \flash, \tasks, \bm, \vect{\decl}}
}
$$
$$
\small{
\inferrule[BaseInit]
{\kappa = \cval{\bn}{\mv} \\ \bn \not\in \dom(\bm)}
{\flash, \tasks, \bm,(\xlet{\t}{x}{\cval{\pi}{\mv}}) \vect{\decl} \compute 
 \flash, \tasks, (\bn : \t = \mv)\bm, \vect{\decl}[\kappa/x]}
}
$$
A contextual evaluation rule for declarations allows variables to be initialized with arbitrary
expressions. This is omitted for brevity but is similar to the expression \TirName{Context}
rule, using a a notion of declaration evaluation contexts denoted $D$.

\subsubsection{Semantics of Boot and Run Time.}

In the nesT machine model, a top level program execution is obtained by loading and running a
fully instantiated module. The codebase, memory layout, and initial machine configuration is
generated at load time by evaluating the declarations in the module. The top level program is
then started at the \texttt{main} entry point.

To differentiate load/boot and run segments of a computation I define $\bootseq{}$ and
$\runseq{}$ constructors to inject declarations and expressions into a uniform datatype. Top
level computation is then defined as a single step reduction relation $\compute$ on
configurations $\flash,\tasks,\bm,X$, where $X$ is of the form $\bootseq{\vect{d}}$ or
$\runseq{e}$ depending on whether the machine is booting or running.

\begin{definition}
  A module of the form $\margs{\varnothing; \varnothing }\lc ; \vect{\decl}; \exports\rc$ is
  \emph{runnable}, and I define:
$$
\bootload(\margs{\varnothing; \varnothing }\lc ; \vect{\decl}; \exports\rc) =
\exports,\varnothing,\varnothing, \bootseq{\vect{\decl}}
$$
\end{definition}

Now, for all computation relations I define $\compute^*$ to be the reflexive, transitive closure
of $\compute$. The concern for type analysis is to rule out modules which, when bootloaded, will
evaluate to semantically ill-formed configurations. In the context of nesT this is defined as
follows. It is important to stipulate that failing casts and out-of-bound array access are not
stuck cases, since run time checks in place with enable graceful failurex.
\begin{definition}
  A configuration $\bm, e$ \emph{fails a run time check} if and only if $e$ is of the form
  $\castto{\t}{\kappa}$ and $\docast{\t}{\kappa}{\blockmem}$ is undefined, or $e$ is of the form
  $\cval{\bn}{\mv}\idx{\cval{\pi}{n}}$ and $n\ge\abs{\vect{\mv}}$.
\end{definition}

\begin{definition}
  \label{def-runnable}
  A configuration $\flash, \tasks, \bm, \ell$ is \emph{stuck} if and only if it is irreducible
  and $\ell$ is neither of the form $\runseq{\context{E}{e}}$ nor $\bootseq{\context{D}{e}}$
  where $\bm, e$ fails a run time check. A runnable module $\mu$ \emph{goes wrong} iff
  $\bootload(\mu) \compute^\star \flash, \tasks, \bm, \ell $ where $\flash, \tasks, \bm, \ell$
  is stuck.
\end{definition}

\subjudgefig

\coretypingfig

\declmodtypingfig

\subsection{nesT Typing} 
\label{section-nesttyping}

The typing rules for nesT combine a standard procedural language typing approach with subtyping
techniques adapted from previous foundational work \cite{FramedML,Ghelli199875}. The goal here
is to specify the typing algorithm used in the Scalaness implementation.

\subsubsection{Subtyping.}
At the heart of our system is a decidable subtyping judgement $\subjudge{\tpdecl}{\t_1}{\t_2}$,
where $\tpdecl$ in the context of typing is called a \emph{coercion} and defines a system of
upper bounds for type variables. Recursive type bounds definitions are not allowed.

The implementation of the subtyping algorithm is based on a classic technique
\cite{Ghelli199875}, with straightforward extensions to accommodate structs and arrays as
defined in \autoref{figure-subjudge}.

A subtyping relation typically called \emph{promotion} is also central to the approach; given a
set of subtyping coercions $\tpdecl$ and a type variable $t$, promotion will return the least
upper bound of $t$ which is also a structured type, i.e.~not a type variable.
\begin{definition}
The relation $\ll$ \emph{promotes} a type variable:
\begin{mathpar}
\figsize
\inferrule
{\tpdecl \vdash \tpdecl(t) \ll \tau}
{\tpdecl \vdash t \ll \tau}

\inferrule
{\neg\exists t . \tau = t}
{\tpdecl \vdash \tau \ll \tau}
\end{mathpar}
\end{definition} 
It is important to observe how promotion and subtyping are used differently. Since any sort of
l-value can be written to via assignment, subtyping invariance \emph{must} be imposed on
l-values occuring in write positions to maintain type soundness. Therefore type subsumption is
allowed only at program points where read-only control flow occurs---for example when an r-value
is directly assigned to an l-value.

\subsubsection{Type Environments and Checking.}
The typing algorithm for source code expressions is based on judgements $\tenv,\tpdecl \vdash e
: \t$, where $\tenv$ is an environment of free term variable typings, syntactically defined
equivalent to value parameters $\vpdecl$ and imports $\imports$. Type environments are also
endowed with the same lookup semantics as memories and codebases. Representative typing rules
for selected expressions are given in \autoref{figure-coretyping}. The derivation of any
judgement $\tenv,\tpdecl \vdash e : \t$ can be interpreted as an algorithm where both $\tenv$
and $\tpdecl$ are given as arguments and $\t$ is returned as a result.

Note that type casting is only statically allowed if the types involved are \emph{compatible} as
specified in rule \TirName{TypeT}. This relation, formalized as $\tpdecl \vdash
\compatible{\t_1}{\t_2}$, is left abstract and user defined, but typical examples include
casting structs to arrays in contiguous memory, and provable lack of pointers in types.
Recalling that the semantics of nesT relies on a $\fundef{docast}$ function that implements cast
conversions, any implementation must be type safe, which allows us to rule out run time cast
failures in well typed programs. Informally, $\fundef{docast}$ is type safe if and only if the
resulting expression has the type of the cast. I refer the reader to \cite{FramedML} for a
thorough formal discussion of type safety for this style of casting.

\subsubsection{Declaration and Module Typings.}
At the module level, one needs to first type check and generate typing environments from
declarations, as specified in \autoref{figure-declmodtyping} (rules for array and struct
declarations omitted for brevity). Given this, a module typing is obtained by type checking
module exports, using a coercion obtained from the module type parameters and a typing
environment obtained from a combination of module value parameters, imports, and variable type
declarations. Module type checking is also specified in \autoref{figure-declmodtyping}. A type
safety conjecture for nesT can then be stated as follows.

\begin{conject}[nesT Type Safety]
  If $\mu : \jmodtcat$ is valid and $\mu$ is runnable, then $\mu$ does not go wrong.
\end{conject}
Here is an example nesT typing. In this and other examples I will take narrative liberties with
function definitions, assuming they can be non-nullary as in \autoref{figure-example}.
\begin{example}
\label{example-nesttyping}
The module $\texttt{nodeC}$ of \autoref{figure-example} can be assigned the type:
$$
\begin{array}{l}
\margs{\tt{adt} \subtype \tt{uint}; \texttt{self : uint}, \tt{neighbor : uint}}\\
\ \ \ \ \lc
\texttt{error\_t\ \ send}(\texttt{s : adt}, \texttt{d : adt}, \texttt{uint8[] : data});\ \ 
\texttt{error\_t\ \  main()} 
\rc
\end{array}
$$
\end{example}


\section{The Scalaness Language}
\label{section-scalaness-theory}

Scalaness serves as the language for nesT module composition in the same manner as nesC
configurations serve to compose nesC modules, but Scalaness is a more powerful metalanguage
since modules are treated as a new category of first class values in Scalaness. Instantiation,
composition (aka wiring), and imaging of modules are defined as operations on module values.
Because instantiation of modules with both types and values is allowed, values and types may
migrate from the Scalaness level to the nesT level after programmatic refinement, realizing a
disciplined form of code specialization.

The goal of this Section is to describe the Scalaness syntax and semantics realized in my
implementation, and justify the prior claims of type safety. Since Scala as implemented is too
large to easily formalize, the formalization here is of a subset of Scala expressed as an
extension of Featherweight Java \cite{FJ}, a core calculus subsumed by Scala. A formalized core
calculus and type analysis for Scala exists \cite{Cremet:2006:CCS:2135978.2135980}, but FJ is
adequate and simpler. Thus many features of Scala have been elided in this Scalaness
formalization, but I adapt all Scala features unchanged in my Scalaness implementation. Here the
focus is primarily on the module metaprogramming operations that have been add. This
presentation ``cleans up'' some implementation details, but is otherwise an accurate description
of the module operation semantics and especially the module operation typing rules.

\subsection{Syntax of Scalaness}

\scalanesssyntaxfig The Scalaness language syntax is presented in
\autoref{figure-scalanesssyntax}. To represent an adequate core calculus of Scala, it subsumes
two Featherweight Java variants: Featherweight Generic Java (FGJ) \cite{FJ} and Assignment
Featherweight Java (AFJ) \cite{AFJ}. The generic class types of FGJ are needed to model type
construction, and the mutation in AFJ is essential to consider since one of our main concerns is
nesT code specialization; nesT programs are run in a separate process space, so specialization
with stateful values, a likely common idiom in a Scala setting, presents a challenge.

I refer the reader to \cite{FJ,AFJ} for details on the FGJ and AFJ object oriented calculi,
which are represented in the languages of class definitions, constructors, methods, and the
first line of expression forms defined in \autoref{figure-scalanesssyntax}. Scalaness extends
these features with a typed variable declaration form $\jdef{x}{T}{e_1}{e_2}$ where the scope of
$\tt{x}$ is $\tt{e_2}$, a dynamic type construction form $\jtlet{x}{T}{e_1}{e_2}$ with similar
scoping rules, and several features for module definition and manipulation. First, nesT modules
$\jmodval$ are included in the Scalaness expression and value spaces: instantiation is obtained
via the form $\jinst{e_1}{\ttvec{e}_1; \ttvec{e}_2}$, where $\ttvec{e}_\texttt{1}$ are type
parameters and $\ttvec{e}_\texttt{2}$ are value parameters. Wiring of modules is denoted
$\jwire{\tt{e_1}}{\tt{e_2}}$. Imaging of modules, denoted $\jimage{\texttt{e}}$, ensures that
$\texttt{e}$ computes to a runnable module, in the sense of \autoref{def-runnable}.

\subsection{Semantics of Scalaness}

The semantics of Scalaness is an extension of the semantics of AFJ and FGJ to incorporate nesT
modules and operations. Computations assume a fixed class table $\CT$ allowing access to class
definitions via class names, which always decorate an object's type. A \emph{store} $\jstore$ is
a function from memory locations $\texttt{p}$ to object representations. Objects are represented
in memory by lists of object references $\ttvec{l}$, which refer to the locations of the objects
stored in mutable field values. A reference $\texttt{l}$ is a pair $\jref{\texttt{p}}{N}$ where
$\texttt{p}$ is the memory location of an object representation and $\texttt{N}$ is the nominal
type of the object, including its class name. Hence, given an object reference
$\jref{\texttt{p}}{\jinst{C}{\ttvec{T}}}$, one can access and mutate its fields $\ttvec{l} =
\jstore(\texttt{p})$, and access and use its methods via the definition $\CT(\texttt{C})$.

Following AFJ, the semantics of Scalaness is defined as a \emph{labeled transition system},
where transitions are of the form $\texttt{e} - \lc s = \jstore, s' = \jstore' \rc \rightarrow
\texttt{e}' $. Intuitively, this denotes that given an initial store $\jstore$ and expression
$\texttt{e}$, one step of evaluation results in a modified store $\jstore'$ and contractum
$\texttt{e}'$. We write $\texttt{e} \rightarrow \texttt{e}'$ as an abbreviation when the store
is not altered.

The primary novelty of Scalaness over FGJ/AFJ is the formal semantics of type and module
construction. I begin with type construction, which is provided to allow programmers to
dynamically construct module type instances. The appropriate behavior is obtained by treating
dynamically constructed types as extensions of a basic class of objects, and declarations of
nesT level types via a \texttt{typedef} construct as syntactic sugar for ordinary object
construction. We define a \texttt{LiftableType} class as the supertype of all types of objects
which can be used to instantiate a module, and dynamically constructed types are defined as
instances of a generic \texttt{MetaType} class.
\begin{definition}
Any Scalaness class table $\CT$ comprises the following definitions:
$$
\begin{array}{l}
\CT(\texttt{LiftableType}) = \gclass{\texttt{LiftableType}}{}{Object}{\ldots}\\
\CT(\texttt{MetaType}) = \gclass{\texttt{MetaType}}{\texttt{X <: LiftableType}}{Object}{ \ldots }
\end{array}
$$
And we take as given the following syntactic sugar:
$$
\jtlet{x}{T}{e_1}{e_2}\ \defeq\ \jdef{x}{\jinst{MetaType}{T}}{e_1}{e_2}
$$
Class type $\texttt{MetaType}$ is generalized on a single type variable. For brevity of
notation, define:
$$
\jinst{\texttt{MetaType}}{\ttvec{T}}\quad \defeq \quad \overline{\jinst{MetaType}{T}}
$$
\end{definition}
A crucial fact of Scalaness type construction is that any dynamically constructed type cannot be
treated as a type at the Scalaness level. This is a more restrictive mechanism than envisioned
in the foundational model \cite{FramedML,FramedMLworkshop}, however it allows Scalaness to be
defined as a straightforward extension to Scala, especially in terms of type checking.

Module instantiation is the only point where specialization of nesT modules is allowed. Since
Scalaness and nesT are two different language spaces, some sort of transformation must occur
when values migrate from Scalaness to nesT via module instantiation. This \emph{lifting}
transformation involves both data mapping and serialization since the process spaces also
differ. We aim to be flexible and allow the user to specify how values are lifted and how types
are transformed. We only require that lifting and type transformation are coherent, in the sense
that the lifting of an object should be typeable at the object's type transformation. This is
formalized in the following definition.
\begin{definition}
\label{def-lifting}
Assume a relation $\ser{\jstore}$ which transforms a Scalaness reference $\texttt{l}$ into nesT
declarations $\vect{d}$ and expression $e$ is provided. Also assume a Scalaness-to-nesT
transformation of types $\codt{\cdot}$ is provided. To preserve type safety, it is required that
in all cases $\jref{p}{N} \ser{\jstore} \vect{d}, e$ implies both of the following for some type
environment $G$:
$$
\varnothing, \varnothing \vdash \vect{d} : \tenv \qquad \text{ and} \qquad
 \tenv, \varnothing \vdash e : \codt{\texttt{N}} 
$$
\end{definition}
The full definition of serialization and an example are given and discussed below in
\autoref{section-serialization}. In brief, when a module $\mu$ is instantiated, serialization
will bind the value parameters of $\mu$ to the lifted values of their instances in a series of
declarations that are added to its own. This is specified in the \TirName{ModInst} rule in
\autoref{figure-jmodsemantics}. Another important detail of the \TirName{ModInst} rule is that
only type information in type parameters is used, and migrates into the module via type
transformation and ordinary substititution.

\jmodsemanticsfig

Module wiring is given a standard component composition semantics. Only the wiring of
instantiated modules is allowed, which is consistent with nesC and simpler to implement. In a
wiring $\jwire{\texttt{e}_1}{\texttt{e}_2}$, the exports of $\texttt{e}_1$ are wired to the
imports of $\texttt{e}_2$. This is specified in the \TirName{ModWire} rule in
\autoref{figure-jmodsemantics}, which relies on the following auxiliary definition of operations
for combining mappings.
\begin{definition}[Special Mapping Operations]
  Let $m$ range over vectors with mapping interpretations, in particular~$\tpdecl$, $\vpdecl$,
  $\imports$, and $\exports$. Binary operator $m_1\maploosemerge m_2$ represents (non-exclusive)
  map merge, i.e.~$m_1 \maploosemerge m_2 = m_1 @ m_2$ with the requirement that $\identifier\in
  \dom(m_1)\cap\dom(m_2)$ implies $m_1(\identifier) = m_2(\identifier)$. The mapping $m / S$ is
  the same as $m$ except undefined on domain elements in set $S$, and the mapping
  $\restrict{m}{S}$ is the same as $m$ except undefined on elements not in ${S}$.
\end{definition}
Finally, the $\TirName{ModImage}$ rule in \autoref{figure-jmodsemantics} shows that imaging it
is an assertion requiring its arguments to be a runnable module.

\begin{example}
\label{example-scalanesssemantics}
Assume given the definitions in \autoref{figure-example}, a module $\texttt{radioC : radioT}$
and an invocation $\texttt{nodeSpecialize}(1,2,50,\texttt{radioC})$. Then $\texttt{ncode}
\ltimes \texttt{scode}$ will evaluate to the following module:
$$
\margs{}\lc\ ; \ldots; \texttt{error\_t\ \ main()} \lc \texttt{call}\ \ \texttt{send}((\texttt{uint8})1, (\texttt{uint8})2, "\texttt{hello}") \rc\rc
$$  
where the elided declarations include a specialized radio and message type:
$$
\texttt{error\_t\ \ radio(\lc \texttt{src : uint8}; \texttt{dest : uint8}; \texttt{data : uint8[\,]} \rc)} \lc \ldots \rc
$$
\end{example}

\subsection{Serialization and Lifting}
\label{section-serialization}

Serialization generates a flattened nesT source code version of a Scalaness object in memory. At
the top level, serialization binds the value parameters of a module to the results of
flattening, aka lifting, via a sequence of declarations. Here is the precise definition.
\begin{definition}[Serialization]
\label{def-serialization}
Assume given a store $\jstore$ which is implicit in the following definitions. We define
serialization of Scalaness references as follows, along with an extension of the user defined
lifting relation to sequences of references:
\begin{mathpar}
\inferrule%[Serialize]
{\vect{\texttt{l}} \ser{\bm} \vect{\decl},\vect{e}}
{\serialize(\vect{x}, \vect{\t}, \vect{\texttt{l}}) = \vect{\decl} @\ {\vect{\t}\ \vect{x} = \vect{e}}}

\inferrule
{}
{\varnothing \ser{\jstore} \varnothing, \varnothing}

\inferrule
{\texttt{l} \ser{\jstore} \vect{d}, e \\ \ttvec{l} \ser{\jstore} \vect{d'}, \vect{e}}
{\texttt{l}\vect{\texttt{l}} \ser{\jstore} \vect{d} @ \vect{d'}, e\vect{e}}
\end{mathpar}
\end{definition}
Although lifting is user defined, a standard strategy is to introduce a new declared variable
for each memory reference in the lifted object, and bind the variable to the lifted referent.
Hence, lifting will typically be defined recursively. In my implementation, I have adapted a
``default'' lifting which follows this strategy, and also transforms objects by just
transforming the fields into a representative struct, and ignoring methods. I illustrate this
with an example in \autoref{section-implementation}.

The essence of this transformation can be formally captured with the following definitions. It
is easy to see that these definitions will satisfy the requirements of \autoref{def-lifting}.
\begin{example}
  In this example I allow lifting of any object references, and transform the object $o$ into a
  structure containing the transformed fields of $o$. Methods are disregarded by the
  transformation. Here is the specification of the type transformation:
\begin{mathpar}
\inferrule[ChapinT] {\CT(\ttt{C}) =
  \gclass{C}{\gbounds{X}{S}}{N}{\fieldvec{R}{f};\ K\ \ttvec{M}}}
          {\codt{\jinst{C}{\ttvec{T}}} = \lc \vect{\ttt{f}} :
            \codt{\ttvec{R}[\ttvec{T}/\ttvec{X}]} \rc}
\end{mathpar}
and here is the specification of lifting.
\begin{mathpar}
\inferrule[Chapin]
{\jstore(\texttt{p}) = \ttvec{l} \\ \fields{C} = \tdecls{T}{f} \\ 
 \ttvec{l} \ser{\jstore} \vect{d}, \vect{e} \\ x \ \text{ fresh}} 
{\jref{p}{\jinst{C}{\ttvec{R}}} \ser{\jstore} \vect{\decl} 
   @ (\xlet{\codt{\jinst{C}{\ttvec{R}}}}{x}{\lc \vect{\ttt{f}} = \vect{e}\rc}) , x}
\end{mathpar}
\end{example}


\subsection{Scalaness Type Checking}
\label{section-scalaness-typing}

\scalanesstypingfig

The Scalaness type checking rules adapt the typing rules of FGJ in their entirety, and I refer
the reader to \cite{FJ} for relevant details. Since type construction via \texttt{typedef} is
syntactic sugar for normal object construction, that is covered by those rules as well. It
remains to define typing rules for nesT modules and module operations.

The nesT module type form at the Scalaness level is $\jmodt{\tpdecl}{\jmodtcat}$, where
$\jmodtcat$ is a nesT module type. The $\tpdecl$ in this form represents the type bounds of
dynamically constructed types that have been used to instantiate the module; we refer to this
part of the type as the \emph{instance coercion}. Because these types are dynamically
constructed, their identity is not known statically, hence the need to treat them as
upper-bounded type names in the static type analysis. It is important to note that the type
names in $\tpdecl$ will be fully resolved at run time, so that any module generated by a
Scalaness program execution will have a fully reified nesT type, i.e.~an empty instance
coercion.

This is reflected in the \TirName{ModT} rule in \autoref{figure-scalanesstyping}, which connects
the nesT typing system with the Scalaness type system. Since in this case an uninstantiated
module definition is being typed its instance coercion is empty. An instance coercion in a
module type is directly populated when a module is instantiated, as in the \TirName{ModInstT}
rule. Here, the type instances $\ttvec{e}_1$ are all dynamically constructed, so they define the
upper bounds of the instantiated module's instance coercion. All type and value parameters are
expected to respect the typing bounds specified in the module definition.

A subtle but significant detail in this rule is the consequence of dynamically constructed types
having no meaning ``as types'' at the Scalaness level. This means that no Scalaness value of
that type can be constructed, so dynamically constructed type names do not occur in the typings
of value parameters. \note{Discuss the practical significance of this?}

The \TirName{ModWireT} typing rule for module wiring is a straightforward reflection of the
operational rule for module wiring, as is the \TirName{ModImageT} rule for module runnability
imaging.
\begin{example}
\label{example-scalanesstyping}
Returning to the code in \autoref{figure-example}, we may assign the following typing, where the
relevant type environment $G$ contains typing for free variables within the function
\texttt{nodeSpecialize}:
$$
  G \vdash \jinst{radioC}{adt}\ :\  
    \jmodt{\texttt{adt} \subtype \texttt{int32}}{\margs{}\lc\ ;
      \tt{error\_t\ radio(\t)} \rc }
$$
$$
\text{where}\ \t = \lc \texttt{src : adt}; \texttt{dest : adt}; \texttt{data : uint8[\,]} \rc 
$$
%and
%$$    
%  G \vdash \jinst{sendC}{adt} \ltimes \jinst{radioC}{adt}\ :\   
%    \jmodt{\texttt{adt} \subtype \texttt{int32}}{\margs{}\lc \ 
%      ;\,
%      \texttt{error\_t\ send(\texttt{adt\ s}, \texttt{adt\ d}, \texttt{data\ uint8[\,]})} \rc}
%$$
%$$
%\text{where}\ \t = \lc \texttt{src : adt}; \texttt{dest : adt}; \texttt{data : uint8[\,]} \rc 
%$$
\end{example}

\subsection{Foundational Insights and Type Safety} 
\label{section-framedml}

Type checking of modules and operations is inspired by the type theory and metatheory developed
for the language \fml\ \cite{FramedML}. Scalaness module instantiation in particular can be
decomposed into a set of \fml\ operations, and typeablity of module instantiation follows from
the typeablity of their composition. The language \fml\ is obtained by extending system \fsub\
with state, dynamic type construction, and staging features. The expression $\langle e \rangle$
is a code value, and the $\mathrm{lift}$ operation takes a value at one stage and ``lifts'' it
to the next, by turning it into code and performing any necessary serialization.

Given this, a Scalaness module with a value and type parameter can be
modeled in \fml\ as a term: 
$$\lambda x : \s_1 . \Lambda t \subtype \s_2 . \langle e \rangle$$
where $x$ and $t$ are value and type parameters for the block of code $\langle e \rangle$. Then,
module instantiation can be modeled as the application of this term to a type and value
parameter, where the latter must be lifted into the next stage:
$$
(\lambda x : \s_1 . \Lambda t \subtype \s_2 . \langle e \rangle)\ (\mathrm{lift}\ e)\ \t
$$ 

This interpretation of modules and module operations for the purposes of typing is evidenced by
the Scalaness type form $\jmodt{\tpdecl}{\jmodtcat}$, where $\tpdecl$ defines the type bounds
for dynamically constructed types used to instantiate a module. This is directly analogous to
$\exists$ type bindings in $\fml$ types, which statically define the upper bounds of dynamically
constructed types.

Observing that AFJ, FGJ, and \fml\ are all proven type safe, and that Scalaness is in essence an
orthogonal composition of these three languages, we conjecture that type safety is maintained in
this composition.

\begin{conject}[Scalaness Type Safety]
  If $\varnothing \vdash \texttt{e} : \texttt{T}$ and $\texttt{e} \rightarrow^* \jimage{\mu}$,
  then $\mu$ is runnable and does not go wrong.
\end{conject}

%%% Local Variables: 
%%% mode: LaTeX
%%% TeX-master: "main"
%%% End: 
