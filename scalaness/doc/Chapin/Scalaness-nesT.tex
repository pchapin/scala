\chapter{Scalaness/nesT}
\label{chapter-scalaness-nest}

In this chapter I cover the specific details of Scalaness and nesT, the practical realization of
DScalaness and DnesT. I first describe nesT along with the details of how nesT programs are
transformed into nesC programs. I also describe how the Scala compiler was modified to provide
Scalaness type checking with a minimum of disruption to the the compiler's existing
functionality. Finally I show a simple example of a Scalaness/nesT program that illustrates how
the languages work in practice. The use of Scalaness on a larger example is discussed in
\autoref{section-scalaness-evaluation}.

\section{NesT}
\label{section-nest-implementation}

\lstset{language=nesC}
\lstMakeShortInline!

NesT is the name give to the second stage language used by the implementation. Roughly speaking
nesT is the practical realization of DnesT. In particular, nesT uses the syntax of nesC to the
greatest extent possible in order to simplify the compiler and to minimize the learning burden
placed on existing nesC programmers. For example a nesT module is specified exactly as a nesC
module except that it can only !use! (import) and !provide! (export) nesC commands. In
particular neither nesC events nor nesC interfaces can appear in a nesT component specification.

NesT also supports DnesT's notion of subtyping and features for safe memory access and safe
casting. Thus while nesT programs are syntactically identical (aside from the new array
increment operator), and semantically similar to programs written in nesC, nesT programs are
more robust than equivalent nesC programs.

NesT type checking implements the rules described in \autoref{section-dnesttyping} and is
largely straight forward. The intention is to follow nesC's type system to the greatest extent
possible with changes to account for stricter rules disallowing implicit conversions. However,
nesT does support the subtyping rules of DnesT. The details of how nesT type checking was
implemented is described by Watson \cite{watson-masters-2013}.

NesT is implemented as a rewriting to nesC. Because of nesT's special relationship with nesC
this rewriting is largely trivial. However, the implementation of the special features of nesT
are described in more detail in this section.

The description that follows assumes the reader is familiar with nesC.

\subsection{Component Specifications}
\label{section-component-specifications}

In nesT components (after specialization) present interfaces that are sequences of imports and
exports. The imports are implemented as nesC commands that are ``used'' by the component and the
exports are implemented as nesC commands that are ``provided'' by the component. NesC-style
events are not part of nesT but can be simulated using commands; an event used in nesC becomes a
command provided in nesT and vice versa.

% TODO: The citation below should really be to the nesC specification.
% It's hard to find an appropriate citation for the specification, however (that isn't ancient).

NesT does not provide separate interfaces as nesC does. Instead all interaction with other
components is done by way of separately declared commands. These bare commands can be wired
together in the usual way by the nesC compiler \cite{Gay-nesC-2003}.
\autoref{figure-example-mininess} shows a simple example of a nesT module that interacts with a
timer. Instead of using an interface with an event !fired!, the module provides a callback
command of the same name.

\singlespace
\begin{fpfig}[tbhp]{Example nesT Module}{figure-example-mininess}
{
\begin{lstlisting}[language=nesC]
module ExampleC {
    uses command void setPeriodic( uint32_t period );
    provides command void fired( );
}
implementation {
    // Written in the nesT subset of nesC.

    void f( T param )
    {
        uint16_t value = x;
    }
}
\end{lstlisting}
}
\end{fpfig}
\primaryspacing

The example in \autoref{figure-example-mininess} also shows the use of an undeclared type !T!
and value !x!. Such types and values are instead declared as parameters of the module in the
Scalaness program. The Scalaness compiler first adds these parameters to the appropriate
environments before it type checks the nesT module \cite{watson-masters-2013}.

A nesT code base consists of a collection of unspecialized nesT modules. These modules do not by
themselves constitute a complete program. It is the job of the first stage Scalaness program to
specialize and compose the nesT modules, along with supporting components written in full nesC,
into full applications.

% Supporting configurations would also require DnesT to have type rules for module wiring.

NesT currently does not support nesC-style configuration components. Such support could
reasonably be added since the implementation of a component is not important to the first stage
code that manipulates it. The runnable module built by the Scalaness program as it specializes
and composes the constituent nesT modules is transformed by the Scalaness runtime system into a
nesC configuration reflecting the result of composition. However, this transformation is
transparent to the Scalaness programmer. Libraries written in full nesC must be wrapped in
components with nesT interfaces as described below in order to become part of a nesT
application.

Each nesT module has a nesT module type implied by its specification element list. This module
type is extracted from the specification element list when the nesT component is type checked.
In most cases, except as described below, it is compared against a module type annotation used
in the Scalaness program for the module (see \autoref{section-module-type}). Any discrepancy is
flagged by the Scalaness compiler as a type error.

\subsection{External Libraries}
\label{section-external-libraries}

My experiments with nesT show that it is expressive enough to write useful program components.
However, any realistic application will need to interact with various libraries written in full
nesC that I call \newterm{external libraries}. It is not my intention to require the whole
program be written in nesT for such a requirement would not be practical. Instead external
libraries could represent low level code such as the TinyOS operating system or high level
application code that wishes to use Scalaness generated nesT modules.

At the time of this writing neither nesT nor Scalaness provide any direct support for
interfacing to external libraries, although such support might be useful future work. However, a
programming technique can be used whereby shim components are manually created that wrap library
interfaces. Here I illustrate that technique using a small example.

Consider first the TinyOS !Boot! interface. This interface is used to indicate when a node is
started; all useful nesC programs must interact with it. Yet nesT does not support interfaces at
all much less some of the entities, such as events, that are commonly declared in interfaces.
Instead the programmer creates a shim component such as !BootShimC! as shown below

\singlespace
\vspace{1.0ex}
\begin{lstlisting}[language=nesC]
module BootShimC {
  uses command void booted( );
  uses interface Boot;
}
implementation {
  event void Boot.booted( )
  {
    call booted( );
  }
}
\end{lstlisting}
\vspace{1.0ex}
\primaryspacing

The shim component is legal nesC but not legal nesT. Its purpose is to expose all the commands
and events in an external library interface as bare commands. To this end wrapper command and
event implementations must be manually created.

Although creating shim components is a burden their form is highly stylized. A future version of
the Scalaness compiler might generate them automatically. However, some shim components are
complex and must do additional transformations on command arguments to interface with the
non-nesT external library commands/events. In any case, shim components can be reused across
nesT applications. Thus it is reasonable to expect programmers to accumulate a library of shims.

The shim components must be wired to the external library components they wrap. This is done by
producing two nesC \newterm{wrapping configurations}. The first, conventionally called
!LibraryIC! encapsulates all nesC components that have imports. The second, conventionally
called !LibraryEC! encapsulates all nesC components that have exports. Normally these are the
only two configurations an application needs. If the programmer has full control over the entire
application he or she can add the necessary external library components (via their shims) to
either !LibraryIC! and/or !LibraryEC! as appropriate.

For example, \autoref{figure-libraryicec} shows an example !LibraryIC! component and an example
!LibraryEC! component for a hypothetical application that uses the external !MainC! component
and a specific instance of the generic timer module, both from the TinyOS library. Notice that
the !SpecficTimerC! component appears in both wrappers since it both provides and uses at least
one command.

\singlespace
\begin{fpfig}[tbhp]{Example LibraryIC/EC configurations}{figure-libraryicec}
{
\begin{lstlisting}[language=nesC]
configuration LibraryIC {
  uses command void booted( );
  uses command void fired( );
}
implementation {
  components MainC, BootShimC, SpecificTimerC;

  BootShimC.booted = booted;
  BootShimC.Boot   -> MainC;

  SpecificTimerC.fired = fired;
}


configuration LibraryEC {
  provides command void startPeriodic( uint32_t period );
}
implementation {
  components SpecificTimerC;

  startPeriodic = SpecificTimerC.startPeriodic;
}
\end{lstlisting}
}
\end{fpfig}
\primaryspacing

NesC generic components must be instantiated in nesC configurations and each instance wrapped in
its own shim. In \autoref{figure-libraryicec} the !SpecificTimerC! component is a shim that
wraps a specific instance of the TinyOS generic timer. This limitation may seem restrictive, but
nesT has its own support for genericity although the two mechanisms are independent.

NesT does not support nesC configurations but Scalaness does allow the components such as shown
in \autoref{figure-libraryicec} to be declared and manipulated in Scalaness code. Such
components are represented as Scala objects that extend the !NesTComponent! trait and that
specify the source file of the nesC configuration using an !external! method as shown in
\autoref{figure-external-components}.

\singlespace
\begin{fpfig}[tbhp]{Representation of External Components}{figure-external-components}
{
\begin{lstlisting}[language=scalaness]
@ModuleType("""{}<;>
               { booted(): Void,
                 fired() : Void; }""")
object LibraryIC extends NesTComponent {
  external("LibraryIC.nc")
}


@ModuleType("""{}<;>
               { ; startPeriodic(period: UInt32): Void }""")
object LibraryEC extends NesTComponent {
  external("LibraryEC.nc")
}
\end{lstlisting}
}
\end{fpfig}
\primaryspacing

The module type of external components can't be determined by examining their definitions since
they are not in nesT. However, as with all nesT modules they must be annotated with their module
type in the Scalaness program as shown in \autoref{figure-external-components} and discussed
further in \autoref{section-module-type}. For external modules this annotation is accepted
without question by the compiler.

The !LibraryIC! and !LibraryEC! objects are then manipulated in the usual way by the Scalaness
program. Because !LibraryIC! has only imports and !LibraryEC! has only exports it is normal for
these components to appear at the ends of a wiring chain. \autoref{figure-full-wire} shows an
example where the result module is runnable. In \autoref{figure-full-wire} the !+>! symbol is
the Scalaness wiring operator.

\singlespace
\begin{fpfig}[tbhp]{Wiring nesT Components}{figure-full-wire}
{
\begin{lstlisting}[language=scalaness]
@ModuleType("""{ checksumType <: UInt32 }
               <;>
               { ; }""")
val resultModule =
  LibraryIC +> formattingModule +> checkingModule +> LibraryEC
\end{lstlisting}
}
\end{fpfig}
\primaryspacing

Although it is not possible to use Scalaness to compose external library components directly,
the programmer is free to create several different wrapping configurations if desired and
represent each of them separately in the Scalaness program. The Scalaness program could then
dynamically select which wrapping configuration is to be used in the final generated code. In
any case the type system will ensure that illegal wirings can never be made.

\subsection{Structure Subtyping}
\label{section-structure-subtyping}

% This covariant subtyping of pointers should probably be brought into DnesT.

DnesT supports width subtyping of structures as shown in \autoref{figure-subjudge}. To implement
this, nesT supports covariant subtyping of pointers to structure types. If $\tau_1$ and $\tau_2$
are structure types and $\tau_1 \subtype \tau_2$ according to DnesT subtype rules, then $\tau_1*
\subtype \tau_2*$. To implement the important case of passing a pointer to a structure to a
function, the Scalaness compiler need only add an appropriate cast as it rewrites the nesT to
nesC. For example, consider the nesT code below

\singlespace
\vspace{1.0ex}
\begin{lstlisting}[language=nesC]
struct X {
  int a;
};

struct Y {
  int a;
  int b;
};

void f( struct X * );
struct Y object;
f( &object );
\end{lstlisting}
\vspace{1.0ex}
\primaryspacing

% TODO: This stuff is currently unimplemented.
% I plan to implement at least some of this before my defense. Perhaps this section should be
% enhanced after that before final submission.

The call to !f! is rewritten to !f( (struct X*)&object )!. The structure layout rules of nesC
guarantee this is safe and !f! will only manipulate the !X! subobject of its parameter.

% Structure assignment is rewritten as a slicing operation where only the individual members of
% the subobject are assigned to the target.

Like DnesT, nesT has no notation to indicate a subtype relation between structures. Instead the
judgment is entirely based on structural considerations.

\subsection{Safe Casts}
\label{section-safe-casts}

Since one of the goals of nesT is to promote type safety no implicit type conversions, aside
from subtype conversions, are provided. Explicit conversions are permitted only when configured
by the programmer. This allows the programmer can define certain casts that are logical even if
they require non-trivial user defined code to execute.

The Scalaness compiler accepts a configuration file that defines a relation on types
\textit{isCompatible}. If \textit{isCompatible($T_1$, $T_2$)} is true then it is permitted to
cast an expression of type $T_1$ into an expression of type $T_2$. There are no restrictions on
the types $T_1$ and $T_2$. However, all such conversions require explicit cast expressions; they
are never applied implicitly. The \textit{isCompatible} relation is the implementation of
\textit{compatible} in the \TirName{CastT} rule of \autoref{figure-coretyping}.

To illustrate the way these casts are implemented in nesT programs consider as an example the
following two structure definitions.

\singlespace
\vspace{1.0ex}
\begin{lstlisting}[language=nesC]
struct UserInfo {
  char name[25];
  int  age;
  int  id;
};

struct UserToken {
  int id;
  int hash;
};
\end{lstlisting}
\vspace{1.0ex}
\primaryspacing

The programmer may wish to allow an object of type !UserInfo! to be explicitly cast into an
object of type !UserToken!. Assuming the Scalaness configuration file has been edited to allow
this, the Scalaness compiler rewrites each cast expression into a call of a conventionally named
conversion command. These conversion commands exist in a nesC interface !DoCast!. For example

\singlespace
\vspace{1.0ex}
\begin{lstlisting}[language=nesC]
token = (struct UserToken)user;
// ... rewritten to ...
token = (call DoCast.UserInfo_UserToken(user));
\end{lstlisting}
\vspace{1.0ex}
\primaryspacing

The programmer is required to provide the !DoCast! interface and a component !DoCastC! that
provides that interface and contains an implementation of the various conversion commands
needed. The Scalaness compiler wires to !DoCastC! automatically without any further programmer
intervention.

% TODO: What about conversions between types that are type parameters?

\subsection{Array Operations}
\label{section-array-operations}

Each nesT expression !a! of array type !Array(T)! for some element type !T! has a corresponding
hidden dynamic value representing the size of the array. Array increment expressions of the form
!a |> e! can nominally be rewritten to nesC using pointer arithmetic as !( (a) + (e) )!. Let $n$
be the dynamic size of expression !a!, then $n_e$, the dynamic size of !a |> e!, is $n_e = n -
e$. This size might be negative but any use of an array expression with a negative size results
in a run time error \emph{at the point of use}.

A statement containing one or more array increment expressions or array indexing expressions is
rewritten as a block enclosed sequence of statements containing Scalaness compiler generated
local variables for the dynamic sizes of the temporary arrays along with appropriate run time
checks.

For each array increment operation $a_i \rhd e_i$ in a statement a variable to hold the value of
$e_i$ is declared and initialized. This is done so that $e_i$ will only be evaluated once; an
important considering in a language, such as nesT, with side effects. Also the dynamic size of
the result of each array operation $d_i$ is declared and initialized appropriately. For example

\singlespace
\vspace{1.0ex}
\begin{lstlisting}[language=nesC]
... (a |> e) ...
\end{lstlisting}
\vspace{1.0ex}
\primaryspacing

Is rewritten without regard to any possible optimizations as:

\singlespace
\vspace{1.0ex}
\begin{lstlisting}[language=nesC]
{
  int __e_1 = e;
  int __d_1 = __d_0 - __e_1;

  ... ((a) + __e_1) ...
}
\end{lstlisting}
\vspace{1.0ex}
\primaryspacing

Here !__d_0! is the dynamic size associated with the array expression !a!. In the common case
where !a! is a declared array the size will be known statically and an appropriate constant can
be used instead of a reference to a dynamic size variable.

For each array indexing operation $a_i[e_i]$ in a statement a variable to hold the value of
$e_i$ is declared and initialized, as before. A run time check is inserted to ensure that the
value of $e_i$ is inside the dynamic size of $a_i$. For example

\singlespace
\vspace{1.0ex}
\begin{lstlisting}[language=nesC]
... a[n] ...
\end{lstlisting}
\vspace{1.0ex}
\primaryspacing

Is rewritten without regard to any possible optimizations as

\singlespace
\vspace{1.0ex}
\begin{lstlisting}[language=nesC]
{
  int __e_1 = n;
  if (__e_1 >= __d_0) call boundsCheckFailed( );
  ... a[__e_1] ...
}
\end{lstlisting}
\vspace{1.0ex}
\primaryspacing

As before !__d_0! is the dynamic size associated with the array expression !a!.

In a statement involving multiple array operations, each operation is rewritten as described
above one at a time. After the first operation is rewritten, the enclosed modified statement is
further expanded with the second rewriting. The checks are issued in the order they are
encountered during a depth first left to right traversal of the nesT abstract syntax tree. For
example a statement such as

\singlespace
\vspace{1.0ex}
\begin{lstlisting}[language=nesC]
x = ((a |> e1) |> e2)[b[i]];
\end{lstlisting}
\vspace{1.0ex}
\primaryspacing

Is first rewritten as:

\singlespace
\vspace{1.0ex}
\begin{lstlisting}[language=nesC]
{
  int __e_1 = e1;
  int __d_1 = __d_0 - __e_1;
  x = (((a) + __e_1) |> e2)[b[i]];
}
\end{lstlisting}
\vspace{1.0ex}
\primaryspacing

The resulting statement still contains three array operations. The second stage of rewriting
yields

\singlespace
\vspace{1.0ex}
\begin{lstlisting}[language=nesC]
{
  int __e_1 = e1;
  int __d_1 = __d_0 - __e_1;
  {
    int __e_2 = e2;
    int __d_2 = __d_1 - __e_2;
    x = ( ((a) + __e_1) + __e_2 )[b[i]];
  }
}
\end{lstlisting}
\vspace{1.0ex}
\primaryspacing

The inner indexing operation is then rewritten

\singlespace
\vspace{1.0ex}
\begin{lstlisting}[language=nesC]
{
  int __e_1 = e1;
  int __d_1 = __d_0 - __e_1;
  {
    int __e_2 = e2;
    int __d_2 = __d_1 - __e_2;
    {
      int __e_3 = i;
      if (__e_3 >= __d_b) call boundsCheckFailed( );
      x = ( ((a) + __e_1) + __e_2 )[b[__e_3]];
    }
  }
}
\end{lstlisting}
\vspace{1.0ex}
\primaryspacing

Finally, the outer indexing operation is rewritten in a similar manner, including an additional
call to !boundsCheckFailed!.

% Notice that declarations can't be enclosed in the block without changing the scope of the
% identifiers they declare. Thus array expressions used in the initialization of a variable
% present a special problem. It might be necessary to first split initializers off from their
% declarations in a separate AST processing pass.

Functions declared to take an array as a parameter are rewritten so that the dynamic size of the
array is passed as an additional parameter. This parameter becomes the $d$ of array expressions
involving only the parameter. Calls to such functions are rewritten to pass the additional
dynamic size information as appropriate.

The command !boundsCheckFailed! must be provided by the programmer in a component named
!BoundsCheckC!. The behavior of this command is unspecified but it should not return. I expect
in most cases it will restart the node after, perhaps, attempting to log the problem. As with
!DoCastC! the Scalaness compiler automatically wires to !BoundsCheckC! as appropriate.

The user defined handling of bounds check failure and of explicit casts as described previously
is where the runtime failure semantics of Definition~\ref{definition-runtime-failure} are
implemented.

\lstDeleteShortInline!

\section{Scalaness}
\label{section-scalaness-implementation}

\lstset{language=scalaness}
\lstMakeShortInline!

Scalaness is implemented as a modified Scala compiler \cite{scalaness-home} based on the open
source development Scala compiler. The Scala compiler has a plug-in architecture and originally
I hoped Scalaness could be implemented as a compiler plug-in. That would have made Scalaness
easier to use and maintain and thus enhanced the systems practicability.

Unfortunately, implementing Scalaness as a plug-in met with difficulties. The main problem was
with extending the type checker of Scala to accommodate the Scalaness type system. The plug-in
approach required a complete reimplementation of Scala typing inside the plug-in. This is
because plug-ins can only gain control either before Scala typing has occurred or after it has
completed. Thus the implementation of the Scalaness typing rules couldn't easily benefit from
the logic in the existing type checker.

In contrast building Scalaness as a modified compiler allowed Scalaness type information to
``piggyback'' on the existing type checker infrastructure. In particular, Scalaness type
information was added to the singleton types already created and maintained by the Scala
compiler for each declared value. This information could then be queried at critical points
during the type checking process where Scalaness rules, as shown in
\autoref{figure-scalanesstyping}, were applied \cite{watson-masters-2013}.

However, in order to facilitate keeping Scalaness synchronized with future developments of the
main Scala compiler every attempt was made to implement Scalaness in the least invasive way
possible. Much of the logic, including the new typing rules themselves are implemented in
separate packages away from the main body of the compiler code base. The places where it was
necessary to insert Scalaness specific code into, for example, the existing type checker, have
been kept to a minimum.

Making radical changes to Scala syntax was not seriously considered. For reasons of simplicity I
didn't want to modify \emph{both} the parser and the type checker. Fortunately Scala has a
general mechanism for adding arbitrary information to declarations, namely
\newterm{annotations}, and I used Scala annotations to express nesT module types as strings
using a syntax of my choosing.

My work is based on Scala 2.10. This version of Scala also provides an extensive reflection API
and experimental support for expression macros. These facilities allow one to do abstract syntax
tree (AST) transformations on Scala programs using ordinary Scala code. Macros are described by
the Scala community as a kind of ``lightweight'' plug-in mechanism. Unfortunately at the time of
this writing type macros are not available so it is not yet possible to write a macro that
outputs a class definition. However, in the future when type macros become available it might be
possible to implement some or all of Scalaness as a macro library.

\subsection{Scala Compiler Organization}
\label{section-scala-compiler-organization}

The Scala compiler is organized as a number of \newterm{phases} that rewrite the input in
successive steps lowering it to JVM bytecode. The precise phases used can be listed with the
command \texttt{scalac -Xshow-phases}. Of primary significance to Scalaness are the first four
phases used by the stock Scala compiler as shown below.

\singlespace
\begin{Verbatim}
parser
namer
packageobjects
typer
...
\end{Verbatim}
\primaryspacing

The bulk of the modifications made by Scalaness are in the typer phase. There hooks were added
at critical points that call into Scalaness specific code in package !edu.uvm.scalaness!. In
addition a new phase was added between the parser and namer phases. This new phase is
responsible for augmenting certain Scalaness constructs with their necessary runtime support.
This is done by inserting material in the AST produced by the parser. In principle that material
could have been manually written by the programmer but instead is automatically generated as a
convenience. It is this ``post-parser'' phase that could potentially be eliminated by type
macros when they become available.

\subsection{Liftable Types}
\label{section-liftable-types}

Certain types and their corresponding values that appear in a Scalaness program are liftable to
types and values in the nesT modules manipulated by that program as described in
\autoref{section-serialization}. Values of these types need to be transformed as they cross the
boundary between the two programming languages due to differences in the way a liftable type and
its nesT counterpart are represented. In this section I will describe which Scalaness types are
liftable and how their values are handled when used to specialize a nesT module.

\subsubsection{Primitive Types}
\label{section-liftable-primitives}

All liftable types except arrays are subtypes of a special marker trait !Liftable!. The
primitive types in nesT have liftable counterparts in Scalaness that are classes extending
!Liftable!. For example the type !uint16_t! in nesT corresponds with class !UInt16! in
Scalaness. In my implementation there are six primitive, liftable integer types: three unsigned
types !UInt8!, !UInt16!, !UInt32!, and three corresponding signed integer types. All of these
types have specific sizes; the implementation does not provide a simple integer type. This
avoids issues associated with the machine dependent size of !int! in nesC. Finally two other
liftable primitive types are also provided: !Char! and !Uninit! (which lifts to !void!).

The nesT subtype relations are preserved in Scalaness where !UInt8 <: UInt16 <: UInt32! and
similarly for the signed integer types. In Scalaness these types are defined in the object
!LiftableTypes! so that they don't conflict with any normal (non-liftable) types defined by the
programmer or the language, such as !Char!. Furthermore the integer liftable types are endowed
with the usual arithmetic operations so they can be manipulated in the Scalaness program in a
natural way.

No conversions are provided between the liftable types and their ordinary Scala analogs. This
means existing libraries that, for example, manipulate Scala !Char! objects won't work with
!LiftableTypes.Char!. I do not consider this a problem for two reasons.
\begin{enumerate}
\item Since values of liftable type will eventually be written into nesT components, they will
  likely be put to very different uses than values of ordinary Scala types. In fact, letting the
  Scala type system catch inadvertent mixing of ordinary primitives and liftable primitives
  could be seen as a desirable feature.
\item Implicit conversions can be easily added by the Scalaness programmer if desired using the
  normal facilities of Scala.
\end{enumerate}

To facilitate the second point, \emph{explicit} conversion methods from each liftable type to
its corresponding non-liftable counterpart are provided as a convenience.

The Scala type system is used to ensure compile-time type safety of the primitive liftable types
in a Scalaness program. For example the type !Int16! can only be constructed using a value of
Scala's type !Short!. Consequently normal Scala type checking prevents a potentially out of
range value from being used.

Unfortunately Scala does not support unsigned types natively. In the current implementation a
sufficiently wide signed type is used to initialize objects of unsigned liftable type. This
makes it possible to use an out-of-range value during the execution of the Scalaness program
resulting in a runtime exception. However, Scala programs are subject to runtime exceptions for
a variety of reasons. It is well outside the scope of this work to address the problem of how to
ensure a Scalaness program never exits by way of an exception.

\subsubsection{Arrays}
\label{section-liftable-arrays}

\emph{An ordinary Scala array type is liftable if and only if its element type is liftable.}
This is an exception to the rule stated in \autoref{section-liftable-primitives} that says all
liftable types must extend !Liftable! and thus arrays are handled in a special way. Yet it is a
significant convenience to the programmer to be able to use ordinary Scala arrays, and not some
special ``liftable array'' class, to hold liftable arrays. This need does not arise for the
other containers in the Scala collections library since those containers have no counterpart in
nesT anyway.

For example, the programmer may wish to create and manipulate a !List[UInt8]! during the
execution of a Scalaness program, but the list itself won't be liftable. In contrast the
programmer may wish to lift an !Array[UInt8]! into nesT.

\subsubsection{Classes}
\label{section-liftable-classes}

A Scala class $C$ that extends the !Liftable! trait is liftable to a nesT structure type
provided it additionally obeys the following inductive rules.

\begin{enumerate}
\item $C$ is not generic.
\item All of $C$'s fields have liftable type.
\item All of $C$'s supertypes (except !AnyRef! and !ScalaObject!) are liftable types.
\end{enumerate}

In this case $C$ is said to be a \textit{liftable class}. Except for the rules mentioned here
there are no restrictions on the definition or use of liftable classes. In particular, they are
able to have methods, although the methods of a liftable class have no manifestation in the
generated nesT code and would exist only as a convenience to the Scalaness programmer.

For example, consider the following Scalaness code:

\singlespace
\vspace{1.0ex}
\begin{lstlisting}[language=scalaness]
class Header
 (val nodeID      : nodeIDType,
  val componentID : UInt8 ) extends Liftable

class TimeStampedHeader
 (val timeStamp   : UInt16) extends Header
\end{lstlisting}
\vspace{1.0ex}
\primaryspacing

Here !nodeIDType! is a previously defined liftable class type. Consequently both of these
classes are liftable and have representations as nesT structure types.

% TODO: Figure out how to properly model Scala's inheritance system in this context.

\subsection{Lifting}
\label{section-lifting}

When nesT modules are specialized by values, the Scala values used to make those specialization
are lifted into the nesT code. For primitive types occurrences of the value parameter in the
nesT code is simply replaced by a constant representing the actual value used to specialize the
module.

Values with array or structure (class) types are handled differently. In that case the Scalaness
compiler writes a global declaration into the nesT module that defines the value along with an
initializer constructed from the Scala value used to specialize the module. This follows the
semantics described in \autoref{section-dscalaness-semantics}.

As an example, consider the following Scalaness class representing a nesT module that does
encryption. The module is parameterized by a key value.

\singlespace
\vspace{1.0ex}
\begin{lstlisting}[language=scalaness]
@ModuleType(
  """{}
     < ; key: Array[UInt8] >
     { ; encrypt(data: Array[UInt8]): Void }""")
class EncryptorC extends NesTComponent {
  "EncryptorC.nt"
}
\end{lstlisting}
\vspace{1.0ex}
\primaryspacing

When this module is instantiated as described in \autoref{section-instantiation} a Scala array
of !UInt8! values is provided. The Scalaness compiler will output, at second stage generation
time, nesT code such as

\singlespace
\vspace{1.0ex}
\begin{lstlisting}[language=nesC]
module EncryptorC {
  provides void encrypt(uint8_t data[]);
}
implementation {
  // Added by the Scalaness compiler.
  uint8_t key[] = { 1, 2, 3, 4 };
  // Uses of 'key' as before.
}
\end{lstlisting}
\vspace{1.0ex}
\primaryspacing

Here !{ 1, 2, 3, 4}! is a sample value of the key parameter used to instantiate the module. The
nesT programmer does not declare the global variable !key! in the nesT module but nevertheless
uses the name !key! freely in the module. In effect the nesT programmer is using the parameter
declaration (in the Scalaness code) to guide her work. This follows the behavior of the nesT
type checker.

The example above shows the result before the final translation to nesC. During that translation
the Scalaness compiler will also augment the parameter list of !encrypt! to include an
additional size parameter for the array as described in \autoref{section-array-operations}.

% TODO: Describe also how types are lifted.

\subsection{MetaType}
\label{section-metatype}

Scalaness allows types to be dynamically constructed. However, the Scala type system does not
directly support using types as values. To work around this limitation I explicitly introduced a
wrapper generic class !MetaType[T]! to represent any liftable type that is a subtype of !T!.

\singlespace
\vspace{1.0ex}
\begin{lstlisting}[language=scalaness]
class MetaType[+Tau <: Liftable]
  (val wrappedType: TypeRepresentation) extends Liftable
\end{lstlisting}
\vspace{1.0ex}
\primaryspacing

Values in the Scalaness program that are intended to hold nesT types that are a subtype of
$\tau$ have a Scala type of !MetaType[Tau]!. A variance annotation is used to ensure that
!MetaType! is covariant in its type parameter. This allows flexibility since, for example, a
!MetaType[UInt16]! value should be usable where a !MetaType[UInt32]! is expected. This is sound
since the subtype relation is transitive and, for example, any type that is a subtype of
!UInt16! is also a subtype of !UInt32!.

Objects of type !MetaType! contain a \emph{representation} of a nesC type. While !MetaType!'s
type parameter is a Scala type that is liftable to nesT, the value it wraps is a representation
of the already lifted type. Thus !MetaType! objects form a bridge between the Scalaness and nesT
type systems.

% TODO: Is it possible for wrappedType to not be a subtype of Tau? It would seem so.
% What happens in that case?

\subsection{Module Type Annotations}
\label{section-module-type}

Values definitions, method parameters, and method results that are intended to manipulate nesT
component values must be explicitly decorated with a module type annotation. In this way nesT
type information can be made known to the Scalaness compiler without modifying the Scala parser
to understand an extended type language directly. Module type annotations are string literals
that obey the abbreviated syntax in \autoref{figure-module-type-syntax}.

\singlespace
\begin{fpfig}[tbhp]{Module Type Syntax}{figure-module-type-syntax}
{
\begin{Verbatim}
module-type ::= '{' existential-binders? '}'
                '<' type-parameters? ';' value-parameters? '>'
                '{' imports? ';' exports? '>'

existential-binder ::= IDENTIFIER '<:' type
type-parameter     ::= IDENTIFIER '<:' type
value-parameter    ::= IDENTIFIER  ':' type
\end{Verbatim}
}
\end{fpfig}
\primaryspacing

The imports and exports in \autoref{figure-module-type-syntax} are nesT declarations written in
a Scala-like syntax using !Array! and !PointerTo! type constructors to define array and pointer
types. Structure types are specified using \{ \ldots \} syntax to enclose the declarations of
structure members and are prefixed by the structure name.

The following shows a sample of a !SendC! component parameterized by an integer type suitable
for use as a network address. The component imports a command !radio! that takes a parameter
!message! of structure type. The component exports a command !send! that returns the TinyOS
standard error type !error_t! which has built in support in module type annotations.

\singlespace
\vspace{1.0ex}
\begin{lstlisting}[language=scalaness]
@ModuleType(
  """{}
     < addrT <: UInt32; >
     { radio(message:
         MessageType{src : addrT,
                     dest: addrT,
                     data: Array[UInt8,64]}): ErrorT;
       send(s: addrT, d: addrT, data: Array[UInt8]): ErrorT }""" )
class SendC extends NesTComponent { ...
\end{lstlisting}
\vspace{1.0ex}
\primaryspacing

Annotations, such as above, that are placed on class or object definitions are checked against
the nesT code that implements that component---except for external library components as
described in \autoref{section-external-libraries}. Method parameters, method results, and !val!
and !var! definitions also need to be explicitly annotated; Scalaness does not support type
inference of nesT types. This does place a considerable burden on the programmer. However, a
type abbreviation scheme has been developed to alleviate this burden \cite{watson-masters-2013}.

In places where module types are required to be annotated, the annotated type is checked against
the actual type derived by the Scalaness type rules in \autoref{figure-scalanesstyping}. Type
errors are reported as necessary.

\subsection{Component Declarations}
\label{section-component-declarations}

Components in nesT can be parameterized by types and values and instantiated multiple times.
These properties are closely modeled by Scala classes. Thus the representation of a nesT
component in Scalaness is by way of a class that extends a special !NesTComponent! marker trait.

One might be tempted to allow a syntax such as

\singlespace
\vspace{1.0ex}
\begin{lstlisting}[language=scalaness]
class SendC
  [Adt <: UInt32, MessageT <: AbstractMessage[Adt]]
  (self: Adt) extends NesTComponent {

  import error_t radio( MessageT * );

  export error_t send( Adt addr, uint8_t *data ) {
    MessageT packet = { self, addr, data };
    radio( &packet );
    return SUCCESS;
  }
}
\end{lstlisting}
\vspace{1.0ex}
\primaryspacing

This intends to define a Scalaness component using Scala syntax for representing type and value
parameters with the body of the component written in nesT. Unfortunately this can't be supported
without modifying the Scala parser to accept nesT as well as Scala, and it was my intention to
avoid this complication.

One way to work around the problem of mixed language syntax is presented by Garcia
\cite{Garcia-Scala-Query} where the ``alien'' language is included as a string literal. The
Scala type checker would treat the nesT program as having type String but the additional
Scalaness type checking could parse the string literal's contents and impose additional typing
rules on those contents. However, this approach leads to a rather ungainly programming style:

\singlespace
\vspace{1.0ex}
\begin{lstlisting}[language=scalaness]
class SendC
  [Adt <: UInt32, MessageT <: AbstractMessage[Adt]]
  (self: Adt) extends NesTComponent {

 """import error_t radio( MessageT * );

    export error_t send( Adt addr, uint8_t *data ) {
      MessageT packet = { self, addr, data };
      radio( &packet );
      return SUCCESS;
    }"""
}
\end{lstlisting}
\vspace{1.0ex}
\primaryspacing

Since the nesT code implementing a component is often long and complex it makes sense to allow
the programmer to edit and manage that code in tools that are nesC-aware such as nesC syntax
highlighting editors. I anticipate that in many cases different programmers with very different
kinds of expertise will be editing the Scalaness and nesT code bases. Thus my implementation
uses a string literal to name an external file containing the nesT contents of a component as
shown below.

\singlespace
\vspace{1.0ex}
\begin{lstlisting}[language=scalaness]
class SendC
  [Adt <: UInt32, MessageT <: AbstractMessage[Adt]]
  (self: Adt) extends NesTComponent {

  "SendC.nt"
}
\end{lstlisting}
\vspace{1.0ex}
\primaryspacing

Conceptually the contents of the named file replace the literal name in the Scalaness program
much as \#include directives work in C programs. This approach allows the normal Scala parser
and type checker to process the program successfully. During compilation the Scalaness extension
locates the specified nesT file (\filename{SendC.nt} above), parses it as nesT and does nesT
type checking on that file using type and value parameters as provided to the Scalaness class.

Using Scala's syntax for specifying type and value parameters as shown above is attractive but
unfortunately it does not work for Scalaness. There are two problems.
\begin{enumerate}
\item I wanted to allow nesT components to be passed around in a Scalaness program in an
  uninstantiated state. In contrast Scala classes are not first class values in Scala.
\item NesT components can have parameters involving dynamically constructed types. In contrast
  Scala class parameters must involve only types that are fully specified statically.
\end{enumerate}

I considered modifying the Scala compiler to allow dynamically constructed types to appear in
declarations and as type parameters but this required extensive modifications to the existing
Scala type checker and so was rejected as an option. Instead the Scalaness representation of a
nesT component includes a special method !instantiate!, automatically generated by the compiler,
that is used to create instantiated nesT components. That method accepts the value parameters as
ordinary Scala parameters using liftable types, and it accepts the type parameters as ordinary
Scala parameters of type !MetaType[T]! where !T! is liftable. The example above becomes:

\singlespace
\vspace{1.0ex}
\begin{lstlisting}[language=scalaness]
class SendC extends NesTComponent {

  def instantiate(
    Adt     : MetaType[UInt32],
    MessageT: MetaType[AbstractMessage],
    self    : UInt32);

  "SendC.nt"
}
\end{lstlisting}
\vspace{1.0ex}
\primaryspacing

% TODO: Talk about the significance of having to declare self as UInt32 instead of as Adt.

\subsection{Runtime Support}
\label{section-runtime-support}

In addition to compile-time analysis and type checking, Scalaness programs require support for
nesT module composition (aka ``wiring'') at runtime. In this section I give an overview of how
this runtime support works. Full details can be found in the documented source code of Scalaness
\cite{scalaness-home}. I note that the current implementation of Scalaness composes nesT modules
and rewrites those modules to nesC in a single, integrated processing step. The runtime system
currently emits nesC directly without the need for any explicit nesT-to-nesC rewriting.

There are two operations to consider: component composition and component instantiation.

\subsubsection{Composition}
\label{section-composition}

Each Scalaness class that represents a nesT module is augmented by the Scalaness compiler to
contain a hidden field that represents a nesC configuration wrapping that single component.
\autoref{figure-checksum-class} shows an example of a Scalaness class representing a nesT module
that provides a command for computing checksums on a given array of bytes. This module is
parameterized by the type used for the checksum and by the size of the arrays that it processes.

\singlespace
\begin{fpfig}[tbhp]{Generated Runtime Support for Composition}{figure-checksum-class}
{
\begin{lstlisting}[language=scalaness]
@ModuleType(
  """{}
     < checksumType <: UInt32; size: UInt16 >
     { ; compute_checksum(data: Array[UInt8]): checksumType }""")
class ChecksumC extends NesTComponent {

  /////////
  // Code generated by the Scalaness compiler.
  /////////
  val configuration =
    new ProgramComponentWrapper(new NamedProgramComponent(
      name            = "ChecksumC",
      enclosingObject = this,
      typeParameters  = Set("checksumType"),
      valueParameters = Set("size"),
      imports         = extractImports(abstractSyntax),
      exports         = extractExports(abstractSyntax),
      abstractSyntax  = abstractSyntax))
    
  def getTypeMap  = Map("checksumType" -> sclnsChecksumType)
  def getValueMap = Map("size" -> sclnsSize)
  ////////
  // END of Scalaness generated code.
  //////// 
           
  "ChecksumC.nt"
}
\end{lstlisting}
}
\end{fpfig}
\primaryspacing

The code marked in \autoref{figure-checksum-class} as being generated by the compiler is not
legal Scala as shown but is presented as an aid to understanding. The compiler actually inserts
during compilation a type-correct AST of the necessary code into the AST of the enclosing
class. The Scalaness runtime library, specifically methods in the inherited !NesTComponent!
trait, makes use of the generated code during module composition.

In particular, the generated field !configuration! holds the reference to an object representing
a nesC configuration that wraps the nesT module. The imports and exports of that module are
extracted from the abstract syntax tree of the nesT code which is parsed by the runtime system
and is represented by !abstractSyntax! in \autoref{figure-checksum-class}. Note that the nesT
code has already been syntax checked and type checked by the Scalaness compiler during the
compilation of the Scalaness program. The reparsing done at runtime is thus guaranteed to
succeed.

The named component wrapped by the configuration is made aware of the names of the type and
value parameters. Furthermore the Scalaness compiler augments the class with methods
!getTypeMap! and !getValueMap! that return maps associating those names with additional hidden
fields (the definitions of which are not shown in \autoref{figure-checksum-class}) holding the
runtime representation of the type and value arguments actually used. This information is used
during module instantiation as described in \autoref{section-instantiation}.

The method !+>! inherited from !NesTComponent! combines the configurations in its operands to
return a new program component representing the overall nesC configuration. It is in the !+>!
method where the operational semantics of wiring is implemented (see
\autoref{figure-jmodsemantics}). The program component returned from !+>! is flattened in the
sense that it is a single configuration that wires all the named program components (nesT
modules) it contains; a hierarchy of configurations is not created.

An !image! method in the !ProgramComponent! class writes the nesC configuration implied by the
current program component and then iterates over all the named components serializing their
abstract syntax trees to nesC. During this serialization the nesT to nesC transformations, for
example array bounds checks, etc., as described in \autoref{section-nest-implementation} are
also done.

\subsubsection{Instantiation}
\label{section-instantiation}

A class representing a nesT component can be instantiated using Scala's operator !new! like any
other class. However, such an instance still has an uninstantiated module type. Instantiation of
a component at the nesT level is done in Scalaness by invoking a compiler generated
!instantiate! method, an example of which is shown in \autoref{figure-checksum-instantiate}.

\singlespace
\begin{fpfig}[tbhp]{Generated Instantiate Method}{figure-checksum-instantiate}
{
\begin{lstlisting}[language=scalaness]
@ModuleType(
  """{}
     < checksumType <: UInt32; size: UInt16 >
     { ; compute_checksum(data: Array[UInt8]): checksumType }""")
class ChecksumC extends NesTComponent {

  /////////
  // Code generated by the Scalaness compiler.
  /////////
  private var sclnsChecksumType: MetaType[UInt32] = null
  private var sclnsSize: UInt16 = null

  def instantiate(checksumType: MetaType[UInt32], size: UInt16) = {
    val result = new ChecksumC
    result.sclnsChecksumType = checksumType
    result.sclnsSize = size
    result
  }
  ////////
  // END of Scalaness generated code.
  //////// 
           
  "ChecksumC.nt"
}
\end{lstlisting}
}
\end{fpfig}
\primaryspacing

As for \autoref{figure-checksum-class} the code shown as generated by the compiler is for
illustration only. The compiler inserts during compilation the AST of the appropriate code into
the AST of the enclosing class.

The !instantiate! method is provided module type parameters as Scala !MetaType! values, and
module value parameters, all of which must have liftable types. The method creates a fresh Scala
instance of the class and stores the module parameters into hidden fields where they are
subsequently used (during imaging) to specialize the module's body. A single instance of the
Scala class could thus create many different specializations of the nesT module by way of
separate invocations of !instantiate! with potentially different parameters.

\lstDeleteShortInline!

%%% Local Variables: 
%%% mode: LaTeX
%%% TeX-master: "main"
%%% End: 
