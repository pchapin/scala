
\chapter{Evaluation}
\label{chapter-evaluation}

In this chapter I present results of evaluating the performance of Sprocket and Scalaness/nesT.
I do this in terms of both simple ``toy'' programs that explore specific issues in isolation and
on a larger, realistic application employing trust management authorization that demonstrates
the applicability of the systems in real world scenarios.

\section{Field Example}
\label{section-field-example}

To evaluate the performance of SpartanRPC and Scalaness in a real application setting, I used
both systems to implement secure versions of data collection and sampling control protocols in
an environmental monitoring system. The Snowcloud system
\cite{frolik-skalka-snowcloudtr,moeser-walker-skalka-frolik-wsc11} is a wireless sensor network
developed at the University of Vermont for snow hydrology research applications. It is based on
the MEMSIC TelosB mote platform running TinyOS, and has seen multiple field deployments. Typical
deployed systems comprise 4-8 sensor nodes but the technology is scalable to arbitrary numbers
of nodes. For data collection and sampling rate control, the system also includes a handheld
``Harvester'' device. This device incorporates a TelosB mote to establish a network connection
when in radio communication with the deployment. Users transport the device to and from
deployment sites, and interact with the sensor node network by issuing commands from a simple
push-button interface. A Harvester device and a deployed Snowcloud sensor tower are pictured in
\autoref{figure-snowcloud}. The scheme described here has been implemented and tested in the UVM
test network, which uses the same software and hardware platforms as in the active deployments.

\snowcloudfig

In the secured version of the Snowcloud system, the goal is to treat data collection and
sampling rate control as protected resources requiring authorization. Furthermore, sampling rate
modifications should require a higher, ``administrator'' level of authorization than data
collection. That is, only system engineers should be able to perform control operations, whereas
data end-users making field visits should be able to collect data. Snowcloud sensor node code in
particular makes use of nearly every resource available on the mote---including timing, sensor
I/O, radio messaging, and flash memory, not to mention CPU and main memory. Thus, it is a robust
example of a realistically scaled application.

The system described here is also informative since it can be easily ported to other similar
application settings. That is, sensor network application settings wherein multiple users of
various authorization levels need to interact with the same network in control or collection
capacities, as mediated by security policy.

\section{\Sprocket}
\label{section-sprocket-evaluation}

In this section I discuss the performance of the programs generated by \Sprocket\ in terms of
both space and time. I begin by evaluating \Sprocket\ using ``toy'' programs that focus on
specific aspects of the system's performance. Next I discuss the performance of \Sprocket\ on
the field example. I show that the combined use of public and private key cryptography in the
underlying security protocol imposes a low amortized cost over time, despite high costs for
initial authorizations.

Since many communication chips now support hardware AES encryption, I was primarily interested
in demonstrating performance using that feature. In particular, the popular Tmote Sky wireless
sensor mote \cite{tmotesky-datasheet} uses a Chipcon CC2420 transceiver with hardware
encryption. Unfortunately, the standard TOSSIM simulation environment does not model hardware
encryption for TinyOS 2.1, so all of my tests were performed on real hardware. I used Tmote Sky
motes, with 16Kb of RAM, 48Kb of ROM and an 8MHz MSP430 microcontroller running TinyOS 2.1.2
\cite{tinyos}.

I exercised my system using several small test programs. These programs consisted of a
client/server pair where the client repeatedly sent a message containing a 16~bit value to the
server. The purpose of these tests was to explore the overhead induced by my system with minimal
obscuring effects from application logic. The percentage overhead observed with the small
programs is thus a worst case overhead.

I also created a demonstration program that implemented the directed diffusion algorithm
\cite{intanagonwiwat-2003} over several nodes. This program allowed me to test the behavior of
the system in a long-running setting, and to exercise my system in a multi-mote, multi-hop
network environment. Although the demonstration program does not perform any significant
function, it does show that useful higher level services can be built on top of SpartanRPC.

\subsection{Memory Overhead}
\label{section-sprocket-memory-performance}

The \Sprocket\ run time system uses several memory caches to hold key material, credential
information, and the minimum model implied by the set of known credentials. These caches are
statically allocated but must be stored in RAM since their contents are dynamic.
Table~\ref{table-ram-consumed} summarizes the RAM consumption of the various storage areas used
by the current implementation.

\begin{table}[!t]
  \newcommand\T{\rule{0pt}{2.1ex}}
  \centering
  \caption{RAM consumed by various storage areas}
  {
  \begin{tabular}{|l|r|r|r|} \hline
    \textit{Storage Area} \T & \textit{\# Items} & \textit{Bytes/Item} & \textit{Total Bytes} \\
    \hline \hline

    Session Keys ($n_k$) \T & 10 & 22 & 220 \\ \hline 
    Public Keys ($n_p$)  \T & 12 & 40 & 480 \\ \hline
    Credentials ($n_c$ ) \T & 12 & 16 & 192 \\ \hline
    Model ($n_m$)        \T & 16 &  6 &  96 \\ \hline \hline
    \textbf{Total} \T & \multicolumn{3}{r|}{ \textbf{988} } \\ \hline
  \end{tabular}
  }
  \label{table-ram-consumed}
\end{table}

The number of items in each cache are tunable parameters. The optimum settings depend on the
intended application. The values in Table~\ref{table-ram-consumed} attempt to strike a balance
between usability and flexibility on one hand and excessive memory consumption on the other. In
applications where these needs are more clearly known a priori, the sizes of the caches can be
adjusted to potentially result in lower memory consumption.

% The paragraphs below attempt to give a pseudo-mathematical justification for the values in the
% table above. I've never been completely happy with it.

The justification for my choice of the number of items in each storage area is as follows.
Assume a node $N_i$ offers $n_{si}$ services and has $m_i$ neighbors. In the worst case a
session key is needed for each service on all of $N_i$'s neighbors and for every neighbor
connecting to $N_i$'s services. The number of session keys $n_k$ is given by
\begin{displaymath}
n_k = \left(\sum_{j = 1}^{m_i} n_{sj}\right) + m_i n_{si}
\end{displaymath}
where $n_{sj}$ represents the number of services on neighbor $j$. For example if $N_i$ had five
neighbors each offering one service and if $N_i$ offered one service, the total number of
session keys required would be 10. In my implementation I use this value presuming a small
number of neighbors with a small number of services on each node.

The number of public keys is related to the complexity of the access policies used by the
services. The intersection credential mentions three public keys so in the worst case the number
of public keys $n_p = 3 n_c$ where $n_c$ is the number of credentials in the credential storage.
However, the intersection credential is rare and all other credentials only mention two public
keys. This suggests an upper bound closer to $n_p = 2 n_c$.

In real policies, however, it is necessary for the same public key to be mentioned in more than
one credential. For example consider a simple credential chain such as $E_1.r_1 \leftarrow
E_2.r_2, \ldots, E_i.r_i \leftarrow E_{i+1}.r_{i+1}, \ldots, E_n.r_n \leftarrow E_{n+1}$. In this
case the number of credentials is $n$ and the number of unique public keys is $n+1$. I feel it
is reasonable to suppose that in realistic policies the number of credentials and the number of
public keys are about the same. For this reason my implementation sets $n_p = n_c$.

Harder to judge is the number of credentials involved in real-world authorization scenarios.
Clearly this will be application specific and will vary widely. However, two or three
credentials needed to establish authorization is a reasonable assumption, since most likely
application designers will avoid complicated policies in a resource constrained setting. Thus
$n_c \approx 3 n_d$ where $n_d$ is the number of interacting domains, assuming each domain
provides a single protected service. I assume only two other domains will be in the immediate
vicinity of a node $N_i$ thus causing $n_d = 3$. I then set $n_c = 12$ to provide some space for
the case when a neighboring domain offers more than once service.

If every entity defines the same roles and if the policies are such that every entity is in
every role, then the number of model tuples required is $n_m = n_r n_p^2$ where $n_r$ is the
total number of roles involved. This value is unrealistically large, however. In a system where
access is widely granted (cooperating domains) a value $n_m = n_r n_d$ would be more
appropriate. My implementation assumes that $n_d$ is about three and that $n_r$ is about four or
five. Thus I used $n_m = 16$.

Table~\ref{table-test-program-ram} shows the overall memory consumption of two small
client/server pairs. The baseline pair handle all communication through normal Active Message
packets that are explicitly programmed by the user. The SpartanRPC pair uses \Sprocket\ which
includes support for certificate distribution and verification, session key management,
authorization logic, and MAC computations. The Directed Diffusion entry shows the memory
consumption of the demonstration program that implements that algorithm.

\begin{table}[!t]
  \newcommand\T{\rule{0pt}{2.1ex}}
  \centering
  \caption{Memory consumption of test programs}
  {
  \begin{tabular}{|l|r|r|} \hline
    \textit{Test Program} \T & \textit{RAM Bytes} & \textit{ROM Bytes} \\
    \hline \hline

    Baseline Client    \T &  349 & 10982 \\ \hline 
    Baseline Server    \T &  283 & 10490 \\ \hline
    SpartanRPC Client  \T & 2222 & 23108 \\ \hline
    SpartanRPC Server  \T & 2126 & 23394 \\ \hline
    Directed Diffusion \T & 3105 & 27826 \\ \hline
  \end{tabular}
  }
  \label{table-test-program-ram}
\end{table}

Although the overhead incurred by the \Sprocket\ runtime system is significant on my test
platform, nearly 80\% of RAM and 50\% of ROM resources are still available. Furthermore, these
memory usage numbers scale well to denser neighborhoods and extended RPC services because many
aspects of the runtime system, in particular the RAM reserved for SpartanRPC, are independent of
the number of RPC services in use.

\subsection{Transient and Steady State Processor Overhead}
\label{section-sprocket-cpu-performance}

The execution performance of my system displays two distinct behaviors. The first is a transient
behavior that occurs after a node boots when certificates are exchanged and session keys are
negotiated between the new node and its neighbors. The second is a steady-state behavior that
occurs during normal operation. The transient overhead of my system is large but the steady
state overhead is not. In a quasi-static environment where new nodes enter the network
infrequently the transient costs are amortized and it is the small, steady state overhead that
dominates.

To explore the steady state overhead three tests were conducted.
\begin{enumerate}
\item A baseline test where the message handling was done explicitly using traditional Active
  Message interfaces.
\item A duties test where the \Sprocket\ system was used but no authorization was requested.
  This is equivalent to using the authorization components \texttt{ACNullC} and \texttt{ASNullC}
  in \autoref{figure-client-server-authorization}.
\item A MAC test where authorization was requested but where the session key storage areas were
  preloaded with appropriate session keys.
\end{enumerate}

Table~\ref{table-steady-state} shows the maximum rate at which messages could be sent and
received by the test programs mentioned above. Note that the MAC test made use of the hardware
assisted AES support provided by the CC2420 radio chip. These results show that maximum message
send rates decrease by a factor of 7\% due to the addition of the duties program logic, and
further decreases by a factor of 25\% due to MAC calculations. I note that the latter overhead
would be incurred in any system using CC2420 MAC calculations.

\begin{table}[!t]
  \newcommand\T{\rule{0pt}{2.1ex}}
  \centering
  \caption{Maximum message transfer rate}
  {
  \begin{tabular}{|l|r|r|} \hline
    \textit{Test} \T & \textit{messages/s} & \textit{\% Reduction} \\
    \hline \hline

    Baseline \T & 128 &   -- \\ \hline 
    Duties   \T & 119 &  7.0 \\ \hline
    MAC      \T &  87 & 32.0 \\ \hline
  \end{tabular}
  }
  \label{table-steady-state}
\end{table}

The transient runtime overhead of my system can be subdivided into three primitive operations:
the time required to transmit and verify a certificate, the time required to build the minimum
model, and the time required to negotiate a session key. Two of these operations require lengthy
public key computations and dominate the transient behavior. Thus the performance in this regard
is closely tied to the performance provided by TinyECC.

TinyECC provides a number of tunable parameters that can be used to optimize performance by
trading off space and time \cite{Liu-Peng-TinyECC-2008}. In my tests, since I had no particular
application constraints in mind, I used the TinyECC ``out of the box.'' However, TinyECC's
optimizations can be used to tune the performance of the system to better match a particular
application. For example, activating the Shamir Trick cut certificate verification time in half
at the expense of increasing RAM usage by nearly 700 bytes.

Table~\ref{table-transient-time} shows the times required for each of the primitive transient
operations in my implementation.

\begin{table}[tbhp]
  \newcommand\T{\rule{0pt}{2.1ex}}
  \centering
  \caption{Processing time for transient operations}
  {
  \begin{tabular}{|l|r|} \hline
    \textit{Operation} \T & \textit{Time} \\ \hline \hline

    Certificate Verification     \T &  82s \\ \hline 
    Minimum Model Construction   \T & 370$\mu$s \\ \hline
    Session Key Negotiation      \T &  80s\\ \hline
  \end{tabular}
  }
  \label{table-transient-time}
\end{table}

The time required to build the minimum model is directly related to the number and nature of the
credentials involved. In my test I used a collection of five representative credentials that
included at least one of each type. In any case this time is entirely negligible compared to the
other transient operations.

% There might be a termination problem with my algorithm for updating the minimum model despite
% the fact that the method I'm using is theoretically guaranteed to terminate. In particular if
% the model is too large it might be possible for the algorithm to run infinitely, alternating
% between replacing two different tuples in the limited space.

The time quoted for session key negotiation represents the time required for both negotiating
partners to compute the session key. In the current implementation the two negotiating nodes do
this sequentially with the server node computing the session key before responding to the client
node. This was done in case the session key computation failed on the server to ensure that the
client does not falsely believe a session key was successfully negotiated.

\subsection{Transient State Times for Directed Diffusion}

As I argue above, the overhead imposed by my system is primarily the time the network spends in
a initial transient state when credentials are verified and session keys are negotiated.
Subsequently, the network enters a steady state during which the main cost is a 32\% reduction
in \emph{maximal} message send rates due to hardware AES encryption. In order to evaluate the
performance of my system in a realistic application, I therefore quantified the transient state
times of the demonstration directed diffusion application. In my experiments I elected a single
node to repeatedly express an interest and observed how long was required for that interest to
flood the network. This time depends on three major factors:
\begin{enumerate}
\item The number of certificates transferred.
\item The number of neighbors for each node.
\item The number of hops to the ``far'' edge of the network.
\end{enumerate}
I conducted two experiments, one on a single hop (star) network and another on a multi-hop
(mesh) network.

In the single hop case, transient time $T$ can be described by the following equation:
\begin{displaymath}
T = n_c B + V + n_n K
\end{displaymath}
where $B$ is the certificate broadcast interval, $V$ is the certificate verification time, $K$
is the session key negotiation time, $n_c$ is the number of certificates and $n_n$ is the number
of neighbors. Since $B$ was set to 90 seconds, which is greater than $V$, certificate
verification for $n_c$ certificates takes time $n_c B + V$ given a 90 second system
initialization period. And since session keys need to be negotiated with $n_k$ neighbors in
turn, $T$ also comprises a $n_nK$ delay. Table~\ref{table-one-hop-transient} shows the transient
time required to flood a network where all nodes are one-hop neighbors of the root node. Values
are given for three different policies with different numbers of certificates transferred from
the root to the neighbors.

\begin{table}[tbhp]
  \newcommand\T{\rule{0pt}{2.1ex}}
  \centering
  \caption{Transient time in single hop directed diffusion}
  {
  \begin{tabular}{|l|r|r|r|} \hline
    \textit{\# neighbors} \T & \textit{1 Cert }
                             & \textit{2 Certs}
                             & \textit{3 Certs} \\ \hline \hline

    1 \T &  4m03s & 5m27s &  6m52s \\ \hline
    2 \T &  5m16s & 6m50s &  8m24s \\ \hline
    3 \T &  6m32s & 7m57s &  9m30s \\ \hline
    4 \T &  7m50s & 9m22s & 10m51s \\ \hline
  \end{tabular}
  }
  \label{table-one-hop-transient}
\end{table}

I explored the behavior of my system in a multi-hop environment by creating a linear mesh
network. Each node (except the root) had a single downstream neighbor. All nodes were booted
simultaneously and the time required for interest information to reach each node was observed.
The policy used required only a single certificate to be transferred between nodes.
Table~\ref{table-multi-hop-transient} shows the results of several runs.

\begin{table}[tbhp]
  \newcommand\T{\rule{0pt}{2.1ex}}
  \centering
  \caption{Transient time in multi-hop directed diffusion}
  {
  \begin{tabular}{|l|r|r|r|} \hline
    \textit{Run} \T & \textit{1 hop }
                    & \textit{2 hops}
                    & \textit{3 hops} \\ \hline \hline

                   1 \T &  4m05s & 7m24s & 9m10s \\ \hline
                   2 \T &  3m12s & 5m12s & 6m30s \\ \hline
                   3 \T &  3m57s & 7m37s & 9m15s \\ \hline
                   4 \T &  4m09s & 7m15s & 8m49s \\ \hline
    \textit{Average} \T &  3m51s & 6m52s & 8m23s \\ \hline
  \end{tabular}
  }
  \label{table-multi-hop-transient}
\end{table}

The reason for variations in transient times over each run was due to a randomized element in
the protocol, specifically a randomized $\pm 10\%$ interval in certificate broadcast times to
avoid collisions. In these results it is essential to note that for hops $> 2$, extra transient
time is comprised solely of session key negotiation times (80s per session key, see
Table~\ref{table-transient-time}) that are forced by duty postings as interests propagate
through the network. Certificates are broadcast and verified in parallel throughout the network
upon system bootup, during the same time period required for the root's interest to propagate
through the first and second hops.

\subsection{Snowcloud with \Sprocket}
\label{section-snowcloud-sprocket}

To explore the real-world feasibility of using SpartanRPC and \Sprocket\ I enhanced the
unsecured versions of the Harvester and sensor node programs described in
\autoref{section-field-example} to use SpartanRPC for access control.

To specify and implement the security policies informally described previously, I considered the
sensor network and the Harvester single node ``network'' as separate security domains, each with
its own set of credentials. The sensor network is always endowed with administrator-level
credentials. If a Harvester is to be used by a system engineer, it is also endowed with
administrator-level credentials, whereas a Harvester to be used by a data end-user is only
endowed with user-level credentials. When a Harvester is introduced to the sensor network, its
resource accesses are mediated by its authorization level. Since credentials are unforgeable, a
user-level Harvester can never be used for sensor network control even if it is reprogrammed.

Sensor nodes within the network possess four credentials, as follows. In these credentials the
Snowcloud domain is abbreviated $\mathit{SC}$. Authority to collect data and control sensors in
the network are governed by the roles $\mathit{SC.Col}$ and $\mathit{SC.Con}$, respectively.
Credential (1), below, says that any node with control authority also has collection authority.
(2) says that nodes in the Snowcloud domain have control authority. (3) says that any entity in
a Snowcloud collaborator's $\mathit{Usr}$ role has collection authority. (4) says that the node
identified by $\mathit{Nid}$ is in the Snowcloud domain.
\begin{mathpar}
(1)\quad \cred{SC.Col}{SC.Con}{}

(2)\quad \cred{SC.Con}{SC.Node}{}

(3)\quad \cred{SC.Col}{SC.Collab.Usr}{}

(4)\quad \cred{SC.Node}{NId}{}
\end{mathpar}
When invoking remote services, the node will do so on behalf of the entity $\mathit{Nid}$. It
will also be imaged with the $\mathit{NId}$ private key for session key negotiation.

Any Harvester within the Snowcloud domain is then provided with the credential
$\cred{SC.Node}{HId}{}$ and the $\mathit{HId}$ private key issued by Snowcloud domain
administration. This will provide that Harvester with collection and control authority in the
domain. For Harvesters to be provided to collaborators, the Snowcloud administrators issue a
credential establishing the institution as a collaborator, while the institution itself may
define and manage policy for their $\mathit{Usr}$ role membership. For example, the University
of New Hampshire ($\mathit{UNH}$) can be established as a collaborator with credential (5),
below, issued by Snowcloud domain administration, and may specify role membership with the
credential (6) issued by UNH domain administration:
\begin{mathpar}
(5)\quad \cred{SC.Collab}{{UNH}}{}

(6)\quad \cred{{UNH}.Usr}{UsrID}{}
\end{mathpar}
These two credentials, along with the $\mathit{UsrID}$ private key, are imaged on Harvesters
used by UNH collaborators for data collection, but which remain unauthorized for control.
Significantly UNH could program their own Harvester nodes without the Snowcloud domain being
involved aside from providing credential (5) above. The policy set by UNH to decide who,
exactly, is in the $\mathit{UNH.Usr}$ role is of no concern to the Snowcloud domain
administrators.

\subsubsection{Implementation}

Resources themselves are accessed through a secure command dissemination protocol, that is
modeled upon the TinyOS Dissemination protocol (as described in TEP 118). In short, protected
RPC services establish network level broadcast channels requiring authorization for use.
Commands are communicated to the network over these channels, and different channels are used
for different sorts of commands.

In more detail, command broadcast services can be specified as a duty in a remotable interface:

\singlespace
\begin{lstlisting}[language=nesC]
interface SpDissemUpdate {
    duty void change( command_t new_value );
}
\end{lstlisting}
\primaryspacing

To implement e.g.~the control command channel, the following module can be defined and included
on sensor nodes in the Snowcloud domain:

\singlespace
\begin{lstlisting}[language=nesC]
module ControlDissemC {
    provides remote interface SpDissemUpdate requires "SC.Con";
    uses            interface SpDissemUpdate as NeighborUpdate;
    provides        interface ComponentManager;
}
implementation { ... }
\end{lstlisting}
\primaryspacing

In the implementation, the provided $\verb+SpDissemUpdate+$ interface accepts command
invocations from neighbors, but requires them to be authorized for the $\mathit{SC.Con}$ role.
Commands are relayed to all other neighbors (i.e.~disseminated) via the used
$\verb+NeighborUpdate+$ interface; those neighbors are identified by the provided
$\verb+ComponentManager+$.

To use this component, both sensor and Harvester nodes can configure it through the following
component instantiation and wiring, where the component's $\verb+NeighborUpdate+$ interface is
wired remotely to neighbors:

\singlespace
\begin{lstlisting}[language=nesC]
components ControlDissemC as ControlChan;
activate "*" for 
  ControlChan.NeighborUpdate -> [ControlChan].SpDissemUpdate;
\end{lstlisting}
\primaryspacing
Note that a node must be endowed with the appropriate credentials for this wiring to be useful.

This same code pattern can be used to implement a data collection request channel, protected by
the $\mathit{SC.Col}$ role instead of $\mathit{SC.Con}$. In response to an authorized control
command invocation, sensor nodes will modify their behavior appropriately, whereas in response
to authorized data collection requests sensor nodes will report their data using collection tree
protocol (TEP 123) to the Harvester.

\subsubsection{Results}

Results can be characterized according to both the user experience and to quantitative aspects.
As detailed in \autoref{section-sprocket-cpu-performance}, a one-time transient overhead is
imposed for initial credential exchange and session key negotiation when a Harvester is first
introduced to the network. However, since data collection for a network after several months of
deployment can take up to an hour, this overhead is relatively insignificant. And steady-state
overhead is small, and does not significantly affect data collection rates. Thus, authorized
user experience is not negatively impacted by the addition of security.

From a quantitative perspective, the most important measurements to consider for this
application are RAM and ROM consumption of the unsecure and secured versions of the Harvester
collection protocol. We have to consider whether layering SpartanRPC security over a realistic
application will overrun the resources available to a mote platform. Relevant measurements are
shown in \autoref{table-sprocket-snowcloud}.

\begin{table}[h]
\centering \newcommand\T{\rule{0pt}{2.1ex}}
\caption{RAM and ROM comparison for SpartanRPC Snowcloud} {
\begin{tabular}{|l|c|c|}
\hline
\emph{Program} \T       & \emph{RAM Bytes} & \emph{ROM Bytes} \\ \hline\hline
Unsecure Harvester \T   &             2274 &            24316 \\ \hline
Secure Harvester \T     &             4771 &            35834 \\ \hline
Unsecure Sensor Node \T &             2868 &            36254 \\ \hline
Secure Sensor Node \T   &             5417 &            48616 \\ \hline
\end{tabular}
}
\label{table-sprocket-snowcloud}
\end{table}

Both RAM and ROM consumption are significantly increased by the addition of SpartanRPC security
to this application. However, these numbers are within operating parameters. Also the Sprocket
implementation of SpartanRPC described in \autoref{section-implementation} has not yet been
optimized so additional improvements could likely be made.

\section{Scalaness/nesT}

The generality of Scalaness makes a full evaluation of the system difficult to interpret.
However, in keeping with my desire to demonstrate trust management in embedded systems I applied
Scalaness to the problem of supporting trust management in the Snowcloud application in a manner
similar to that described in \autoref{section-snowcloud-sprocket}. It should be noted, however,
that as a general staged programming system, Scalaness can be used for many purposes; building
authorization systems is only one application. Furthermore Scalaness could be used to support
authorization in various ways depending on the trade offs needed between node efficiency,
deployment frequent, and system functionality.

\subsection{Snowcloud with Scalaness}

To demonstrate a staged solution to providing trust management in Snowcloud, I developed a
Scalaness program called \newterm{Snowstorm} intended to be run by each security domain
participating in a deployment. Snowstorm targets a conventional machine with Internet
connectivity and arbitrary resources.

\autoref{fig:running-snowstorm} shows two instances of Snowstorm running, $S_H$ and $S_N$, one
by each of two administrative domains. $S_N$ is run by the sensor network administrators and is
only interested in generating the sensor node application. $S_H$ is run by the collaborating
domain and is only interested in generating the Harvester application. The two domains would
probably run completely independent Scalaness programs, perhaps using a common library, but as a
convenience during development I created a single program to serve the needs of both domains.

\begin{figure}[t]
  \input{Figures/SnowStorm}
  \centerline{\raise 1em\box\graph}
  \caption{Running Snowstorm}
  \label{fig:running-snowstorm}
\end{figure}

$S_N$ reads the access policy from suitable configuration files (or as entered by the user)
consisting of $RT_0$ credentials in a convenient syntax. $S_N$ and $S_H$ run continuously and
communicate via the Internet. Both programs provide an interactive user interface with features
for generating and managing keys, issuing credentials, and storing policy statements and
credentials from its peers. As directed by its user $S_H$ requests access to node collection
resources causing authorization and session key negotiation to all take place automatically.
Once session keys are available, the user can direct Snowstorm to generate the appropriate node
level program, with $S_N$ generating the sensor node program and $S_H$ generating the Harvester
program. The nesT modules that will communicate during stage two execution are specialized with
the previously computed session key values.

The development of Snowstorm was a straight forward exercise in software engineering; most of
the program is ordinary Scala. Snowstorm makes use of widely used third party Java libraries for
Internet communication and ECC cryptographic operations. Thanks to the expressive power of the
Scala language I was able to implement the core $RT_0$ authorization decision in just 90 lines.
Furthermore, although Snowstorm has only a text-mode interface it would have been a simple
matter to endow it with a fully fledged graphical interface if desired. A majority of Snowstorm
development was done without any specialized knowledge of embedded systems development, a point
of significance since embedded systems programming often requires different training and
experience from that used by general application developers.

When asked to generate their node level programs, Snowstorm specializes a few key nesT modules
with key information and then composes those modules to form  fully functioning node programs.
When deployed to the nodes, these programs behaved as did the original implementation.
Anecdotally the Scalaness type system proved its worth several times during the development of
Snowstorm. The compiler detected improper wirings as type errors, thus preventing nonsense
compositions of nesT modules. 

Snowstorm's implementation also made extensive use of external nesC libraries. In fact, the bulk
of the original, tested sensor node and Harvester programs were wrapped as external libraries in
the manner described in \autoref{section-external-libraries}. NesT modules were created
primarily to hold key material and to interact with the AES encryption hardware on the CC2420.
No significant changes were needed to the existing code base.

\subsection{Memory Usage}

To explore the efficiency of Scalaness generated programs, the memory consumption of the
generated code was measured. \autoref{table-scalaness-snowcloud} shows the results with the
memory values of the \Sprocket\ version shown in \autoref{table-sprocket-snowcloud} duplicated
in the ``Unstaged'' column as a convenience.

\begin{table}[h]
\centering \newcommand\T{\rule{0pt}{2.1ex}}
\caption{RAM and ROM comparison for Scalaness Snowcloud} {
\begin{tabular}{|r||c|c|c|c|} \hline \hline
              & Unsecured & Unstaged & Staged & Savings\\ \hline
Sensor ROM    &     36254 &    48616 &  36596 & 25\% \\
Sensor RAM    &      2868 &     5417 &   3038 & 44\% \\ \hline
Harvester ROM &     24316 &    35834 &  24436 & 32\% \\
Harvester RAM &      2274 &     4771 &   2402 & 50\% \\ \hline
\end{tabular}
}
\label{table-scalaness-snowcloud}
\end{table}

The ``Savings'' are the percent reduction from unstaged to staged secure implementation, and
these numbers show the potential for saving both RAM and ROM space is significant.
Unsurprisingly the memory consumed by the Scalaness generated code is virtually identical to
that used by the unsecured programs. The only overhead injected into the staged node programs is
that required to interact with the AES encryption hardware and, of course, to hold the
negotiated session key material.

From the perspective of user experience, the staged version of this application is more
convenient, since no initial authorization period is needed when the harvester is first
introduced to the network. The staged version also exposes the system to fewer bugs and failures
that would be obstacles to the primary goal of data collection. On the other hand the staged
version requires the presence, somewhere in the deployment cycle, of a powerful machine on which
the first stage program can be executed.

%%% Local Variables: 
%%% mode: LaTeX
%%% TeX-master: "main"
%%% End: 
