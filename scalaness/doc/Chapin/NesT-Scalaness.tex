
\chapter{NesT and Scalaness}
\label{chapter-nest-scalaness}

% Theory from the CC paper...

Programming wireless sensor networks (WSNs) and other small embedded
devices presents multiple challenges due to severe resource
constraints.  This paper describes a programming language system
supporting type safe dynamic code generation for WSNs. It features
programming abstractions for specializing sensor code, allowing
on-the-fly adaptation to current WSN deployment conditions that can achieve greater
efficiency. %  The system has been implemented as an extension to Scala \cite{PiS2},
 % through modification of the Scala compiler. 
% everyone knows Scala -SS
%Scala is a powerful
% functional and object oriented hybrid language for the Java Virtual
% Machine .

Our system targets a scenario where a relatively powerful hub device
can automatically combine dynamically specialized libraries and deploy
them to a WSN. We use a restricted form of \emph{staging}
\cite{Taha-MetaML,DBLP:conf/icess/Taha04,289140} to achieve
well founded dynamic program generation.  \emph{First stage} code is
written in an extended version of Scala, called Scalaness, which is
programmer friendly and suitable for running on powerful hubs.
Scalaness program execution yields a residual \emph{second stage} WSN
node program written in nesT, a type safe variant of the popular nesC
WSN programming language \cite{Gay-nesC-2003}.  The second stage
program is constructed from module components treated as first class
values, which may be type and value specialized during the course of
first stage computation.

A central contribution of Scalaness is static type safety. The
Scalaness implementation incorporates a type checking algorithm
ensuring that typeable Scalaness programs will always generate type
safe nesT code.  Since type generalization is allowed to be
cross-stage, our technology supports a novel form of cross-stage type
specialization.  In existing strongly typed staged sensor programming
environments the type correctness of second stage programs must be
verified after execution of first stage code
\cite{Mainland-Flask-2008}, and could in fact error which would
invalidate the deployment.  Such type errors are always caught at
first stage compilation time in our system.  Our previous work on the
staged programming calculus $\langle \text{ML} \rangle$
\cite{FramedML} provides a theoretical foundation for Scalaness/nesT
type safety.

\subsection{Scalaness/nesT Implementations}

The goal of our work is to build a practical programming system for
writing real WSN applications. Here we present language designs and
proof-of-concept implementations that are contributions in this
direction. Scala is an appropriate choice as a basis for our Scalaness
first stage language because its compiler is open source and easy to
modify and maintain, and the Scala language offers a flexible and
user friendly feature set. 
% Commented out bits about Scalas formal foundation, its not
%  verified so seems best to keep silent as its not a major selling
%   point. - SS
%
%  Scala also has a strong formal foundation. In particular,
% although full Scala has not been formally proven type safe in the
% manner of its Java subset (via Featherweight Java \cite{XXX}), its
% type checking algorithm has been formalized and shown to be decidable
% \cite{XXX}. 
Our nesT WSN language is implemented by translation into nesC,
which can in turn be compiled for TinyOS platforms.
% nesC characterized previously - SS
%  This is an appropriate
% choice due to the popularity of TinyOS in WSN settings, and the domain
% specificity of nesC.
% We also allow nesT code to reference nesC libraries in our
% implementation. 
nesT type checking has been implemented from the ground up since it
attains stronger type safety guarantees than nesC.  Scalaness can
compose and specialize nesT modules programmatically, and the compiler
integrates type checking for nesT.  nesT is by necessity a lower level
language than Scala, arguably a required gap since sensors are too small to
support VMs or automated memory management.

\subsection{Overview of Scalaness/nesT Design}

We introduce our system through an informal discussion of its novel
features, illustrated by the example in
\autoref{figure-example}. This toy example highlights how staged
programming techniques can reduce WSN energy consumption by allowing
network node messaging code to be specialized on the hub. In
particular, we focus on how radio packets can be specialized; since
each bit of transmission is known to consume energy similar to 800
instructions \cite{tag}, this can lead to big savings. The example is
written in a user friendly form that is a minor variation on the
formal abstract syntax of Scalaness and nesT that we subsequently present.

In the example, the Scalaness function \tt{node\_specialize} configures a simple
application to be run on individual sensor nodes-- to tell a one-hop
neighbor ``hello''. To support configuration of the the application
for a variety of node platforms, the function is abstract wrt the
nesT radio module \tt{radioC} which defines the physical layer interface,
although the required type signature \tt{radioT} of any particular
radio module is known. The nesT \tt{sendC} module is explicitly defined,
which implements a message send abstraction on the node, as is the
nesT \tt{nodeC} module which implements the top level node application. The
\tt{node\_specialize} code instantiates the \tt{nodeC} module with
source and destination addresses, as well as an address type \tt{adt}
which is determined dynamically based on the value \tt{nmax} which is
a neighborhood size; we assume \emph{a priori} that the address space
size is bounded by \tt{nmax}. Hence, instantiation of \tt{nodeC}
enables a type specialization to use the minimal needed
bit sizes. 
% (note that all addresses are appropriately type cast leading to
% reduced space consumption for those parameters). 
%As a first
%approximations, TinyOS pracitioners can view Scalaness programs as a
%significantly enriched form of nesC module configuration.

\subsubsection{Modules as Staging Elements.}

In Scalaness, nesT modules can be treated as data to be composed,
following traditional staged programming languages \cite{Taha-MetaML}.
The so-called ``runnable'' modules -- ones without imports or generic
parameters -- define an initial machine configuration. This supports a
TinyOS mote reprogramming model, where the entire OS is recompiled and
target nodes are reimaged and rebooted. We define an \tt{image}
operation (invoked in line \ref{l:image} of the example) which asserts
its argument to be runnable and dumps the module code for subsequent
sensor deployment.

nesT modules specify a list of imported function signatures, and a list of
exported functions implemented by the module. Module genericity is
obtained via a sequence of type and value parameter definitions. For
example, as specified in \autoref{figure-example}, any module
\tt{radioC} satisfying \tt{radioT} has an address type parameter
\tt{adt} which specializes the message type declared in the exported
\tt{radio} function.
%, whereas
%\tt{sendC} imports a \tt{radio} function and exports a \tt{send}
%function.  
All generic type parameters are assigned an upper
\emph{type bound} via the subtyping symbol $\subtype$. 
%The \tt{nodeC}
%module additionally has value parameters \tt{self} and \tt{neighbor},
%which are type cast during the call to the imported \tt{send}
%function. 
(The concrete syntax used in \autoref{figure-example} 
precedes each import/export definition with keyword
$\tt{import}$/$\tt{export}$ for a more readable presentation; this is
not part of the formal nesT syntax.)
  
The other Scalaness operations on nesT modules are instantiation and composition. In
line \ref{l:lift}, module \tt{nodeC} is instantiated with arguments
\tt{adt}, \tt{self}, and \tt{neighbor}. In lines \ref{l:scode} and
\ref{l:image}, modules are composed using the $\ltimes$ operator.
The semantics of module composition $\mu_1 \ltimes \mu_2$ is standard
\cite{Cardelli-1997}; imports of one module are connected
to exports of the other. nesT module composition is analogous to nesC
\emph{configuration wiring}.%  In \autoref{example-scalanesssemantics}
% we describe these semantics in detail.
\begin{fpfig*}[t]{An Example \tt{Scalaness}/{\bf \texttt{nesT}} Program; fonts distinguish the two grammars}{figure-example}
\lstset{numbers=left, numberstyle=\tiny, stepnumber=1, numbersep=5pt, basicstyle=\ttfamily} 
%\lstset{classoffset=1, 
%morekeywords={post, typedef,image,uint,uint8,uint4,uint16,uint64,error_t,import,export,def,val},keywordstyle=\color{black}\bfseries, classoffset=0}
\lstset{moredelim=*[is][\color{red}]{(((}{)))}} 
\lstset{escapeinside={(*@}{@*)}}

{\scriptsize\bf
$$
\tt{radioT} \defeq \texttt{<adt $\subtype$ uint> \{ export error\_t radio ({src:adt; dest:adt; data:uint8[]})\}} 
$$
\begin{lstlisting}
 (*@\tt{sendC\ \  =}@*)  <adt (*@$\subtype$@*) uint>  
           { import error_t radio( {src:adt; dest:adt; data: uint8[]} );       
             export error_t send (adt s, adt d, uint8[] data) 
               { radio({src = s, dest = d, data = data}); (*@\label{l:struct}@*) }
           }
 
 (*@\tt{nodeC\ \  =}@*) < adt (*@$\subtype$@*) uint, uint self, uint neighbor>.
          { import error_t send (adt s, adt d, uint8[] data); 
            export error_t main() 
              { call send((adt)self, (adt)neighbor, "hello"); }   
          }
 
 (*@\tt{def\ nodeSpecialize(self: uint, neighbor: uint, nmax: uint16, radioC: radioT)}@*) {
   (*@\tt{typedef\ adt <: uint16 =\ if\ (nmax <= 256)\ uint8\ else\ uint16;}@*)  (*@\label{l:typedef}@*)   
   (*@\tt{val\ scode = \jinst{sendC}{adt} \ltimes \jinst{radioC}{adt};}@*)  (*@\label{l:scode}@*)    
   (*@\tt{val\ ncode = \jinst{nodeC}{adt, self, neighbor};}@*) (*@\label{l:lift}@*)    
   (*@\tt{image(ncode  \ltimes scode);}@*) (*@\label{l:image}@*)   
 }
\end{lstlisting}}%small
\end{fpfig*}


\subsubsection{Typing.} 

The most novel feature of our type system is dynamic type
construction. We allow dynamic construction of nesT types at the
Scalaness level for module instantiation and specialization.  On line
\ref{l:typedef} in \autoref{figure-example}, the address type \tt{adt}
is dynamically constructed via a conditional expression.

% Scalaness incorporates two distinct type checking algorithms-- one for
% the first stage extended Scala language, and one for nesT
% code.  
Scalaness type checking is an extension of Scala type checking; we
introduce a new module type form to the Scala type language, and add
type checking cases for the three module operations: instantiation,
composition, and imaging. No other part of Scala type checking needs
to be modified.  NesT type checking is defined as a standalone type system, and
yields first stage module types.

% stuff about soundness cut, was already mentioned in intro and is in
% scalaness section below as well. SS
%
% Scala type checking has been formally studied and shown to be
% decidable \cite{Cremet:2006:CCS:2135978.2135980}.
% % , and in this work it
% % is also conjectured that Scala enjoys type safety.
% % no need to say - SS
% % , which is supported
% % by practical experience
% The typing rules we add to accommodate staged module
% operations are based on our previous foundational work
% \cite{FramedML}, lending strong support for the safety of Scalaness typing.
% % Not sure what this review of nesT is doing here, nesT was already
% % briefly described earlier. - SS
% %
% % The nesT type system is a
% % mostly standard procedural language type system, but with type safety
% % restrictions on features (e.g.~no pointer arithmetic) and bounded
% % subtyping polymorphism for greater precision of the analysis.  We
% % illustrate nesT and Scalaness type checking in
% % \autoref{example-nesttyping} and \autoref{example-scalanesstyping},
% % respectively.



%\subsubsection{TinyOS Compliant Programming Model.} 
%
%Like nesC, the nesT langage comprises a subset of the C language of
%expressions and statements to define sequential computations. The
%model also includes pseudo-concurrent task postings with a
%run-to-completion semantics. Indeed, all nesT code has an
%interpretation in nesC, and residual nesT code is translated to nesC
%by the Scalaness compiler for subsequent compilation and deployment.
%
%Although Scalaness is intended for real applications, in this document
%and in our implementation we have focused on a minimal language to
%isolate fundamental issues and simplify implementation details. Thus
%for example, our module language does not separate interfaces and
%module implementations as in nesC. Nevertheless it is an adequate
%``featherweight'' representation of the nesC module language and
%integrates naturally with the TinyOS programming model.
%
%Since it is not realistic (or convenient) to imagine that all TinyOS
%libraries can be recreated in nesT, it is essential to be able to use
%nesC libraries as needed. In \autoref{section-libraries} we describe a
%nesT mechanism in the implementation for importing nesC modules.

\subsubsection{Cross-Stage Migration of Types and Values.} 

A crucial feature of our programming model is \emph{process
  separation} between stages \cite{FramedML}.  Since
  first and second stage code are 
to be run on separate devices, state is not shared between these
stages. Thus, serialization may be required when modules are
instantiated. Furthermore, types and base values may be represented
differently on the first and second stages, requiring some sort of
transformation during module instantiation.  An example transformation
is discussed in \autoref{section-serialization}.

\section{The nesT language}
\label{section-nest}
 
\syntaxfig

The nesT design aims to distill a production language, nesC,
into its fundamental elements, yielding a language that is amenable to
formal analysis but also practical. However, since our focus is on type
safety, we enhance, and in some instances restrict, those fundamental
elements to obtain a type safe language, as discussed below.

\subsubsection{Notation.}  \emph{Sequences} are notated
$x_1,\ldots,x_n$, and are abbreviated $\vect{x}$; $\vect{x}_\mapidx{i}$ is the $i$-th
element, $\emptyset$ denotes the empty sequence, and $\abs{\vect{x}}$
is the size. We write $x \in \vect{x}$ to denote membership in
sequences, and $x\vect{x}$ denotes a sequence with head $x$ and tail
$\vect{x}$.  We denote append as $\vect{x}@\vect{y}$.  For relational
symbols $R \in \{ \subtype, =, : \}$, we use the abbreviation:
$\vect{x}\, R\, \vect{y} = x_1\, R\, y_1, \ldots, x_n\, R\, y_n$. So
for example, $\tbindvec{x}{\t} = x_1 : \t_1,\ldots,x_n : \t_n$.

We also use the following naming conventions for various language 
constructs.
We use metavariable $\fname$ (of set $\mathcal{F}$) for function
names, $\fdname$ (of set $\mathcal{L}$) for field names, $\VAR$ (of
set $\mathcal{V}$) for term variables, $\TVAR$ (of set $\mathcal{T}$)
for type variables, $ \blockno$ (of set $\mathcal{M}$) for memory
locations, $\neight$ (of set $\mathbb{Z}_{2^8}$) for 8-bit unsigned
integers, and $\nsixtn$ (of set $\mathbb{Z}_{2^{16}}$) for 16-bit
unsigned integers.  We use $n$ to range over both types of integers
when their type is irrelevant.
\subsection{Syntax and Features of nesT} The syntax of nesT is presented 
in \autoref{figure-syntax}. It comprises a core language of
expressions for defining computations, a language of declarations for
defining variables and functions, and a language for defining modules.

\semanticssyntaxfig

\coresemanticsfig

%\subsubsection{Expressions.}  headers elided to save space - SS
The nesT language includes standard C-like 
constructs, such as conditional branching, looping, sequencing of
expressions, and function calls, arrays, structs, numeric base
datatypes (and operations on them). % Familiar syntax $e \idx e$ and
% $e.\fdname$ is used for array indexing and struct field selection,
% respectively.
%For the purpose of this presentation,
%array-out-of-bound access can happen.  
A ``null'' value $\undefv$ is also provided.  Function definitions and
calls are available only in a limited nullary form to simplify our
model, in particular the expression $\blok{e}$ is a function body.
Since functions are not recursive there is no expressiveness lost, a
global variable can be used to encode a function
parameter.  Memory locations $\blockno$ are program values (\emph{\'a la}
pointers).  As in all C dialects, assignment can only be performed on
so-called \emph{l-values} $\lvalue$, a restricted subset of expression syntax.

As in nesC, no dynamic memory allocation is possible; all memory
layout is established by static variable declarations.  Also, the core
language we present here does not allow pointer arithmetic to support
type safety. Type casting and array access have run time checks
imposed, as we explain in \autoref{section-nestsemantics}.  

Also as in nesC, we provide a $\kwpost$ operation for posting tasks.
Our semantics will account for tasks using the ``run-to-completion''
model of TinyOS.  Interrupts are omitted from our language to avoid
concurrency issues in the semantics.  Typical WSN
applications do not need interrupts; they are only needed in a few
low level nesC libraries.

%\subsubsection{Declarations.} 
Programs in nesT may refer to declarations of values 
and functions which are scoped at the module level and establish the
statically fixed memory layout of nesT.
% More explanation than is needed - SS
%  We provide a convenient form
% for explicit initialization of array and struct values, though neither
% arrays nor struct values may be treated anonymously (i.e.~they must be
% declared and referred to by name).
Besides convenience, declarations
are useful to support serialization of program objects passed in from
the Scala level in Scalaness, as we discuss in
\autoref{section-serialization}.

%\subsubsection{Modules.} 
nesT Modules are written $\margs{\tpdecl; \vpdecl}\lc \imports; \decls
\exports \rc $ with $\tpdecl$ and $\vpdecl$ being generic type and
term parameters, $\vect{d}$ being module scope identifier
declarations, including function definitions, and $\imports$ and
$\exports$ being imports and exports.  Exports are explicitly defind
in the module. A ``runnable'' module -- one without imports or generic
parameters -- defines an initial machine configuration. In
\autoref{figure-example}, the wiring $\tt{ncode} \ltimes \tt{scode}$
on line \ref{l:image} yields a runnable module.

\tasksemanticsfig

%\declsemanticsfig

\subsection{Semantics of nesT} 
\label{section-nestsemantics}

We define an operational semantics for nesT as a small step relation
$\compute$ on dynamic configurations in
\autoref{figure-semanticssyntax}.  We decompose the semantics into
several distinct $\compute$ relations; each computation
``sub-relation'' can be distinguished by the arity of the relevant
configurations.
The semantics for the non-module portions follows standard C-like
language formalizations \cite{Leroy-compcert-06,grossman03}.

\subsubsection{Semantics of Expressions.}
% At the heart of nesT is a C-like 
% language of expressions built from l- and r-values.  An
% l-value is an object in memory that can be assigned to, in particular
% a variable, a struct or array field, or a pointer. An r-value is a
% value resulting from expression computation and may be a value
% that is not in memory. 

The nesT syntax for necessary dynamic entities is given in
\autoref{figure-semanticssyntax}.  In nesT, we represent computed
values as pairs $\cval{\pi}{\mv}$, where $\mv$ is a base, pointer,
array, or struct value, and $\pi$ is a tag indicating whether or not
the value is in memory.  Computed r-values not in memory are denoted
$\cval{\circ}{\mv}$, e.g. $\cval{\circ}{2}$ is the result of computing
1 + 1. The l-value object $\cval{\bn}{\mv}$ indicates that the value
$\mv$ is in memory at location $\bn$.

We model memories as sequences of definitions $\bn : \t = \mv$;
observe that each memory location $\bn$ is typed at $\t$ and assigned
a value $\mv$. We interpret memories as mappings, writing
$\blockmem(\bn) = \mv$ when there exists some $\t$ such that $\bn : \t
= \mv$ is the leftmost definition of $\bn$ in $\blockmem$, and writing
$\blockmem[\bn \mapsto \mv]$ to denote $(\bn : \t = \mv)$ where $\bn :
\t = \mv'$ is the leftmost definition of $\bn$ in $\blockmem$.%  (this
% latter notation is used when a memory location is updated with a new
% value).

Evaluation rules for selected expressions are given in
\autoref{figure-coresemantics}. Here, computation is on pairs of
memories and expressions. 
%We assume existence of a function
%$\fundef{opsem}$ that interprets operations $\mathit{op}$. We also
We assume existence of a function $\fundef{docast}$ which performs a
casting conversion. We allow this function to be defined by users.
% space saver - SS
% , and in
% certain cases may be a no-op (e.g.~casting pointers to arrays when the
% latter are contiguous in memory). But in any case, i
If $\docast{\t}{\mv}{\blockmem}$ is defined, we require that the cast
conversion is type safe, in that the result must be of type $\t$.
This is discussed more in \autoref{section-nesttyping}. 
% space saver - SS
%  Note that a
% pointer is modeled by an object of the form $\cval{\pi}{\bn}$. The
% operation $\kwstar\cval{\pi}{\bn}$ looks up the value at address $\bn$
% in memory.  The operation $\&\cval{\bn}{\mv}$ returns the address
% $\bn$ of the object in memory as an r-value. Functions are defined in
% an assumed-given codebase $\flash$ with a lookup semantics defined
% similarly to that for memories $\bm$.

%\bootloadsemanticsfig

\subsubsection{Semantics of Tasks.}  NesC uses a simple scheduling model 
of serial, run-to-completion execution of queued \emph{tasks}. We thus
supplement the base semantics of nesT with a corresponding \emph{task
  collection} $\tasks$ of the tasks yet to run, and define a
single step transition relation on configurations extended with task
collections.  We leave undetermined the definition of task
collections, and also how tasks are added and retrieved -- this
because it is unspecified whether tasks are treated in e.g.~a FIFO
manner by the scheduler.  We let $\addt(\tasks,\fname\blok{})$ denote
$\tasks'$ which is $\tasks$ plus the task consisting of the function
call $\fname\blok{}$, and let $\nextt(\tasks)$ denote a pair
$\tasks',\fname\blok{}$ which comprises the ``next'' task
$\fname\blok{}$ in $\tasks$, and $\tasks'$ which is $\tasks$ with
$\fname\blok{}$ removed.  We define the task semantics, integrated
with the expression semantics defined previously, in
\autoref{figure-tasksemantics}. As for expressions, we assume the
existence of a given codebase $\flash$. When it is necessary to be
explicit about which codebase is given for a computation, we will
write $\flash \vdash \tasks, \bm, e \compute \tasks', \bm', e'$.

\subsubsection{Semantics of Declarations.} The operational behavior of
declarations is fairly straightforward.
%  Since sunction definitions may
% be declared, we must now add \emph{codebases} $\flash$ to the
% execution model. 
Functions and first class mutable variables may be declared and
initialized.  At run time mutable variables are bound (via 
substitution) to an l-value $\cval{\bn}{\mv}$, where $\bn$ is the
address of the variable. Thus, for base and function type declarations 
we have the following rules, respectively:
$$
\small{
\inferrule[FDecl]
{}
{\flash, \tasks, \bm, (\fname : \t = \blok{e}) \vect{\decl} \compute 
 (\fname : \t = \blok{e}) \flash, \tasks, \bm, \vect{\decl}}
}
$$
$$
\small{
\inferrule[BaseInit]
{\kappa = \cval{\bn}{\mv} \\ \bn \not\in \dom(\bm)}
{\flash, \tasks, \bm,(\xlet{\t}{x}{\cval{\pi}{\mv}}) \vect{\decl} \compute 
 \flash, \tasks, (\bn : \t = \mv)\bm, \vect{\decl}[\kappa/x]}
}
$$ A contextual evaluation rule for declarations allows variables to
be initialized with arbitrary expressions. This is omitted for brevity
but is similar to the expression \TirName{Context} rule, using a 
a notion of declaration evaluation contexts denoted $D$.
%Two different initialization rules are given,
%since since arrays and structs use special initialization syntax and
%must be treated as special cases in the \TirName{StructInit} rule.
%
%There are some other subtleties involved with initialization of struct
%and array variables. First, since array and struct fields are
%l-values, we constrain the syntax of array and struct values in memory
%as follows.
%\begin{definition}
%Given any $\arr{\vect{\kappa}}$ or $\{\vect{\fdname} = \vect{k}\}$ and 
%$\kappa\in\vect{\kappa}$, we have $\kappa = \cval{\bn}{\mv}$ for some $\bn$ and
%$\mv$.
%\end{definition}
%Given this requirement, we see that no base values can be inlined in
%array and struct initializations, rather only bound variables can be
%used. This is awkward in practice but simplifies our formal
%presentation, and is easily relaxed in our implementation.

\subsubsection{Semantics of Boot and Run Time.}

In the nesT machine model, a top level program execution is obtained
by loading and running a fully instantiated module. The codebase,
memory layout, and initial machine 
configuration is generated at load time by evaluating the declarations
in the module. The top level program is then started at the
\texttt{main} entry point.

To differentiate load/boot and run segments of a computation we define
$\bootseq{}$ and $\runseq{}$ 
constructors to inject declarations and
expressions into a uniform datatype. Top level computation is 
then defined as a single step reduction relation $\compute$ on
configurations  $\flash,\tasks,\bm,X$, where $X$ is of the form
$\bootseq{\vect{d}}$ or $\runseq{e}$ depending on whether the
machine is booting or running.
\begin{definition}
A module of the form $\margs{\varnothing; \varnothing }\lc ;
\vect{\decl}; \exports\rc$  is \emph{runnable}, and we define:
$$\bootload(\margs{\varnothing; \varnothing }\lc ; \vect{\decl}; \exports\rc) =
\exports,\varnothing,\varnothing, \bootseq{\vect{\decl}}$$
\end{definition}

Now, for all computation relations we define $\compute^*$ to be the
reflexive, transitive closure of $\compute$. The concern for our type
analysis is to rule out modules which, when bootloaded, will evaluate
to semantically ill-formed configurations. In the context of nesT this
is defined as follows. It is important to stipulate that failing casts
and out-of-bound array access are not stuck cases, since run time
checks in place with enable graceful failurex.
\begin{definition}
A configuration $\bm, e$ \emph{fails a run time check} iff $e$ is of
the form $\castto{\t}{\kappa}$ and $\docast{\t}{\kappa}{\blockmem}$ is
undefined, or $e$ is of the form $\cval{\bn}{\mv}\idx{\cval{\pi}{n}}$ and 
$n\ge\abs{\vect{\mv}}$.
\end{definition}

\begin{definition}
\label{def-runnable}
A configuration $\flash, \tasks, \bm, \ell$ is \emph{stuck} iff it is
irreducible and $\ell$ is neither of the form $\runseq{\context{E}{e}}$ 
nor $\bootseq{\context{D}{e}}$ 
where $\bm, e$ fails a run time check. A runnable
module $\mu$ \emph{goes wrong} iff
$\bootload(\mu) \compute^\star \flash, \tasks, \bm, \ell $ 
where $\flash, \tasks, \bm, \ell$ is stuck.
\end{definition}

\subjudgefig

\coretypingfig

\declmodtypingfig

\subsection{nesT Typing} 
\label{section-nesttyping}

The typing rules for nesT combine a standard procedural language
typing approach with subtyping techniques adapted from previous
foundational work \cite{FramedML,Ghelli199875}. Our goal here is to specify
the typing algorithm used in our Scalaness implementation.

\subsubsection{Subtyping.} At the heart of our system is a decidable
subtyping judgement $\subjudge{\tpdecl}{\t_1}{\t_2}$, where $\tpdecl$
in the context of typing is called a \emph{coercion} and defines a
system of upper bounds for type variables. 
% space saver - SS
% We do not allow recursive
% type bounds definitions. 
%
% more space savers - SS
% We implement subtyping with an algorithm
% based on a classic technique \cite{Ghelli199875}
% , with straightforward
% extensions to accommodate structs and arrays as defined in
% \autoref{figure-subjudge}
A subtyping relation typically called
\emph{promotion} is also central to our approach; given a set of
subtyping coercions $\tpdecl$ and a type variable $t$, promotion will
return the least upper bound of $t$ which is also a structured type,
i.e.~not a type variable.
\begin{definition}
The relation $\ll$ \emph{promotes} a type variable:
\begin{mathpar}
\figsize
\inferrule
{\tpdecl \vdash \tpdecl(t) \ll \tau}
{\tpdecl \vdash t \ll \tau}

\inferrule
{\neg\exists t . \tau = t}
{\tpdecl \vdash \tau \ll \tau}
\end{mathpar}
\end{definition} 
It is important to observe how promotion and subtyping are used
differently in our system.  Since any sort of l-value can be written
to via assignment, subtyping invariance \emph{must} be imposed on
l-values occuring in write positions to maintain type soundness.
Therefore type subsumption is allowed only at program points where
read-only control flow occurs -- for example when an r-value is
directly assigned to an l-value. 

\subsubsection{Type Environments and Checking} Our typing algorithm 
for source code expressions is based on judgements $\tenv,\tpdecl
\vdash e : \t$, where $\tenv$ is an environment of free term variable
typings, syntactically defined equivalent to value parameters
$\vpdecl$ and imports $\imports$. We also endow type environments with
the same lookup semantics as memories and codebases.  Representative
typing rules for selected expressions are given in
\autoref{figure-coretyping}.  The derivation of any judgement
$\tenv,\tpdecl \vdash e : \t$ can be interpreted as an algorithm where
both $\tenv$ and $\tpdecl$ are given as arguments and $\t$ is returned
as a result.

Note that type casting is only statically allowed if the types
involved are \emph{compatible} as specified in rule \TirName{TypeT}.
This relation, formalized as $\tpdecl \vdash \compatible{\t_1}{\t_2}$,
is left abstract and user defined, but typical examples include
casting structs to arrays in contiguous memory, and provable lack of
pointers in types.  Recalling that the semantics of nesT relies on a
$\fundef{docast}$ function that implements cast conversions, any
implementation must be type safe, which allows us to rule out run time
cast failures in well typed programs. Informally, $\fundef{docast}$ is
type safe iff the resulting expression has the type of the cast.  We
refer the reader to \cite{FramedML} for a thorough formal discussion of
type safety for this style of casting.

\subsubsection{Declaration and Module Typings.} At the module level, 
we need to first type check and generate typing environments from
declarations, as specified in \autoref{figure-declmodtyping} (rules
for array and struct declarations omitted for brevity). Given this, a
module typing is obtained by type checking module exports, using a
coercion obtained from the module type parameters and a typing
environment obtained from a combination of module value parameters,
imports, and variable type declarations. Module type checking is also
specified in \autoref{figure-declmodtyping}. Our type safety conjecture 
for nesT can then be stated as follows.

\begin{conject}[nesT Type Safety]
If $\mu : \jmodtcat$ is valid and $\mu$ is runnable, then $\mu$ does not 
go wrong.
\end{conject}
Here is an example nesT typing. In this and other examples we will
take narrative liberties with function definitions, assuming they can
be non-nullary as in \autoref{figure-example}.
\begin{example}
\label{example-nesttyping}
The module $\tt{nodeC}$ of \autoref{figure-example} can be assigned the type:
$$
\begin{array}{l}
\margs{\tt{adt} \subtype \tt{uint}; \tt{self : uint}, \tt{neighbor : uint} \tt}\\
\ \ \ \ \lc
\tt{error\_t\ \ send}(\tt{s : adt}, \tt{d : adt}, \tt{uint8[] : data});\ \ 
\tt{error\_t\ \  main()} 
\rc
\end{array}
$$
\end{example}


\section{The Scalaness Language}
\label{section-scalaness}

Scalaness serves as the language for nesT module composition in the
same manner as nesC configurations serve to compose nesC modules, but
Scalaness is a more powerful metalanguage since modules are treated as
a new category of first class values in Scalaness.  Instantiation,
composition (aka wiring), and imaging of modules are defined as
operations on module values.  Because instantiation of modules with
both types and values is allowed, values and types may migrate from
the Scalaness level to the nesT level after programmatic refinement,
realizing a disciplined form of code specialization.

%\snote{Scalaness formalized vs Scalaness implemented was not clear,
%  reworded this to bring out the difference.  May need some polishing,
%  take a pass at it.}  
Our goal in this Section is to describe the
Scalaness syntax and semantics realized in our implementation, and
justify our claims of type safety. Since Scala as implemented is too
large to easily formalize, we formalize here a subset of Scala
expressed as an extension of Featherweight Java \cite{FJ}, a core
calculus subsumed by Scala. A formalized core calculus and type
analysis for Scala exists \cite{Cremet:2006:CCS:2135978.2135980}, but
FJ is adequate and simpler.  We thus elide many features of Scala in
the Scalaness formalization, but we adapt all Scala features unchanged
in our Scalaness implementation.  Here we primarily focus on the
module metaprogramming operations that we add.  The presentation here
``cleans up'' some implementation details, but is otherwise an
accurate description of the module operation semantics and especially
the module operation typing rules.

% Scalaness type checking is grounded in a previous foundational
% work, as discussed in \autoref{section-framedml}. The type checking
% algorithm presented here statically supports type safety at both
% programmatic levels, i.e.~if a Scalaness program type checks, then
% running it will not cause a type failure, \emph{and} residual nesT
% code will also be type-safe when run on its own. 

\subsection{Syntax of Scalaness}

\scalanesssyntaxfig The Scalaness language syntax is presented in
\autoref{figure-scalanesssyntax}. To represent an adequate core
calculus of Scala, it subsumes two Featherweight Java variants:
Featherweight Generic Java (FGJ) \cite{FJ} and Assignment Featherweight Java
(AFJ) \cite{AFJ}. The generic class types of FGJ are needed to model
type construction, and the mutation in AFJ is essential to consider
since one of our main concerns is nesT code specialization; nesT
programs are run in a separate process space, so specialization with
stateful values, a likely common idiom in a Scala setting, presents a
challenge.

We refer the reader to \cite{FJ,AFJ} for details on the FGJ and AFJ
object oriented calculi, which are represented in the languages of
class definitions, constructors, methods, and the first line of expression
forms defined in \autoref{figure-scalanesssyntax}. Scalaness extends
these features with a typed variable declaration form
$\jdef{x}{T}{e_1}{e_2}$ where the scope of $\ttt{x}$ is $\tt{e_2}$, a
dynamic type construction form $\jtlet{x}{T}{e_1}{e_2}$ with similar
scoping rules, and several features for module definition and
manipulation. First, we include nesT modules $\jmodval$ in the
Scalaness expression and value spaces: instantiation is
obtained via the form $\jinst{e_1}{\ttvec{e}_1; \ttvec{e}_2}$, where
$\ttvec{e}_\tt{1}$ are type parameters and $\ttvec{e}_\tt{2}$ are
value parameters. Wiring of modules is denoted
$\jwire{\tt{e_1}}{\tt{e_2}}$.  Imaging of modules, denoted
$\jimage{\tt{e}}$, ensures that $\tt{e}$ computes to a runnable
module, in the sense of \autoref{def-runnable}.

\subsection{Semantics of Scalaness}

The semantics of Scalaness is an extension of the semantics of AFJ and
FGJ to incorporate nesT modules and operations.  Computations assume a
fixed class table $\CT$ allowing access to class definitions via class
names, which always decorate an object's type. A \emph{store}
$\jstore$ is a function from memory locations $\tt{p}$ to object
representations.  Objects are represented in memory by lists of object
references $\ttvec{l}$, which refer to the locations of the objects
stored in mutable field values.  A reference $\tt{l}$ is a pair
$\jref{\tt{p}}{N}$ where $\tt{p}$ is the memory location of an object
representation and $\tt{N}$ is the nominal type of the object,
including its class name. Hence, given an object reference
$\jref{\tt{p}}{\jinst{C}{\ttvec{T}}}$, we can access and mutate its
fields $\ttvec{l} = \jstore(\tt{p})$, and access and use its methods
via the definition $\CT(\tt{C})$.

Following AFJ, the semantics of Scalaness is defined as a \emph{labeled
  transition system}, where transitions are of the form
$\tt{e} - \lc s = \jstore, s' = \jstore' \rc \rightarrow \tt{e}'
$. Intuitively, this denotes that given an initial store $\jstore$ and
expression $\tt{e}$, one step of evaluation results in a modified
store $\jstore'$ and contractum $\tt{e}'$. We write $\tt{e}
  \rightarrow \tt{e}'$ as an abbreviation when the store is not altered.
% space saver - SS
%  When discussing the
% semantics of module operations in Scalaness we will use this
% abbreviated form, since they are purely functional.

% We refer the reader to \cite{XXX} for details on the semantics of AFJ
% and FGJ subsumed by Scalaness.
The primary novelty of Scalaness over FGJ/AFJ is the formal semantics
of type and module construction. We begin with type construction,
which is provided to allow programmers to dynamically construct module
type instances.  The appropriate behavior is obtained by treating
dynamically constructed types as extensions of a basic class of
objects, and declarations of nesT level types via a \tt{typedef}
construct as syntactic sugar for ordinary object construction. We
define a \tt{LiftableType} class as the supertype of all types of
objects which can be used to instantiate a module, and dynamically
constructed types are defined as instances of a generic \tt{MetaType}
class.
\begin{definition}
Any Scalaness class table $\CT$ comprises the following definitions:
$$
\begin{array}{l}
\CT(\tt{LiftableType}) = \gclass{\tt{LiftableType}}{}{Object}{\ldots}\\
\CT(\tt{MetaType}) = \gclass{\tt{MetaType}}{\tt{X <: LiftableType}}{Object}{ \ldots }
\end{array}
$$
And we take as given the following syntactic sugar:
$$
\jtlet{x}{T}{e_1}{e_2}\ \defeq\ \jdef{x}{\jinst{MetaType}{T}}{e_1}{e_2}
$$
Class type $\tt{MetaType}$ is generalized on a single type variable. For brevity of 
notation, we define:
$$
\jinst{\tt{MetaType}}{\ttvec{T}}\quad \defeq \quad \overline{\jinst{MetaType}{T}}
$$
\end{definition}
A crucial fact of Scalaness type construction is that any dynamically
constructed type cannot be treated as a type at the Scalaness
level. This is a more restrictive mechanism than envisioned in our
foundational model \cite{FramedML,FramedMLworkshop}, however it allows us to define
Scalaness as a straightforward extension to Scala, especially in terms
of type checking.

Module instantiation is the only point where specialization of nesT
modules is allowed.
% , since non-local variables are disallowed in module
% definitions
Since Scalaness and nesT are two different language spaces, some sort
of transformation must occur when values migrate from Scalaness to
nesT via module instantiation. This \emph{lifting} transformation
involves both data mapping and serialization since the process spaces
also differ.  We aim to be flexible and allow the user to specify how
values are lifted and how types are transformed. We only require that
lifting and type transformation are coherent, in the sense that the
lifting of an object should be typeable at the object's type
transformation.  We formalize this in the
following definition.%  Note that lifting is defined on references,
% since this is how objects are referred to at run-time and we never
% want to allow lifting of modules.
\begin{definition}
\label{def-lifting}
We assume given a relation $\ser{\jstore}$ which transforms a
Scalaness reference $\tt{l}$ into nesT declarations $\vect{d}$ and 
expression $e$.  We also assume given a Scalaness-to-nesT
transformation of types $\codt{\cdot}$. To preserve type safety, we
require in all cases that $\jref{p}{N} \ser{\jstore} \vect{d}, e$
implies both of the following for some type environment $G$:
$$
\varnothing, \varnothing \vdash \vect{d} : \tenv \qquad \text{ and} \qquad
 \tenv, \varnothing \vdash e : \codt{\tt{N}} 
$$
\end{definition}
The full definition of serialization and an example are given and
discussed below in \autoref{section-serialization}.
% \snote{check that cut here is OK, just trying to save space.}
%  In brief, when a
% module $\mu$ is instantiated, serialization will bind the value
% parameters of $\mu$ to the lifted values of their instances in a
% series of declarations that are added to its own. This is specified in
% the \TirName{ModInst} rule in \autoref{figure-jmodsemantics}. Another
% important detail of the \TirName{ModInst} rule is that only type
% information in type parameters is used, and migrates into the module
% via type transformation and ordinary substititution.  

\jmodsemanticsfig

Module wiring is given a standard component composition semantics. We
only allow wiring of instantiated modules, which is consistent with
nesC and simpler to implement. In a wiring
$\jwire{\tt{e}_1}{\tt{e}_2}$, the exports of $\tt{e}_1$ are wired to
the imports of $\tt{e}_2$.  This is specified in the \TirName{ModWire}
rule in \autoref{figure-jmodsemantics}, which relies on the following
auxiliary definition of operations for combining mappings.
% space saver - SS
%  of various
% sorts (e.g.~lists of imports and exports)
\begin{definition}[Special Mapping Operations]
Let $m$ range over vectors with mapping interpretations, in
particular~$\tpdecl$, $\vpdecl$, $\imports$, and $\exports$. Binary
operator $m_1\maploosemerge m_2$ represents (non-exclusive) map merge,
i.e.~$m_1 \maploosemerge m_2 = m_1 @ m_2$ with the requirement that
$\identifier\in \dom(m_1)\cap\dom(m_2)$ implies $m_1(\identifier) =
m_2(\identifier)$.
%Binary operator $\mu_1\uplus\mu_2$
%represents mutually exclusive map merge; for this operation to be
%defined it must be that $\dom(\mu_1)\cap\dom(\mu_2) = \emptyset$.
%Given $\vect{\mu} = [\mu_1,\ldots,\mu_n]$ we write
%$\maploosemerge\vect{\mu}$ to denote $\mu_1 \maploosemerge \cdots
%\maploosemerge \mu_n$, similarly $\uplus\vect{\mu}$. 
The mapping $m / S$ is the same as $m$ except undefined on domain
elements in set $S$, and the mapping $\restrict{m}{S}$ is the same
as $m$ except undefined on elements not in ${S}$.
\end{definition}
Finally, the $\TirName{ModImage}$ rule in
\autoref{figure-jmodsemantics} shows that imaging it is an
assertion requiring its arguments to be a runnable module.

\begin{example}
\label{example-scalanesssemantics}
Assume given the definitions in \autoref{figure-example}, a module
$\tt{radioC : radioT}$ and an invocation 
$\tt{nodeSpecialize}(1,2,50,\tt{radioC})$. 
Then $\tt{ncode} \ltimes \tt{scode}$ will evaluate to the following module:
$$
\margs{}\lc\ ; \ldots; \tt{error\_t\ \ main()} \lc \tt{call}\ \ \tt{send}((\tt{uint8})1, (\tt{uint8})2, "\tt{hello}") \rc\rc
$$  
where the elided declarations include a specialized radio and message type:
$$
\tt{error\_t\ \ radio(\lc \tt{src : uint8}; \tt{dest : uint8}; \tt{data : uint8[\,]} \rc)} \lc \ldots \rc
$$
\end{example}

\subsection{Serialization and Lifting}
\label{section-serialization}

Serialization generates a flattened nesT source code version of a
Scalaness object in memory. At the top level, serialization binds the
value parameters of a module to the results of flattening, aka
lifting, via a sequence of declarations. Here is the precise 
definition.
\begin{definition}[Serialization]
\label{def-serialization}
Assume given a store $\jstore$ which is implicit in the following definitions.
We define serialization of Scalaness references as follows, along with an 
extension of the user defined lifting relation to sequences of references:
\begin{mathpar}
\inferrule%[Serialize]
{\vect{\tt{l}} \ser{\bm} \vect{\decl},\vect{e}}
{\serialize(\vect{x}, \vect{\t}, \vect{\tt{l}}) = \vect{\decl} @\ {\vect{\t}\ \vect{x} = \vect{e}}}

\inferrule
{}
{\varnothing \ser{\jstore} \varnothing, \varnothing}

\inferrule
{\tt{l} \ser{\jstore} \vect{d}, e \\ \ttvec{l} \ser{\jstore} \vect{d'}, \vect{e}}
{\tt{l}\vect{\tt{l}} \ser{\jstore} \vect{d} @ \vect{d'}, e\vect{e}}
\end{mathpar}
\end{definition}
Although lifting is user defined, a standard strategy is to introduce
a new declared variable for each memory reference in the lifted
object, and bind the variable to the lifted referent. Hence, lifting
will typically be defined recursively. In our implementation, we have
adapted a ``default'' lifting which follows this strategy, and also 
transforms objects by just transforming the fields into a representative
struct, and ignoring methods. We will illustrate this with an example in 
\autoref{section-implementation}.
% , which also shows how base values such 
% as integers are lifted
We can formally capture the essence of this 
transformation with the following definitions. It is easy to see that 
these definitions will satisfy the requirements of \autoref{def-lifting}.
\begin{example} In this example we allow lifting of any object references,
and transform the object $o$ into a structure containing the transformed 
fields of $o$. Methods are disregarded by the transformation. Here is the 
specification of the type transformation:
\begin{mathpar}
\inferrule[ChapinT] {\CT(\ttt{C}) =
  \gclass{C}{\gbounds{X}{S}}{N}{\fieldvec{R}{f};\ K\ \ttvec{M}}}
          {\codt{\jinst{C}{\ttvec{T}}} = \lc \vect{\ttt{f}} :
            \codt{\ttvec{R}[\ttvec{T}/\ttvec{X}]} \rc}
\end{mathpar}
and here is the specification of lifting.
\begin{mathpar}
\inferrule[Chapin]
{\jstore(\tt{p}) = \ttvec{l} \\ \fields{C} = \tdecls{T}{f} \\ 
 \ttvec{l} \ser{\jstore} \vect{d}, \vect{e} \\ x \ \text{ fresh}} 
{\jref{p}{\jinst{C}{\ttvec{R}}} \ser{\jstore} \vect{\decl} 
   @ (\xlet{\codt{\jinst{C}{\ttvec{R}}}}{x}{\lc \vect{\ttt{f}} = \vect{e}\rc}) , x}
\end{mathpar}
\end{example}


\subsection{Scalaness Type Checking}
\label{section-typing}

\scalanesstypingfig
% boilerplate cut to save space - SS
% The Scalaness Type judgements are of the form $\Gamma \vdash \tt{e} :
% \tt{T}$, where $\Gamma$ is a free variable typing environment and
% judgement validity is obtained via a deduction system. 
We adapt the
typing rules of FGJ in their entirety, and refer the reader to
\cite{FJ} for relevant details. Since type construction via 
\texttt{typedef} is syntactic sugar for normal object construction, 
that is covered by those rules as well. It remains to define typing 
rules for nesT modules and module operations.

%\begin{theorem}[AFGJ Type Safety]
%If $\varnothing \vdash \tt{e} : \tt{N}$ and $\tt{e} \rightarrow^*
%\tt{e}'$ with $\tt{e}'$ irreducible\footnote{This is defined as ``normal
%form'' in \cite{XXX}.}, then $\tt{e}'$ is either a
%reference $\jref{\tt{p}}{N'}$ with $\tt{N'} <: \tt{N}$, or an
%expression containing $(\tt{N}')\tt{\jref{\tt{p}}{N''}}$ where
%$\tt{N''}\not<:\tt{N}'$.
%\end{theorem}

The nesT module type form at the Scalaness level is
$\jmodt{\tpdecl}{\jmodtcat}$, where $\jmodtcat$ is a nesT module
type. The $\tpdecl$ in this form represents the type bounds of
dynamically constructed types that have been used to instantiate the
module; we refer to this part of the type as the \emph{instance
  coercion}. Because these types are dynamically constructed, their
identity is not known statically, hence the need to treat them as
upper-bounded type names in the static type analysis. It is important
to note that the type names in $\tpdecl$ will be fully resolved at
run time, so that any module generated by a Scalaness program
execution will have a fully reified nesT type, i.e.~an empty instance
coercion.

This is reflected in the \TirName{ModT} rule in
\autoref{figure-scalanesstyping}, which connects the nesT typing
system with the Scalaness type system. Since in this case we are
typing an uninstantiated module definition its instance coercion is
empty.  An instance coercion in a module type is directly populated
when a module is instantiated, as in the \TirName{ModInstT} rule.
Here, the type instances $\ttvec{e}_1$ are all dynamically
constructed, so they define the upper bounds of the instantiated
module's instance coercion. We also expect all type and value
parameters to respect the typing bounds specified in the module
definition. 
%A subtle but significant detail in this rule is the
%consequence of dynamically constructed types having no meaning ``as
%types'' at the Scalaness level. This means that no Scalaness value of
%that type can be constructed, so dynamically constructed type names do
%not occur in the typings of value parameters. This detail is of
%practical importance, and will be discussed at greater length in
%\autoref{section-futurework}.
The \TirName{ModWireT} typing rule for module wiring is a
straightforward reflection of the operational rule for module wiring,
as is the \TirName{ModImageT} rule for module runnability
imaging.
\begin{example}
\label{example-scalanesstyping}
Returning to the code in \autoref{figure-example}, we may assign the
following typing, where the relevant type environment $G$ contains 
typing for free variables within the function \tt{nodeSpecialize}:
$$
  G \vdash \jinst{radioC}{adt}\ :\  
    \jmodt{\tt{adt} \subtype \tt{int32}}{\margs{}\lc\ ;
      \tt{error\_t\ radio(\t)} \rc }
$$
$$
\text{where}\ \t = \lc \tt{src : adt}; \tt{dest : adt}; \tt{data : uint8[\,]} \rc 
$$
%and
%$$    
%  G \vdash \jinst{sendC}{adt} \ltimes \jinst{radioC}{adt}\ :\   
%    \jmodt{\tt{adt} \subtype \tt{int32}}{\margs{}\lc \ 
%      ;\,
%      \tt{error\_t\ send(\tt{adt\ s}, \tt{adt\ d}, \tt{data\ uint8[\,]})} \rc}
%$$
%$$
%\text{where}\ \t = \lc \tt{src : adt}; \tt{dest : adt}; \tt{data : uint8[\,]} \rc 
%$$
\end{example}

\subsection{Foundational Insights and Type Safety} 
\label{section-framedml}

Type checking of modules and operations is inspired by the type theory
and metatheory developed for the language \fml\ \cite{FramedML}. Scalaness
module instantiation in particular can be decomposed into a set of
\fml\ operations, and typeablity of module instantiation follows from
the typeablity of their composition. The language \fml\ is obtained by
extending system \fsub\ with state, dynamic type construction, and
staging features. The expression $\langle e \rangle$ is a code value,
and the $\mathrm{lift}$ operation takes a value at one stage
and ``lifts'' it to the next, by turning it into code and performing
any necessary serialization.

Given this, a Scalaness module with a value and type parameter can be
modeled in \fml\ as a term: 
$$\lambda x : \s_1 . \Lambda t \subtype \s_2 . \langle e \rangle$$
where $x$ and $t$ are value and type parameters for the block of code
$\langle e \rangle$.  Then, module instantiation can be modeled as the
application of this term to a type and value parameter, where the
latter must be lifted into the next stage:
$$
(\lambda x : \s_1 . \Lambda t \subtype \s_2 . \langle e \rangle)\ (\mathrm{lift}\ e)\ \t
$$ 
% This interpretation of modules and module operations for the
% purposes of typing is evidenced by the Scalaness type form
% $\jmodt{\tpdecl}{\jmodtcat}$, where $\tpdecl$ defines the type bounds
% for dynamically constructed types used to instantiate a module. This
% is directly analogous to $\exists$ type bindings in $\fml$ types,
% which statically define the upper bounds of dynamically constructed
% types. 

Observing that AFJ, FGJ, and \fml\ are all proven type safe, and that
Scalaness is in essence an orthogonal composition of these three
languages, we conjecture that type safety is maintained in this
composition. 
% Furthermore, since static type safety in \fml\ implies
% type safety of all computational stages, we expect that static type
% safety in Scalaness implies type safety of any residual nesT code in
% Scalaness computations. This conjecture can be stated formally as
% follows.

\begin{conject}[Scalaness Type Safety]
If $\varnothing \vdash \tt{e} : \tt{T}$ and $\tt{e} \rightarrow^*
\jimage{\mu}$, then $\mu$ is runnable and does not go wrong.
\end{conject}

% Below is stuff I threw together...

\section{Scalaness Design}
\label{section-scalaness-design}

\lstset{language=scalaness}
\lstMakeShortInline!

\subsection{Liftable Types}
\label{section-liftable-types-design}

Certain types that appear in a Scalaness program correspond to types in the Mininess components
manipulated by that program. These types are said to be \textit{liftable}. Values of these types
can be computed in the Scalaness program and then used to specialize Mininess components. These
values need to be transformed as they cross the boundary between the two programming languages
in response to differences in the way a liftable type and its Mininess counterpart are
represented. In this section I will describe which Scalaness types are liftable and how their
values are handled when used to specialize a Mininess component.

\subsubsection{Primitive Types}
\label{section-primitive-types-design}

All liftable types are subtypes of a special marker trait !NesCType!. The primitive types in
Mininess have liftable counterparts in Scalaness that are classes extending !NesCType!. These
classes and their corresponding Mininess types are shown in \autoref{table-primitive-liftable}.

\begin{table}
  \centering
  \caption{Primitive Liftable Types}
  \label{table-primitive-liftable}
  \begin{tabular}{|l|l|} \hline
    \textit{Scalaness Type} & \textit{Mininess Type} \\ \hline
    Uninit & void      \\ \hline
    Char   & char      \\ \hline
    UInt32 & uint32\_t \\ \hline
    UInt16 & uint16\_t \\ \hline
    UInt8  & uint8\_t  \\ \hline
    Int32  & int32\_t  \\ \hline
    Int16  & int16\_t  \\ \hline
    Int8   & int8\_t   \\ \hline
  \end{tabular}
\end{table}

The Mininess subtype relations are preserved in Scalaness where !UInt8 <: UInt16 <: UInt32! and
similarly for the signed integer types. In Scalaness these types are defined in the object
!LiftableTypes! so that they don't conflict with any normal (non-liftable) types defined by the
programmer or the language, such as !Char!. Furthermore the integer liftable types are endowed
with the usual arithmetic operations so they can be manipulated in the Scalaness program in a
natural way.

No conversions are provided between the liftable types and their ordinary Scala analogs. This
means existing libraries that, for example, manipulate Scala !Char! objects won't work with
!LiftableTypes.Char!. I do not consider this a problem for two reasons.
\begin{enumerate}
\item Since values of liftable type will eventually be written into Mininess modules, they will
  likely be put to very different uses than values of ordinary Scala type.
\item Conversions (including implicit conversions) can be easily added by the Scalaness
  programmer if desired using the normal facilities of Scala.
\end{enumerate}

To facilitate the second point, explicit conversion methods from each liftable type to its
obvious non-liftable counterpart are provided.

% TODO: The following two paragraphs should be moved to the Scalaness-Implementation section.

The Scala type system is used to ensure compile-time type safety of the primitive liftable types
in a Scalaness program. For example the type !Int16! can only be constructed using a value of
Scala's type !Short!. Consequently normal Scala type checking prevents a potentially out of
range value from being used.

Unfortunately Scala does not support unsigned types natively. In the current implementation a
sufficiently wide signed type is used to initialize objects of unsigned liftable type. This
makes it possible to use an out-of-range value during the execution of the Scalaness program
resulting in a runtime exception. I don't believe this can be avoided without adding support for
unsigned types to the Scala compiler.

\subsubsection{Arrays}
\label{section-arrays-design}

\subsubsection{Structures}
\label{section-structures-design}

A Scala class $C$ that extend the !NesCType! trait is liftable to a Mininess structure type
provided it additionally obeys the following inductive rules.

\begin{enumerate}
\item $C$ is not generic.
\item All of $C$'s fields have liftable type.
\item All of $C$'s supertypes (except !AnyRef! and !ScalaObject!) are liftable types.
\end{enumerate}

In this case $C$ is said to be a \textit{liftable class}. Except for the rules mentioned here
there are no restrictions on the definition or use of liftable classes. In particular, they are
able to have convenience methods, although the methods of a liftable class have no manifestation
in the generated Mininess code.

For example, consider the following Scalaness code:

\singlespace
\begin{lstlisting}[language=scalaness]
class Header
 (val nodeID      : nodeIDType,
  val componentID : UInt8 ) extends NesCType

class TimeStampedHeader
 (val timeStamp   : UInt16) extends Header
\end{lstlisting}
\primaryspacing

Here !nodeIDType! is a previously defined liftable class type. Consequently both of these
classes are liftable and have representations as Mininess structure types. \note{TODO: Figure
  out how to properly model Scala's inheritance system in this context.}

\section{Mininess Design}
\label{section-mininess-design}

\lstset{language=nesC}

The second stage language used by the system described here is a reduced dialect of nesC that I
call Mininess. Unlike full nesC, Mininess is restricted in a way that allows one to make a
stronger statement of the language's type and memory safety. \note{Be specific} In this respect
Mininess closely follows \newterm{nesT}, a theoretical language described in \cite{XXX}.
However, Mininess is not exactly the same as nesT in certain respects. \note{Be specific}

In this chapter I will describe the design of Mininess with a focus on how it differs from nesC
and from nesT. Roughly Mininess is a practical implementation of nesT that can be compiled with
an ordinary nesC compiler.

The description that follows assumes the reader is familiar with nesC.

% TODO: I need to describe the Mininess subset somewhat exhaustively and carefully.

\subsection{Component Specifications}
\label{section-component-specifications-design}

In Mininess components (after specialization) present interfaces that are sequences of imports
and exports. The imports are implemented as nesC commands that are ``used'' by the component and
the exports are implemented as nesC commands that are ``provided'' by the component. NesC-style
events are not part of Mininess but can be simulated using commands.

% TODO: The citation below should really be to the nesC specification.

Mininess does not provide separate interfaces as nesC does. Instead all interaction with other
components is done by way of separately declared commands. These bare commands can be wired
together in the usual way by the nesC compiler \cite{Gay-nesC-2003}. This approach follows nesT
as described in in \cite{FramedML,nesT}. \autoref{figure-example-mininess} shows a simple
example of a Mininess module that interacts with a timer. Instead of using an interface with an
event, the module provides a callback command.

\singlespace
\begin{fpfig}[tbhp]{Example Mininess Module}{figure-example-mininess}
{
\begin{lstlisting}[language=nesC]
module ExampleC {
    uses command void setPeriodic(uint32_t period);
    provides command void fired();
}
implementation {
    // Written in the Mininess subset.
}
\end{lstlisting}
}
\end{fpfig}
\primaryspacing

A Mininess code base consists of a collection of nesC modules. These modules do not by
themselves constitute a complete program. It is the job of the first stage Scalaness program to
compose Mininess modules, along with supporting components written in full nesC, into full
applications.

Mininess currently does not support nesC-like configuration components, although such support
could be added without significantly changing the system. The Scalaness program that composes
and specializes Mininess modules is responsible for generating the application's top level (and
only) configuration. That configuration defines how the modules are wired together. \note{Say
  something about what happens when Scalaness is used on only a portion of the application}

The only units of Mininess code that Scalaness programs are allowed to manipulate are entire
Mininess modules. The component oriented nature of nesC makes this level of granularity both
sufficient and appropriate. There is no need for the first stage program to manipulate, for
example, individual Mininess functions or Mininess expressions. Also module level granularity is
easier for both the implementation of Scalaness and it's theoretical development. \note{Why?}

Each Mininess module has a nesT module type implied by its specification element list. The
handling of this module type is described in more detail in Section~\ref{XXX}.

\subsection{External Libraries}
\label{section-external-libraries-design}

My experiments with Mininess show that it is expressive enough to write useful program
components. However, any realistic application will need to interact with various libraries
written in full nesC that I call \newterm{external libraries}. It is not my intention to require
the whole program be written in Mininess and generated by Scalaness. These libraries could
represent lower level code such as the TinyOS operating system or represent higher level
application code that wishes to use Scalaness generated Mininess modules.

Neither Mininess nor Scalaness provide any direct support for interfacing to external libraries.
However, a programming technique can be used whereby shim components are manually created that
wrap library interfaces.

As an example, consider the TinyOS \lstinline!Boot! interface. This interface is used to
indicate when a node is started; all useful nesC programs must interact with it. Yet Mininess
does not support interfaces at all much less some of the entities, such as events, that are
commonly declared in nesC interfaces. Instead the programmer creates a shim component such as
\lstinline!BootShimC! as shown below

\singlespace
\vspace{1.0ex}
\begin{lstlisting}[language=nesC]
module BootShimC {
    uses command void booted( );
    uses interface Boot;
}
implementation {
    event void Boot.booted( )
    {
        call booted( );
    }
}
\end{lstlisting}
\vspace{1.0ex}
\primaryspacing

The shim component is legal nesC but not legal Mininess. Its purpose is to expose all the
commands and events in an external library interface as bare commands. To this end wrapper
command and event implementations must be manually created.

Although creating shim components is a burden their form is highly stylized. Conceivably a tool
could be created that generates them automatically. A future version of the Scalaness compiler
may perform this function. However, some shim components are complex and must do additional
transformations on command arguments to interface with the non-Mininess external library
commands. In any case, shim components can be reused across Mininess applications. I expect
programmers to accumulate a library of such shims.

The shim components must be wired to the external library components they wrap. This is done by
producing two standard nesC configurations. The first, conventionally called
\lstinline!LibraryIC! encapsulates all full nesC components that have imports. The second,
conventionally called \lstinline!LibraryEC! encapsulates all full nesC components that have
exports. Normally these are the only two configurations an application needs. If the programmer
has full control over the entire application he or she can add the necessary external library
components (via their shims) to either \lstinline!LibraryIC! or \lstinline!LibraryEC! as
appropriate.

For example, \autoref{figure-libraryic} shows an example \lstinline!LibraryIC! component for a
hypothetical application that uses the external \lstinline!MainC! component and a specific
instance of the generic timer module, both from the TinyOS library. \note{Probably need to say
  something more about how generic modules are handled}

\singlespace
\begin{fpfig}[tbhp]{Example LibraryIC configuration}{figure-libraryic}
{
\begin{lstlisting}[language=nesC]
configuration LibraryIC {
    uses command void booted( );
    uses command void fired( );
}
implementation {
    components MainC, BootShimC, SpecificTimerC;

    BootShimC.booted = booted;
    BootShimC.Boot   -> MainC;

    SpecificTimerC.fired = fired;
}
\end{lstlisting}
}
\end{fpfig}
\primaryspacing

Mininess does not (currently) support nesC configurations but Scalaness does allow the component
such as the one shown in \autoref{figure-libraryic} to be declared and manipulated in Scalaness
code. Such components are represented as Scala objects that extend the
\lstinline!MininessComponent! trait as shown below

\singlespace
\vspace{1.0ex}
\begin{lstlisting}[language=scalaness]
object LibraryIC extends MininessComponent {
  external("LibraryIC.nc")
}
\end{lstlisting}
\vspace{1.0ex}
\primaryspacing

The module type of this component can't be determined by examining its nesC definition since it
is not necessarily in the Mininess subset of nesC. Instead the module type on such external
components must be annotated by the programmer as described in XXX, and is accepted without
question by the Scalaness compiler.

In a typical program there are two such objects created, one representing the
\lstinline!LibraryIC! configuration and one representing the \lstinline!LibraryEC!
configuration. These objects are then manipulated in the usual way by the Scalaness program.

Although it is not possible to use Scalaness to compose external library components directly,
the programmer is free to create several different wrapping configurations if desired and
represent each of them separately in the Scalaness program. The Scalaness program could then
dynamically select which wrapping configuration is to be used in the final generated code. In
any case the type system will ensure that illegal wirings can never be made.

\subsection{Safe Casts}
\label{section-safe-casts-design}

Since one of the goals of Mininess is to promote type safety, no implicit converstion operations
are provided. Assignment and initialization from a subtype to a supertype is permitted and
explicit casts are permitted only when configured by the programmer. This allows the programmer
to disallow casts that are unsafe or illogical while also supporting casts that are logical even
if they require non-trivial user defined code to execute.

The Scalaness compiler accepts a configuration file that defines a relation on types
\textit{isCompatible}. If \textit{isCompatible($T_1$, $T_2$)} is true then it is permitted to
cast an expression of type $T_1$ into an expression of type $T_2$. There are no restrictions on
the types $T_1$ and $T_2$.

By giving the developer explicit control over casts, conventional type conversions are permitted
without compromising the logical consistency of the program. Furthermore all such conversions
require explicit cast expressions; they are never applied implicitly.

The implementation of each allowed cast operation is made manifest by a command written by the
developer. This is described more in \autoref{section-safe-casts-implementation}.

\lstDeleteShortInline!

%%% Local Variables: 
%%% mode: LaTeX
%%% TeX-master: "main"
%%% End: 
