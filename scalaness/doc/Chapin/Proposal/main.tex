%%%%%%%%%%%%%%%%%%%%%%%%%%%%%%%%%%%%%%%%%%%%%%%%%%%%%%%%%%%%%%%%%%%%%%%%%%%%
% FILE    : Proposal.tex
% AUTHOR  : (C) Copyright 2011 by Peter C. Chapin
% SUBJECT : Dissertation proposal
%
% Send comments or bug reports to:
%
%    Peter C. Chapin <pchapin@cems.uvm.edu>
%    Department of Computer Science
%    University of Vermont
%    Burlington, VT
%%%%%%%%%%%%%%%%%%%%%%%%%%%%%%%%%%%%%%%%%%%%%%%%%%%%%%%%%%%%%%%%%%%%%%%%%%%%

%+++++++++++++++++++++++++++++++++
% Preamble and global declarations
%+++++++++++++++++++++++++++++++++
\documentclass{article}
\usepackage{listings}
\usepackage{graphicx}
\usepackage{url}
\usepackage{hyperref}
\setlength{\parindent}{0em}
\setlength{\parskip}{1.75ex plus0.5ex minus0.5ex}

%%%%%%%%%%%%%%%%%%%%%%%%%%%%%%%%%%%%%%%%%%%%%%%%%%%%%%%%%%%%%%%%%%%%%%%%%%%%
% FILE    : macros.tex
% AUTHOR  : (C) Copyright 2013 by Peter C. Chapin
% SUBJECT : Various macro definitions that can be used in my dissertation.
%%%%%%%%%%%%%%%%%%%%%%%%%%%%%%%%%%%%%%%%%%%%%%%%%%%%%%%%%%%%%%%%%%%%%%%%%%%%

\newtheorem{condition}{Condition}[section]
\newtheorem{conject}{Conjecture}[section]
\newtheorem{corollary}{Corollary}[section]
\newtheorem{definition}{Definition}[section]
\newtheorem{example}{Example}[section]
%\newtheorem{exmp}{Example}[section]
\newtheorem{lemma}{Lemma}[section]
\newtheorem{proposition}{Proposition}[section]
\newtheorem{theorem}{Theorem}[section]

\def\arr#1{\textup{\textbf{[}}\mskip -.5mu\textup{\textbf{|}}\, #1 \,\textup{\textbf{|}}\mskip -.5 mu\textup{\textbf{]}}}
\def\blockno{m}
\def\blok#1{\textbf{(}#1\textbf{)}}
\def\castto#1#2{(#1)\,#2}
\def\defassign{::=}
\def\dom{{\rm Dom}}
\def\envletter{E}
\def\figsize{\small}                      % Use in figures to set the size of the figure.
\def\fundef#1{\mathit{#1}}
\def\lc{\textup{\textbf{\{}}}             % Set brackets used in code.
\def\mapidx#1{{(\mskip -2.5mu #1\mskip -2.5mu)}}
\def\maploosemerge{\curlyveedownarrow}
\def\margs#1{\mathrm{<}#1\mathrm{>}}
\def\neight{n^{8}}
\def\nsixtn{n^{16}}
\def\ran{{\rm Ran}}
\def\rc{\textup{\textbf{\}}}}
\def\s{\varsigma}
\def\t{\tau }
\def\undefv{\ttbf{uninit}}
\def\VAR{\textit{x} }

\newcommand{\abs}[1]{|#1|}
\newcommand{\activation}[2]{#1\,\mathit{as}\,#2}
\newcommand{\addt}{\mathit{add}}
\newcommand{\bit}{\ttbf{bit}}
\newcommand{\blockmem}{M}
\newcommand{\bm}{\blockmem}
\newcommand{\bn}{\blockno}
\newcommand{\bootload}{\fundef{bootload}}
\newcommand{\bootseq}[1]{\mathbf{boot}(#1)}
\newcommand{\cast}[2]{\tt{(#1)#2}}
\newcommand{\cedge}[1]{\stackrel{#1}{\longleftarrow}}

% The \code macro should only be used when italic font is needed in inline code snippets, for
% example to deal with metavariables in code examples. Otherwise the listings package should
% be used (the short inline form should be turned on in most chapters). The listings package
% is more configurable and automatically bolds keywords.
\newcommand{\code}[1]{\texttt{#1}}

\newcommand{\codt}[1]{\llbracket #1 \rrbracket}
\newcommand{\compatible}[2]{\mathit{compatible}(#1,#2)}
\newcommand{\compute}{\leadsto}
\newcommand{\context}[2]{#1\lc#2\rc}
\newcommand{\cred}[3]{\mathit{#1} \cedge{#3} \mathit{#2}}
\newcommand{\creds}{\mathcal{C}}
\newcommand{\CT}{{CT}}
\newcommand{\cval}[2]{\lfloor #1,#2 \rfloor}
\newcommand{\datalogc}{$\text{Datalog}_\mathcal{C}$}
\newcommand{\datalog}{\text{Datalog}}
\newcommand{\decl}{d}
\newcommand{\decls}{\vect{\decl};}
\newcommand{\defeq}{\triangleq}
\newcommand{\defvec}[2]{\vect{#1} = \vect{#2}}
\newcommand{\delcred}[3]{#1 \stackrel{#3}{\longrightarrow} #2}
\newcommand{\docast}[3]{\fundef{docast}(#1,#2,#3)}
\newcommand{\exportsty}{\varepsilon}
\newcommand{\exports}{\xi}
\newcommand{\fdname}{\textsf{l}}
\newcommand{\fields}[1]{\mathit{fields}(\tt{#1})}
\newcommand{\fieldvec}[2]{\ttvec{#1}\ \ttvec{#2}}
\newcommand{\filename}[1]{\texttt{#1}}    % File names.
\newcommand{\flash}{F}
\newcommand{\fml}{\ensuremath{\langle \text{ML} \rangle}}
\newcommand{\fname}{\textsf{f}}
\newcommand{\fsub}{\ensuremath{F_\le}}
\newcommand{\gbounds}[2]{\ttvec{#1} <: {\ttvec{#2}}}
\newcommand{\gclass}[4]{\tt{class\ #1\langle #2\rangle\ extends\ #3\ \{#4 \}}}
\newcommand{\gdesc}[1]{\text{\textit{#1}}}
\newcommand{\gnew}[3]{\tt{new\ #1\langle#2\rangle(#3)}}
\newcommand{\identifier}{\mathit{id}}
\newcommand{\id}{\identifier}
\newcommand{\idx}[1]{[#1]}
\newcommand{\imports}{\iota}
\newcommand{\init}[3]{\tt{#1(#2)\{ #3 \}}}
\newcommand{\inteight}{\ttbf{uint64}}
\newcommand{\intfour}{\ttbf{uint32}}
\newcommand{\inthalf}{\ttbf{uint4}}
\newcommand{\intone}{\ttbf{uint8}}
\newcommand{\intt}{\ttbf{uint}}
\newcommand{\inttwo}{\ttbf{uint16}}
\newcommand{\jdef}[4]{\tt{def\ #1 : #2 = #3\ in\ #4}}
\newcommand{\jimage}[1]{\tt{image}\ #1}
\newcommand{\jinst}[2]{\tt{#1\ensuremath \langle #2 \ensuremath \rangle}}
\newcommand{\jmodt}[2]{#1 \circ #2}
\newcommand{\jmodtcat}{\mu\!\tau}
\newcommand{\jmodval}{\mu}
\newcommand{\jref}[2]{\tt{(}#1,\tt{#2)}}
\newcommand{\jstore}{ST}
\newcommand{\jtlet}[4]{\tt{typedef\ #1 <: #2 = #3\ in\ #4}}
\newcommand{\jwire}[2]{#1 \ltimes #2}
\newcommand{\kwelse}{\ttbf{else}}
\newcommand{\kwif}{\ttbf{if}}
\newcommand{\kwpost}{\ttbf{post}}
\newcommand{\kwreturn}{\ttbf{return}}
\newcommand{\kwstar}{\texttt{*}}
\newcommand{\kwthen}{\ttbf{then}}
\newcommand{\kwtlet}{\ttbf{typedef}}
\newcommand{\kwtypet}{\ttbf{type}}
\newcommand{\kwwhile}{\ttbf{while}}
\newcommand{\lvalue}{\ell e}
\newcommand{\mathgraph}[1]{\mathcal{G}_{#1}}
\newcommand{\meth}[4]{\tt{#1\ #2(#3)\{ #4 \}}}
\newcommand{\mutate}[2]{\tt{#1 = #2}}
\newcommand{\mv}{{\nu}}
\newcommand{\nesT}{\text{nesT}}
\newcommand{\newterm}[1]{\emph{#1}}                  % Newly introduced terms.
\newcommand{\nextt}{\mathit{next}}
\newcommand{\op}{\ \textit{op}\ }
\newcommand{\prolog}{\text{Prolog}}
\newcommand{\promote}{\ll}
\newcommand{\restrict}[2]{#1\!\mid\!_{#2}}
\newcommand{\return}[1]{\tt{return\ #1}}
\newcommand{\RT}{\textit{RT}}
\newcommand{\runseq}[1]{\mathbf{run}(#1)}
\newcommand{\select}[2]{\tt{#1.#2}}
\newcommand{\semantics}[1]{\llbracket #1 \rrbracket}
\newcommand{\send}[3]{\tt{#1.#2(#3)}}
\newcommand{\ser}[1]{\overset{\text{lift}}{\hookrightarrow}}
\newcommand{\serialize}{\mathrm{serialize}}
\newcommand{\Sprocket}{Sprocket}                     % The name of my SpartanRPC implementation.
\newcommand{\subjudge}[3]{#1 \vdash #2 \subtype #3}
\newcommand{\subtvec}[2]{\vect{#1} \subtype \vect{#2}}
\newcommand{\subtype}{\preccurlyeq}
\newcommand{\super}{\tt{super}}
\newcommand{\tasks}{P}
\newcommand{\tbindvec}[2]{\vect{#1} : \vect{#2}}
\newcommand{\tcompute}[1]{\stackrel{#1}\leadsto}
\newcommand{\tdecls}[2]{\ttvec{#1}\ \ttvec{#2}}
\newcommand{\tdefvec}[3]{\vect{#1} : \vect{#2} = \vect{#3}}
\newcommand{\tenv}{G}
\newcommand{\this}{\tt{this}}
\newcommand{\tpdecl}{\textit{T}}
\newcommand{\ttbf}[1]{\mbox{\bf \texttt{#1}}}        % New version to match lstlisting.
\newcommand{\ttt}[1]{{\tt #1}}
\newcommand{\ttvec}[1]{{\tt{\bar{#1}}}}
\newcommand{\TVAR}{\textit{t}}
\newcommand{\vect}[1]{\overline{#1}}
\newcommand{\vpdecl}{\textit{V}}
\newcommand{\xlet}[3]{#1\ #2\ =\ #3}

\renewcommand{\tt}[1]{\ensuremath{\mathtt{#1}}}

% The \note macro is useful for creating easy to see notes.
\long\def\note#1{\marginpar{NT}{\small \ \ $\langle\langle\langle$\
{#1}\
    $\rangle\rangle\rangle$\ \ }} 


% The macro below was being used for some of the program listings. However, all listings are now
% being handled by the listings package so the macro below is no longer needed. I'm keeping it
% for possible future reference in case, for some reason, it proves useful again (perhaps for
% other kind of Verbatim displays?). It is commented out to be sure it is not used accidentally.

% This macro is for text figures (program listings). Use as follows:
% \begin{figure}[htbp]
% \begin{textbox}{3in}  % The width of the text.
% \begin{Verbatim}
% ...
% \end{Verbatim}
% \end{textbox}
% \caption{...}
% \label{...}
% \end{figure}
%
%\newsavebox{\savetextbigbox}
%\newenvironment{textbox}[1]{
%  \begin{lrbox}{\savetextbigbox}
%  \begin{minipage}{#1}
%  \vspace{0.6em}
%}
%{
%  \vspace{0.3em}
%  \end{minipage}
%  \end{lrbox}\framebox[\textwidth]{\hfill\usebox{\savetextbigbox}\hfill}
%}
% Surround the \usebox{} call with \fbox{} to make the centering boxes visible.


%++++++++++++++++++++
% The document itself
%++++++++++++++++++++
\begin{document}

%%%%%%%%%%%%%%%%%%%%%%%%%%%%%%%%%%%%%%%%%%%%%%%%%%%%%%%%%%%%%%%%%%%%%%%%%%%%
% FILE    : languages.tex
% AUTHOR  : (C) Copyright 2011 by Peter C. Chapin
% SUBJECT : Various language definitions for the listings package.
%%%%%%%%%%%%%%%%%%%%%%%%%%%%%%%%%%%%%%%%%%%%%%%%%%%%%%%%%%%%%%%%%%%%%%%%%%%%

% Define Scalaness language support.
\lstdefinelanguage{scalaness}{keywords=
  {class, object, trait, module, extends, with, new, if, while, for, def, val, var, this,
    import, export},
  otherkeywords={->, =>},
  sensitive=true,
  morecomment=[l]{//},
  morecomment=[s]{/*}{*/}
  morestring=[b]"}

% Define nesC language support.
% TODO!

% Default settings for code listings.
\lstset{language=scalaness,
  basicstyle={\small},  % If you turn off \ttfamily you get bold keywords.
  stringstyle=\ttfamily,
  commentstyle=\ttfamily,
  xleftmargin=0.25in,
  showstringspaces=false}


%-----------------------
% Title page information
%-----------------------
\title{Implementation of Type Safe Staged Programming for Embedded Systems}
\subtitle{Dissertation Proposal}
\author{Peter C. Chapin\thanks{pchapin@cs.uvm.edu}}
\date{March 30, 2011}
\maketitle

\begin{abstract}
  Building programs for constrained embedded devices is challenging due to their severe
  limitations on processing speed, memory, and network bandwidth. Staged programming can help
  bridge the abstraction gap between high level algorithms and device level programs by allowing
  a first stage program to specialize device level code. This proposed work entails implementing
  a two stage programming system for wireless sensor network nodes where type errors in all
  possible second stage programs are detected during the compilation of the first stage program.
  The implementation is intended to be both practical and grounded in a solid theoretical
  foundation. The system will then be used to implement a non-trivial access control application
  for sensor networks. The performance of the generated application will be compared to that of
  a non-staged solution.
\end{abstract}

\section{Introduction}
\label{sec:introduction}
% Describe in high level terms what I want to do and provide motivation.

A wireless sensor network node is an example of a small scale, resource constrained embedded
system. In such systems efficiency is of the utmost concern. Processing power, memory space,
electrical energy, and network bandwidth are all very limited and must be carefully conserved.
Yet the applications envisioned for these systems are becoming increasingly complex. For
example, as wireless sensor networks are deployed in progressively more sensitive environments
they must provide support for elaborate security services so that their sensitive data and
control functions can be protected.

Previous work done by me, and described in more detail below, shows that doing complex security
computations on sensor nodes is only barely feasible. An alternative strategy for implementing
such systems is to use a staged programming approach. The idea is to execute a program on a
resource endowed system that performs as much computation as possible ``off line'' and then
generates an appropriately specialized second stage program that actually executes on the
resource constrained nodes.

In this environment the first stage programming language should be different than the second
stage programming language. As an example nesC \cite{Gay-nesC-2003} is an excellent embedded
systems language. However the first stage program is not executing on an embedded system.
Accordingly it will want to take advantage of the resources normally accessible to ordinary
desktop software. For example, in a security application the first stage program may need to
communicate over the Internet with other security domains to access certificates. There is no
need to restrict the first stage program to the much more limited resources that have been
developed for the highly constrained environments targeted by embedded systems languages.

Staged programming by itself is not new. It has been used for years by parser generators, macro
preprocessors, and other similar systems. What is new is the application of staging to embedded
systems development. Flask \cite{Mainland-Flask-2008} is an existing staged programming language
for sensor networks. It uses Haskell as the first stage language and is capable of generating
nesC programs. However, in Flask the type consistency of the generated programs is checked
during the \emph{execution} of the first stage. I believe this is too late. In the scenarios I
envision, the first stage program will be executed in the field, for example on a sensor network
base station. Failures to generate a valid second stage program can not be easily fixed at that
time. In addition, Flask does not allow the first stage program to dynamically compute second
stage types. Yet specializing types based on inputs obtained in the field is an essential
optimization point.

The work I propose here will extend the state of the art by providing assurance during the
\emph{compilation} of the first stage that all possible second stage programs will be type
correct. In addition, the work I propose here will allow specialization of second stage types.
Finally, I propose to build a system that is both theoretically well founded and useful for the
practical construction of real systems.

\section{Previous Work}
\label{sec:PreviousWork}
% Describe what I have done.

In previous work I implemented SpartanRPC, a remote procedure call system for nesC that provides
a form of role based access control for wireless sensor networks
\cite{chapin-skalka-SpartanRPC}. SpartanRPC hides the details of the radio communications and
the details of the security computations from the programmer. My implementation of the system,
called Sprocket, demonstrates the feasibility and usefulness of SpartanRPC.

The simple role based access control model used by Sprocket might be sufficient for certain
applications. However, in large multi-domain sensor networks\footnote{A multi-domain sensor
  network is a network consisting of several overlapping subnetworks managed by different
  security domains.} a more flexible form of distributed access control is needed. Building on
my previous work with trust management systems \cite{skalka-wang-chapin-jcs06,
  chapin-skalka-wang-acmcs08}, I have been exploring the feasibility of using such a system in
wireless sensor networks. Specifically, I have been developing an enhanced version of Sprocket,
called Sprocket$_{RT}$, that uses the $RT_0$ trust management system \cite{Li:DRBTMF, Li:DCDTM,
  Li:RRBTMF} for authorization.

The implementation of Sprocket$_{RT}$ is underway but it is well enough along to see that it
will work and, under some circumstances, could be feasible. However, the necessary computations
include public key signature verification to validate the certificates, and the evaluation of a
logic program to do the authorization. In Sprocket$_{RT}$ the sensor nodes do all of these
computations. This puts a large burden on the sensor nodes. My preliminary results with
Sprocket$_{RT}$ show that the security computations consume a significant fraction of the node's
resources. More detailed analysis of Sprocket$_{RT}$'s performance is pending and is a part of
this proposal.

Previous work by Liu, Skalka, and Smith describe a foundational framework for staged computation
in wireless sensor networks where the first and second stage programs execute in separate
process spaces \cite{FramedML}. They extended this work by defining a first stage programming
language called nesT that allows composition of components written in a simplified dialect of
nesC I call \textit{minimal nesC}, along with a type safety result guaranteeing that well typed
nesT programs will never generate a minimal nesC program that goes wrong \cite{nesT}. One
important feature of nesT is that it allows minimal nesC types to be treated as first class
values in the first stage program. This distinguishes nesT from many other staged programming
systems and has significant impact on the implementation.

\section{Proposed Work}
\label{sec:ProposedWork}
% Describe what I want to do specifically and how it builds on what I have done.

The central work that I propose is the implementation of a nesT-like language as an extension to
the Scala programming language \cite{PiS2}. To distinguish my work from nesT, which has
nesC-like syntax and semantics, I am calling the first stage language proposed here
\textit{Scalaness}. My proposed work will extend the previous work in three ways.

\begin{enumerate}
\item It will provide a practical implementation of a staged programming language for embedded
  systems that can detect second stage type errors during first stage compilation. To my
  knowledge no such system currently exists.

\item Since the first stage language will be a dialect of Scala, a very different language than
  nesC, various cross stage compatibility issues will need to be addressed. For example data
  types known to Scalaness need to be mapped into corresponding data types known to minimal
  nesC. Also the typing rules in \cite{nesT} may need to be adapted for a Scalaness context.

\item The previous work on nesT treats the second stage language abstractly. For example,
  minimal nesC modules are composed directly. In contrast actual nesC programs use
  configurations to describe the way in which components are connected. The minimal nesC
  constructs need to be translated into equivalent constructs provided by full nesC so that the
  generated programs are acceptable to the standard nesC compiler.
\end{enumerate}

In addition I also propose to use Scalaness to develop a library that allows the sensor network
developer to create SpartanRPC programs. In effect this library would be a re-implementation of
Sprocket$_{RT}$ in Scalaness. The goal would be to have the Scalaness program do the public key
and authorization computations so that the specialized minimal nesC output would contain
preconfigured session keys.

Once the demonstration Scalaness program is complete, I then plan to measure the performance of
the Scalaness-generated SpartanRPC programs and compare them against those generated by
Sprocket$_{RT}$ where all computations are done directly on the sensor nodes. This would
illustrate the effectiveness of staged computation in a sensor network context but it is
important to realize that Scalaness will be general and could be used for many other
applications.

\subsection{Scalaness Sample}

In this section I will show some hypothetical Scalaness code to illustrate a few of its novel
features. The syntax shown here is experimental and subject to change.

In Scalaness minimal nesC modules are first class values with module types as described in
\cite{nesT}. Modules are parameterized by both (bounded) types and (typed) terms.
Figure~\ref{fig:module-definition} shows an example message sending module template named sendM.
Minimal nesC code can only exist in a Scalaness program inside a module definition. However
certain Scalaness types, such as UInt in Figure~\ref{fig:module-definition} can exist in both
the pure Scala and the minimal nesC portions of a program. These types are marked in a special
way so the Scalaness compiler can serialize them properly across the two sublanguages.

\begin{figure}[htbp]
  \centering
  \begin{bigbox}
    \begin{lstlisting}
      @nesc
      class sendM
        [Adt      <: UInt,
         MessageT <: AbstractMessage[Adt]](self: Adt) {

        import error_t radio(MessageT);
        export error_t send(Adt addr, UInt8 data[]) {
          ...
        }
      }
    \end{lstlisting}
  \end{bigbox}
  \caption{Sample module definition in Scalaness}
  \label{fig:module-definition}
\end{figure}

Scalaness modules can be instantiated using natural Scala syntax with the result bound to a name
in the usual way. For example
\begin{lstlisting}
val senderModule = new sendM[UInt8, Message[UInt8]](1)
\end{lstlisting}

This creates a specific module where the address type has been specialized to UInt8. The term
parameter self, representing the node address of the sender, is specialized to $1$ in this
example. Modules can be composed using $+$ as follows
\begin{lstlisting}
val result = moteModule + senderModule
\end{lstlisting}

These compositions are type checked to ensure the consistency of imports and exports. The type
of \texttt{result} is inferred to have an appropriate module type. When the overall module is
\textit{validated} \cite{nesT} an \textit{implied configuration} is created that defines the
nesC wiring between modules. The Scalaness run time system generates this configuration when the
second stage program is output. The additional type rules in Scalaness ensure that if the
validation operation type checks, no type errors can arise in any generated program.

In addition, Scalaness programs can manipulate minimal nesC types as values. These manipulations
can be roughly encoded in conventional Scala as shown in Figure~\ref{fig:type-computation}.

\begin{figure}[htbp]
  \centering
  \begin{bigbox}
    \begin{lstlisting}
      class Typedef
        [+T <: NesCBase](t: CovariantClass[T])
         extends NesCBase

      val addrt = new Typedef[UInt](
        if (topo.length <= 256)
          classOf[UInt8] else classOf[UInt16] )
    \end{lstlisting}
  \end{bigbox}
  \caption{Type computation in Scalaness}
  \label{fig:type-computation}
\end{figure}

In this case the dynamically computed type can be used to instantiate the minimal nesC module in
a natural way.
\begin{lstlisting}
val senderModule = new sendM[addrt, Message[addrt]](1)
\end{lstlisting}

The ability to use a computed type to instantiate a module is another of Scalaness's novel
features.

Scala was chosen as a basis for the first stage language for reasons that are largely pragmatic.
As a Java-like language I anticipate that it will be relatively easy for sensor application
developers to learn and use. Yet as a functional language Scala provides the usual collection of
features that facilitate functional analysis. I speculate that this will make it easier to add
the new type rules required for this proposed work. Also Since Scala targets the Java virtual
machine, Scala programs have direct access to the extensive collection of Java libraries that
are available. These features should make it an excellent vehicle for the creation of complex,
yet practical first stage programs.

\section{Expected Results}
\label{sec:expected-results}
% Describe expected results.

The following items represent the intended artifacts of this work.

\begin{enumerate}
\item A complete version of Sprocket$_{RT}$ along with a performance assessment of the programs
  it generates to use as a baseline.
\item An implementation of the Scalaness language. This implementation will do type checking as
  discussed in \cite{nesT}, modified to account for the different first and second stage
  languages.
\item A re-implementation of Sprocket$_{RT}$ (or is semantic equivalent) as a Scalaness library.
  This library will allow the programmer to generate SpartanRPC-like programs using the $RT_0$
  trust management system where the authorization computations are done in the Scalaness library
  rather than on the sensor nodes.
\item A sample Scalaness program demonstrating the library above, along with a performance
  assessment of the programs it generates to compare with the Sprocket$_{RT}$ baseline obtained
  previously.
\end{enumerate}

Performance assessment of the node programs will borrow techniques outlined in
\cite{Malan:2008:IPI:1387663.1387668}.

In addition I anticipate that this work will lead to the publication of \cite{nesT} and/or a
paper focusing on Scalaness specifically.

\section{Related Work}
\label{sec:RelatedWork}
% Describe related work to show that I've done a reasonable literature search.

% * MetaML
% * MetaOCaml

There is a large body of literature related to staged programming. One of the seminal works on
the subject is MetaML \cite{Taha-MetaML}. A practical implementation of MetaML concepts can be
found in MetaOCaml \cite{Calcagno-MetaOCaml}. These systems support a general ability, called
\emph{cross stage persistence}, for a later stage of computation to use values created in
earlier stages. The nesT system on which my work is based, uses a restricted notion of cross
stage persistence that is more appropriate for sensor network applications. Also MetaML does not
address the issue of dynamic type construction. Finally MetaML and MetaOCaml, like many other
staged programming systems, requires that the same language be used in all stages. That is not a
feasible requirement in my case.

% * Flask
% * Template Haskell

As discussed previous, the most closely related work to this proposal is Flask
\cite{Mainland-Flask-2008}, a domain specific language embedded in Haskell that allows the
generation of second stage programs that run on sensor nodes. Like my proposed work Flask allows
the second stage program to be in nesC. It uses a quasiquoting scheme similar to Template
Haskell \cite{Sheard-TemplateHaskell} but extended to allow for non-Haskell second stage
languages \cite{Mainland-2007}.

% * Extending Scala with Database Query Capability

A Scala compiler plugin that processes a database query language embedded in Scala is described
in \cite{Garcia-Scala-Query}. Embedded queries can make use of data objects defined in the
enclosing Scala program and can bind query results to names accessible by the enclosing program.
Thus, as with my proposed work, issues of mapping types and values from one language to another
arise. In addition, similar to my proposed work, the embedded query processor statically ensures
that the queries are well typed and guaranteed to evaluate without run time error.

In order to implement support for embedded queries as a compiler plugin, it was necessary to
require the programmer to express queries as string literals or, alternatively, in a subset of
plain Scala. The query processor plugin converts the query language into Scala abstract syntax
trees so that it can be manipulated by later compiler phases. This technique is also similar to
my anticipated implementation strategy. However, there are some important differences. My
proposed work requires that certain types be treated as values, along with associated changes in
type checking. In addition my proposed work must support parametric polymorphism in the style of
System F$_\le$, a feature not explored in \cite{Garcia-Scala-Query}.

% * Lightweight Modular Staging

Scala has also been used to implement a staged programming system, called Lightweight Modular
Staging, entirely as a library \cite{Rompf-LMS}. In this system the representation of code is
hidden inside a type constructor that is parameterized by the result type of the code it
represents. In this respect the system is similar to, for example, MetaML. However, in
\cite{Rompf-LMS} the representation of code is under the control of the programmer, and
operations on code fragments can be optimized (for example to reduce code duplication). Also the
second stage language can be, but does not need to be, the same as the first stage language.

Some of the techniques described in \cite{Rompf-LMS} will be useful in my proposed work.
However, Lightweight Modular Staging by itself does not allow types to be manipulated as values.
Also it provides fine grained staged operations and would require a sensor network programmer to
write node level code in Scala syntax. My proposed work would allow the sensor network
programmer to write the bulk of the node level code in nesC and use Scala strictly for higher
level processing.

\bibliographystyle{plain}
\bibliography{../../references-Meta,../../references-WSN,../../references-AccessControl,../../references-Scala}

\end{document}
