%%%%%%%%%%%%%%%%%%%%%%%%%%%%%%%%%%%%%%%%%%%%%%%%%%%%%%%%%%%%%%%%%%%%%%%%%%%%
% FILE    : macros.tex
% AUTHOR  : (C) Copyright 2011 by Peter C. Chapin
% SUBJECT : Various macro definitions that can be used in many documents.
%%%%%%%%%%%%%%%%%%%%%%%%%%%%%%%%%%%%%%%%%%%%%%%%%%%%%%%%%%%%%%%%%%%%%%%%%%%%

% I want to typeset newly introduced terms in a special way.
\newcommand{\newterm}[1]{\emph{#1}}

% Used for bits of code in-line.
\newcommand{\code}[1]{\texttt{#1}}

\newcommand{\filename}[1]{\texttt{#1}}

% The \note macro is useful for creating easy to see notes.
\long\def\note#1{\marginpar{NT}{\small \ \ $\langle\langle\langle$\
{#1}\
    $\rangle\rangle\rangle$\ \ }} 

% I really should put this in a package file! Note how I set up some parameters before opening
% the listing. This used to be necessary when I was using verbatim environments. Is it still
% necessary with the listings package? Are those other settings just being overridden?
%
\newsavebox{\savebigbox}
\newenvironment{bigbox}{\begin{lrbox}{\savebigbox}
  \begin{minipage}{0.95\columnwidth}%
    \small\setlength{\baselineskip}{0.9\baselineskip}}
{\end{minipage}\end{lrbox}\fbox{\usebox{\savebigbox}}}

% This version should be used for full width boxes.
\newsavebox{\savewbigbox}
\newenvironment{wbigbox}{\begin{lrbox}{\savewbigbox}
  \begin{minipage}{0.9\textwidth}%
    \small\setlength{\baselineskip}{0.9\baselineskip}}
{\end{minipage}\end{lrbox}\fbox{\usebox{\savewbigbox}}}

% This horrifying hack is for making subtitles. Surely there is a better way.
% See: http://groups.google.com/group/comp.text.tex/msg/3aa4f67d8b3a979b?hl=en-EN&
\makeatletter
\def\s@btitle{\relax}
\def\subtitle#1{\gdef\s@btitle{#1}}
\def\@maketitle{%
  \newpage
  \null
  \vskip 2em%
  \begin{center}%
    \let \footnote \thanks
    {\LARGE \@title \par}%
    \if\s@btitle\relax
    \else\typeout{[subtitle]}%
    \vskip .5pc
    \begin{large}%
      \textsl{\s@btitle}%
      \par
    \end{large}%
    \fi
    \vskip 1.5em%
    {\large
      \lineskip .5em%
      \begin{tabular}[t]{c}%
        \@author
      \end{tabular}\par}%
    \vskip 1em%
    {\large \@date}%
  \end{center}%
  \par
  \vskip 1.5em}
\makeatother 
