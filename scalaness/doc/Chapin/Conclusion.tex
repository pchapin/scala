
\chapter{Conclusion}
\label{chapter-conclusion}

This dissertation has described two language-level approaches for providing, for the first time,
trust management style authorization to resource constrained embedded systems. One approach,
SpartanRPC, is based on a remote procedure call discipline with primitives for specifying
authorization requests and requirements. The other approach, Scalaness, makes use of staged
programming to off-load complicated security computations to a higher powered machine.

As a method for providing distributed trust management to resource constrained systems, both
approaches are feasible. SpartanRPC demands considerable resources on the devices, limiting the
amount of memory and processor time available for application logic. In particular, SpartanRPC
enabled applications exhibit transient start-up times measured in minutes, although maximum
steady-state message transfer rates exhibit a degradation of only about 30\%. In addition, the
SpartanRPC runtime system consumes approximately 13\,KiB of ROM and 2\,KiB of RAM. Despite these
significant overheads, realistic applications can nevertheless use the system as evidenced by
the SpartanRPC-enabled version of the Snowcloud application.

SpartanRPC is fundamentally a link-level protocol. Since the number of neighbors in a typical
sensor network remains small as the network grows, the applicability of SpartanRPC is only
weakly impacted by the total size of the network. The main issue is in the relatively long
session key negotiation time; the first time a message floods the network an extremely long time
may pass before the message reaches the network frontier since session key negotiations must
occur sequentially at each hop.

Scalaness has the potential of greatly reducing the load on the embedded devices. In a trust
management context, with a Scalaness program pre-computing session keys, the long transient
start-up time and large memory overheads of SpartanRPC are all but eliminated. The very slow
network flooding time experienced by SpartanRPC applications is also removed. However, The 30\%
reduction in maximum message transfer rate remains.

Scalaness does require a deployment scenario where a more powerful machine is available to
specialize the device programs. In some scenarios the time required to generate and deploy the
specialized node programs might be significant, negating somewhat the advantage in transient
start-up time Scalaness has relative to SpartanRPC. However, Scalaness is a far more flexible
system, admitting other kinds of deployment scenarios and application use-cases besides those
available to the more limited SpartanRPC.

Indeed, Scalaness represents a more principled approach to generating efficient embedded systems
software in general, as evidenced by the formal description of the system in
\autoref{chapter-dscalaness-dnest} and in the foundational \fml\ work \cite{FramedML}. Scalaness
provides a unique combination of staging with process separation, dynamic type construction, and
a cross-stage type safety conjecture that enable the robust and efficient generation of many
embedded systems applications. In a Scalaness context, the embedded trust management problem is
nothing more than a demonstration application.

Both SpartanRPC and Scalaness are tied to the nesC programming language, either by extending
nesC in the case of SpartanRPC or by translating a specialized language into nesC in the case of
Scalaness. However, the systems described here are not specific to sensor networks and would be
applicable in any environment where nesC is used. Furthermore, although nesC was developed for
sensor networks, it could be used as a general purpose embedded systems language.

\section{Future Work}
\label{section-future-work}

Possible future directions of this work can be divided into two broad categories: generalizing
the systems and providing additional safety guarantees.

The Sprocket implementation of SpartanRPC is already modular enough to support alternate (and
even multiple, simultaneous) authorization mechanisms. It would be interesting to experiment
with richer trust management languages such as $RT_1$ and its variations to see how expressive a
trust management language could be supported on constrained devices. Currently the $RT_0$
authorization logic uses minimal time and space so conceivably fairly complex trust management
languages could be supported without significantly increasing the overall overhead of the
system. Notice that the current version of Scalaness already supports arbitrary trust management
languages because the first stage program runs in an environment with relatively infinite
resources.

Sprocket currently supposes that neighboring nodes communicate over a radio link. However, this
assumption is only reflected in the code generated by Sprocket for the stubs and skeletons. It
would be a simple engineering matter to modify Sprocket to generate stubs and skeletons for some
other communication technology such as TCP/IP or the Controller Area Network (CAN) bus widely
used in automotive embedded systems \cite{pazul1999controller}.

SpartanRPC is, however, closely tied to nesC because of the way it defines and uses dynamic
wires. In contrast, the current implementation of Scalaness formally translates nesT to nesC as
it generates the second stage program. This final translation step could be modified to produce
a different language, such as C, with no change to the foundational semantics. This would make
the system applicable to a larger group of embedded developers.

The type safety guarantee provided by Scalaness is valuable but embedded systems have other
correctness needs as well. Many embedded systems are used in safety critical applications where
assurance of freedom from runtime errors, such as array bounds overflow, is essential. Systems
exist that can analyze Ada or C programs to prove freedom from such errors
\cite{Barnes:2000:SWC:369264.369271,Cuoq:2012:FSA:2404232.2404250} and those systems could
conceivably be applied to the output of Scalaness now. However, it would be an interesting and
challenging direction for future work to extend Scalaness so the programmer could be assured
that \emph{all possible} generated programs were free of important classes of runtime errors.

%%% Local Variables: 
%%% mode: LaTeX
%%% TeX-master: "main"
%%% End: 
