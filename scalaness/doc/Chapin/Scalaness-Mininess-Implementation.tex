\chapter{Scalaness/Mininess Implementation}
\label{chapter-scalaness-mininess-implementation}

In this chapter I cover the specific details of the Scalaness implementation. I first describe
Mininess, the practical realization of nesT along with the details of how Mininess programs are
transformed into nesC programs. I also describe how the Scala compiler was modified to provide
Scalaness type checking with a minimum of disruption to the the compiler's existing
functionality.

\section{Mininess}
\label{section-mininess-implementation}

\lstset{language=nesC}
\lstMakeShortInline!

\note{Talk about Mininess type rules and type checking? I need to reference Michael's thesis in
  here somewhere.}

Mininess is the name give to the second stage language used by the implementation. Roughly
speaking Mininess is the practical realization of nesT. In particular, Mininess uses the syntax
of nesC to the greatest extent possible in order to simplify the compiler and to minimize the
learning burden placed on existing nesC programmers. For example a Mininess module is specified
exactly as a nesC module except that it can only use (import) and provide (export) nesC
commands. In particular neither nesC events nor nesC interfaces can appear in a Mininess
component specification.

In principle a Mininess component could, after suitable specialization, be directly compiled by
the nesC compiler. However, Mininess also supports nesT's notion of subtyping and features for
safe memory access and safe casting. Thus while Mininess programs are syntactically identical,
and semantically similar to programs written in nesC, Mininess programs are more robust than
equivalent nesC programs.

Mininess is implemented as a rewritting to nesC. Because of Mininess's special relationship with
nesC this rewriting is largely trivial. However, the implementation of the special features of
Mininess are described in more detail in this section.

The description that follows assumes the reader is familiar with nesC.

% TODO: I need to describe the Mininess subset somewhat exhaustively and carefully.
\subsection{Component Specifications}
\label{section-component-specifications}

In Mininess components (after specialization) present interfaces that are sequences of imports
and exports. The imports are implemented as nesC commands that are ``used'' by the component and
the exports are implemented as nesC commands that are ``provided'' by the component. NesC-style
events are not part of Mininess but can be simulated using commands.

% TODO: The citation below should really be to the nesC specification.

Mininess does not provide separate interfaces as nesC does. Instead all interaction with other
components is done by way of separately declared commands. These bare commands can be wired
together in the usual way by the nesC compiler \cite{Gay-nesC-2003} and follows the design of
nesT. \autoref{figure-example-mininess} shows a simple example of a Mininess module that
interacts with a timer. Instead of using an interface with an event, the module provides a
callback command.

\singlespace
\begin{fpfig}[tbhp]{Example Mininess Module}{figure-example-mininess}
{
\begin{lstlisting}[language=nesC]
module ExampleC {
    uses command void setPeriodic( uint32_t period );
    provides command void fired( );
}
implementation {
    // Written in the Mininess subset of nesC.

    void f( T param )
    {
        int value = x;
    }
}
\end{lstlisting}
}
\end{fpfig}
\primaryspacing

The example in \autoref{figure-example-mininess} also shows the use of an undeclared type !T!
and value !x!. Such types and values are instead declared as parameters of the module in the
Scalaness program. The Scalaness compiler first adds these parameters to the appropriate
environments before it type checks the Mininess module.

A Mininess code base consists of a collection of unspecialized Mininess modules. These modules
do not by themselves constitute a complete program. It is the job of the first stage Scalaness
program to specialize and compose the Mininess modules, along with supporting components written
in full nesC, into full applications.

Mininess currently does not support nesC-style configuration components, although such support
could be added without significantly changing the system since the implementation of a module is
not important to the first stage code that manipulates it. The Scalaness program that composes
and specializes Mininess modules is responsible for generating the application's top level (and
only) configuration reflecting the result of composition. Libraries written in full nesC must be
wrapped in components with Mininess interfaces as described in
\autoref{section-external-libraries} in order to become part of a Mininess application.

Each Mininess module has a nesT module type implied by its specification element list. This
module type is extracted from the specification element list when the Mininess component is type
checked. In most cases, except as described below, it is compared against the module type
annotation used in the Scalaness program for the module. Any discrepancy is flagged by the
Scalaness compiler as a type error.

\subsection{External Libraries}
\label{section-external-libraries}

My experiments with Mininess show that it is expressive enough to write useful program
components. However, any realistic application will need to interact with various libraries
written in full nesC that I call \newterm{external libraries}. It is not my intention to require
the whole program be written in Mininess for such a requirement would not be practical. Instead
external libraries could represent low level code such as the TinyOS operating system or high
level application code that wishes to use Scalaness generated Mininess modules.

At the time of this writing neither Mininess nor Scalaness provide any direct support for
interfacing to external libraries, although such support might be useful future work. However, a
programming technique can be used whereby shim components are manually created that wrap library
interfaces. Here I illustrate that technique using a small example.

Consider first the TinyOS !Boot! interface. This interface is used to indicate when a node is
started; all useful nesC programs must interact with it. Yet Mininess does not support
interfaces at all much less some of the entities, such as events, that are commonly declared in
interfaces. Instead the programmer creates a shim component such as !BootShimC! as shown below

\singlespace
\vspace{1.0ex}
\begin{lstlisting}[language=nesC]
module BootShimC {
    uses command void booted( );
    uses interface Boot;
}
implementation {
    event void Boot.booted( )
    {
        call booted( );
    }
}
\end{lstlisting}
\vspace{1.0ex}
\primaryspacing

The shim component is legal nesC but not legal Mininess. Its purpose is to expose all the
commands and events in an external library interface as bare commands. To this end wrapper
command and event implementations must be manually created.

Although creating shim components is a burden their form is highly stylized. A future version of
the Scalaness compiler might generate them automatically. However, some shim components are
complex and must do additional transformations on command arguments to interface with the
non-Mininess external library commands/events. In any case, shim components can be reused across
Mininess applications. Thus it is reasonable to expect programmers to accumulate a library of
shims.

The shim components must be wired to the external library components they wrap. This is done by
producing two standard nesC configurations. The first, conventionally called !LibraryIC!
encapsulates all nesC components that have imports. The second, conventionally called
!LibraryEC! encapsulates all nesC components that have exports. Normally these are the only two
configurations an application needs. If the programmer has full control over the entire
application he or she can add the necessary external library components (via their shims) to
either !LibraryIC! and/or !LibraryEC! as appropriate.

For example, \autoref{figure-libraryicec} shows an example !LibraryIC! component and an example
!LibraryEC! component for a hypothetical application that uses the external !MainC! component
and a specific instance of the generic timer module, both from the TinyOS library. Notice that
the !SpecficTimerC! component appears in both wrappers since it both provides and uses at least
one command. \note{Probably need to say something more about how generic modules are handled}

\singlespace
\begin{fpfig}[tbhp]{Example LibraryIC/EC configurations}{figure-libraryicec}
{
\begin{lstlisting}[language=nesC]
configuration LibraryIC {
  uses command void booted( );
  uses command void fired( );
}
implementation {
  components MainC, BootShimC, SpecificTimerC;

  BootShimC.booted = booted;
  BootShimC.Boot   -> MainC;

  SpecificTimerC.fired = fired;
}


configuration LibraryEC {
  provides command void startPeriodic( uint32_t period );
}
implementation {
  components SpecificTimerC;

  startPeriodic = SpecificTimerC.startPeriodic;
}
\end{lstlisting}
}
\end{fpfig}
\primaryspacing

Mininess does not (currently) support nesC configurations but Scalaness does allow the
components such as shown in \autoref{figure-libraryicec} to be declared and manipulated in
Scalaness code. Such components are represented as Scala objects that extend the
!MininessComponent! trait and that specify the source file of the nesC configuration using an
!external! method as shown in \autoref{figure-external-components}.

\singlespace
\begin{fpfig}[tbhp]{Representation of External Components}{figure-external-components}
{
\begin{lstlisting}[language=scalaness]
@ModuleType("""{}<;>
               { booted(): Void,
                 fired() : Void; }""")
object LibraryIC extends MininessComponent {
  external("LibraryIC.nc")
}


@ModuleType("""{}<;>
               { ; startPeriodic(period: UInt32): Void }""")
object LibraryEC extends MininessComponent {
  external("LibraryEC.nc")
}
\end{lstlisting}
}
\end{fpfig}
\primaryspacing

The module type of external components can't be determined by examining their definitions since
they are not in Mininess. However, as with all Mininess modules they must be annotated with
their module type in the Scalaness program as shown in \autoref{figure-external-components}. For
external modules this annotation is accepted without question by the Scalaness compiler.

The !LibraryIC! and !LibraryEC! objects are then manipulated in the usual way by the Scalaness
program. Because !LibraryIC! has only imports and !LibraryEC! has only exports it is normal for
these components to appear at the ends of a wiring chain. \autoref{figure-full-wire} shows an
example where the final result module is runnable. In \autoref{figure-full-wire} the !+>!
symbol is the Scalaness wiring operator.

\singlespace
\begin{fpfig}[tbhp]{Wiring Mininess Components}{figure-full-wire}
{
\begin{lstlisting}[language=scalaness]
@ModuleType("""{ checksumType <: UInt32 }<;>
               { ; }""")
val resultModule =
  LibraryIC +> formattingModule +> checkingModule +> LibraryEC
\end{lstlisting}
}
\end{fpfig}
\primaryspacing

Although it is not possible to use Scalaness to compose external library components directly,
the programmer is free to create several different wrapping configurations if desired and
represent each of them separately in the Scalaness program. The Scalaness program could then
dynamically select which wrapping configuration is to be used in the final generated code. In
any case the type system will ensure that illegal wirings can never be made.

\subsection{Structure Subtyping}
\label{section-structure-subtyping}

\note{Include something useful here}

\subsection{Safe Casts}
\label{section-safe-casts}

Since one of the goals of Mininess is to promote type safety no implicit type conversions are
provided, and explicit conversions are permitted only when configured by the programmer.
However, assignment and initialization from a subtype to one of its supertypes \emph{is}
permitted. This provides convenience for safe conversions while allowing the programmer to
disallow casts that are unsafe or illogical. Furthermore, the programmer can allow certain casts
that are logical even if they require non-trivial user defined code to execute.

The Scalaness compiler accepts a configuration file that defines a relation on types
\textit{isCompatible}. If \textit{isCompatible($T_1$, $T_2$)} is true then it is permitted to
cast an expression of type $T_1$ into an expression of type $T_2$. There are no restrictions on
the types $T_1$ and $T_2$. However, all such conversions require explicit cast expressions; they
are never applied implicitly.

To illustrate the way these casts are implemented in Mininess programs consider as an example
the following two structure definitions.

\singlespace
\begin{lstlisting}[language=nesC]
struct UserInfo {
  char name[25];
  int  age;
  int  id;
};

struct UserToken {
  int id;
  int hash;
};
\end{lstlisting}
\primaryspacing

The developer may wish to allow an object of type !UserInfo! to be explicitly cast into an
object of type !UserToken!. Assuming the Scalaness configuration file has been edited to allow
this, the Scalaness compiler rewrites each cast expression into a call of a conventionally named
conversion command. These conversion commands exist in a nesC interface !DoCast!. For example

\singlespace
\begin{lstlisting}[language=nesC]
token = (struct UserToken)user;

// ... rewritten to ...

token = (call DoCast.UserInfo_UserToken(user));
\end{lstlisting}
\primaryspacing

The programmer is required to provide the !DoCast! interface and a component !DoCastC! that
provides that interface and contains an implementation of the various conversion commands
needed. The Scalaness compiler wires to !DoCastC! automatically without any further programmer
intervention.

\note{What about conversions between types that are type parameters?}

\subsection{Array Operations}
\label{section-array-operations}

Each Mininess expression !a! of array type !Array(T)! for some element type !T! has a
corresponding hidden dynamic value representing the size of the array. Array increment
expressions of the form !a |> e! can nominally be rewritten to nesC using pointer arithmetic as
!( (a) + (e) )!. Let $n$ be the dynamic size of expression !a!, then $n_e$ the dynamic size of
!a |> e! is $n_e = n - e$. This size might be negative but any use of an array expression with a
negative size results in a run time error \emph{at the point of use}.

A statement containing one or more array increment expressions or array indexing expressions is
rewritten as a block enclosed sequence of statements containing Scalaness compiler generated
local variables for the dynamic sizes of the temporary arrays along with appropriate run time
checks.

For each array increment operation $a_i \rhd e_i$ in a statement a variable to hold the value of
$e_i$ is declared and initialized. This is done so that $e_i$ will only be evaluated once; an
important considering in a language, such as nesC, with side effects. Also the dynamic size of
the result of each array operation $d_i$ is declared and initialized appropriately. For example

\singlespace
\begin{lstlisting}[language=nesC]
... (a |> e) ...
\end{lstlisting}
\primaryspacing

Is rewritten without regard to any possible optimizations as:

\singlespace
\begin{lstlisting}[language=nesC]
{
  int __e_1 = e;
  int __d_1 = __d_0 - __e_1;

  ... ((a) + __e_1) ...
}
\end{lstlisting}
\primaryspacing

Here !__d_0! is the dynamic size associated with the array expression !a!. In the common case
where !a! is a declared array the size will be known statically and an appropriate constant can
be used instead of a reference to a dynamic size variable.

For each array indexing operation $a_i[e_i]$ in a statement a variable to hold the value of
$e_i$ is declared and initialized, as before. A run time check is inserted to ensure that
the value of $e_i$ is inside the dynamic size of $a_i$. For example

\singlespace
\begin{lstlisting}[language=nesC]
... a[n] ...
\end{lstlisting}
\primaryspacing

Is rewritten without regard to any possible optimizations as

\singlespace
\begin{lstlisting}[language=nesC]
{
  int __e_1 = n;
  if (__e_1 >= __d_0) call boundsCheckFailed( );
  ... a[__e_1] ...
}
\end{lstlisting}
\primaryspacing

As before !__d_0! is the dynamic size associated with the array expression !a!.

In a statement involving multiple array operations, each operation is rewritten as described
above one at a time. After the first operation is rewritten, the enclosed modified statement is
further expanded with the second rewriting. The checks are issued in the order they are
encountered during a depth first left to right traversal of the Mininess abstract syntax tree.
For example a statement such as

\singlespace
\begin{lstlisting}[language=nesC]
x = ((a |> e1) |> e2)[b[i]];
\end{lstlisting}
\primaryspacing

Is first rewritten as:

\singlespace
\begin{lstlisting}[language=nesC]
{
  int __e_1 = e1;
  int __d_1 = __d_0 - __e_1;
  x = (((a) + __e_1) |> e2)[b[i]];
}
\end{lstlisting}
\primaryspacing

The resulting statement still contains three array operations. The second stage of rewriting
yields

\singlespace
\begin{lstlisting}[language=nesC]
{
  int __e_1 = e1;
  int __d_1 = __d_0 - __e_1;
  {
    int __e_2 = e2;
    int __d_2 = __d_1 - __e_2;
    x = ( ((a) + __e_1) + __e_2 )[b[i]];
  }
}
\end{lstlisting}
\primaryspacing

The inner indexing operation is then rewritten

\singlespace
\begin{lstlisting}[language=nesC]
{
  int __e_1 = e1;
  int __d_1 = __d_0 - __e_1;
  {
    int __e_2 = e2;
    int __d_2 = __d_1 - __e_2;
    {
      int __e_3 = i;
      if (__e_3 >= __d_b) call boundsCheckFailed( );
      x = ( ((a) + __e_1) + __e_2 )[b[__e_3]];
    }
  }
}
\end{lstlisting}
\primaryspacing

Finally, the outer indexing operation is rewritten

\singlespace
\begin{lstlisting}[language=nesC]
{
  int __e_1 = e1;
  int __d_1 = __d_0 - __e_1;
  {
    int __e_2 = e2;
    int __d_2 = __d_1 - __e_2;
    {
      int __e_3 = i;
      if (__e_3 >= __d_b) call boundsCheckFailed( );
      {
        int __e_4 = b[__e_3];
        if (__e_4 >= __d_2) call boundsCheckFailed( );
        x = ( ((a) + __e_1) + __e_2 )[__e_4];
      }
    }
  }
}
\end{lstlisting}
\primaryspacing

% Notice that declarations can't be enclosed in the block without changing the scope of the
% identifiers they declare. Thus array expressions used in the initialization of a variable
% present a special problem. It might be necessary to first split initializers off from their
% declarations in a separate AST processing pass.

Functions declared to take an array as a parameter are rewritten so that the dynamic size of the
array is passed as an additional parameter. This parameter becomes the $d$ of array expressions
involving only the parameter. Calls to such functions are rewritten to pass the additional
dynamic size information as appropriate.

The command !boundsCheckFailed! must be provided by the programmer in a component named
!BoundsCheckC!. The behavior of this command is unspecified but it should not return. I expect
in most cases it will restart the node after, perhaps, attempting to log the problem. As with
!DoCastC! the Scalaness compiler automatically wires to !BoundsCheckC! as appropriate.

\lstDeleteShortInline!

\section{Scalaness}
\label{section-scalaness-implementation}

\lstset{language=scalaness}
\lstMakeShortInline!

Scalaness is implemented as a modified Scala compiler based on the development Scala compiler on
GitHub \cite{XXX}. The Scala compiler has a plug-in architecture and originally I hoped that
Scalaness could be implemented as a compiler plug-in. That would have made Scalaness easier to
use and maintain and thus enhanced the systems practicability.

Unfortunately, implementing Scalaness as a compiler plug-in met with difficulties. The main
problem was with extending the typing rules of Scala to accommodate the Scalaness type system.
The plug-in approach seemed to require a complete reimplementation of Scala typing inside the
plug-in. This is because plug-ins can only gain control either before Scala typing has occurred
or after it has completed. Thus the implementation of the nesT typing rules couldn't easily
benefit from the logic in the existing type checker. In contrast building Scalaness as a
modified compiler allowed me to ``piggyback'' nesT type information on the existing type checker
infrastructure.

Yet in order to facilitate keeping Scalaness synchronized with future developments of the main
Scala compiler every attempt was made to implement Scalaness in the least invasive way possible.
Much of the logic, including the nesT typing rules themselves are implemented in separate
packages away from the main body of the compiler code base. The places where it was necessary to
insert Scalaness specific code into, for example, the existing type checker, have been kept to a
minimum.

I should also note that making radical changes to Scala syntax was not seriously considered. I
didn't want to modify \emph{both} the parser and the type checker. Fortunately Scala has a
general mechanism for adding arbitrary information to declarations, namely
\newterm{annotations}. Thus I used Scala annotations to express nesT module types as strings
using a syntax of my choosing.

My work is based on Scala 2.10. This version of Scala also provides an extensive reflection API
and experimental support for expression macros. These facilities allow one to do AST
transformations on Scala programs using ordinary (macro) Scala code. Macros are described by the
Scala community as a kind of ``lightweight'' plug-in mechanism. Unfortunately at the time of this
writing type macros are not available so it is not yet possible to write a macro that outputs a
class definition. However, in the future when type macros become available it might be possible
to implement some or all of Scalaness as a macro library.

\subsection{Scala Compiler Organization}
\label{section-scala-compiler-organization}

The Scala compiler is organized as a number of \newterm{phases} that rewrite the input in
successive steps lowering it to JVM bytecode. The precise phases used can be listed with the
command \texttt{scalac -Xshow-phases}. Of primary significance to Scalaness are the first four
phases used by the stock Scala compiler. These phases are shown below.

\singlespace
\begin{Verbatim}
parser
namer
packageobjects
typer
...
\end{Verbatim}
\primaryspacing

The bulk of the modifications made by Scalaness are in the typer phase. There hooks where added
at critical points that call into Scalaness specific code in package !uvm.edu.scalaness!. In
addition a new phase was added between the parser and namer phases. This new phase,
\texttt{scalanesspostparser}, is responsible for augmenting certain Scalaness constructs with
their necessary runtime support. This is done by inserting new material in the AST produced by
the parser. In principle that material could have been manually written by the programmer but
instead is automatically generated as a convenience. It is this postparser phase that could
potentially be eliminated by type macros when they become available.

\subsection{Liftable Types}
\label{section-liftable-types}

Certain types that appear in a Scalaness program are liftable to types in the Mininess
components manipulated by that program. Values of these types can be computed in the Scalaness
program and then used to specialize Mininess components. These values need to be transformed as
they cross the boundary between the two programming languages due to differences in the way a
liftable type and its Mininess counterpart are represented. In this section I will describe
which Scalaness types are liftable and how their values are handled when used to specialize a
Mininess component.

\subsubsection{Primitive Types}
\label{section-liftable-primitives}

All liftable types are subtypes of a special marker trait !Liftable!. The primitive types in
Mininess have liftable counterparts in Scalaness that are classes extending !Liftable!. For
example the type !uint16\_t! in Mininess corresponds with class !UInt16! in Scalaness. In my
implementation there are six primitive, liftable integer types: three unsigned types !UInt8!,
!UInt16!, !UInt32!, and three corresponding signed integer types. All of these types have
specific sizes; the implementation does not provide a simple integer type. This avoids issues
associated with the machine dependent size of !int! in nesC. Finally two other liftable
primitive types are also provided: !Char! and !Uninit! (which lifts to !void!).

The Mininess subtype relations are preserved in Scalaness where !UInt8 <: UInt16 <: UInt32! and
similarly for the signed integer types. In Scalaness these types are defined in the object
!LiftableTypes! so that they don't conflict with any normal (non-liftable) types defined by the
programmer or the language, such as !Char!. Furthermore the integer liftable types are endowed
with the usual arithmetic operations so they can be manipulated in the Scalaness program in a
natural way.

No conversions are provided between the liftable types and their ordinary Scala analogs. This
means existing libraries that, for example, manipulate Scala !Char! objects won't work with
!LiftableTypes.Char!. I do not consider this a problem for two reasons.
\begin{enumerate}
\item Since values of liftable type will eventually be written into Mininess components, they
  will likely be put to very different uses than values of ordinary Scala types. In fact,
  letting the Scala type system catch inadvertant mixing of ordinary primitives and liftable
  primitives could be seen as a desirable feature.
\item Conversions (including implicit conversions) can be easily added by the Scalaness
  programmer if desired using the normal facilities of Scala.
\end{enumerate}

To facilitate the second point, explicit conversion methods from each liftable type to its
corresponding non-liftable counterpart are provided.

The Scala type system is used to ensure compile-time type safety of the primitive liftable types
in a Scalaness program. For example the type !Int16! can only be constructed using a value of
Scala's type !Short!. Consequently normal Scala type checking prevents a potentially out of
range value from being used.

Unfortunately Scala does not support unsigned types natively. In the current implementation a
sufficiently wide signed type is used to initialize objects of unsigned liftable type. This
makes it possible to use an out-of-range value during the execution of the Scalaness program
resulting in a runtime exception. However, Scala programs are subject to runtime exceptions for
a variety of reasons. It is outside the scope of this work to address the problem of how to
ensure a Scalaness program never exits by way of an exception.

\subsubsection{Arrays}
\label{section-liftable-arrays}

\emph{An ordinary Scala array type is liftable if and only if its element type is liftable.}
This is an exception to the rule stated in \autoref{section-liftable-primitives} that says all
liftable types must extend !Liftable! and thus arrays are handled in a special way. Yet it is a
significant convenience to the programmer to be able to use ordinary Scala arrays, and not some
special ``liftable array'' class, to hold liftable arrays. This need does not arise for the
other containers in the Scala collections library since those containers have no counterpart in
Mininess anyway.

For example, the programmer may wish to create and manipulate a !List[UInt8]! during the
execution of a Scalaness program, but the list itself won't be liftable. In contrast the
programmer may wish to lift an !Array[UInt8]! into Mininess.

\subsubsection{Classes}
\label{section-liftable-classes}

A Scala class $C$ that extends the !Liftable! trait is liftable to a Mininess structure type
provided it additionally obeys the following inductive rules.

\begin{enumerate}
\item $C$ is not generic.
\item All of $C$'s fields have liftable type.
\item All of $C$'s supertypes (except !AnyRef! and !ScalaObject!) are liftable types.
\end{enumerate}

In this case $C$ is said to be a \textit{liftable class}. Except for the rules mentioned here
there are no restrictions on the definition or use of liftable classes. In particular, they are
able to have convenience methods, although the methods of a liftable class have no manifestation
in the generated Mininess code.

For example, consider the following Scalaness code:

\singlespace
\begin{lstlisting}[language=scalaness]
class Header
 (val nodeID      : nodeIDType,
  val componentID : UInt8 ) extends Liftable

class TimeStampedHeader
 (val timeStamp   : UInt16) extends Header
\end{lstlisting}
\primaryspacing

Here !nodeIDType! is a previously defined liftable class type. Consequently both of these
classes are liftable and have representations as Mininess structure types. \note{Figure out how
  to properly model Scala's inheritance system in this context.}

\subsection{Lifting}
\label{section-lifting}

\note{Talk about how lifting is actually done.}

\subsection{MetaType}
\label{section-metatype}

Scalaness allows types to be dynamically constructed. However, the Scala type system does not
directly support using types as values. To work around this limitation I explicitly introduced
a wrapper generic class !MetaType[T]! to represent any liftable type that is a subtype of !T!.

\singlespace
\begin{lstlisting}[language=scalaness]
class MetaType[+Tau <: Liftable]
  (val wrappedType: TypeRepresentation)
    extends Liftable
\end{lstlisting}
\primaryspacing

Objects of type !MetaType! contain a representation of a nesC type. They are covariently
parameterized by an upper bound type. Note that !MetaType!'s type parameter is a Scala type that
is liftable to Mininess. However the value it wraps is a representation of the already lifted
type. Thus !MetaType! objects form a bridge between the Scala and nesC sides of the system.
\note{Is it possible for wrappedType to not be a subtype of Tau? It would seem so. What happens
  in that case?}

\subsection{Component Definition}
\label{section-component-definition}

Components in Mininess can be parameterized by types and values and instantiated multiple times.
These properties are closely modeled by Scala classes. Thus the representation of a Mininess
component in Scalaness is by way of a class that extends a special marker trait.

One might be tempted to allow a syntax such as

\singlespace
\begin{lstlisting}[language=scalaness]
class SendC
  [Adt <: UInt32, MessageT <: AbstractMessage[Adt]]
  (self: Adt) extends MininessCompont {

  import error_t radio( MessageT * );

  export error_t send( Adt addr, uint8_t *data ) {
    MessageT packet = { self, addr, data };
    radio( &packet );
    return SUCCESS;
  }
}
\end{lstlisting}
\primaryspacing

This defines a Scalaness component using Scala syntax for representing type and value
parameters. The body of the component, however is written in Mininess/nesT. Unfortunately this
can't be supported without modifying the Scala parser to accept Mininess as well as Scala. It
was my intention to avoid this complication.

One way to work around the problem is presented in \cite{Garcia-Scala-Query} where the alien
language is included as a string literal. The Scala typer will treat the Mininess program as
having type String but the additional Scalaness type checking code could parse the string's
contents (provided it is required to be a literal) and impose additional typing rules on those
contents. However, this approach leads to a rather ungainly programming style:

\singlespace
\begin{lstlisting}[language=scalaness]
class SendC
  [Adt <: UInt32, MessageT <: AbstractMessage[Adt]]
  (self: Adt) extends MininessComponent {

 """import error_t radio( MessageT * );

    export error_t send( Adt addr, uint8_t *data ) {
      MessageT packet = { self, addr, data };
      radio( &packet );
      return SUCCESS;
    }"""
}
\end{lstlisting}
\primaryspacing

Since the Mininess code implementing a component is often long and complex it makes sense to
allow the programmer to edit and manage that code in tools that are nesC-aware such as nesC
syntax highlighting editors. I anticipate, in fact, that in many cases different programmers
with very different kinds of expertise will be editing the Scalaness and Mininess code bases.
Thus my implementation uses a string literal to name an external file containing the Mininess
contents of a component as shown below.

\singlespace
\begin{lstlisting}[language=scalaness]
class SendC
  [Adt <: UInt32, MessageT <: AbstractMessage[Adt]]
  (self: Adt) extends MininessComponent {

  "SendC.nc"
}
\end{lstlisting}
\primaryspacing

Conceptually the contents of the named file replace the literal name in the Scalaness program.
This approach allows the normal Scala parser and type checker to process the program
successfully. During compilation the Scalaness extension locates the specified Mininess file
(\filename{SendC.nc} above), parses it as Mininess and does Mininess type checking on that file
using type and value parameters as provided to the Scalaness class.

Using Scala's syntax for specifying type and value parameters as shown above is attractive but
unfortunately it does not work for Scalaness. There are two problems.
\begin{enumerate}
\item Mininess components can be passed around in a Scalaness program in an uninstantiated
  state. In contrast Scala classes are not first class values in Scala.
\item Mininess components can have parameters involving dynamically constructed types. In
  contrast Scala class parameters must involve only types that are fully statically specified.
\end{enumerate}

I considered modifying the Scala compiler to allow dynamically constructed types to appear in
declarations and as type parameters but this required extensive modifications to the existing
Scala type checker and so was rejected as an option. Instead the Scalaness representation of a
Mininess component includes a special method !instantiate! that is used to create instantiated
Mininess components. That method accepts the value parameters as ordinary Scala parameters using
liftable types, and it accepts the type parameters as ordinary Scala parameters of type
!MetaType[T]! where !T! is liftable. The example above becomes:

\singlespace
\begin{lstlisting}[language=scalaness]
class SendC extends MininessComponent {

  def instantiate(
    Adt: MetaType[UInt32],
    MessageT: MetaType[AbstractMessage],
    self : ?);

  "SendC.nc"
}
\end{lstlisting}
\primaryspacing

\note{What's up with this example? It doesn't seem expressible with the current system.}

%\note{I have to talk about ModuleType annotations somewhere!}

\subsection{Component Operations}
\label{section-component-operations}

\note{This section needs to be overhauled. I'm not sure how much detail I really should go into
  here.}

In this section I describe how the Scalaness-level operations on Mininess components are
implemented. There are two basic operations to consider: component composition and component
instantiation.

\subsubsection{Composition}
\label{section-composition}

\textit{blah, blah, blah}

Before delving into the details of how the conversion to nesC works, I will start with a simple
example. Below is the Scalaness runtime representation of three Mininess components, each as an
instance of class NamedProgramComponent.

\singlespace
\begin{lstlisting}[language=scalaness]
val A = new NamedProgramComponent(
  name            = "A",
  typeParameters  = Set(),
  valueParameters = Set(),
  imports         = Set("c1", "c2"),
  exports         = Set())

val B = new NamedProgramComponent(
  name            = "B",
  typeParameters  = Set(),
  valueParameters = Set(),
  imports         = Set(),
  exports         = Set("c1"))

val C = new NamedProgramComponent(
  name            = "C",
  typeParameters  = Set(),
  valueParameters = Set(),
  imports         = Set("c1"),
  exports         = Set("c2"))
\end{lstlisting}
\primaryspacing

The parameters, imports, and exports are all represented (for now) as sets of strings. In this
example A uses commands c1 and c2, B provides command c1, etc.

These components can be composed as follows:

\singlespace
\begin{lstlisting}[language=scalaness]
val composedComponent1 =
  A.toProgramComponent + B.toProgramComponent

val composedComponent2 =
  composedComponent1 + C.toProgramComponent

composedComponent2.display(outputFile)
\end{lstlisting}
\primaryspacing

The code above is plain Scala. The Scalaness programmer would use a more convenient syntax but
the compiler, essentially, rewrites that syntax into the above. The first stage program executes
with a Scalaness runtime support library that provides the necessary classes and related
infrastructure to make such code work as intended.

A, B, and C are NamedProgramComponents. These are the components defined by the Scalaness
programmer. Internally, however, all operations on components are done using anonymous
components from class ProgramComponent. I describe the reasons for this more fully below. The
method toProgramComponent is a conversion operator. The method '+' is overloaded here to do
component composition.

In the example above the material that is written to the outputFile is:

\singlespace
\begin{lstlisting}[language=nesC]
configuration AppC {
    provides c1;
    provides c2;
}
implementation {
    components A, B, C;
    A.c1 -> B.c1;
    A.c2 -> C.c2;
    C.c1 -> B.c1;
}
\end{lstlisting}
\primaryspacing

There are two kinds of components: those with names and those that are anonymous. Anonymous
components are the result of component composition and can contain a collection of named
components embedded inside of them. The imports and exports of a component are maps that
associate the name of the import or export with the embedded named component that uses or
provides it, respectively. A conversion exists from named to anonymous components. That
conversion returns an anonymous component that has the (one) named component embedded inside of
it.

The types used to represent imports and exports are as shown below.

\singlespace
\begin{lstlisting}[language=scalaness]
type ImportsType =
  Map[String, List[NamedProgramComponent]]

type ExportsType =
  Map[String, NamedProgramComponent]

type WireType =
  (String, NamedProgramComponent, NamedProgramComponent)
\end{lstlisting}
\primaryspacing

The ImportsType maps an overall import name to a list of NamedProgramComponents because, after
composition the same import might be used by more than one embedded component. That is not the
case for exports since exports are combined with a mutually exclusive map merge. Each overall
export can thus be provided by exactly one embedded component.

The type WireType is used to represent the "wires" between embedded components. It is a triple
where the first element is the name of the command being wired and the other two elements are
the wire endpoints (user first).

In addition each (anonymous) ProgramComponent contains a ProgramConfiguration member. This
configuration describes the embedded components and the nesC wiring used to connect them.

ProgramComponents are immutable instances from the following class

\singlespace
\begin{lstlisting}[language=scalaness]
class ProgramComponent(
  private val typeParameters: Set[String],
  private val termParameters: Set[String],
  private val imports       : ImportsType,
  private val exports       : ExportsType,
  private val configuration : ProgramConfiguration) { ...
\end{lstlisting}
\primaryspacing

\subsubsection{Instantiation}
\label{section-instantiation}

A module such as

\singlespace
\begin{lstlisting}[language=scalaness]
class BlinkC[A <: Int32](period: A) extends NesCComponent {
  "BlinkC.nc"
}
\end{lstlisting}
\primaryspacing

Is modified by the Scalaness plug-in to hold some additional fields and methods

\singlespace
\begin{lstlisting}[language=scalaness]
class BlinkC[A <: Int32](period: A) extends NesCComponent {
  /////////
  // Code generated by the Scalaness compiler.
  /////////
  var AType: MetaType[Int32] = null
  def getTypeMap  = Map("A" -> AType)
  def getValueMap = Map("period" -> period)
  ////////
  // END of Scalaness generated code.
  ////////

  "BlinkC.nc"
}
\end{lstlisting}
\primaryspacing

The generated field !AType! holds a reference to the type used to instantiate the module. In
Scalaness this type can only be known at runtime. The methods !getTypeMap! and !getValueMap! are
used during module specialization. The method that does so is inherited from !NesCComponent! but
it calls these methods through their abstract interface.

Notice that at the Scala level the type parameter must be some type that extends
!LiftableTypes.Int32! (in this example). However, when the module is instantiated the Scalaness
programmer might provide a variable of type !MetaType[Int32]!. For example

\singlespace
\begin{lstlisting}[language=scalaness]
val desiredType: MetaType[Int32] = ...
val blinkModule = new BlinkC[desiredType](new Int8(100))
\end{lstlisting}
\primaryspacing

The Scala parser will accept this construction but the Scala type checker will reject the use of
a val as a type argument. Thus expressions involving component instantation are rewritten during
the post-parser phases into

\singlespace
\begin{lstlisting}[language=scalaness]
val blinkModule = {
  val temp = new BlinkC[Int32](new Int8(100))
  temp.AType = desiredType
  temp
}
\end{lstlisting}
\primaryspacing

Unfortunately this rewritting is difficult because of the minimal amount of information
available to the plug-in immediately after parsing.

\lstDeleteShortInline!

%%% Local Variables: 
%%% mode: LaTeX
%%% TeX-master: "main"
%%% End: 
