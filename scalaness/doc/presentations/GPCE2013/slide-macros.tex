%
% For landscape mode slides:
%  1) use \documentclass[landscape]{slides}
%  2) latex main
%  2) dvips -tlandscape main -o 
%
% You can also try
%  1) use \documentclass[landscape]{slides}
%  2) latex main
%  3) texdvipdfm -l main.dvi
%
\usepackage{graphics}
\usepackage{color}
\usepackage{epsfig}
%\setlength{\oddsidemargin}{-30pt}
%\setlength{\evensidemargin}{-30pt}
\setlength{\topmargin}{-0.75in}
\setlength{\textheight}{7in}
% colour defs
\input colordvi
\definecolor{headcolor}{rgb}{0.55,0,0}
%\definecolor{itemcolor}{rgb}{0.1,0.1,0.55}
%\definecolor{itemcolor}{named}{SkyBlue}
%\definecolor{tickmarkcolor}{rgb}{0.5,0.1,0.5}
%\definecolor{tickmarkcolor}{named}{SkyBlue}
%\definecolor{titlecolor}{rgb}{0.65,0.1,0.1}
% \definecolor{titlecolor}{rgb}{0.7,0.5,0.9}
% \definecolor{tickmarkcolor}{rgb}{0.7,0.5,0.9}
% \definecolor{emphcolor}{rgb}{0.7,0.5,0.9}
%\definecolor{titlecolor}{rgb}{0.8,0.7,0.2}
%\definecolor{tickmarkcolor}{rgb}{0.85,0.84,0.1}
\definecolor{titlecolor}{cmyk}{0.98,.3,0,0.43}
\definecolor{tickmarkcolor}{cmyk}{0.98,.3,0,0.43}
%\definecolor{emphcolor}{rgb}{1.0,0.6,0.3}
\definecolor{emphcolor}{named}{BrickRed}
\definecolor{refcolor}{rgb}{0.,.4,.4}
%\def\Red#1{\Color{0 0.70 0.70 0.2}{#1}}
%\def\Green#1{\Color{0.70 0 0.5 0.2}{#1}}
%\def\Blue#1{\Color{0.70 0 0 0.2}{#1}}
\definecolor{Beige}   {rgb}{0.96,0.96,0.86}
%\definecolor{Background}   {rgb}{.97,.94,0.82}
\definecolor{Background}   {rgb}{1.,1.,1.}
\definecolor{MyYellow}{rgb}{1.,0.84,0.8}
\definecolor{Gold}  {rgb}{1.0,0.84,0.}
\definecolor{White} {named}{White}
%\definecolor{DarkGold}  {rgb}{.5,0.42,0.}
\definecolor{DarkGold}  {rgb}{.3,0.28,0.}
\definecolor{Blue}{rgb}{0.,0.,1.}
\definecolor{Pink}{rgb}{1.,0.75,0.8}
\definecolor{light-blue}{rgb}{0.8,0.85,1}
\definecolor{mygrey}{gray}{0.75}
\definecolor{darkgrey}{gray}{0.40}
\definecolor{greenyellow}{named}{GreenYellow}
\definecolor{myblack}{named}{Black}
% existing color names:
%GreenYellow, Yellow, Goldenrod, Dandelion, Apricot, Peach, Melon,
%YellowOrange, Orange, BurntOrange, Bittersweet, RedOrange, Mahogany,
%Maroon, BrickRed, Red, OrangeRed, RubineRed, WildStrawberry, Salmon,
%CarnationPink, Magenta, VioletRed, Rhodamine, Mulberry, RedViolet,
%Fuchsia, Lavender, Thistle, Orchid, DarkOrchid, Purple, Plum, Violet,
%RoyalPurple, BlueViolet, Periwinkle, CadetBlue, CornflowerBlue,
%MidnightBlue, NavyBlue, RoyalBlue, Blue, Cerulean, Cyan, ProcessBlue,
%SkyBlue, Turquoise, TealBlue, Aquamarine, BlueGreen, Emerald,
%JungleGreen, SeaGreen, Green, ForestGreen, PineGreen, LimeGreen,
%YellowGreen, SpringGreen, OliveGreen, RawSienna, Sepia, Brown, Tan,
%Gray, Black, White.

% new tick marks for itemize
\renewcommand{\labelitemi}{ {\color{tickmarkcolor}$\bullet$} }

% new tick marks for itemize
\renewcommand{\labelenumi}{
        {\color{tickmarkcolor} \arabic{enumi}}
}


% color versions of itemize/enumerate with less spacing
\newenvironment{citemize}{
        \vspace{-0.2in}
        \begin{itemize}
        \itemsep 0in
        %\color{itemcolor}
}{
        \end{itemize}
}
\newenvironment{cenumerate}{
        \vspace{-0.2in}
        \begin{enumerate}[\hspace{5mm}1.]
        \itemsep 0in
        %\color{itemcolor}
}{
        \end{enumerate}
}

% headed slide takes 1 parameter: the heading of the slide, which will
% appear at the topright of the page
\newenvironment{hslide}[1]{
        \begin{slide}
        \begin{flushright}
                {\color{headcolor} \tiny #1}
        \end{flushright}
}{
        \end{slide}
}

% titled slide takes 1 parameter: the heading of the slide, which will
% appear at the topright of the page
\newenvironment{tslide}[1]{
        \begin{slide}
        \begin{flushleft}
                {\bf \color{headcolor} #1}
        \end{flushleft}
}{
        \end{slide}
}

% title - displayed as emphasised
\newcommand{\stitle}[1]{
        { \bf \color{titlecolor} #1 }
}

% comments
\newcommand{\mknote}[1]{
        \begin{note}
                \begin{flushright}
                        {\it NOTES}
                \end{flushright}
                {\scriptsize #1
                
                end}
        \end{note}
}

% end web borrowed macros

\def\cemph#1{{\color{emphcolor}\emph{#1}}}
\def\ucemph#1{{\color{emphcolor}#1}}
\def\cref#1{{\color{refcolor}\small\emph{#1}}}
\def\bigref#1{{\color{refcolor}\emph{#1}}}
\def\cdesc#1{{\color{titlecolor}\small\emph{#1}}}
%\usepackage{times}
%\usepackage{mathptmx}
% beat slides class fonts into submission
% first, use most of package times
\renewcommand{\sfdefault}{phv}
\renewcommand{\rmdefault}{ptm}
\renewcommand{\ttdefault}{pcr}
% then, use most of package mathptm
\DeclareSymbolFont{operators}   {OT1}{ptmcm}{m}{n}
\DeclareSymbolFont{letters}     {OML}{ptmcm}{m}{it}
\DeclareSymbolFont{symbols}     {OMS}{pzccm}{m}{n}
\DeclareSymbolFont{largesymbols}{OMX}{psycm}{m}{n}
\DeclareSymbolFont{bold}        {OT1}{ptm}{bx}{n}
\DeclareSymbolFont{italic}      {OT1}{ptm}{m}{it}
%\setlength{\oddsidemargin}{-30pt}
%\setlength{\evensidemargin}{-30pt}
%\setlength{\textwidth}{7.4in}

\long\def\inlong#1{{\def\titlefont{\it}{\it #1}}}
\long\def\inlong#1{}
% slide macro abbreviations
\def\centerheader#1{\begin{center}
 {{\large\titlefont\color{titlecolor}#1}\par\vskip 1.5ex}
 \end{center}}
\def\titlefont{\bf}

\def\startslide#1{
\begin{slide}
\if !#1
 \else \centerheader{#1}
\fi
}
\def\stopslide{\end{slide}}

\newcounter{section} % make macro files load without complaining
