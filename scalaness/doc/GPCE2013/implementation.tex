\section{Implementation}
\label{section-implementation}

Scalaness is implemented as a modification to the open source Scala compiler. Scala was chosen
as a basis for our first stage language because the Scala compiler supported a plug-in
architecture, and we originally envisioned implementing Scalaness as a plug-in. We also wanted
to create a practical programming system, and Scala's easy access to Java libraries and broad
community support were attractive.

Keeping the Scala and Scalaness type checking well separated in the implementation had useful
software engineering benefits as well. It simplifies the problem of tracking the evolving Scala
compiler. It also promotes a clear separation of first and second stage concerns in the mind of
the Scalaness programmer. Unfortunately we found that the needed modifications to the type
checker could only be made by direct modification of the compiler code base.

In addition to type checking, runtime support is needed to implement Scalaness module
operations. Also, facilities are required to read nesT modules from the file system and parse
them into ASTs, and to write TinyOS image source code files defined by constructed nesT modules
at \tt{image} invocations.

\subsection{Online Repository and Examples}

The Scalaness/nesT compiler and several code examples, including applications discussed in
\autoref{section-overview} and \autoref{section-application}, are available for download from 
% Space saver - SS
%a GitHub repository, accessible via an
%informative webpage at 
the following URL:
\vspace{0.8em}

\centerline{\texttt{http://tinyurl.com/a85z8cu}}

% presentation allows perusal of Scalaness and nesT code examples,
% without the need to build the compiler.

\subsection{nesT Type Checking and Program Transformation}

The nesT language is treated by two major components in the implementation, the type checker and
the nesT-to-nesC rewriting transformation. The nesT type checker was written from the ground up,
in contrast to the Scalaness type checker which was defined as an extension to the Scala type
checker. The rewriting transformation yields TinyOS2-compliant source code, which can be
separately compiled.

The nesT language is defined as a subset of the nesC language. An AST yielded by parsing is type
checked by our algorithm, which incorporates subtyping and other features not present in nesC
type checking. This algorithm is a nearly direct encoding of the type discipline described in
\autoref{section-nest}. Following type checking, the AST is submitted to a rewriting
transformation that imposes semantic disciplines discussed in \autoref{section-nest}, in
particular type safe casting and array bounds checks, also in nesC. For example, a statement of
the form $\tt{x = a[e]}$ will be rewritten to:
\begin{Verbatim}[fontfamily=cmtt,fontsize=\scriptsize]
    int _x = e; if (_x >= a_SIZE) fail(); x = a[_x];  
\end{Verbatim}
where $\tt{a\_SIZE}$ is an automatically generated variable containing the size of $\tt{a}$ and
$\tt{fail}$ is some user-defined function that handles array bounds check
failure. % The expression $\tt{e}$ is bound to
% $\tt{x}$ to ensure that it is evaluated only once in case of side effects.

Source code for nesT module definitions is written in separate files that are included in
Scalaness code, as discussed below. This separation is mainly for software engineering purposes,
to avoid modifying the Scala compiler to parse intermingled Scala and nesT syntax.

\subsection{Scalaness Module Language Syntax}

In order to limit modifications of the Scala compiler and reduce engineering problems in our
implementation, we have avoided modifying Scala syntax to represent Scalaness features. Hence,
modules are represented as class instances, which must satisfy the following trait:
\begin{Verbatim}[fontfamily=cmtt,fontsize=\scriptsize]
  trait NesTModule {
    def image(): Unit    // Generates residual nesC program.
    def +>(m: NesTModule): NesTModule    // Wires this to m.
  }
\end{Verbatim}
This trait is implemented by a \tt{NesTModule} class that provides the appropriate semantics for
wiring and TinyOS image generation, including translation to nesC and file output. This class
also manages parsing and storage of nesT ASTs from source code files, and type checking of nesT
ASTs.

Any nesT module definition is a subclass of \tt{NesTModule}. Some subtleties are involved in
supporting first class \emph{generic} modules. Instantiation is implemented by method call, but
since type and value parameters vary per module, particular modules must define their own
parameters and instantiation methods. For example, we would represent the \tt{authSend}
component definition from \autoref{section-overview}, line \ref{l:authsend} as follows:
\begin{Verbatim}[fontfamily=cmtt,fontsize=\scriptsize]
 class authSend extends NesTModule {
   var mt     : MetaType[LiftableType] = _
   var sendk  : LiftableType = _
   def instantiate(m: MetaType[LiftableType], k: LiftableType) =
     { val result=new authSend; result.mt=m; result.sendk=k }
   "authSend.nt"
 }
\end{Verbatim}
Although the \tt{instantiate} method and parameter fields must be defined in the implementation
at the time of this writing, compiler generation of these definitions is a topic for future
work; any modules instantiate method can be easily inferred from its type annotation. Note that
the types at which parameters are declared are as general as possible (e.g.~\tt{s} and \tt{n}
are not declared as \tt{uint}s but as \tt{LiftableType}s). This is because class definitions
support the semantics of Scalaness, not Scalaness type checking (discussed below), and declaring
generic parameters at a maximally general type removes interference related to Scala type
checking. Finally, note the string literal \tt{``authSend.nt"} at the end of the definition.
This is the file containing the nesT source code definition of the module. The Scala compiler
has been modified to input and parse the specified source code when this literal is encountered
during the Scala type checking phase.

%Here is an example usage of these constructs, which is an inlining of
%the invocation described in \autoref{example-scalanesssemantics}. In
%this snippet we assume \tt{scode} has already been defined
%appropriately, and for clarity we omit the type annotations described
%in the following section:
%\begin{Verbatim}[fontfamily=cmtt,fontsize=\scriptsize]
%  val nodeM = new NodeC
%  typedef adt <: uint8 = new MetaType[uint8]()
%  val ncode = nodeM.instantiate(adt, 1, 2)
%  (ncode +> scode).image()
%\end{Verbatim}
%Evaluation of this code will generate a TinyOS node image, written in
%nesC, supporting the application defined by the wiring of \tt{scode}
%to \tt{ncode}.

\subsection{Type Annotation and Checking}

Scalaness typing relies on native Scala syntax for terms, specifically Scala annotations and
singleton types are utilized. Scala annotations allow metadata to be associated with
definitions. A module type annotation is of the form \tt{@ModuleType(\tt{``}\jmodtcat\tt{"})},
where $\jmodtcat$ is defined using the syntax of \autoref{figure-modulesyntax}. The
compiler-defined \tt{ModuleType} class automatically associates the type with the identifier
immediately following it. In the case of module class definitions, the type is assigned as a
class field. In the case of variable definitions, the type is assigned as a Scala singleton type
of the object. For example, the declaration of \tt{authSend} on line \ref{l:authsend} in as in
\autoref{section-overview} would be preceded by such an annotation where $\jmodtcat$ is the type
specified in \autoref{example-nesttyping}, and \tt{sendM} as on line \ref{l:inst1} would be
annotated with an instance of that type. Similarly, annotations are required on method parameter
and result types, if those methods expect nesT modules as arguments or return them, as for the
\tt{radioC} parameter of the \tt{authSpecialize} method defined in \autoref{section-overview},
and the method's \tt{commT} return type. These requirements reflect the type discipline in
Scalaness as specified in \autoref{section-scalaness}, which requires module type annotations at
these points.

Scalaness type checking has been implemented as an analysis of these annotations during Scala
type checking, piggybacking on that process. When type checking a class that extends
\tt{NesTModule}, the compiler uses its type annotation to perform nesT type checking on the
underlying AST representation of the module. When type checking module operations (i.e.~at
invocations of \tt{instantiate}, \verb!+>!, or \tt{image}), the Scala compiler has been modified
to examine operand types for Scalaness type annotations, and to decorate resultant singleton
types of these operations with new Scalaness annotations, reflecting the typing rules in
\autoref{figure-scalanesstyping}. A type checking exception is raised in case this analysis
fails. Scalaness type checking does not modify Scala type checking in any other way, so it is a
conservative extension of Scala typing.

%\begin{verbatim}
%@ModuleType("""{ countType <: UInt32 }
%               < ; >
%               { set_count(x: countType): Void;
%                 configure(): Void }""")
%\end{verbatim}


\subsection{Importing nesC Libraries}
\label{section-libraries}

Our preliminary experiments with nesT show that it is expressive enough to write useful program
components. However, any realistic application will need to interact with various libraries
written in nesC. One library of critical importance is the TinyOS operating system itself. Our
current solution is to allow non-generic nesC components to be treated as nesT modules as long
as they only \tt{use} or \tt{provide} commands, which are interpreted as nesT imports and
exports. Support for specializable generic nesC library components is a topic for future work.
Events can be accessed through ``shim'' modules provided by the user, since used or provided
events are really just syntactic sugar for provided or used commands respectively. A library
component defined in a file \tt{LibraryC.nc} can be defined as a nesT module as follows:
\begin{Verbatim}[fontfamily=cmtt,fontsize=\scriptsize]
 object LibraryC extends NesTModule { external("LibraryC.nc") }
\end{Verbatim}
Note that nesC code imported in this way is not type checked by the Scalaness/nesT compiler,
since nesT is a strict subset of nesC. Rather, the programmer type annotates the shimmed module
using a $\tt{@ModuleType}$ annotation as for other module definitions, and the compiler trusts
that the annotation is correct. This introduces a possibility for type safety failure in our
system, if the imported code contains a type error. A possible long term goal would be a
complete re-write of TinyOS in nesT, yielding full type safety of all sensor code, but this is
well beyond our current scope.
