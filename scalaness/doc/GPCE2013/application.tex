\section{Example: Authorization and Access Control}
\label{section-application}

In \cite{chapin-skalka-SpartanRPCTR} the SpartanRPC architecture for link-layer resource
authorization in TinyOS-based WSNs is developed (as an extension of
\cite{chapin-skalka-SpartanRPC}). In SpartanRPC, resources are accessed by link-layer remote
procedure calls (RPC) which require authorization. Users are authorized by communicating
credentials to the provider, expressed in an authorization logic based on RT \cite{Li:RRBTMF}
and implemented using TinyECC \cite{Liu-Peng-TinyECC-2008} public key signatures. SpartanRPC
supports an ``open world'' security model, allowing WSNs in different security domains to
interact without sharing secrets \emph{a priori}. However, public key encryption and signature
verification is very expensive in a WSN. Hence, session keys are negotiated for ongoing resource
access (using a TinyECC-based Diffie-Hellman protocol).

In this section we describe a re-implementation of the SpartanRPC protocol in Scalaness/nesT
that addresses several shortcomings of SpartanRPC, and will thus serve to illustrate the power
of Scalaness/nesT. The central idea, illustrated in \autoref{figure-spartanrpc}, is that
responsibility for authorization on the basis of public key credentials is offloaded from the
WSN to a Scalaness program running on a hub device or lab computer. We assume a WSN comprising
two subnetworks under control of distinct security domains $A$ and $B$. Each domain also
controls a lab or hub device which is in communication with WSN nodes in their domain, either
prior to or during deployment. These devices are in communication with each other over the
Internet, and exchange authorization credentials for their domain over that medium in the
first-stage Scalaness program. Each device then confirms authorization for resource access
according to their own domain's policies, and subsequently they negotiate session keys over the
Internet. These keys are then used to specialize nesT code for imaging on WSN nodes. The overall
architecture of this application represents a concrete realization of the ideas of
\autoref{figure-scalaness}, and also expands on and implements the idealized example presented
in \autoref{section-overview}.% In
% particular, each node can be specialized with the keys it uses to access resources on
% neighboring nodes, and the keys it requires for its own resource access.

% Of course, whether nodes are programmed in the lab or over the air (OTA) \emph{in situ} from a
% hub device is a signficant concern. In the latter case, it is clearly insecure to transmit
% session key material over the air in plaintext. However, as we have already discussed, the
% Deluge protocol can be used for OTA program dissemination, and extensions to Deluge for secure
% OTA reprogramming have been developed \cite{XXX} that could be applied here.

Note that our current implementation assumes nodes are programmed in the lab since we have no
secure OTA program dissemination library; the Deluge protocol has such an extension
\cite{Dutta:2006:SDN:1127777.1127826} that we plan on using in the future.

\spartanrpcfig

\paragraph{Evaluation on Snowcloud} To empirically evaluate the staged implementation of
SpartanRPC in Scalaness/nesT, we have implemented and tested both the original SpartanRPC as
well as the Scalaness/nesT staged version in our deployed Snowcloud WSN system architecture.
% The sensor node software in this system is complex and uses virtually every available hardware
% feature, so it is a good test setting.
Mobile gateway devices as pictured in \autoref{figure-snowcloud} are provided to Snowcloud
system users for data gathering, and are also used by system administrators for controlling
sampling rates. The hardware for both of these so-called ``harvester'' devices, the same for
users and administrators, is equipped with a mote for establishing network communication. When
the device is introduced to the sensor network, the two together comprise a single network with
two distinct security domains -- the sensor node subnetwork, and the subnetwork of the single
device mote. The mote on harvester devices provided to system users is supplied with credentials
for collecting data, but not modifying network control, whereas system administration harvester
motes are supplied with stronger credentials for both functions.

The original and Scalaness/nesT versions of this application can be compared both in terms of
performance and user experience. In the unstaged version, the SpartanRPC protocol requires an
initial network configuration period when credentials are exchanged and verified. Since a single
TinyECC signature requires at least 90 seconds to verify on the Crossbow TelosB platform
\cite{chapin-skalka-SpartanRPCTR} with a fully dedicated processor, there is an initial network
``warmup'' period of at least a few minutes. Also, in the unstaged version, upon first
invocation of an RPC service Diffie-Hellman is used in the network to negotiate a session key.
In the staged version, credential exchange, validation, and session key negotiation are all
performed on the high-powered hub. For this reason, mote code size in ROM is significantly
reduced. There are differences in RAM usage as well, due to authorization overhead in the
unstaged version and also the storage of key material in RAM vs.~ROM, since specialization of
code with key material in the staged version allows the latter. Note that this difference is
intensified by scale and the number of keys (i.e.~RPC services) needed by an application. Lower
RAM and ROM usage can have significant performance impacts on deployed code. In the following
table we summarize RAM and ROM usage for the harvester and sensor node images for three software
versions: one with no security mechanisms in place, one with unstaged SpartanRPC protocols in
place, and one generated by Scalaness evaluation in our staged version of the SpartanRPC
protocol. \medskip

\noindent
\hspace{-5pt}
{\small
\begin{tabular}{r@{\ }r}
 & 
\begin{tabular}{|r||c|c|c|c|}
\hline
& & & & \\[-7pt]
\phantom{RAM} & Unsecured & Unstaged & Staged & Savings\\
\hline\hline
\end{tabular}
\\
\emph{Sensor:} & 
\begin{tabular}{|r||c|c|c|c|}
& \phantom{Unsecured} & \phantom{Unstaged} & \phantom{Staged} & \phantom{Savings}\\[-7pt]
ROM & 36254 & 48616 & 36596 & 25\% \\
RAM & 2868 & 5417 & 3038 & 44\%\\
\end{tabular}
\\[2mm]
\emph{Harvester:} & 
\begin{tabular}{|r||c|c|c|c|}
\hline
%& & & \\[-2.5mm]
% & Unsecured & Unstaged & Staged \\
%\hline
%\hline
& \phantom{Unsecured} & \phantom{Unstaged} & \phantom{Staged} &\phantom{Savings}\\[-7pt]
ROM & 24316 & 35834 & 24436 & 32\%\\
RAM & 2274 & 4771 & 2402 & 50\%\\
\hline
\end{tabular}
\end{tabular}}
\medskip
\medskip

The ``Savings'' are the percent reduction from unstaged to staged secure implementation, and
these numbers show the potential for saving both RAM and ROM space is significant. From the
perspective of user experience, the staged version of this application is more convenient, since
no initial authorization period is needed when the harvester is first introduced to the network.
The staged version also exposes the system to fewer bugs and failures that would be obstacles to
the primary goal of data collection.
