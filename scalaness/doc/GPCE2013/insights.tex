\section{Scalaness/nesT Foundations} 
\label{section-framedml}

The Scalaness/nesT type system design is based on principles studied in the foundational
calculus \fml\footnote{Pronounced ``framed ML.''} \cite{FramedML}. \fml\ comprises \fsub, state,
dynamic type construction, and staging features. In this section we describe how the design of
modules and module operations in Scalaness can be modeled in \fml. Although the correspondence
is informal, these models directed the design of Scalaness semantics and type checking, and
provide confidence in its soundness. While our choice of modules as the basic unit of nesT code
is based on obvious software engineering concerns and the need for a tight relation with nesC,
Scalaness modules are well correlated with certain structures in \fml\ and so are also
technically appealing.

\paragraph{The model of a module.} Code as a datatype is available in \fml\ as expressions of
the form $\langle e \rangle$. While code as a datatype is a standard feature of
staged/generative programming, \fml\ has adapted staged programming to a setting where different
code levels are intended for execution on different machines with distinct process spaces. In
particular, values, including code values, must be closed. If a type or term variable occurs
free in $\fmlcod{e}$, it must be $\Lambda$ or $\lambda$ bound, respectively, for closure. Hence,
if a type variable $t$ is free in $\fmlcod{e}$, then $\Lambda t \subtype \t . \fmlcod{e}$ binds
it, and provides parametric subtyping polymorphism for \fml\ terms.

If the term variable $x$ is free in $\fmlcod{e}$, then $\lambda x : \t. \fmlcod{e}$ binds it.
Furthermore, the type $\t$ in the term $\lambda x : \t. \fmlcod{e}$ \emph{must} be of the form
$\fmlcod{\s}$, because the type discipline requires that $x$ is of code type, since it occurs
within code. If the programmer wishes to pass a value residing at the current execution stage to
such a function, it must be explicitly ``lifted'' in the now-standard style of
\cite{Taha-MetaML}. However, in \fml, lifting a value entails serialization of it, which is
non-trivial in case the value is stateful.

We use \fml\ type and term bindings to model Scalaness type and term parameters. This is a
standard strategy, in fact FGJ typing \cite{FJ} is based on it as well. Hence the basic analog
of a module is:
$$
\Lambda t \subtype \t . \lambda x : \fmlcod{\s} . \fmlcod{e}
$$
where $t$ is a bounded type parameter and $x$ is a value parameter.

\paragraph{The model of instantiation.} Most of the interesting parts of Scalaness typing happen
at instantiation. Given the above model of a module, the \fml\ analog of instantiation is a term
of the form:
$$
(\Lambda t \subtype \t . \lambda x : \fmlcod{\s} . \fmlcod{e})(\t')(\text{lift}\ v)
$$
where all parameters are instantiated. Note in particular that the value parameter $v$ must be
explicitly lifted, since the model must reflect that values passed in to modules are always
constructed at the first stage in a Scalaness program. This means that $v$ must be assumed to
not be a code value, while the type annotation on $x$ requires that it be lifted. There is no
explicit lift operation in Scalaness, but the DSCalaness semantics
(\autoref{figure-jmodsemantics}) specifies that serialization is always implicit at module
instantiation. Scalaness typing of instantiation thus treats value instantiation as $\lambda$
application with implicit lifting of the argument, and type instantiation as $\Lambda$
application, i.e.~a form of bounded $\forall$-elimination.

\emph{Type construction and variable escape.} A central technical novelty and core feature of
DScalaness is dynamic type construction for module instantiation. As we discussed in
\autoref{section-overview}, this feature is technically challenging since constructed types can
escape their scope of declaration. Similarly, in \fml, types may be dynamically constructed that
can escape their declaration scope, in particular if they are used as function type annotations.
An $\exists$ type binder was introduced in \fml\ for this purpose; intuitively a type of the
form $\exists t \subtype \t . \s$ is a type containing a dynamically constructed type term $t$
with upper bound $\t$. \fml\ includes a ``tlet'' expression form for constructing types, so for
example:
\begin{eqnarray*}
&\text{tlet\ } t \subtype \finttwo = \metaite{e}{\fintone}{\finttwo} 
\text{\ in\ }(\lambda x : t . x) \ \ : \\
&\exists t \subtype \finttwo . t \rightarrow t
\end{eqnarray*}
Here a type $t$ is dynamically constructed to be either $\fintone$ or $\finttwo$, and then used
in the type annotation of a type-specialized identity function. Furthermore, $t$ escapes its
declaration scope since it annotates a function argument. Since $e$ is some arbitrary
computation, we cannot statically predict what $t$ will be, other than ``some type with upper
bound $\finttwo$''. Note also that since $t$ can appear in contravariant positions, it is
unsound to perform a covariant substitution of $\finttwo$ for $t$, so the $\exists$ bound is
needed. Although this usage of $\exists$ types is somewhat non-standard, an eigenvariable
interpretation of the bound type variable is sound and also consistent with standard existential
type interpretations.

Inspired by these foundations, in DScalaness the type form:
$$
\jmodt{\tpdecl_1}{\margs{\tpdecl_2;\vpdecl}\{\imports;
  \exportsty\}}
$$ 
captures the same typing mechanisms, in particular the instance coercion $\tpdecl_1$ is the
analog of $\exists$ bound type variables, in contrast to the type parameters $\tpdecl_2$ which
are implicitly $\forall$ bound, as discussed above. The static semantics of $\tpdecl_1$ and
$\tpdecl_2$ are distinguished appropriately, especially in the treatment of the typing rules for
module instantiation and module wiring in \autoref{figure-scalanesstyping}.

%For example, putting together the pieces of parametric polymorphism
%and dynamic type construction discussed above, let $\mu$ be a 
%module defined as follows:
%$$
%\margs{t \subtype \intt; x : \intt}\{ \mathit{id} : t = \blok{(t)x}\}
%$$
%The following typing can then be assigned in DScalaness:
%$$
%\mu \quad : \quad \jmodt{\varnothing}{\margs{t \subtype \intt; x : \intt}\{\varnothing; \mathi%t{id} : \blok{t}\}}
%$$ where $t$ is here serving a similar purpose to the $\Lambda$-bound
%$t$ in the \fml\ example above.  Then, the following module
%instantiation could be performed with a dynamically constructed type:
%$$
%\jtlet{t}{\inttwo}{e}{\mu\margs{t;1}}
%$$ Assuming that $e$ is a (well-typed) DScalaness expression which
%chooses a type, this whole expression can be assigned the following
%type:
%$$
%\jmodt{t \subtype \inttwo}{\margs{\varnothing; \varnothing}\{\varnothing; \mathit{id} : \blok{%t}\}}
%$$ Observe that the type and value parameters are eliminated in this
%typing due to the module instantiation. But an instance coercion $t
%\subtype \inttwo$ is introduced, to represent the type of the
%dynamically constructed type instance which in this case is more
%refined than the type parameter bound (since $\inttwo \subtype
%\intt$), though it must always be at least as refined.
