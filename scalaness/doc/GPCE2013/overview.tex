\section{An Example: Authenticated Messaging}
\label{section-overview}

In this section we provide a high-level overview of Scalaness/nesT usage and applications via an
example. (This example is written in DScalaness/DnesT, a simplified formalization of the
implemented Scalaness/nesT, which is defined later in this paper.) The example illustrates both
the type and value specializations that can occur in our system. This example needs to be very
small to fit in this overview which means it alone is not complete evidence of the advantage of
mote code staging, but it should point the reader to the promise of the approach.
% added disclaimer above to address reviewer 2's criticism - SS

\paragraph{Program description.} To illustrate type specialization, we refine address size. It
is well-known that minimizing address space size in WSN message packets can obtain significant
energy savings by reducing message sizes, since each bit of transmission is known to consume
energy similar to 800 instructions \cite{tag}. However, WSNs are ``ad hoc'' precisely in the
sense that positions and densities of nodes in space are unpredictable, so ``minimal'' address
space is an environmental property, where minimality may need to be determined \emph{in situ}.

To illustrate value specialization, we define a DnesT code template that can be instantiated
with specific session keys for secure communication in a WSN. We imagine that the template is
instantiated on high powered hub or lab device, where session keys are generated. In previous
work it has been shown how symmetric key signatures can be used to support language based
resource authorization in WSNs \cite{chapin-skalka-SpartanRPC,chapin-skalka-SpartanRPCTR}. In
particular, communication between security domains in a WSN is mediated by credentials
implemented as keys, and nodes lying at domain frontiers can use different keys to send (to the
other domain) and receive (from the other domain) over secured link layer channels. Since it is
unpredictable where nodes will be physically distributed in space, appropriate keys for each
node need to be established \emph{in situ}. Defining node functionality using generic code that
must be instantiated with specific keys allows adaptation to a deployment environment, and
allows expensive computations for establishing session keys to be offloaded from the WSN to a
higher powered device. Experience with an actual implementation of this application is discussed
\autoref{section-application}.

\paragraph{The Code.} To distinguish Scalaness and nesT code in examples we will use a darker
font for Scalaness code and a lighter font for nesT code, and line numbers for reference. We
begin with the definition of a parameterized type $\tt{mesgT(t)}$ using the Scalaness
$\tt{abbrvt}$ binder, where an instance $\tt{mesgT}(\t)$ denotes the ground type obtained by
substituting $\t$ for $\tt{t}$ in the definition of $\tt{mesgT}$.

\lstset{numbers=left,
  numberstyle=\tiny, firstnumber=1, stepnumber=1, numbersep=0pt,
  basicstyle=\ttfamily} \lstset{escapeinside={(*@}{@*)}} {\scriptsize
\begin{lstlisting}
 (*@\tt{abbrvt\ mesgT(t)\  =}@*) { src : (*@\tt{t}@*); dest : (*@\tt{t}@*); data : uint8[] };  (*@\label{l:mesgt}@*)
\end{lstlisting}
}
\noindent Next, we define a type $\tt{radioT}$,
which is the type of nesT modules that provide an API to the radio. 
\lstset{numbers=left, numberstyle=\tiny, firstnumber=2, stepnumber=1,
  numbersep=0pt, basicstyle=\ttfamily}
\lstset{escapeinside={(*@}{@*)}}
{\scriptsize
\begin{lstlisting}
 (*@\tt{abbrvt\ radioT\  =}@*) < mt (*@$\subtype$@*) (*@\tt{mesgT}@*)(uint) > (*@\label{l:radiot}@*)
                 { export radio_x(mt*); 
                   import handle_radio_r(mt*); };
\end{lstlisting}
}
\noindent The nesT module language is a simplified version of the nesC component language. In
this example, any module of type $\tt{radioT}$ exports a $\tt{radio\_x}$ function for sending
messages, and imports a $\tt{handle\_radio\_r}$ function that allows received messages to be
handled in a user-defined manner. Both functions take message references as
arguments\footnote{For brevity the return type on all commands is omitted. In all cases it is
  the TinyOS error type $\tt{error\_t}$ }. Furthermore, the module is parameterized by the type
of messages $\tt{mt}$, where the address type is upper-bounded by 32-bit unsigned integer. Thus,
any module of type $\tt{radioT}$ can be dynamically specialized to a 32, 16, or 8 bit address
space by type instantiation. Module type parameters are always defined with brackets
$\tt{<...>}$.
 
Now we define another type $\tt{commT}$ which is the type of modules providing a QOS layer over
a specialized radio.
\lstset{numbers=left, numberstyle=\tiny, firstnumber=5, stepnumber=1,
  numbersep=0pt, basicstyle=\ttfamily}
\lstset{escapeinside={(*@}{@*)}}
{\scriptsize
\begin{lstlisting}
 (*@\tt{abbrvt\ commT\  =}@*) (mt (*@$\subtype$@*) (*@\tt{mesgT}@*)(uint)) (*@$\circ$@*)  
                 < > (*@\label{l:commt}@*)
                 { export send(mt*); 
                   import handle_receive(mt*); };
\end{lstlisting}} 
\noindent Although this type is also parameterized by a bounded message type $\tt{mt}$, as is
$\tt{radioT}$, the parameterization is subtly different syntactically and semantically, since
$\tt{commT}$ expects a program context where the radio has been specialized. Thus, in
$\tt{commT}$, $\tt{mt}$ is understood as being ``some'' type with an upper bound of
$\tt{mesgT(uint)}$ which occurs in the module signature, whereas the module itself has no
parameters to be instantiated-- note the empty instance parameter brackets $\tt{< >}$ in the
module type after the $\circ$ delimiter. This sort of type is needed in the presence of
\emph{dynamic type construction}, a useful Scalaness feature we exemplify below.

Next we define modules for sending and receiving messages that provide a layer of authentication
security over the radio.
\lstset{numbers=left, numberstyle=\tiny, firstnumber=9, stepnumber=1,
  numbersep=0pt, basicstyle=\ttfamily}
\lstset{escapeinside={(*@}{@*)}} {\scriptsize
\begin{lstlisting}
 (*@\tt{authSend\ \  =}@*) < mt (*@$\subtype$@*) (*@\tt{mesgT}@*)(uint);(*@\label{l:authsend}@*) sendk : uint8[],>  
             { import radio_x(mt*);
               export send(m : mt*) 
                      { radio_x(AES_sign(m, sendk)); } };

 (*@\tt{authRecv\ \  =}@*) < mt (*@$\subtype$@*) (*@\tt{mesgT}@*)(uint);(*@\label{l:authrecv}@*) recvk : uint8[] >  
             { import handle_recv(mt*);
               export handle_radio_r(m : mt*) 
                      { if AES_signed(m, recvk) 
                         handle_recv(m); } }; 
\end{lstlisting}
}
Observe that in the implementation of $\tt{send}$ in module \tt{auth\-Send}, messages are
signed with a key $\tt{sendk}$, whereas when messages are received they must be signed with a
possibly different key $\tt{recvk}$ before being passed on to the user's receive handler, as
specified in module \tt{authRecv}. These modules are parameterized by a message type $\tt{mt}$,
and also the $\tt{sendk}$ and $\tt{recvk}$ key values.

To generalize a technique for composing these modules with a radio to yield a module of type
\tt{commT}, that is abstract wrt neighborhood sizes, radio implementations, and session key
material, we define the Scalaness $\tt{authSpecialize}$ function as follows:
\lstset{numbers=left, numberstyle=\tiny, firstnumber=19, stepnumber=1,
  numbersep=0pt, basicstyle=\ttfamily}
\lstset{escapeinside={(*@}{@*)}} {\scriptsize
\begin{lstlisting}
 (*@\tt{def\ authSpecialize}@*) (*@\label{l:authspecialize}@*)
  (*@ \tt{(nmax : uint16, radioM : radioT, keys : uint8[][]) : commT }@*) { (*@\label{l:args}@*)
     (*@\tt{typedef\ adt \subtype uint = \ if\ (nmax  \le 256)\ uint8\ else\ uint16;}@*)  (*@\label{l:typedef}@*) 
     (*@\tt{val\ sendM = \jinst{authSend}{mesgT(adt); keys[0]};}@*)  (*@\label{l:inst1}@*)
     (*@\tt{val\ recvM = \jinst{authRecv}{mesgT(adt); keys[1]};}@*)  (*@\label{l:inst2}@*)    
     (*@\tt{(sendM \ltimes \jinst{radioM}{mesgT(adt)}) \ltimes recvM;}@*) (*@\label{l:wire}@*)
    }
\end{lstlisting}} 

The first-class status of nesT modules in Scalaness is apparent here. On line \ref{l:args} the
function is specified to take a module parameter $\tt{radioM}$ of type $\tt{radioT}$ among its
arguments, and to return a module of type $\tt{commT}$ as a result. It also takes an array of
keys as an argument, and on lines \ref{l:inst1} and \ref{l:inst2} it instantiates
$\tt{sendMesg}$ and $\tt{recvMesg}$ with the keys in the array. It also uses the type $\tt{adt}$
in the instantiations, which in line \ref{l:typedef} is dynamically constructed on the basis of
the input variable $\tt{nmax}$ which defines the needed address space size and bound using the
Scalaness $\tt{typedef}$ construct. This illustrates a key novelty of our system, the ability to
\emph{dynamically} set a type to use on a mote based on a decision made in the Scalaness
runtime. Since the value of $\tt{nmax}$ cannot be statically determined, the type analysis only
knows that $\tt{adt}$ is some subtype of $\tt{uint}$. Finally, on line \ref{l:wire} the
instantiated radio module is composed with the instantiated send and receive modules via the
Scalaness $\ltimes$ operator. The semantics of module composition here is standard
\cite{Cardelli-1997}; in a composition aka wiring $\mu_1 \ltimes \mu_2$, the exports of $\mu_2$
are connected to imports of $\mu_1$. The function result is a module of type \tt{commT}.

To obtain a module defining a mote OS image in a program context where neighborhood size is
known, a radio implementation has been provided, and session keys have been computed. We can
then compose the results of an \tt{authSpecialize} function with modules specifying top-level
message send and receive behaviors, and a \texttt{main} application entry point as follows (here
we assume it is known that address sizes can be limited to 8 bits, so \tt{nmax} < 256). At line
\ref{l:image} a closed module is defined and a binary mote image can be produced by a call to
\tt{image}. \lstset{numbers=left, numberstyle=\tiny, firstnumber=26, stepnumber=1,
  numbersep=0pt, basicstyle=\ttfamily}
\lstset{escapeinside={(*@}{@*)}} {\scriptsize
\begin{lstlisting}
 (*@\tt{appMR\ =}@*) 
  < > { export handle_recv(m : (*@\tt{mesgT}@*)(uint8)*) {(*@\ldots@*)} }; 
 (*@\tt{appM\ =}@*) 
  < > { import send((*@\tt{mesgT}@*)(uint8)*); export main() {(*@\ldots@*)} };  
 (*@\tt{image(appM \ltimes (authSpecialize(nmax, radioM, keys) \ltimes appMR)); } @*) (*@\label{l:image}@*)  
\end{lstlisting}}
\noindent In DScalaness, \tt{image} is an assertion that its argument is a \emph{runnable}
module, with no unresolved parameters or imports. In the Scalaness implementation, this is the
point where nesT source code is actually generated.
% As future work, this is also a natural point to integrate automated code deployment in the
% Scalaness runtime.
Successful Scalaness/nesT type checking (which occurs during stage 1 compilation as per
\autoref{figure-scalaness}) statically guarantees that specialized code generated at the point
of \tt{image} will run in a type-safe manner when it is eventually loaded and run on a mote.
