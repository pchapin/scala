\section{The nesT Language Distilled}
\label{section-nest}

In this section we summarize a \textit{D}istilled version of nesT, called DnesT, that isolates
novel elements of nesT, specifically parametric types, subtyping, type safety, and modules.
DnesT serves as a formal specification for the nesT implementation---given the novel type theory
a specification is crucial as a guide for the implementation.
% space saver cut SS
% , and DnesT also serves as documentation for the language design.
Due to limited space we summarize only the top-level structure of DnesT modules and our type
checking algorithm, in order to focus more on the more central technical issues of module
composition, instantiation, and typing at the Scalaness level.
% space saver, this point made in next pgph. SS
% The DnesT language syntax is a reduced version of C
% which is largely standard.

The goal of nesT is to be a type-safe variant of nesC, and DnesT serves as the specification for
how type safety is achieved. Our approach is another species of ``safe C'' language design
projects such as \cite{ccured02}. In particular, in DnesT all array bound accesses are checked
at run-time, and pointer arithmetic and casting are restricted to safe forms only. We have
developed a new type checking algorithm that incorporates subtyping, which supports bounded type
parameters in DnesT module definitions and a more accurate static analysis of Scalaness code in
the presence of type construction and nesT module instantiation.

\subsection{Syntax and Semantics of DnesT}

Module definitions rely on a notion of lists aka sequences of syntactic entities, so we begin
with a definition of relevant notation.

\paragraph{Notation and identifiers.} \emph{Sequences} are notated $x_1,\ldots,x_n$, and are
abbreviated $\vect{x}$; $\vect{x}_\mapidx{i}$ is the $i$-th element, $\emptyset$ denotes the
empty sequence, and $\abs{\vect{x}}$ is the size. We write $x \in \vect{x}$ to denote membership
in sequences, and $x\vect{x}$ denotes a sequence with head $x$ and tail $\vect{x}$. We denote
append as $\vect{x}@\vect{y}$. For relational symbols $R \in \{ \subtype, =, : \}$, we use the
abbreviation: $\vect{x}\, R\, \vect{y} = x_1\, R\, y_1, \ldots, x_n\, R\, y_n$. So for example,
$\tbindvec{x}{\t} = x_1 : \t_1,\ldots,x_n : \t_n$.
% space saver SS
%
%Syntactic sorts of identifiers are partitioned as follows. 
We will use metavariable $\fname$ (of set $\mathcal{F}$) for function names, $\fdname$ (of set
$\mathcal{L}$) for field names, $\VAR$ (of set $\mathcal{V}$) for term variables, $\TVAR$ (of
set $\mathcal{T}$) for type variables.

\modulesyntaxfig

\paragraph{Module syntax.} The syntax of DnesT modules is defined in
\autoref{figure-modulesyntax}. Modules $\mu$ are written $\margs{\tpdecl; \vpdecl}\lc \imports;
\decls \exports \rc $ with $\tpdecl$ and $\vpdecl$ being generic type and term parameters,
$\vect{d}$ being module scope identifier declarations, including function definitions, and
$\imports$ and $\exports$ being imports and exports. In \autoref{section-overview} and elsewhere
we use the keywords $\texttt{import}$ and $\texttt{export}$ as sugar indicating this
categorization.
% space saver, shunk this point SS
% in module and module type
% definitions that do not exist in the syntactic definition, but merely
% make explicit the categorization of module elements.

All type parameters are assigned an upper bound, and term parameters are explicitly typed.
Imports and export types are sequences of imported and exported command type signatures. Exports
are sequences of command definitions. Exports are defined in terms of expressions $e$, the
syntax of which we omit here for brevity. Declarations $\decls$ are a sequence of typed variable
declarations. Base values, arrays (in brackets $\arr{\!\cdot\!}$), structs (in braces
$\{\cdot\}$), and commands may all be declared, and the scope of declared variable names is
restricted to the module. Declarations are important to include in DnesT, as they support
serialization of value parameters during Scalaness instantiation as we describe in
\autoref{section-serialization}.

While we have elided the specifics of DnesT syntax from this shortened presentation, we now give
a high-level summary of its largely standard features. Expressions include standard C-like
conditional branching, looping, sequencing of expressions, function calls, arrays, structs,
numeric base data types and basic arithmetic operations. As in nesC, no dynamic memory
allocation is possible; all memory layout is established by static variable declarations. DnesT
disallows pointer arithmetic, to support stronger type safety guarantees. Type casting and array
access have run time checks imposed: types may never be cast to a pointer, and array accesses
are always checked to be in bounds at runtime. As in nesC, DnesT includes a $\kwpost$ operation
for posting tasks, although we make no syntactic distinction between tasks and commands. The
meaning of $\kwpost$ corresponds to the ``run-to-completion'' model of TinyOS tasks. Interrupts
are omitted from DnesT since they do not significantly affect the typing issues we are concerned
with here.

\paragraph{Module semantics} A ``runnable'' module---one without imports or generic
parameters---is the DnesT model of a node OS image. The declarations in the module defines a
\emph{load sequence} establishing an initial machine configuration, and the application entry
point is defined in a required command $\tt{main}$.
\begin{definition}
\label{def-runnable}
A module of the form $\margs{\varnothing; \varnothing }\lc ;
\vect{\decl}; \exports\rc$,  where $\tt{main}() : \undefv \in 
\exports$, is \emph{called runnable}.
\end{definition}
This model is consistent with nesC, where an application is defined as a top-level component
that establishes an initial configuration through variable declarations, and requires user
definition of an entry point (an event handler called $\tt{Booted}$). Formally speaking, type
safety in nesT is a dynamic property of runnable modules.

\subsection{Type Checking and Subtyping}
\label{section-netttyping}

The type system for DnesT combines a standard procedural language typing approach with subtyping
techniques adapted from previous foundational work \cite{FramedML,Ghelli199875}.

At the heart of our system is a decidable subtyping judgment $\subjudge{\tpdecl}{\t_1}{\t_2}$,
where $\tpdecl$ is a \emph{coercion} and defines a system of upper bounds for type variables.
This establishes a subtype ordering on base types, and also allows for width subtyping of
records. The relation is defined in \autoref{figure-subjudge}. Algorithms for deciding the
relation and integrating it with dynamic type construction and other Scalaness (stage 1) type
features was a central topic of \cite{FramedML}.

\subjudgefig

The type checking algorithm for DnesT expressions is a combination of standard procedural type
systems and standard subtyping systems. Module typing is obtained by type checking module
exports, using a coercion obtained from the module type parameters and a typing environment
obtained from a combination of module value parameters, imports, and variable type declarations.
A valid module type checking judgment is written as:
\begin{mathpar}
\margs{\tpdecl, \vpdecl}\lc \imports; \vect{\decl}; \exports \rc : 
  \margs{\tpdecl, \vpdecl}\lc \imports; \exportsty \rc
\end{mathpar}
Where $\exportsty$ is just the type signatures of $\exports$, and each of the command bodies in
$\exports$ is proven to respect its type signature.

\begin{exmp}
\label{example-nesttyping}
The module $\tt{authSend}$ defined in \autoref{section-overview} code line \ref{l:authsend} can
be assigned the following type in DnesT:
\end{exmp}
\vspace{-2mm}
\lstset{numbers=none, basicstyle=\ttfamily} 
\lstset{escapeinside={(*@}{@*)}}\begin{lstlisting}
 < mt (*@$\subtype$@*) mesgT(uint); sendk : uint8[] >  
  { import radio_x(mt*), export send(mt*) }
\end{lstlisting}
