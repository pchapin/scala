
\special{papersize=8.5in,11in}
\setlength{\pdfpageheight}{\paperheight}
\setlength{\pdfpagewidth}{\paperwidth}

\conferenceinfo{GPCE~'13}{October 27--28, 2013, Indianapolis, Indiana, USA} 
\copyrightyear{2013} 
\copyrightdata{978-1-4503-2373-4/13/10} 
\doi{2517208.2517217}
% The full doi URL is: http://dx.doi.org/10.1145/2517208.2517217
% However, the template document seems to indicate a format of 7 digits dot 7 digits.

% Uncomment one of the following two, if you are not going for the 
% traditional copyright transfer agreement.

\exclusivelicense                % ACM gets exclusive license to publish, 
                                 % you retain copyright

%\permissiontopublish             % ACM gets nonexclusive license to publish
                                  % (paid open-access papers, 
                                  % short abstracts)

%\titlebanner{banner above paper title}        % These are ignored unless
%\preprintfooter{short description of paper}   % 'preprint' option specified.

\title{Scalaness/nesT: Type Specialized Staged Programming for Sensor Networks}

\authorinfo{Peter Chapin}{University of Vermont}{pchapin@cs.uvm.edu}
\authorinfo{Christian Skalka\thanks{This author's work was supported by a YIP grant from the 
Air Force Office of Scientific Research (AFOSR).}}{University of Vermont}{skalka@cs.uvm.edu}
\authorinfo{Scott Smith}{The Johns Hopkins University}{scott@cs.jhu.edu}
\authorinfo{Michael Watson}{University of Vermont}{mpwatson@cs.uvm.edu}

\maketitle

\begin{abstract}
  Programming wireless embedded networks is challenging due to severe limitations on processing
  speed, memory, and bandwidth. Staged programming can help bridge the gap between high level
  code refinement techniques and efficient device level programs by allowing a first stage
  program to specialize device level code. Here we introduce a two stage programming system for
  wireless sensor networks. The first stage program is written in our extended dialect of Scala,
  called Scalaness, where components written in our type safe dialect of nesC, called nesT, are
  composed and specialized. Scalaness programs can dynamically construct TinyOS-compliant nesT
  device images that can be deployed to motes. A key result, called cross-stage type safety,
  shows that successful static type checking of a Scalaness program means no type errors will
  arise either during programmatic composition and specialization of WSN code, or later on the
  WSN itself. Scalaness has been implemented through direct modification of the Scala compiler.
  Implementation of a staged public-key cryptography calculation shows the sensor memory
  footprint can be significantly reduced by staging.
\end{abstract}

%\category{CR-number}{subcategory}{third-level}
\category{D.3.4}{Programming Languages}{Processors}---Compilers

% general terms are not compulsory anymore, 
% you may leave them out
%\terms
%term1, term2

\keywords
Staged programming; Scala; nesC
