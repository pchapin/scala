\section{Introduction}

A wireless sensor network (WSN) is a network of small nodes, also called \textit{motes},
equipped with sensors or actuators, and that communicate with each other via short range radio
links. Programming WSNs is challenging because the nodes are severely resource constrained in
terms of memory and processor speed. This paper describes a programming language designed to
support the automatic generation of runtime-efficient code for WSN nodes. The language enables
\emph{dynamic specialization} of node code on a nearby hub or other more resource-rich device,
allowing adaptation to properties of a node's deployment environment such as neighborhood
characteristics, network interference factors, \emph{etc}.

Our system supports dynamic generation of programs for the TinyOS operating system
\cite{TinyOS}, a popular platform for WSNs. It features programming abstractions for
specializing WSN code, allowing on-the-fly adaptation to current WSN deployment conditions. We
use a restricted form of \emph{staging} \cite{Taha-MetaML,DBLP:conf/icess/Taha04,289140} to
achieve well founded dynamic program generation. \emph{First stage} code is written in an
extended version of Scala \cite{PiS2}, called Scalaness. Scalaness program execution yields a
residual \emph{second stage} WSN node program written in nesT, a variant of the popular nesC
programming language \cite{Gay-nesC-2003} with a stronger type checking analysis. The second
stage program is constructed from module components treated as first class values, which may be
\emph{type and value specialized} during the course of first stage computation to yield more
compact and efficient code. A code rewriting strategy in the implementation transforms nesT code
into nesC code, which can be compiled using standard TinyOS tools.

While staging is well-studied and has been explored in a WSN context \cite{Mainland-Flask-2008},
our work is novel in that we achieve stronger static safety guarantees than previous work. At
the point of Scalaness program compilation, our compiler can statically verify that any nesT
program produced by the Scalaness runtime will be statically type-safe when deployed and run on
a network node, even if module parameters are specialized during the course of nesT module
composition. We call this property \emph{cross-stage type safety}, which has been previously
studied in a foundational language context \cite{FramedML}. In this paper we apply these
concepts to the more practical Scalaness/nesT language, and illustrate how they support the
implementation of efficient real-life WSN applications.

\subsection{Application Setting and Contributions}

\scalanessfig

The diagram in \autoref{figure-scalaness} provides an overview of the Scalaness/nesT language
architecture. Scalaness source code is compiled in a modified Scala compiler to Java bytecode,
and run in a standard JVM. At runtime this Scalaness program may generate nesT code, which is
subsequently rewritten to nesC and compiled using the standard TinyOS compiler. The resulting
image can then be installed on nodes in a WSN.
% space saving cut. SS
%
% Observe that more than one code image can be generated during program execution, so code can
% be specialized for each node in a network, or a single Scalaness program can refine and
% redeploy network code, allowing programmatic specification of evolving network behavior.

Another interesting feature of our intended application setting, captured in
\autoref{figure-scalaness}, is the physical platform on which different elements of the
Scalaness/nesT ``workflow'' may be executed. Scalaness source code will typically be compiled in
the lab, prior to deployment. There are two distinct deployment scenarios where compiled
bytecode execution, TinyOS image generation, and mote (re)programming (the rightmost two boxes
in \autoref{figure-scalaness}) can occur. Clearly these activities can take place in the lab,
where WSN motes can be easily imaged over e.g.~USB connections prior to deployment. But the more
interesting scenario we aim to support is generation of TinyOS images on a ``hub'' device
\emph{in situ}, and then to automatically reprogram WSN nodes over the air (OTA) from that hub.

In WSN applications such as our Snowcloud snow telemetry system
\cite{moeser-walker-skalka-frolik-wsc11,frolik-skalka-realwsn13}, sensor motes report data to
higher powered hubs, pictured in \autoref{figure-snowcloud}. The hub device in the figure uses a
low-powered embedded Linux platform that supports a JVM, and is in direct radio communication
with the sensor network via a physically attached mote. These types of devices are readily
available and cheap. Such a system can execute compiled Scalaness code, and compile and deploy
generated TinyOS images to nearby motes. Since the hub is in communication with the network,
Scalaness/nesT is uniquely positioned to evolve network behavior based on a global view of
observed data, a technique called \emph{backcasting} \cite{Willett:2004:BAS:984622.984641} when
used specifically for network control. In this context the benefit of cross-stage type safety is
clear: type-checked Scalaness compilation in the lab ensures type safety of bytecode execution
on the hub, \emph{and} type-safety of dynamically generated TinyOS image execution on the WSN.
Manual correction of type errors in generated TinyOS images in this scenario is infeasible since
these systems run automatically in remote settings.

\snowcloudfig

\paragraph{Paper Outline.} The main contributions of the work presented here are the
specification, implementation, and proto\-type application of the Scalaness and nesT languages,
including their syntax, semantics, and type analysis. In \autoref{section-overview} we summarize
Scalaness/nesT via discussion of an extended example. Formal specifications of the nesT and
Scalaness languages are presented in \autoref{section-nest} and \autoref{section-scalaness},
respectively. Their semantics and type theory possess several novel and challenging features,
which we show in \autoref{section-framedml} are grounded in principles studied in a previous
foundational setting \cite{FramedML}. We describe our implementation and present an extended
example application of our system to resource access control in WSNs in
\autoref{section-implementation} and \autoref{section-application}, along with some empirical
results demonstrating efficiency benefits of our approach. We conclude with remarks on related
work in \autoref{section-conclusion}

% following gets in the way of flow in the intro. If its important could be put in conclusion,
% but I think its OK to just leave out -SS
%
% Note that certain elements of \autoref{figure-scalaness} are not addressed in our current
% implementation, in particular we have not integrated OTA reprogramming nor a WSN communication
% API into the Scalaness runtime. However, these are more routine engineering challenges outside
% the scope of this paper.
%
% the more text we write on a shortcoming the worse reviewers conclude it really is -SS
% , since existing TinyOS Java
% APIs for WSN communcation can easily be integrated, and protocols for
% OTA reprogramming in WSNs are well-studied \cite{XXX}.

